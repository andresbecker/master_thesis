%% Magic command to compile root document
% !TEX root = ../../thesis.tex

%% Reset glossary to show long gls names
\glsresetall

Nothing new here, only a short recap of the project, it's results, as well as possible future work.

%% Future work
% interpretability
As future work, it would be interesting to explore other interpretability  methods or go more in-depth with the current one by, for instance, exploring different baseline for \acrlong{ig} and analyzing their impact on the importance maps.
% Perturbed dataset
It also would be interesting to extend this work to the perturbed dataset and to see how the pharmacological and metabolic perturbations change score maps. To see if instead of relaying on the shape of the nucleolus and/or splicing speckles, the \gls{cnn} looks at other subnuclear structures. It also would be interesting to see how the similarities between score map channels (or score maps channels and cell image channels) change.

% Dataset proportions
Run the analysis with a dataset that has the same number of cells in phase $G_1$, $S$, $G_2$, and see if the model accuracy improve. Also validate if the results of the score maps change.

% Dataset with less EU incubation time
Because it has been observed that the EU marker also binds to DNA molecules after some incubation time \cite{jao2008exploring}, \cite{bao2018capturing}, as a future work another \gls{mpm} dataset could be analyzed, either with a shorter or longer incubation time for the UE marker. Then, it would be interesting to validate if the results obtained with both datasets are consistent.

Hannah's recommendations:
\begin{itemize}
  \item Summarise previous chapters and list main results
  \item Depending on which suits your project better, you can also do the discussion in this chapter instead of in the results chapter
  \item Last paragraph of the discussion should include important outlook with future impact.
\end{itemize}
