\glsresetall
% define where the images are
\graphicspath{{./Sections/Basics/Resources/}}

Cells are considered the smallest unit of life. There are two types of cells, \textit{prokaryotic} and \textit{eukaryotic}. The main difference between these, is that prokaryotic cells do not contain nucleus and that that prokaryotes are considered single-celled organisms, while eukaryotes organisms can be either single-celled or multicellular. For multicellular organisms, like plants or mammals, eukaryotic cells are the \textit{building-blocks} of life. This work focuses on a process of eukaryotic cells. Therefore, in the subsequent when we refer to cells, we will be referring to eukaryotic cells only.

Multicellular organisms (like us) have different cell types, where each one of them can have many or an specific function. For instance, red blood cells are responsible for carrying the oxygen in the body. In order to carry as much oxygen as possible they lack a nucleus, and therefore they are unable to undergo \textit{mitosis}\footnote{Mitosis is the process through which eukaryotic cells reproduce themselves and give rise to new organisms.}.

However, there are also cells aimed to produce (\textit{synthesize}) certain substances that regulate process in our body. For instance, \textit{Alpha cells} are pancreatic cells responsible for synthesizing the \textit{glucagon} hormone, which elevates the glucose levels in the blood \cite{1e48f81ce88f4602a25a4ebbcea3a6e7}. The process in which cells produce this substances is called \textit{cellular expression} or \textit{gene expression}. The reason why this process is also called gene expression, is because the instructions to synthesize every substance (or any functional product, like hormones or proteins) are encoded in a specific gene\footnote{A \textit{gene} is defined as a region of the \textit{DNA} that encodes a function. DNA is contained in \textit{chromosomes}, which are long DNA strands containing many genes.}.

There are two key steps involved in gene expression, \textit{transcription} and \textit{translation}. Roughly speaking, transcription is the process in which the instructions to synthesize a product (like proteins) are copied from a gene in the DNA, to a single strand molecule called \gls{mrna}. On the other hand, Translation is the process in which the instructions in the \gls{mrna} are interpreted to produce a functional product. Figure \ref{fig:BB:tt} shows a simple representation of this process.

\begin{figure}[htb]
  \centering
  \includegraphics[width=\linewidth]{central-dogma-large.png}
  \caption{Simple representation of the gene expression process. Image source \cite{transcript_translation_diagram}.}
  \label{fig:BB:tt}
\end{figure}

The transcription process happens inside the cell nucleus, while translation happens in the \textit{ribosome} (outside the nucleus). The reason why transcription is necessary, is because the instructions needed to build a product are encoded in the DNA, which is inside the nucleus. Since DNA is too big to pass the membrane that covers the nucleus (nuclear envelop) to travel to the ribosome (which is the organelle in charge of building the product), the necessary instructions in the DNA are copied into a smaller strand (\gls{mrna}), which is now able to escape the nucleus and travels to the ribosome to start the translation process. Figure \ref{fig:BB:euka} shows a diagram of an eukaryotic cell and some of its parts. This work focuses on the transcription process and the factors that seed up or slow down this process.

\begin{figure}[htb]
  \centering
  \includegraphics[width=0.7\linewidth]{Animal_cell_structure_en.png}
  \caption{Animal eukaryotic cell diagram. Image source \cite{eukacell}.}
  \label{fig:BB:euka}
\end{figure}
