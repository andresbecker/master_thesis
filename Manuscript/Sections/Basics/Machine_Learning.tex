%% Magic command to compile root document
% !TEX root = ../../thesis.tex

%% Reset glossary to show long gls names
\glsresetall

%% Set path to look for the images
\graphicspath{{./Sections/Basics/Resources/}}

Road map. Here we only explain what is a NN and maybe mention the over and under fitting.
\begin{enumerate}
  \item What is a ANN
  \item Why this work (universal approx th.)
  \item what is a CNN
  \item training logic: Data -> training process -> prediction
\end{enumerate}

\glspl{ann} are universal approximators widely used in the field of \gls{ml} and an important part of this work. This is a very broad subject and there are entire books that cover this in detail, like  \cite{Goodfellow-et-al-2016} or \cite{bishop2006pattern}. However, in this section we will give a small introduction to \glspl{ann}, specially to a specific kind of \gls{ann} known as \glspl{cnn}.

Before defining what exactly is a \gls{ann}, lest first recall the definition of machine learning. We refer as \gls{ml} to the group of algorithms that automatically improve (learn) through experience. Among this algorithms, we could say that there are three main classes (which depend on the kind of experience we provide):

\begin{itemize}
  \item \textbf{Supervised Learning}: The experience is given in the form of input and output examples, and the goal is to learn a general rule that maps inputs to outputs.
  \item \textbf{Unsupervised Learning}: The experience is given in the form of data (no outputs provided) and the goal is to discover hidden patterns in data.
  \item \textbf{Reinforcement Learning}: No experience (data) is given, instead a dynamic \hl{environment} is provided and an \hl{agent} must learn how to interact with it in order to achieve a goal.
\end{itemize}

\glspl{ann} can be applied in any of the 3 kinds of learning algorithms listed above. However, \hl{supervised learning} is the one that best illustrates \glspl{ann} and it is also the kind of challenge we're dealing with. Recall that we seek to approximate a function (a \gls{cnn}), such that when it is fed with an images of a cell nucleus (input data), it can approximate the \gls{tr} (output data) of the corresponding cell.
