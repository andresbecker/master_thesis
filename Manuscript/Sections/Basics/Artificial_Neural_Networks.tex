%% Magic command to compile root document
% !TEX root = ../../thesis.tex

%% Reset glossary to show long gls names
\glsresetall

%% Set path to look for the images
\graphicspath{{./Sections/Basics/Resources/}}

% what is a ANN

Roughly speaking, an \gls{ann} is a non-linear function $f:\mathbb{R}^D \rightarrow \mathbb{R}^L$, that maps an input $\bs{x}\in\mathbb{R}^D$ with an output $\bs{y}\in\mathbb{R}^L$. Of course, to consider $f$ as a \gls{ann}, $f$ must have a specific form that will be address later. However, for the sake of this explanation, let us start by defining a simple function as follow

\begin{equation}
  \begin{split}
    f(\bs{x},\bs{w}) &:= h \left(w_0 + \sum_{j=1}^{M-1}w_j\phi_j(\bs{x}) \right) \\
    &= h(\bs{w}^T\bs{\phi}(\bs{x})) \\
    &:= h(z)
  \end{split}
  \label{eq:basics:slp}
\end{equation}

\noindent where $\bs{\phi}:\mathbb{R}^{D+1} \rightarrow \mathbb{R}^M$ is an element-wise function, with $\phi_0:=1$, know as \hl{basis function}, $h:\mathbb{R} \rightarrow \mathbb{R}$ is a function know as \hl{activation function} and $\bs{w}\in\mathbb{R}^M$ is the parameter vector. The parameters $w_j$, with $j\in\{1,\dots M-1\}$ are known as \hl{weights}, while the parameter $w_0$ is know as \hl{bias}.

Then, an \gls{ann} is composition of functions of the same form as \ref{eq:basics:slp}, with non-linear \hl{activation functions}, and where the basis functions are also of the same form as \ref{eq:basics:slp} (\cite{bishop2006pattern})

\begin{equation}
  F(\bs{x}, \bs{W}) :=
  h_K(\bs{w}^T_K h_{K-1}(\bs{w}^T_{K-1} \dots h_{0}(\bs{w}^T_0 \bs{x}) \dots ))
  \label{eq:basics:ann}
\end{equation}

The subscript in the parameter vectors $\bs{w}_k$ and the activation functions $h_k$, with $k\in\{0, \dots, K\}$, of \ref{eq:basics:ann} represents the depth of the layers. Note that unlike the other layers, the base function of the \hl{input layer} (k = 0) is the identity function. Furthermore, the activation function of the \hl{output layer} $h_K$ does not necessarily have to be non-linear. Instead, it is chosen based on the type of function we want to approximate. In our case, since we have a regression problem (predicting \gls{tr}), $h_K$ is chosen as the identity function.

There are different non-linear activation functions that can be chosen for the hidden units. However, all the models showed in this work use the \gls{relu}

\begin{equation}
  ReLU := max\{0, x\}
\end{equation}

Figure \ref{fig:basics:ann:relu} shows the \gls{relu} activation function.

% Figure made in notebook Preprocessing_resources.ipynb
\begin{figure}[!ht]
  \centering
  \includegraphics[width=0.6\linewidth]{ReLU.jpg}
  \caption{\gls{relu} activation function.}
  \label{fig:basics:ann:relu}
\end{figure}

Figure \ref{fig:basics:ann:ann} shows a graphical representation of a \gls{ann}. The circles represent the activation function applied to what is inside it. Black colored circles represent the identity function, red colored circles the non-linear activation function for the hidden layers, while green any function for the output layer that suits the problem we want to solve. Note that values inside the circles of the hidden and output layers $z^k_i$, for $k\in\{0, \dots, K\}$ and $i$ representing one of the units of the $k$ layer, are the output of a function of the same form as \ref{eq:basics:slp}. The lines connecting the circles represent the weights and biases corresponding to each layer $\bs{W}_k$, for $k\in\{0, \dots, K\}$. The circles in the \hl{hidden layers} are known as \hl{hidden units}.

\begin{figure}[!ht]
  \centering
  \includegraphics[width=0.8\linewidth]{Diagrams/ANN.jpg}
  \caption{Graphical representation of an \gls{ann}. The color of the circles represents the type of activation function. Black means the identity, red a non-linear function for the hidden layers and green any function for the output layer.}
  \label{fig:basics:ann:ann}
\end{figure}

Strictly speaking, equation \ref{eq:basics:ann} and figure \ref{fig:basics:ann:ann} represent a \hl{fully connected feedforward neural network}. However, in this work we will refer to it just as \gls{ann}, which in some literature is also known as \gls{mlp}.

\subsubsection{Update rule}

Sow far we have introduced the general form an \gls{ann} must have. Moreover, equation \ref{eq:basics:ann} shows that an \gls{ann} is simply a non-linear function controlled by a set of adjustable parameters $\bs{W}$. Therefor the question is, how can we approximate this parameters?

Recall that we are dealing with a supervised learning problem, which means that we can use both the input data (images of cell nucleus, $\bs{X}$) and the output data (the \glspl{tr}, $\bs{Y}$) to approximate $\bs{W}$. Therefore, we can fed the \gls{ann} with $\bs{X}$, and then measure its performance by comparing its output $\hat{\bs{Y}}$ against the true values $\bs{Y}$.

This comparison is made by means of a \hl{loss function} $\mathcal{L}$ that must be chosen beforehand. The choice of $\mathcal{L}$ depends mainly on the type of problem you are solving (regression, classification, etc.). However, even for each type, there are many different options. For now, let us just say that $\mathcal{L}$ should return high values when $\hat{\bs{Y}}$ is far from the true values $\bs{Y}$, and low when they are close.

Then, we can fit the values of $\bs{W}$, by minimizing the loss function $\mathcal{L}$ each time the model is fed with an input value $\bs{x}$. Since the gradient of $\mathcal{L}$ with respect to $\bs{W}$ (i.e., $ \nabla_{\bs{W}} \mathcal{L}$) returns the direction in which the loss function grows the fastest, then we choose $- \nabla_{\bs{W}} \mathcal{L}$ as the direction of our update rule

\begin{equation}
  \bs{W}_{new} = \bs{W}_{old} - \alpha \nabla_{\bs{W}} \mathcal{L}(\bs{W}_{old})
  \label{eq:basics:ann:learn_rule}
\end{equation}

\noindent where $\alpha \in \mathbb{R}^+$ (known as \hl{learning rate}) controls how much we move in the direction of $-\nabla_{\bs{W}} \mathcal{L}(\bs{W}_{old})$ on every step.

The iterative method in which \ref{eq:basics:ann:learn_rule} is applied over elements of $\bs{X}$ to optimize $\bs{W}$ is known as \gls{gd} \cite{bishop2006pattern}. However, in practice \ref{eq:basics:ann:learn_rule} is not applied for a single element of $\bs{X}$ every time, but to a random subset of $\bs{X}$ (known as a \hl{Batch}) instead. This process is known as \gls{sgd} and computationally is less  expensive than \gls{gd}.

However, \gls{gd} (\gls{sgd}) has a downside, the choice of its hyperparameter $\alpha$ (learning rate). In practice, it has been shown that the correct choice of $\alpha$ is essential to train an \gls{ann} successfully. Therefore, other algorithms (\hl{optimizers}) have been proposed to mitigate this problem. The revision of these optimizers is out of the scope to this work. However, all of them follow the same idea proposed by \gls{gd}. For example, instead of having a fixed learning rate $\alpha$ as in \gls{gd}, the \gls{adam} optimizer adapts its learning rate dynamically during training depending on the mean and variance of the loss function \cite{kingma2014adam}.

\subsubsection{Back propagation}

Nevertheless, there is still one question that needs to be answered, which is how to efficiently calculate the derivative of the loss function ($\nabla_{\bs{W}} \mathcal{L}$) with respect to all the parameters of the \gls{ann}. The answer to this is through an algorithm called \hl{backpropagation}, which is performed during the \hl{training process}. Again, there is a lot of literature that explains this in depth(for instance \cite{Goodfellow-et-al-2016} or \cite{bishop2006pattern}). Therefor, here we will just provide the intuition behind it.

Recall that $\mathcal{L}$ is a function of the true values $y$ and $\hat{y}$ i.e., $\mathcal{L}(y, \hat{y})$. Also from equation \ref{eq:basics:ann} and figure \ref{fig:basics:ann:ann} note that

\begin{equation}
  \begin{split}
    y &:= F(\bs{x}, \bs{W}) \\
    &= h_K(\bs{z}^K) \\
    &= h_K(\bs{W}_K^T h_{K-1}(\bs{z}^{K-1}))
  \end{split}
  \label{eq:basics:ann:backprop_1}
\end{equation}

and therefore

\begin{equation}
  \begin{split}
    \nabla_{\bs{W}_K} \mathcal{L} &= \frac{\partial \mathcal{L}}{\partial \bs{W}_K} \\
    &= \frac{\partial \mathcal{L}}{\partial \hat{y}}
    \frac{\partial \hat{y}}{\partial \bs{z^K}}
    \frac{\partial \bs{z^K}}{\partial \bs{W}_K} \\
  \end{split}
  \label{eq:basics:ann:backprop_2}
\end{equation}

\noindent which is just the product of the derivative of the loss function w.r.t. $\hat{y}$ (i.e., $\frac{\partial \mathcal{L}}{\partial \hat{y}}$), the derivative of the activation function of the output layer w.r.t the argument of the last layer (i.e., $\frac{\partial \hat{y}}{\partial \bs{z^K}}$) and the output of the layer $K-1$ (i.e., $\frac{\partial \bs{z^K}}{\partial \bs{W}_K}=h_{K-1}(\bs{z^{K-1}})$).

Note that we can easily compute the gradient of $\mathcal{L}$ w.r.t deeper parameters $\bs{W}_k$ (for $k\in\{0, \dots, K-1\}$), just by extending \ref{eq:basics:ann:backprop_1} and \ref{eq:basics:ann:backprop_2}.

This shows how by means of the \hl{chain rule}\footnote{$(f \circ g)'=(f \circ g) \cdot g'$, or equivalently $h'(x)=f'(g(x))g'(x)$, for $h(x):=f(g(x))$.}, the backpropagation algorithm can compute the gradient of the loss function w.r.t. a specific parameter, just by multiplying the derivative of the loss function, the derivative of the activation functions and some values computed during the evaluation of the \gls{ann}.

\subsubsection{Model development}

The properties of \glspl{ann} have been studied extensively before (\cite{cybenko1989approximation}, \cite{hornik1989multilayer}, \cite{funahashi1989approximate}) and established in the \hl{Universal approximation theorem}

\begin{theorem}[Universal approximation theorem]
  An \gls{mlp} with a linear output layer and one hidden layer can approximate any continuous function defined over a closed and bounded subset of $\mathbb{R}^D$, under mild assumptions on the activation function (\hl{squashing} activation function) and given the number of hidden units is large enough.
\end{theorem}

For this reason \gls{ann} are known as \hl{universal approximators}, since they are able to approximate any continuous function on a compact\footnote{A set $A$ in a metric space is said to be \hl{compact} if it is close (i.e., it contain all its limit points) and bounded (i.e., all its points lie within some fixed distance of each other) \cite{bartle2000introduction}.} input domain with an arbitrary accuracy \cite{bishop2006pattern}.

These means that, as long as a \gls{ann} has a sufficiently large number of hidden units, the loss function can be reduced as much as desired. However, this nice property can also lead to an unwanted one known as \hl{overfitting}.
Intuitively this means that the \gls{ann} \hl{memorize} the data used to train it (low error/bias), and therefore it is not able to perform (or \hl{generalize}) well when it is fed with new data (high error/bias and variance). This happens mainly when the \gls{ann} is optimized/fed too many times with the same data.

On the other hand, \hl{underfitting} means that the \gls{ann} performs poorly on both new data and data used to train the network (high bias and low variance). This usually happens when the training time is insufficient or the \gls{ann} is not complex enough (too few hidden units and/or layers).

Figure \ref{fig:basics:ann:fitting} shows synthetic data (blue circles), generated from a sine function (green line) and random noise sampled from a normal distribution.
The red line in figure \ref{fig:basics:ann:fitting:under} represents a fitted model with high bias and low variance (underfitting), while in figure \ref{fig:basics:ann:fitting:over} a model with low bias and high variance (overfitting). The red line in figure \ref{fig:basics:ann:fitting:good}, represents a model with low bias and variance (good fit and good generalization).

% this plots were extracted from page 7 (Bishop) and adapted in the file overfitting.odg
\begin{figure}[htb]
  \centering
  \begin{subfigure}[t]{.3\linewidth}
    \includegraphics[width=\linewidth]{Diagrams/underfitting.jpg}
    \caption{Underfitted model.}
    \label{fig:basics:ann:fitting:under}
  \end{subfigure}
  \vspace{3mm}
  \begin{subfigure}[t]{.3\linewidth}
    \includegraphics[width=\linewidth]{Diagrams/goodfit.jpg}
    \caption{Model with good fit and generalization.}
    \label{fig:basics:ann:fitting:good}
  \end{subfigure}
  \vspace{3mm}
  \begin{subfigure}[t]{.3\linewidth}
    \includegraphics[width=\linewidth]{Diagrams/overfitting.jpg}
    \caption{Overfitted model.}
    \label{fig:basics:ann:fitting:over}
  \end{subfigure}
  \caption{Representation of a model (red line) with underfitting \subref{fig:basics:ann:fitting:under}), good fit \subref{fig:basics:ann:fitting:good}) and overffiting \subref{fig:basics:ann:fitting:over}), trained over synthetic data (blue small circles). The synthetic data was generating by adding random noise to a sine function (green line) on the interval $[0,1]$. Image source \cite{bishop2006pattern}.}
  \label{fig:basics:ann:fitting}
\end{figure}

In practice, we seek to fit models that has low bias and low variance (i.e., good accuracy and good generalization). Therefore, to prevent overfitting we split the data into 3 different sets; \hl{training}, \hl{validation} and \hl{test}, and train the model using only the first set. Then, during model training, we measure how well the model is generalizing by comparing the value of the loss function when it is evaluated in the training and validation set \footnote{This is usually known as the \hl{bias–variance tradeoff}.}.
During the model development, the \hl{test} set is never evaluated and is only used at the end, to report the model performance. This methodology is shown in figure \ref{fig:basics:model_train_process}

Figure \ref{fig:basics:bias_variance} shows this \hl{bias–variance tradeoff} between training and validation set. In practice, multiple versions of the model are saved during training and then the one with the lowest validation error is chosen (red dot on figure \ref{fig:basics:bias_variance}).

% figure taken from BL_070121_0942.ipynb
\begin{figure}[!ht]
  \centering
  \includegraphics[width=0.8\linewidth]{bias_variance.png}
  \caption{Bias–variance tradeoff. In orange (respectively blue) the loss function curve when it is evaluated in the validation (respectively training) set. The red dot shows the lowest loss for the validation set.}
  \label{fig:basics:bias_variance}
\end{figure}

\begin{figure}[!ht]
  \centering
  \includegraphics[width=0.8\linewidth]{Diagrams/Model_methodologt.jpg}
  \caption{Model development methodology.}
  \label{fig:basics:model_train_process}
\end{figure}

The methodology shown in figure \ref{fig:basics:model_train_process} is also used to optimize the hyperparameters of the model, like the number hidden units/layers or the activation function of the hidden layers.

\subsubsection{Batch Normalization}

However, overfitting is not the only problem we may encounter when training an \gls{ann}. Training \gls{ann} with several layers can be complicates, since the distribution of the data can change from layer to layer. This means that the input and output distribution of a layer will not necessarily be the same. It has been empirically proven that this can affect the training performance, since it require the use of lower learning rates \cite{ioffe2015batch}. This can also lead to \hl{saturation}\footnote{\hl{Saturation} is a commonly used term to refer to the situation when the evaluation of a "squashing" function returns values close to some of its horizontal asymptotes most of the time. Remember that these "squash" functions (like \hl{Sigmoid} or \hl{tanh}) compress the real line $(-\inf, \inf)$ into an interval of finite length $(a, b)$.} of the activation functions, so a more careful initialization of the \gls{ann} parameters is required. To address this problem Ioffe et al. \cite{ioffe2015batch} proposed to normalize the layer inputs.

Roughly speaking, batch normalization consist of two main steps; 1) the standardization of the layer input and 2) the normalization of the standardized data. For the first step the layer input is standardized using parameters extracted from the \hl{batch}

\begin{equation}
    \bs{z}'_k := \frac{\bs{z}_k-\bs{\mu}_k}{\sqrt{\bs{\sigma}_k^2-\epsilon}}
\end{equation}
\noindent with
\begin{equation}
  \begin{split}
    \bs{\mu}_k &= \frac{1}{M}\sum_{m=1}^M \bs{z}_k \\
    \bs{\sigma}_k^2 &= \frac{1}{M}\sum_{m=1}^M (\bs{z}_k - \bs{\mu}_k)^2 \\
  \end{split}
\end{equation}

\noindent where $M$ is the \hl{Batch} size and $k$, with $k \in \{0 \dots K\}$, denotes the layer.

Note that for each layer $k$ we have different normalization parameters $\bs{\mu}_k$ and $\bs{\sigma}_k$. Moreover, this normalization parameters are vectors of the same shape as the layer size (i.e., one pair of normalization parameters per unit/neuron).

The second step in batch normalization consist on normalizing the standardized data $\bs{z}'_k$ using parameters $\bs{\gamma}_k$ and $\bs{\beta}_k$ learned during training

\begin{equation}
    \overset{\sim}{\bs{z}}_k := \bs{\gamma}_k \odot \bs{z}'_k + \bs{\beta}_k
\end{equation}

\noindent where $\odot$ denotes \hl{element-wise} multiplication. At the beginning of the training $\bs{\gamma}_k=1$ and $\bs{\beta}_k=0$ are used for all the layers and units.

During training, the normalization parameters of each epoch are stored, so the average ($\bar{\bs{\gamma}}_k$ and $\bar{\bs{\beta}}_k$) can be used during evaluation (when the model is not training).

\subsubsection{Residual Block V2}

As already mentioned, the \hl{Universal approximation theorem} guarantees that the training error can be reduced by adding more layer to an \gls{ann}. However, in practice it is not that simple. As we add layers to an \gls{ann}, the training becomes more unstable and difficult as we can face vanishing or exploding gradients (when the value of the gradients become very close to 0 or $\inf$ respectively during back propagation). To overcome this problem, He et al. (\cite{he2015deep} and \cite{he2016identity}) proposed the \hl{residual blocks}, which have been empirically shown to make deep \gls{ann} training more stable.
The core idea of residual blocks is to reformulate the layers as \hl{learning residual functions} with reference to the layer inputs, by adding an \hl{identity connection}. Then, if a layer is not longer beneficial to the \gls{ann} (e.g. in case of gradient vanishing), the \gls{ann} can just "skip" it. Figure \ref{fig:basics:residual_block} shows a diagram of the second version of a residual block \cite{he2016identity}.

\begin{figure}[!ht]
  \centering
  \includegraphics[width=\linewidth]{Diagrams/Residual_block_v2.jpg}
  \caption{Residual block V2.}
  \label{fig:basics:residual_block}
\end{figure}
