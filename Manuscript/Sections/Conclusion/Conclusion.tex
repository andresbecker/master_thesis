%% Magic command to compile root document
% !TEX root = ../../thesis.tex

%% Reset glossary to show long gls names
\glsresetall

In this work we use a dataset generated by means of the \gls{mpm} protocol to predict the \gls{tr} of a cell, using spatial information from its nucleus and a \gls{cnn}. In order to do this, the first step was to reconstruct the multichannel images of cell nucleus from pixel intensity readouts stored in text files, and then build a dataset that allows to train models easily and efficiently.
To reduce as much as possible the information contained in the intensity of the pixels and the correlation of these between the channels, different preprocessing and data augmentation techniques were tried. This allowed the \gls{cnn} to focus mainly on the information contained in the location, distribution, shape and size of elements within the cell nucleus to predict its \gls{tr}.

For this work different architectures were tried. From a \hl{simple CNN} (with only 160k parameters), to more complex architectures such as the \hl{ResNet50V2} and the \hl{Xception} (with more than 20m parameters).
Besides, for all the implemented architectures, $L_1$ and $L_2$ regularization was tried for both dense and convolution layers. However, in practice this did not significantly improve the generalization of the models, but rather increased the number of hyperparameters. For this reason the use of regularization was discarded from the final results.
For the \hl{ResNet50V2} and \hl{Xception} architectures, the use of pre-trained weights and biases to initialize the model parameters (transfer learning) was also tried. However, since this did not significantly improve the performance of the models, its use was also discarded.

However, predicting \gls{tr} was not the only objective of this work.
Through the interpretability methods \gls{ig} and \gls{vg}, for each cell nucleus image we obtained score maps that allowed us to observe the pixels that the \gls{cnn} considered as relevant to predict the \gls{tr}.
This allowed us to know which proteins and nuclear bodies were most relevant for the prediction of \gls{tr}, and how this changes as the \gls{tr} decreases or increases.
This led us to formulate the hypothesis that as the \gls{tr} grows, the \gls{cnn} goes from focusing on the synthesis of \gls{pmrna} to the synthesis of mature \gls{mrna}.
This shows that interpretability methods can help us to understand how a \gls{cnn} makes its predictions and learn from it, which has the potential to help us make biological discoveries.
In addition to this, the analysis of the dynamics of the score maps showed that for the model the \hl{SP100} channel (which is closely related to the \hl{PML} channel) has a high influence on the prediction of the \gls{tr}. This may suggest that cells with a high number of PML bodies in the cell nucleus, could be an indicative of a high transcription. However, this is a hypothesis that would have to be rigorously studied and validated.

In order to reduce the noise in the score maps, the use of random noise in the input images during model training was also tried. However, this did not show any apparent improvement in the scores more, so its use was discarded.

For each of the three proposed architectures, the score map of each cell nucleus image was computed. However, the score maps corresponding to the \hl{simple CNN} architecture were less noisy and showed more defined nuclear structures. For this reason, and besides the fact that the difference in performance against the more complex architectures was minimal, the \hl{simple CNN} was selected to generate the results shown in section \ref{sec:results:model_interpretation}.
