%==================================================
% abstract.tex
% Beispieldatei für tumthesis.cls und thesis.tex
% Michael Ritter, 2012
% Lizenz:
% This work may be distributed and/or modified under the
% conditions of the LaTeX Project Public License, either version 1.3
% of this license or (at your option) any later version.
% The latest version of this license is in
% http://www.latex-project.org/lppl.txt
% and version 1.3 or later is part of all distributions of LaTeX
% version 2005/12/01 or later.
%==================================================

%% Magic command to compile root document
% !TEX root = ../thesis.tex

%% Reset glossary to show long gls names
\glsresetall

%\cleardoublepage

\selectlanguage{english}
\section*{Abstract}

By means of fluorescent antibodies it is possible to observe the amount of nascent RNA within the nucleus of a cell, and thus estimate its \gls{tr}. But what about the other molecules, proteins, organelles, etc. within the nucleus of the cell? Is it possible to estimate the \gls{tr} using only the shape and distribution of these subnuclear components? By means of multichannel images of single cell nucleus (obtained through the \gls{mpm} protocol \cite{Guteaar7042}) and \glspl{cnn}, we show that this is possible. 
Applying pre-processing and data augmentation techniques, we reduce the information contained in the intensity of the pixels and the correlation of these between the different channels. This allowed the \gls{cnn} to focus mainly on the information provided by the location, size and distribution of elements within the cell nucleus.
For this task different architectures were tried, from a simple \gls{cnn} (with only 160k parameters), to more complex architectures such as the ResNet50V2 or the Xception (with more than 20m parameters).
Furthermore, through the interpretability methods \gls{ig} and \gls{vg}, we could obtain score maps that allowed us to observe the pixels that the \gls{cnn} considered as relevant to predict the \gls{tr} for each cell nucleus input image. The analysis of these score maps reveals how as the \gls{tr} changes, the \gls{cnn} focuses on different proteins and areas of the nucleus. This shows that interpretability methods can help us to understand how a \gls{cnn} make its predictions and learn from it, which has the potential to provide guidance for new discoveries in the field of biology.
