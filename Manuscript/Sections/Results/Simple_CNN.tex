%% Magic command to compile root document
% !TEX root = ../../thesis.tex

\glsresetall
% define where the images are
\graphicspath{{./Sections/Results/Resources/}}

% Experiments parameter files:
%BL_RIV2_test4.json
%BL_RIV2_test8.json wo RCI

Figure \ref{fig:results:bl_performance} provides a more in-depth look at what was mentioned in section \ref{sec:model_performance}, with respect to the \hl{simple CNN} model. All the subfigures in \ref{fig:results:bl_performance} correspond to the test set and are divided/colored by cell cycle.

\begin{figure}[!ht]
  \centering
  \begin{subfigure}[b]{.5\linewidth}
    \includegraphics[width=\linewidth]{BL_cs_y_boxplot.png}
    \caption{Distribution of $y$ and $\hat{y}$ (boxplots) for data with color and structure.}
    \label{fig:results:bl_performance:cs_dist}
  \end{subfigure}
  \begin{subfigure}[b]{.27\linewidth}
    \includegraphics[width=\linewidth]{BL_cs_y_vs_y_hat.png}
    \caption{$y$ vs. $\hat{y}$ for data with color and structure.}
    \label{fig:results:bl_performance:cs_y_vs_y_hat}
  \end{subfigure}%
  \vspace{3mm}
  \begin{subfigure}[b]{.5\linewidth}
    \includegraphics[width=\linewidth]{BL_s_y_boxplot.png}
    \caption{Distribution of $y$ and $\hat{y}$ (boxplots) for data with spatial information (structure only).}
    \label{fig:results:bl_performance:s_dist}
  \end{subfigure}
  \begin{subfigure}[b]{.27\linewidth}
    \includegraphics[width=\linewidth]{BL_s_y_vs_y_hat.png}
    \caption{$y$ vs. $\hat{y}$ for data with spatial information.}
    \label{fig:results:bl_performance:s_y_vs_y_hat}
  \end{subfigure}
  \caption{Comparison between the true and predicted \gls{tr} ($y$ and $\hat{y}$ respectively) for the \hl{simple CNN} model on the test set, divided by cell cycle. The first row of figures corresponds to the linear model trained with data containing pixel intensity and spatial information (color and structure), while the second row to the linear model trained with spatial data only (structure). The boxes in figures \subref{fig:results:bl_performance:cs_dist} and \subref{fig:results:bl_performance:s_dist} show the first and third quartiles of the data ($25\%$ and $75\%$ respectively), while the whiskers extend to show the rest of the distribution, except for points that are determined to be \hl{outliers} using a function of the inter-quartile range. The line inside the boxes shows the second quartile (median) of the data. Figures \subref{fig:results:bl_performance:cs_y_vs_y_hat} and \subref{fig:results:bl_performance:s_y_vs_y_hat} shows the true vs. predicted \gls{tr}.}
  \label{fig:results:bl_performance}
\end{figure}

Unlike the linear model, subfigures \ref{fig:results:bl_performance:cs_y_vs_y_hat} and \ref{fig:results:bl_performance:s_y_vs_y_hat} show that the simple CNN model training with structure data was able to approximate a function almost as good as that of the model trained with color and structure data.
This can also be seen in subfigures \ref{fig:results:bl_performance:cs_dist} and \ref{fig:results:bl_performance:s_dist}, which show that even after removing the pixel intensity information from the data, the model was still able to approximate fairly well the \gls{tr} distribution.

Again, this reinforces our hypothesis that it is possible to describe cell expression to some extent, based solely on spatial information from the cell nucleus.
