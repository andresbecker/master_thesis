%% Magic command to compile root document
% !TEX root = ../../thesis.tex

\glsresetall
% define where the images are
\graphicspath{{./Sections/Results/Resources/}}

This chapter is mainly divided into two parts

\begin{enumerate}
    \item The model performance analysis
    \item The model interpretation analysis
\end{enumerate}

In the first part, we show the results of the models introduced in section \ref{sec:methodology:models}, and compare the performance of each model against the others based on the metrics introduced in section \ref{sec:methodology:metrics}. Besides, the performance of each model is compared to the reference values of the metrics, which validate that the models are capable of learning something meaningful from the data. This section ends by analyzing more in-depth the performance of the linear model against one of the \glspl{cnn} models, and shows that the latter is still capable of predicting fairly well the \gls{tr}, even after significantly reducing the pixel intensity information of each channel and the correlation between them (by means of the data augmentation techniques introduced in section \ref{sec:dataset:data_augmentation} and \ref{sec:methodology:tfds}).

On the other hand, the Model interpretation section focuses on analyzing the results obtained using the interpretability methods introduced in section \ref{sec:basics:interpretability_methods}. The analysis begins with the division of the cells into three levels of transcription (low, medium and high), and ends by analyzing how the model changes the areas of interest in the input image as the \gls{tr} changes. 
Furthermore, the analysis shows that as the \gls{tr} increases, the model relies more on regions of the nucleus that are directly related to the genesis of mature \gls{mrna}, which shows that interpretation methods can be used as tools to discover unknown biological relationships when applied to black-box models like \glspl{cnn}.

The results of both sections also show that it is possible to predict (to some extent) the \gls{tr} of a cell, based mainly on spatial information within the nucleus.