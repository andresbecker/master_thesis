%% Magic command to compile root document
% !TEX root = ../../thesis.tex

\glsresetall
% define where the images are
\graphicspath{{./Sections/Results/Resources/}}

This intro is still under construction....

Hannah's recommendations:
\begin{itemize}
  \item Results for each method/model in subsection
  \item Discuss results; what does this mean? Does this prove your hypothesis? Disprove it?
  \item Ideally final paragraph connecting different results
\end{itemize}


%----------------------------------------------------------------------
In this chapter the results of the models will be presented and analyzed. The main objective is to answer the question

El prinsipal objetivo de este capitulo es intntar contestar la pregunta de si es posible predecir el TR de una celula basandonos principalmente en informacion espacial del nucleo de la celula.

Este capitulo tambien mustra los resultados obtenidos de los interpretabiliy methods VG and IG (score maps). Estos score maps permiten apreciar los pixeles de alta relevancia para la prediccion del TR. Al ser analizados, descubrimos que la CNN is relaying en lugares muy especificos para hacer su prediccion, especificamente en la zona correspondiente al nucleolus y a las zonas donde se lleva a cabo el proceso de splicing (splicing speakles).

Este capitulo es concluido con analizis que mustra la relacion entre los differentes score maps channel y entre score maps y los canales originales.
