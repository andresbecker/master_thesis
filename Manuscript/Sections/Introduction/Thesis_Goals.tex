%% Magic command to compile root document
% !TEX root = ../../thesis.tex

\glsresetall

The main objective of this work is to prove that it is possible to predict the \gls{tr} of a cell, using a deep learning model and the spatial information of its nucleus. More specifically, this means to design a \gls{cnn} architecture for regressing the \gls{tr} value of a cell, using only the information encoded in the location, distribution and shape of subnuclear components (like molecules, proteins and nuclear bodies) in multichannel images of cell nucleus.
In order to do it, we implement preprocessing and data augmentation techniques aimed to reduce the information contained in the intensity of the pixels and its correlation among the channels. This would encourage the model to focus mainly on spatial information.

The second goal of this work is to apply recent gradient-based interpretability methods \cite{adebayo2020sanity}, to understand which molecules/proteins and nuclear bodies were most relevant for the prediction of the mode.
Understanding how a \gls{cnn} model works gives us the possibility to learn from it, which has the potential to provide guidance for new discoveries in the field of biology.

%Finally, we test the veracity of the information in the score maps, by confronting them with a validation method.
