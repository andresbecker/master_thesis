A small overview. Explain how CNN can be used to predict Transcription Rate and how we can use interpretability methods to learn from this models.

Hannah's recommendations about intro chapter:
\begin{itemize}
  \item Problem statement and challenges
  \item Main objectives / research questions
  \item List contributions
  \item last section of intro is important; give summary of content + key impact
\end{itemize}

Hannah's general recommendations:
\begin{itemize}
  \item Content: You do not need to include all models you trained and details of all the bugs you fixed, but the thesis should contain a consistent story (detailled in the contributions in the introduction) where every result you show adds something to the story. For every experiment/figure ask yourself: is this necessary to answer the research question? If the answer is no, don’t include it
  \item Assume that some people will only read the introduction and summary, and look at the figures -> the main points should be understandable from this information alone
  \item The thesis should be written at a level where a masters student from your course of study can understand it. All concepts that are not know at this level, need to be introduced
  \item Prefer descriptive section and paragraph headings that ideally tell the reader what this section is about rather than the method that was used
\end{itemize}

Hannah's recommendations about figures:
\begin{itemize}
  \item Give references for every figure you use
  \item Describe all figures in the main text, but also make sure that each figure has a descriptive caption.
  \item Captions should help describe what is being shown, i.e. what do the colors stand for etc. They should help the reader to understand the flow of the figure. They shouldn’t contain too much methods stuff (i.e. how this was done -> belongs to the methods) neither should they contain results stuff (how do we interpret these results? what do they mean? -> belongs to the results). The figure and caption should stand alone and be interpretable without the rest of the text. Ideally you should be able to redraw the figure from the caption and recreate the caption from the figure (but that is a very high bar).
\end{itemize}
