%% Magic command to compile root document
% !TEX root = ../../thesis.tex

%% Reset glossary to show long gls names
\glsresetall

We can interpret \gls{tr} as the amount of new RNA molecules inside a cell nucleus in a given period of time. By means of a fluorescent marker, it is possible to identify these new RNA molecules and thus approximate \gls{tr}. But, what about the morphology of other molecules and organelles within the cell nucleus? The distribution, shape and location of molecules, proteins and organelles within the nucleus could potentially encode relevant information for cellular expression. This has been the main motivation for this work. By means of a \gls{cnn}, we seek to predict \gls{tr} base mainly in spacial information encoded on images of cell nucleus.

In this section we introduce the process used to generate the data for this work, the \gls{mpm} protocol. In addition to this, we introduce the preprocessing and data augmentation techniques used. These techniques aim to improve the model's training performance, prevent overfitting and remove non-relevant information from the images. With this, we seek to encourage the model to base its prediction mainly on the spatial information encoded in the images of cell nucleus.
