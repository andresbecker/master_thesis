%% Magic command to compile root document
% !TEX root = ../../thesis.tex

%% Reset glossary to show long gls names
%\glsresetall

\graphicspath{{./Sections/Methodology/Resources/}}

% experiment
%LM_RIV2_test7.json

As we already mentioned, the objective of this work is to explain cell expression using spatial information in multichannel images of cell nucleus.

To compare the performance of the \glspl{cnn}, and have an idea about how much pixel intensity information was still contained in the data, we also fitted a  \hl{linear model}

\begin{equation}
  y = w_0 + w_1x_1 + \cdots + w_{33}x_{33}
\end{equation}

\noindent where $x_i, \in \mathbb{R}$, for $i \in \{1, \cdots, 33\}$, is the average pixel intensity corresponding to channel $i$ and $w_i, \in \mathbb{R}$, for $i \in \{0, \cdots, 33\}$ are the model coefficients.

The linear model architecture is specified in table \ref{table:metho:models:lm}. The rows of the table represent each layer of the model, which are evaluated from top to bottom. This model has only $34$ free (learnable) parameters in total.

% set table lengths
\setlength{\mylinewidth}{\linewidth-7pt}%
\setlength{\mylengtha}{0.35\mylinewidth-2\arraycolsep}%
\setlength{\mylengthb}{0.25\mylinewidth-2\arraycolsep}%
\setlength{\mylengthc}{0.18\mylinewidth-2\arraycolsep}%

\begin{longtable}{m{\mylengtha} | m{\mylengthb} | m{\mylengthc}}
    \hline
    Layer & Output Shape & Number of parameters \\
    \hline
    Input & $(bs, 224, 224, 38)$ & 0 \\
    \hline
    Channel filtering & $(bs, 224, 224, 33)$ & 0 \\
    \hline
    Global Average Pooling & $(bs, 33)$ & 0 \\
    \hline
    Dense & $(bs, 1)$ & 34 \\
    \hline
  \caption{Linear model architecture. The rows represent each layer of the model. The flow of the model is from top to bottom. The $bs$ on the \hl{Output Shape} column stands for \hl{Batch size}.}
  \label{table:metho:models:lm}
\end{longtable}

The learning rate used to train the baseline model was $0.1$.
