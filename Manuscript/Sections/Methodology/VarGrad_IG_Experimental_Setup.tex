\glsresetall
\graphicspath{{./Sections/Methodology/Resources/}}

There are several hyper-parameters that need to be chosen in order to compute the score map for each cell image.

For the \gls{ig} attribution map, recall that in practice computing $\phi^{IG}$ could be unfeasible or computationally very expensive. However, we can approximate $\phi^{IG}$ by means of $\phi^{Approx\ IG}$ (see equation \ref{eq:ig:approx}). Therefore, we need to define the number of steps $m$ for the Riemann sum approximation. In section \ref{sec:basics:IG} we also mentioned the necessity to set a baseline image $x'$, which should contain no information about the image, in order to compute the \gls{ig}. There are several options that can be used, each one of them with different advantages and disadvantages. However, for this work we only implemented two of them: 1) a simple black image (image containing only zeros) and 2) an image filled with Gaussian noise ($\mu=0,\ \sigma=1$). A very good analysis on the choice of the baseline can be found in this reference \cite{sturmfels2020visualizing}.

In section \ref{sec:basics:VarGrad} we saw that for \gls{vg} we need to define 2 parameters, the number of noisy images $n$ (sample size) and the standard deviation $\sigma$ for the the noise distribution.

As a rule of thumbs, a sample should not be smaller than 30, so this could be a feasible option. However, since Smilkov et al. \cite{Smilkov_smoothgrad} showed empirically that no further improvemnt (less noise) in score maps is observed for sample sizes greater than 50, we chose this bound as sample size.

Table \ref{table:VGIG_exp_set:params} shows a summary of the parameters chosen to calculate the \gls{vgig} score maps.

\begin{table}[!ht]
  \centering
  \begin{tabular}{c|c|c}
    Method & Hyperparameter & Value \\
    \hline
    \hline
    \multirow{2}{*}{\gls{ig}} & $m$ & 70 \\
    \cline{2-3}
     & $x'$ & black image \\
    \hline
    \multirow{2}{*}{\gls{vg}} & $n$ & 50 \\
    \cline{2-3}
     & $\sigma$ & 1 \\
    \hline
  \end{tabular}
  \caption{Parameters to compute score maps.}
  \label{table:VGIG_exp_set:params}
\end{table}

In section \ref{sec:basics:IG}, we mentioned that the \gls{ig} algorithm holds the \textit{Completeness Axiom}, which means that the sum of all the components of the \gls{ig} attribution map must be equal to the difference between the model's output evaluated at the image and the model's output evaluated at the baseline (see equation \ref{eq:ig_completeness}). This property allow us to check empirically if the number of steps $m$ selected for the Riemann sum approximation is sufficiently large. Figure \ref{fig:VGIG_exp_set:m_sanity} shows that for our baseline\fxnote{after finishing, check that the baseline model is still called baseline} model, a random image and $m=70$, the  completeness axiom is satisfied sufficiently well.

\begin{figure}[!ht]
  \centering
  \includegraphics[width=0.8\linewidth]{sanity_check_for_m.jpg}
  \caption{Sanity check for the number of steps $m$ in the Riemann sum to approximate $\phi^{IG}$. The red dotted line represent the difference $f(x)-f(x')$. The blue line represents the value of $\sum_i \phi^{Approx\ IG}_i(f, x, x', m)$ over $\alpha$.}
  \label{fig:VGIG_exp_set:m_sanity}
\end{figure}
