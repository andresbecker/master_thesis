%% Magic command to compile root document
% !TEX root = ../../thesis.tex

%% Reset glossary to show long gls names
\glsresetall
\graphicspath{{./Sections/Methodology/Resources/}}

In this section we specify all the hyperparameters needed to execute the process explained on chapter \ref{ch:dataset}. This contemplates the raw data processing, the quality control, the \gls{tfds} creation, the image preprocessing, as well as data augmentation.

\subsection{Data preprocessing}

As we explained in section \ref{sec:dataset:data_pp}, the data preprocessing consist of 4 main steps; 1) the raw data processing, 2) the quality control, 3) the creation of the dataset and 4) the image preprocessing.

A complementary explanation of the data preprocessing parameters, as well as implementation references, can be found in the appendices \ref{sec:appendix:raw_data} and \ref{sec:appendix:tfds}.

\subsubsection{Raw data processing}

As we explained in section \ref{sec:dataset:data_pp:dataset_creation}, to build the \gls{tfds} it is necessary to specify the perturbations that will be included in the dataset. For this reason, all the available wells were processed and transformed into images. This included wells exposed to pharmacological and metabolic perturbations, control wells and unperturbed wells. This allows the user to easily create new datasets without having to run the raw data processing first. Table \ref{table:methodology:dataset:raw_data} shows the processed wells separated by perturbation.

\begin{table}[!ht]
  \centering
  \begin{tabular}{>{\centering\arraybackslash}m{0.35\linewidth} | >{\centering\arraybackslash}m{0.2\linewidth} | >{\centering\arraybackslash}m{0.3\linewidth}}
    \hline
    Perturbation type & Perturbation name & Well names \\
    \hline
    \multirow{5}{*}{pharmacological/metabolic} & CX5461 & I18, J22, J09 \\
    \cline{2-3}
     & AZD4573 & I13, J21, J14, I17, J18 \\
    \cline{2-3}
     & meayamycin & I12, I20 \\
    \cline{2-3}
    & triptolide & I10, J15 \\
    \cline{2-3}
    & TSA & J20, I16, J13 \\
    \hline
    control & DMSO & J16, I14 \\
    \hline
    unperturbed & normal & J10, I09, I11, J18, J12 \\
    \hline
  \end{tabular}
  \caption{Well names divided by perturbation name and type.}
  \label{table:methodology:dataset:raw_data}
\end{table}

Another hyperparameter that needs to be specified during the raw data processing, is the size of the output images $I_s$. This size applies to both, the width and height of the image (square images). Since some prebuilt architectures use a standard image size of 224 by 224, we define $I_s$ as 224.

\subsubsection{Quality control}

As it is mentioned in section \ref{sec:dataset:data_pp:qc}, the quality control is meant to exclude cells with undesirable features. In our case we discriminate mitotic and border cells. The information used by the quality control is contained in the metadata of each well. Table \ref{table:methodology:dataset:qc} shows the metadata columns and the discriminated values. If a cell has any of these values, then it is excluded.

\begin{table}[!ht]
  \centering
  \begin{tabular}{>{\centering\arraybackslash}m{0.3\linewidth} | >{\centering\arraybackslash}m{0.3\linewidth} | >{\centering\arraybackslash}m{0.2\linewidth}}
    \hline
    Feature & Metadata column name & Discriminated value \\
    \hline
    \multirow{3}{*}{Cell in mitosis phace} & \texttt{is\_polynuclei\_HeLa} & 1 \\
    \cline{2-3}
     & \texttt{is\_polynuclei\_184A1} & 1 \\
    \cline{2-3}
     &  \texttt{cell\_cycle} & \texttt{NaN} \\
    \hline
    Border cell & \texttt{is\_border\_cell} & 1 \\
    \hline
  \end{tabular}
  \caption{Discrimination characteristics for quality control.}
  \label{table:methodology:dataset:qc}
\end{table}

\subsubsection{Dataset creation and image preprocessing}

As it is explained in section \ref{sec:dataset:data_pp:dataset_creation}, in this work we decided to use a custom \gls{tfds}. Table \ref{table:methodology:dataset:tfds} shows the parameters used to build the dataset employed in this work, together with the image preprocessing parameters.

% set table lengths
\setlength{\mylinewidth}{\linewidth-7pt}%
\setlength{\mylengtha}{0.25\mylinewidth-2\arraycolsep}%
\setlength{\mylengthb}{0.65\mylinewidth-2\arraycolsep}%

\begin{longtable}{>{\centering\arraybackslash}m{\mylengtha} | >{\centering\arraybackslash}m{\mylengthb}}
    \hline
    Parameter & Description \\
    \hline
    Perturbations to be included in the dataset & \hl{normal} and \hl{DMSO} \\
    \hline
    Cell phases to be included in the dataset & $G_1$, $S$, $G_2$ \\
    \hline
    Training set split fraction & 0.8 \\
    \hline
    Validation set split fraction & 0.1 \\
    \hline
    Seed & 123 (for reproducibility of the train, val and test split) \\
    \hline
    Percentile & 98 (for clipping / linear scaling / standardization) \\
    \hline
    Clipping flag & 1 \\
    \hline
    Mean extraction flag & 0  \\
    \hline
    Linear scaling flag & 0 \\
    \hline
    Standardization (z-score) flag & 1 \\
    \hline
    Model input channels & All of them except for channel \texttt{00\_EU} (see table \ref{table:tfds_in:channels})  \\
    \hline
    Channel used to compute target variable (output channel) & \texttt{00\_EU} (channel id 35, see table \ref{table:tfds_in:channels}) \\
    \hline
  \caption{Parameters used to biuld \gls{tfds} and image preprocessing.}
  \label{table:methodology:dataset:tfds}
\end{longtable}

The custom \gls{tfds} created with the parameters specified in table \ref{table:methodology:dataset:tfds} is called \\
\texttt{mpp\_ds\_normal\_dmso\_z\_score}.
The Python script that builds the custom \gls{tfds}, also returns a file with the image preprocessing parameters (\texttt{channels\_df.cvs}) (as this is applied at a per-channel level) and information about the channels (channel name, id, etc.). It also returns another file with the metadata of each cell included in the \gls{tfds} (\texttt{metadata\_df.csv}). These files are stored in the same directory as the \gls{tfds} files.

In table \ref{table:methodology:dataset:tfds} we also specify the channel used to compute the target variable (ground truth), which is the channel corresponding to the marker \hl{EU} (channel id 35, see tables \ref{table:tfds_in:channels} and \ref{table:apendix:if_markers}).
Recall that this channel contains nuclear readouts of nascent RNA (\gls{pmrna}) in a given period of time.
For the data provided, this time period was the same for all the cells (30 minutes) and is specified in the \hl{duration} columns of the metadata.
Since channel 35 is used to compute the target variable (ground truth), it is removed from the prediction/input channels.

\subsection{Data augmentation}
\label{sec:methodology:data:augm}

In this section we specify the data augmentation techniques (see section \ref{sec:dataset:data_augmentation}) and its hyperparameters used to train all the models of this work. Recall that the techniques are either aimed to remove non-relevant characteristics of the data (color shifting, central zoom in/out) or to improve model generalization (horizontal flipping, 90 degree rotations). Table \ref{table:methodology:dataset:augm} shows this techniques and its hyperparameters grouped by objective and technique. In practice, the augmentation techniques are applied as shown in table \ref{table:methodology:dataset:augm} from top to bottom.

% set table lengths
\setlength{\mylinewidth}{\linewidth-7pt}%
\setlength{\mylengtha}{0.2\mylinewidth-2\arraycolsep}%
\setlength{\mylengthb}{0.2\mylinewidth-2\arraycolsep}%
\setlength{\mylengthc}{0.25\mylinewidth-2\arraycolsep}%
\setlength{\mylengthd}{0.22\mylinewidth-2\arraycolsep}%

\begin{longtable}{>{\centering\arraybackslash}m{\mylengtha} | >{\centering\arraybackslash}m{\mylengthb} | >{\centering\arraybackslash}m{\mylengthc} | >{\centering\arraybackslash}m{\mylengthd}}
    \hline
    Objective & Technique & Hyperparameter & Description \\
    \hline
    \multirow{2}{\mylengtha}{\centering Remove non-relevant features} & random color shifting & distribution & $U(-3,3)$ \\
    \cline{2-4}
     & random central zoom in/out & distribution\footnotemark & $N(\mu=0.6, \sigma=0.1)$ \\
    \hline
     \multirow{2}{\mylengtha}{\centering Improve generalization} & random horizontal flipping & NA & NA \\
    \cline{2-4}
     & random 90 degrees rotations & NA & NA \\
    \hline
  \caption{Parameters used for data augmentation techniques. The NA means that there are no hyperparameters for this technique or that there is no further description.}
  \label{table:methodology:dataset:augm}
\end{longtable}

\footnotetext{This distribution is used to sample the \hl{cell nucleus size ratio} $S_{ratio}$ (see section \ref{sec:dataset:data_aug:zoom}) of each cell. However, the parameters for this distribution (mean and standard deviation) were not provided by us. Instead, they were estimated using the information in column \texttt{cell\_size\_ratio} of the \gls{tfds} metadata file. Therefore, the \texttt{return\_cell\_size\_ratio} flag must be set to 1 (True) during raw data processing, so this column is created (see section \ref{sec:dataset:data_pp:raw_data_p} and appendix \ref{sec:appendix:raw_data}).}

Even thought we specify the data augmentation hyperparameters here, in practice these are selected for each model and applied during training. However, all the models showed in this work were trained using the techniques and values shown in table \ref{table:methodology:dataset:augm}. A complementary explanation can be found in appendix \ref{sec:appendix:Model_training_IN}.

In section \ref{sec:dataset:data_augmentation} we mentioned that data augmentation techniques can be applied to both the training set and the validation set. However, we also mentioned that only horizontal flips and 90 degree rotations are applied for the validation set. Furthermore, for the training set these techniques are applied randomly, while for the validation set they are applied deterministically. Therefore, table \ref{table:methodology:dataset:augm} only applies to the training set.
