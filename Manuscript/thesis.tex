% ==============================================================================
% thesis.tex
% Example file for tumthesis.csl
% Michael Ritter, 2012
% Licence:
% This work may be distributed and/or modified under the
% conditions of the LaTeX Project Public License, either version 1.3
% of this license or (at your option) any later version.
% The latest version of this license is in
% http://www.latex-project.org/lppl.txt
% and version 1.3 or later is part of all distributions of LaTeX
% version 2005/12/01 or later.
% ==============================================================================
\documentclass[biblatexBackend=bibtex]{tumthesis}

% ------------------------------------------------------------------------------
%FixMe-Status: final (no FixMe comments) or draft (comments visible)
\fxsetup{draft}
%\fxsetup{final}
% ------------------------------------------------------------------------------

% ------------------------------------------------------------------------------
%  Language selection for metadata and main text (can be changed at any point in
%  the main text.
%\selectlanguage{ngerman}
\selectlanguage{english}
% ------------------------------------------------------------------------------

% ------------------------------------------------------------------------------
% Data for the bibliography
\addbibresource{thesis.bib}
% ------------------------------------------------------------------------------

% ------------------------------------------------------------------------------
% Further packages and TikZ libraries can be incorporated here.
\usetikzlibrary{arrows}
% ------------------------------------------------------------------------------

% ------------------------------------------------------------------------------
% PDF-Metadaten
\hypersetup{
 pdfauthor={Andres Alberto Becker Sanabria},
 pdftitle={Predicting transcription rate from multiplexed protein maps using deep learning},
 pdfsubject={Predicting transcription rate from multiplexed protein maps using deep learning},
 pdfkeywords={Master's Thesis},
 colorlinks=true, %coloured links (for the PDF version)
% colorlinks=false, % no coloured links (for the print version)
}
% ------------------------------------------------------------------------------

% ------------------------------------------------------------------------------
% Data for the title page and declaration
\author{Andres Becker}
\title{Predicting transcription rate from multiplexed protein maps using deep learning}
%\subtitle{A Tutorial for Theses}
%\faculty{Fakultät für Mathematik}
\faculty{Department of Mathematics}
%\institute{Lehrstuhl für Mathematische Modelle biologischer Systeme}
\institute{Chair of Mathematical Modeling of Biological Systems}
\subject{master}
%\subject{bachelor}
%\subject{diploma}
%\subject{project}
%\subject{seminar}
%\subject{idp}
%\subject{Short Overview}
\professor{Prof. Dr. Fabian J. Theis} %Themensteller
\advisor{Dr. Hannah Spitzer} %Betreuer
\date{April 14, 2021} %Submission Date
\place{München} %Place where document is signed
% ------------------------------------------------------------------------------

% ==============================================================================
% Acronyms
% ==============================================================================

% Packages should be called in the preamble document (preamble.tex). However,
% glossaries package requires to be called after hyperref, babel, polyglossia,
% inputenc and fontenc. Therefore, calling glossaries package in preamble doc
% produce errors during compilation (and no generation of Acronyms section).

%\usepackage[style=long,nonumberlist, toc,acronym,nomain]{glossaries}
% style=long: center the acronyms in acronym section
% nonumberlist: remove the list all places in docu where the acron was called
% toc: Add Acronyms section to table of content
% acronym, nomain: necessary
\usepackage[acronym, toc, nomain]{glossaries}
\makeglossaries

%\newacronym[<options>]{<label>}{<short>}{<long>}
\newacronym{4i}{4i}{iterative indirect immunofluorescence imaging}
\newacronym{tr}{TR}{transcription rate}
\newacronym{scmos}{sCMOS}{scientific complementary metal oxide semiconductor}
\newacronym{cnn}{CNN}{Convolutional Neural Network}
\newacronym{mrna}{mRNA}{messenger RNA}
\newacronym{ig}{IG}{Integrated Gradient}
\newacronym{vg}{VG}{VarGrad}
\newacronym{sg}{SG}{SmoothGrad}
\newacronym{vgig}{VGIG}{VarGrad Integrated Gradient}
\newacronym{dnn}{DNN}{Deep Neural Network}
% ------------------------------------------------------------------------------

% ==============================================================================
% Costume commands
% ==============================================================================
\newcommand{\hl}[1]{\textbf{#1}} % Defined to highlight new words

% ------------------------------------------------------------------------------

% ==============================================================================
% Main part of the document
% ==============================================================================
\makeindex[title=Index,options=-s myindex]
\begin{document}
\pagestyle{empty}
\frontmatter%
%\selectlanguage{ngerman}
\selectlanguage{english}
\maketitlepage%
%\maketitlepageDissertation % a more elegant one, but for PHd dissertation
%\makedeclaration%

%% ==============================
\chapter*{Declaration}
%% ==============================

\vfill

I hereby declare that this thesis is my own work and that no other sources have been used except those clearly indicated and referenced.

\vspace{3em}

\noindent Andres Alberto Becker Sanabria, München, 14.05.2021

%==================================================
% abstract.tex
% Beispieldatei für tumthesis.cls und thesis.tex
% Michael Ritter, 2012
% Lizenz:
% This work may be distributed and/or modified under the
% conditions of the LaTeX Project Public License, either version 1.3
% of this license or (at your option) any later version.
% The latest version of this license is in
% http://www.latex-project.org/lppl.txt
% and version 1.3 or later is part of all distributions of LaTeX
% version 2005/12/01 or later.
%==================================================

%% Magic command to compile root document
% !TEX root = ../thesis.tex

%% Reset glossary to show long gls names
\glsresetall

%\cleardoublepage

\selectlanguage{english}
\section*{Abstract}

By means of fluorescent antibodies it is possible to observe the amount of nascent RNA within the nucleus of a cell, and thus estimate its \gls{tr}. But what about the other molecules, proteins, organelles, etc. within the nucleus of the cell? Is it possible to estimate the \gls{tr} using only the shape and distribution of these subnuclear components? By means of multichannel images of single cell nucleus (obtained through the \gls{mpm} protocol \cite{Guteaar7042}) and \glspl{cnn}, we show that this is possible. 
Applying pre-processing and data augmentation techniques, we reduce the information contained in the intensity of the pixels and the correlation of these between the different channels. This allowed the \gls{cnn} to focus mainly on the information provided by the location, size and distribution of elements within the cell nucleus.
For this task different architectures were tried, from a simple \gls{cnn} (with only 160k parameters), to more complex architectures such as the ResNet50V2 or the Xception (with more than 20m parameters).
Furthermore, through the interpretability methods \gls{ig} and \gls{vg}, we could obtain score maps that allowed us to observe the pixels that the \gls{cnn} considered as relevant to predict the \gls{tr} for each cell nucleus input image. The analysis of these score maps reveals how as the \gls{tr} changes, the \gls{cnn} focuses on different proteins and areas of the nucleus. This shows that interpretability methods can help us to understand how a \gls{cnn} make its predictions and learn from it, which has the potential to provide guidance for new discoveries in the field of biology.

\selectlanguage{english}
\section*{Acknowledgments}

To my father, my partner in my wildest adventures, best friend in life and who taught me what are the important things in life. He may never read this, but let the world known he is a loved and admired man.


%%% Local Variables:
%%% mode: latex
%%% TeX-master: "thesis"
%%% End:


\tableofcontents%

\mainmatter%
\pagestyle{headings}

%% =============================================================================
%% Basics chapter
%% =============================================================================
\chapter{Introduction}
\label{ch:introduction}
%% Magic command to compile root document
% !TEX root = ../../thesis.tex

%% Reset glossary to show long gls names
\glsresetall

We can interpret \gls{tr} as the amount of new RNA molecules inside a cell nucleus in a given period of time. By means of a fluorescent marker, it is possible to identify these new RNA molecules and thus approximate \gls{tr}. But, what about the morphology of other molecules and organelles within the cell nucleus? The distribution, shape and location of molecules, proteins and organelles within the nucleus could potentially encode relevant information for cellular expression. This has been the main motivation for this work. By means of a \gls{cnn}, we seek to predict \gls{tr} base mainly in spacial information encoded on images of cell nucleus.

In this section we introduce the process used to generate the data for this work, the \gls{mpm} protocol. In addition to this, we introduce the preprocessing and data augmentation techniques used. These techniques aim to improve the model's training performance, prevent overfitting and remove non-relevant information from the images. With this, we seek to encourage the model to base its prediction mainly on the spatial information encoded in the images of cell nucleus.


\section{Motivation}
\label{sec:intro:motivation}

\section{Literature review}
\label{sec:intro:literature_review}
%% Magic command to compile root document
% !TEX root = ../../thesis.tex

%% Set path to look for the images
\graphicspath{{./Sections/Introduction/Resources/}}

\glsresetall

Can we fully describe gene expression using only information about concentration of proteins and/or molecules like RNA inside the cell nucleus? Accordingly to Buxbaum et al. \cite{Buxbaum_2014}, the location of \gls{mrna} within the cell plays an important role in protein synthesis. In \cite{Korolchuk2011}, Korolchuk et al. show that cellular response to nutrient levels is a mechanism that needs to contemplate the position of Lysosomes (dynamic intracellular organelles) in order to be fully understood. However, the need for localization information to explain cellular mechanisms is not only limited to a subcellular level, but also at a subnuclear level. For instance, in \cite{van2019role} van Steensel et al. argue that the spatial organization of subnuclear components can have an important role in the regulation of gene expression. In \cite{vogel2010sequence}, Vogel et al. shows that in human cells the concentration of \gls{mrna} can only explain protein abundance to a certain extent, which could indicate the need to consider spatial information to predict protein expression.

In recent years, the implementation of new imaging technologies has made it possible to access subnuclear spatial information. In \cite{Guteaar7042}, Gut et al. introduce the \gls{4i} protocol, which is a process that allows to efficiently capture thousands of single cell multichannel images from a cell culture without degrading it. The \gls{4i} protocol is part of the \gls{mpm} protocol also introduced in \cite{Guteaar7042}, which allows the segmentation of the tissue images into single cell nucleus images (Multiplexed single cell analysis).
The \gls{mpm} protocol also introduces two other features that are not used in this work, but are still worth mentioning.
The first one is the \gls{mcu} analysis, which segments the cell nucleus image into regions.
These regions can be then used to identify subnuclear bodies or protein complexes. The segmentation is done through two unsupervised clustering algorithms\footnote{To identify clusters in an unsupervised manner, \hl{Self Organizing Maps} algorithm and \hl{Phenograph} analysis are used over a very large number of pixels sampled from all the single cells images.}, applied over the measured pixel intensities. The \gls{mcu} analysis is shown on figure \ref{fig:mcu}.

\begin{figure}[htb]
  \centering
  \begin{subfigure}[t]{.3\linewidth}
    \includegraphics[width=\linewidth]{mcu_1.png}
    \caption{Extraction of pixel intensities.}
    \label{fig:mcu:1}
  \end{subfigure}
  \hspace{4mm}
  \begin{subfigure}[t]{.3\linewidth}
    \includegraphics[width=\linewidth]{mcu_2.png}
    \caption{Pixel clustering by Self Organizing Maps and Phenograph.}
    \label{fig:mcu:2}
  \end{subfigure}
  \hspace{4mm}
  \begin{subfigure}[t]{.3\linewidth}
    \includegraphics[width=\linewidth]{mcu_3.png}
    \caption{Cell subdivision base on the \gls{mcu}.}
    \label{fig:mcu:3}
  \end{subfigure}
  \caption{Figure \subref{fig:mcu:1} shows the pixel intensity extraction for a single cell. The pixel intensity is a vector containing the readout of that 2D location for each protein, one specific protein readout per entrance. Figure \subref{fig:mcu:2} shows the clusters found by Self Organizing Maps algorithm and Phenograph analysis over the pixel intensities. Figure \subref{fig:mcu:3} shows a cell masked with the clusters found by the \gls{mcu} analysis. Images source \cite{Guteaar7042}.}
  \label{fig:mcu}
\end{figure}

The second feature of the \gls{mpm} protocol that is not discussed here, but could be used in future work, is the application of pharmacological and metabolic perturbations to some sections of the cell culture. The analysis shown in \cite{Guteaar7042} revealed expected and unexpected changes in the concentration and distribution of molecules inside the cell.

% why NN
\gls{ann} are very robust tools widely used in the field of \gls{ml} that can potentially approximate any function \cite{cybenko1989approximation}, \cite{hornik1989multilayer}, \cite{funahashi1989approximate}.
In the field of biology, \glspl{ann} have proven capable of solving very complex and high-impact problems.
One of the best examples in recent years is the three-dimensional prediction of the structure of a protein using amino acid sequences encoded in the genes \cite{AlphaFold}, which is a very important problem since the structure of a protein largely determines its function. In \cite{chen2016gene}, Chen et al. introduced a deep \gls{ann} model known as \hl{D-GEX}, which outperformed previous linear model approaches when trained using gene expression profiling data.

However, in \cite{krizhevsky2012imagenet} Krizhevsky et al. show that \glspl{cnn} are powerful tools in the recognition of spatial patterns, achieving outstanding results in ImageNet LSVRC-2010 contest.
This makes \glspl{cnn} suitable models to analyze spatial information embedded in images of single cell nucleus, like the ones provided by the \gls{mpm} protocol.

% what is the problem with NN
However, in many fields of study and industries, the interpretation of the models is essential. For example, in the medical field, \gls{cnn} architectures have achieved remarkable results in the segmentation of brain tumors \cite{saleem2021visual}. However, to successfully implement deep learning models in the diagnosis of patients, it is not enough only to know what the model predicts, but also how it does it.

% How to solve it? Interpretability methods
Many researchers have proposed different techniques to explain what happens inside black-box models like \glspl{cnn}.
The difference between these methods is basically whether they are applicable to any type of model (model-agnostic/model-independent) or only to a specific group (model-specific).
An example of a model-independent method is the \gls{lime}, which basically aims to approximate the underlying model $f$ (not interpretable) by means of an interpretable model $g$ (e.g. a linear model) for a specific region of the input \cite{ribeiro2016model}. As the name suggest, \gls{lime} provides a local and individual explanation of each input.
However, there are other methods that provide a general (global) explanation of the model. An example of a global method (and also model-specific) would be the visualization of the learned filters/kernels of a \gls{cnn}, which can indicate the features in the data that are important for the model prediction \cite{zeiler2014visualizing}.

% Gradient based methods
However, in this work we use \hl{attribution methods}, which are aimed to rank each input feature based on how much they contribute to the output of the model.
These methods create an importance (or score) map for each element of the input data. There are several ways to compute these score maps \cite{JMLR:v11:baehrens10a}, \cite{ShrikumarGSK16}. However, most of these methods base the importance assignment of each input feature on the gradient of the model output with respect to the input (gradient-based methods) \cite{SimonyanVZ13}, \cite{BinderMBMS16} and \cite{Springenberg}.

% why IG
Nevertheless, just using the gradient as a feature importance designation method is not enough. As a model learns the relationship between an input and its output, the gradient of the model's output with respect to the input features will approximate to 0 (saturation).
To alleviate this issue, Sundararajan et al. \cite{sundararajan2017axiomatic} proposed \gls{ig}, which accumulates the gradient of the output with respect to the input when it goes from an uninformative value to the original input.

% why VarGrad
However, in practice attribution methods like \gls{ig} often produce noisy and diffuse score maps, and in some cases they are not even better than a random designation of feature importance \cite{hooker2018benchmark}.
For this reason Smilkov et al. \cite{Smilkov_smoothgrad} proposed an ensemble interpretability method known as \gls{sg}, which in practice reduces noise in score maps and can be easily combined with other attribution methods such as  \gls{ig}.
In this work we use a slightly different version proposed by Adebayo et al. \cite{adebayo2018local} known as \gls{vg}, which is inspired by \gls{sg} and has been shown to empirically outperform such a random assignment of importance \cite{hooker2018benchmark}.


%% =============================================================================
%% Basics chapter
%% =============================================================================
\chapter{Basics}
\label{ch:basics}
%% Magic command to compile root document
% !TEX root = ../../thesis.tex

%% Reset glossary to show long gls names
\glsresetall

We can interpret \gls{tr} as the amount of new RNA molecules inside a cell nucleus in a given period of time. By means of a fluorescent marker, it is possible to identify these new RNA molecules and thus approximate \gls{tr}. But, what about the morphology of other molecules and organelles within the cell nucleus? The distribution, shape and location of molecules, proteins and organelles within the nucleus could potentially encode relevant information for cellular expression. This has been the main motivation for this work. By means of a \gls{cnn}, we seek to predict \gls{tr} base mainly in spacial information encoded on images of cell nucleus.

In this section we introduce the process used to generate the data for this work, the \gls{mpm} protocol. In addition to this, we introduce the preprocessing and data augmentation techniques used. These techniques aim to improve the model's training performance, prevent overfitting and remove non-relevant information from the images. With this, we seek to encourage the model to base its prediction mainly on the spatial information encoded in the images of cell nucleus.


\section{Biology Background}
\label{sec:basics:bio_back}

\section{Artificial Neural Networks}
\label{sec:basics:Cellular_Expression}

\subsection{Convolutional Neural Networks}
\label{sec:basics:CNN}

\section{Interpretability Methods}
\label{sec:basics:interpretability_methods}
\glsresetall
% Motivation and problem
In recent years, \glspl{dnn} have been used to solve a wide variety of problems and gained popularity. Amazing results such as those achieved by Deep Mind's Alpha Fold team, have shown the great potential \gls{dnn} has to solve complex problems. However, the difficulty to interpret \glspl{dnn} has become one of the main obstacles to their acceptance in applications where the interpretability of the model is necessary.

% solution
To understand how the \glspl{dnn} predict the \gls{tr} of a cell, we use \textit{Attribution Methods}. This methods are meant to measure how much each component of the input image contributes to the model's prediction by creating a \textit{Score Map} (also known as \textit{Importance Map, Sensitivity Map} or \textit{Saliency Map}) of the same shape as the model's input. In particular, in this work we use a combination between \gls{ig} \cite{sundararajan2017axiomatic} and \gls{vg} \cite{adebayo2020sanity} as attribution method. In general we will denote attribution method as $\phi$.

% other advantages
Attribution methods are not only used to interpret black-box models like \gls{dnn}, the can also be used to debug models or as a sanity check to validate that the model base its prediction on the relevant features of the input.

% in our case
In our case, this interpretability techniques will show us which parts of the cell image are relevant for the prediction of the \gls{tr}. However, this will not just help us to interpret the results of the model, this also have the potential to help us understand unknown cellular processes.


\subsection{Integrated Gradients}
\label{sec:basics:IG}

% what it IG
%\glsresetall
\glsfirst{ig} is an interpretability technique (attribution method) proposed by Sundararajan et al. \cite{sundararajan2017axiomatic}, aimed to assign an importance to the input features (in our case pixels from a cell image) with respect to the model prediction. The attribution problem have been studied before in other papers \cite{JMLR:v11:baehrens10a}, \cite{SimonyanVZ13}, \cite{ShrikumarGSK16}, \cite{BinderMBMS16} and \cite{Springenberg}.

In our case, we seek to predict \gls{tr} given a cell image $x \in \mathbb{R}^{d \times d \times c}$, where $d$ is the height and width of the image and $c$ is the number of channels. Therefore, our \gls{dnn} would be a function $f:\mathbb{R}^{d \times d \times c} \rightarrow \mathbb{R}$ and an attribution method should be a function $\phi:\mathbb{R}^{d \times d \times c} \rightarrow \mathbb{R}^{d \times d \times c}$ having an input and output of the same shape as the model's input image.

Early interpretability methods only use gradients to assign importance to each input feature

\begin{equation}
  \begin{split}
    \phi(f,x) &:= \nabla f(x) \\
    &= \frac{\partial f}{\partial x}
  \end{split}
\end{equation}

Mathematically speaking, $\phi_i(f,x)$ assign an importance score to the pixel $i$ (out of the $d \times d \times c$ there are), representing how much it adds or subtract from the model output. However, this score maps have some drawback when they are used to interpret deep neural networks \cite{sturmfels2020visualizing}. Recall that the gradient with respect to the input indicate us the pixels that have the steepest local slope with respect to the model's output. This means that it only describes local changes in the input, and not the whole prediction model. Another mayor problem is saturation\footnote{In the context of artificial neural networks, a neuron is said to be saturated when the predominant output value of a neuron is close to the asymptotic ends of the bounded activation function. This behavior can potentially damage the learning capacity of a neural network.}.
As the model learns the relationship between an input image and its \gls{tr}, the gradient of the most important pixels will approximate to 0, i.e. the pixel's gradient saturates.

To overcome this problems, Sundararajan et al. proposed \gls{ig} as an attribution method, where the importance of the input feature $i$ is defined as follow
\begin{equation}
  \phi^{IG}_i(f, x, x') := (x_{i} - x'_{i})\int_{\alpha=0}^1\frac{\partial f(x'+\alpha (x - x'))}{\partial x_i}{d\alpha}
  \label{eq:ig:definition}
\end{equation}

Intuitively speaking, \gls{ig} accumulates the input gradient when it goes from a baseline $x'$, which should represents \textit{absence} of information, to the actual input image $x$. With this, we avoid losing information about relevant pixels for the model's prediction in the importance map, even if they saturate eventually.

For a better understanding, we can divide the \gls{ig} definition as follow
\begin{equation}
  \phi^{IG}_i(f, x, x') := \overbrace{(x_{i} - x'_{i})}^\text{Difference from baseline}
  \underbrace{\int_{\alpha=0}^1}_\text{From baseline to input...}
  \overbrace{\frac{\partial f(x'+\alpha (x - x'))}{\partial x_i}{d\alpha}}^\text{…accumulate local gradients}
  \label{eq:ig:explanation}
\end{equation}

The integral in equation \ref{eq:ig:explanation} accumulate the gradients for the interpolated images $x'+\alpha (x - x'))$ between the baseline $x'$ and the image $x$. On the other hand, the difference $(x_i - x_i')$ outside the integral comes from the chain rule and the fact that we are interested in integrating over the path between the baseline and the image.

%https://arxiv.org/pdf/1806.03000.pdf
\gls{ig} is very simple and easy to implement, since it does not require any modification to the model and it only require some calls to the gradient operator.

The \gls{ig} satisfy several properties and axioms that are addressed in detail in the paper. However, there is one axiom satisfied by \gls{ig} that is of special importance for us, \textit{completeness}. Completeness means that the value of the summed attributes will be equal to difference between the model's output when it is evaluated at the image and the model's output when it is evaluated at the baseline
\begin{equation}
  \sum_i \phi(f, x, x')^{IG} = f(x) - f(x')
  \label{eq:ig_completeness}
\end{equation}

In practice, computing the analytic expression for the integral in equation \ref{eq:ig:definition} would be complicated, and in some cases unfeasible.
However, luckily we can numerically approximate $\phi(f, x, x')^{IG}$ using a Riemann sum
\begin{equation}
  \phi^{Approx\ IG}_i(f, x, x', m) := (x_{i} - x'_{i})\sum_{k=1}^m\frac{\partial f(x'+\frac{k}{m} (x - x'))}{\partial x_i} \frac{1}{m}
  \label{eq:ig:approx}
\end{equation}

\noindent where $m$ is number of steps for the Riemann sum approximation.

This is when the completeness axiom comes into scene, which is a good value for the parameter $m$? 10, 100, 500? To answer this question, we can simply apply the completeness axiom as a sanity check for the election of $m$. If $m$ is good enough, then the value of $\sum_i \phi^{Approx\ IG}_i(f, x, x', m)$ should be close to $f(x)-f(x')$, or equivalently, the value of $|\sum_i \phi^{Approx\ IG}_i(f, x, x', m) - (f(x)-f(x'))|$ should be close to 0.

Figures \ref{fig:vg:img_gradients} and \ref{fig:vg:img_IG} show a comparative between the gradient of a model output with respect to a cell image, and the \gls{ig}. One can see that either for score maps computed using \gls{ig} or vanilla gradients, the output is noisy and diffuse.


\subsection{VarGrad}
\label{sec:basics:VarGrad}
% define where the images are
\graphicspath{{./Sections/Basics/Resources/}}
\glsresetall

As we can see in figure \ref{fig:vg:img_IG}, \gls{ig} attribution maps can be noisy and diffuse. To improve their empirical quality, Smilkov et al. \cite{Smilkov_smoothgrad} proposed \gls{sg}, which tends to reduce noise in practice and can be combined with other attribution map algorithms (like \gls{ig}). The idea behind \gls{sg} is pretty simple, given an input image $x$, you create a sample of similar images by adding noise, then compute the attribution map for each one of them using the algorithm you prefer (in our case \gls{ig}), and take the average of the attribution maps.
Although Smilkov et al. do not provide a mathematical proof of why \gls{sg} reduce noise in score maps, they provide a conjecture and empirical evidence.
For this work we use a slightly difference version called \gls{vg}, proposed by Adebayo et al. \cite{adebayo2018local} but inspired by \gls{sg}, which takes the variance of the attribution maps instead of the mean. The reason for this choice is that Seo et al. \cite{Seo_noise} analyzed theoretically \gls{vg}, and concluded that it is independent to the gradient and capture higher order partial derivatives.

In general, \gls{vg} is defined as follow

\begin{equation}
  \phi^{SG}(f, x) := Var(\phi(f, x + z_j))
\end{equation}

\noindent where $x \in \mathbb{R}^{d \times d \times c}$ is the input image, $f:\mathbb{R}^{d \times d \times c} \rightarrow \mathbb{R}$ a model, $\phi$ an attribution method to get preliminary score maps and $z_j \sim \mathcal{N}(0, \sigma^2)$, with $j\in\{1, \dots, n\}$, are i.i.d. noise images of same shape as the input image.

Since we use \gls{ig} to get preliminary score maps, in our case \gls{vg} (in the subsequent defined as \gls{vgig}) looks as follow

\begin{equation}
  \phi^{SG}(f, x) := Var(\phi^{IG}(f, x + z_j, x'))
\end{equation}

\noindent where $x' \in \mathbb{R}^{d \times d \times c}$ is a given baseline needed to compute the \gls{ig} score maps.

Figures \ref{fig:vg:img_IG} and \ref{fig:vg:img_VG_IG} show a comparative between \gls{ig} and \gls{vgig} score maps. One can see that \gls{vgig} produces less noisy score maps than vanilla \gls{ig}.

% this plots were created using the notebook ~/Documents/Master_Thesis/Project/workspace/Interpretability/Integrated_Gradient_Sanity_check.ipynb
\begin{figure}[!ht]
  \centering
  \begin{subfigure}[b]{.45\linewidth}
    \includegraphics[width=\linewidth]{Cell_Image.jpg}
    \caption{Original cell image.}
    \label{fig:vg:cell_img}
  \end{subfigure}
  \begin{subfigure}[b]{.45\linewidth}
    \includegraphics[width=\linewidth]{Image_Gradient.jpg}
    \caption{Gradient wrt the input image.}
    \label{fig:vg:img_gradients}
  \end{subfigure}%
  \vspace{3mm}
  \begin{subfigure}[b]{.45\linewidth}
    \includegraphics[width=\linewidth]{Integrated_Gradient.jpg}
    \caption{Integrated Gradient.}
    \label{fig:vg:img_IG}
  \end{subfigure}
  \begin{subfigure}[b]{.45\linewidth}
    \includegraphics[width=\linewidth]{VarGrad_Integrated_Gradient.jpg}
    \caption{VarGrad with Integrated Gradients.}
    \label{fig:vg:img_VG_IG}
  \end{subfigure}
  \caption{Comparative between a cell image and the different attribution methods. All the figures show the same 3 channels taken from a cell image. \subref{fig:vg:cell_img}) cell image, i.e. no attribution method. \subref{fig:vg:img_gradients}) score map using only the gradient of the model with respect to the input image. \subref{fig:vg:img_IG}) \acrlong{ig} score map. \subref{fig:vg:img_VG_IG}) \acrlong{vgig} score map.}
  \label{fig:vg:comparative}
\end{figure}


\section{Interpretability Methods Evaluation}
\label{sec:basics:vgig_eval}

%% =============================================================================
%% Dataset chapter
%% =============================================================================
\chapter{The Dataset}
\label{ch:dataset}
%% Magic command to compile root document
% !TEX root = ../../thesis.tex

%% Reset glossary to show long gls names
\glsresetall

We can interpret \gls{tr} as the amount of new RNA molecules inside a cell nucleus in a given period of time. By means of a fluorescent marker, it is possible to identify these new RNA molecules and thus approximate \gls{tr}. But, what about the morphology of other molecules and organelles within the cell nucleus? The distribution, shape and location of molecules, proteins and organelles within the nucleus could potentially encode relevant information for cellular expression. This has been the main motivation for this work. By means of a \gls{cnn}, we seek to predict \gls{tr} base mainly in spacial information encoded on images of cell nucleus.

In this section we introduce the process used to generate the data for this work, the \gls{mpm} protocol. In addition to this, we introduce the preprocessing and data augmentation techniques used. These techniques aim to improve the model's training performance, prevent overfitting and remove non-relevant information from the images. With this, we seek to encourage the model to base its prediction mainly on the spatial information encoded in the images of cell nucleus.


\section{Multiplexed Protein Maps}
\label{sec:dataset:multiplexed_protein_maps}
%% Magic command to compile root document
% !TEX root = ../../thesis.tex

%% Reset glossary to show long gls names
\glsresetall

%% Set path to look for the images
\graphicspath{{./Sections/Dataset/Resources/}}

% A small motivation to create Multiplexed Protein Maps
The amount of protein or \gls{mrna} inside a cell may not be enough to fully describe cellular function. Accordingly to Buxbaum et al. \cite{Buxbaum_2014} and Korolchuk et al. \cite{Korolchuk2011}, cellular function can heavily depends on the specific intracellular location and interaction with other molecules and intracellular structures. Therefore, cellular expression is determined by the functional state, abundance, morphology, and turnover of its intracellular organelles and cytoskeletal structures. This means that having the ability to look at the concentration and distribution of different molecules within a cell, is an important technological achievement that can significantly leverage scientific discoveries in biomedicine.
This is exactly what \gls{mpm} allows us to do (\cite{Guteaar7042}). \gls{mpm} are protein readouts from cell cultures, that simultaneously captures different properties of the cell, like its shape, cycle state, detailed morphology of organelles, nuclear subcompartments, etc. It also captures highly multiplexed subcellular protein maps, which can be used to identify functionally relevant single-cell states, like \gls{tr}. These maps can also identify new cellular states and allow quantitative comparisons of intracellular organization between single cells in different cell cycle states, microenvironments, and drug treatments \cite{Guteaar7042}.

So, let us explain more in deept what are these \gls{mpm}. Accordingly to Gabriele Gut et al. \cite{Guteaar7042}, \gls{mpm} is a nondegrading protocol that allows to capture efficiently thousands of single cell multichannel images, where each channel contains captures the distribution and concentration of a protein of interest inside each cell. To achieve this, the protocol is made up of different steps that will be briefly explained here.

% 4i explanation
\subsubsection{Iterative indirect immunofluorescence imaging}
The \gls{mpm} protocol starts with a process called \gls{4i} developed by the same group. The \gls{4i} is a complete protocol by itself, and it allows to capture the concentration and distribution of individual proteins in thousands of different cells in a tissue\footnote{The tissues are made from cell cultures, which were made using the HeLa Kyoto cell line. \hl{HeLa} is the oldest and most commonly used immortal human cell line used in scientific research. The story behind it quite interesting, so it's worth checking out.}.
Before applying the \gls{4i} protocol, the \hl{plate} where the cell culture is must to be divided into squared sections called \hl{wells}. Then, the \gls{4i} protocol is applied over each well and photographed in sections called \hl{sites}.

Roughly speaking, \gls{4i} works as follow
\begin{enumerate}
  % 1
  \item The selected well is prepared for the staining-elution process.
  %2
  \item A specific protein inside the cells is photographed by saturating a well with a liquid containing \hl{antibodies}\footnote{An antibody is a Y-shaped protein that can recognize and bind to a unique molecule (its antigen, e.g. a specific protein).} stained with a fluorescent ink (\gls{if}), which binds to the targeted protein.
  %3
  \item The well is exposed to a high-energy light and photographed using a light microscopy.
  %4
  \item The antibodies inside the tissue are washed-out using a chemical elution substrate.
  %5
  \item Steps 2 and 4 are repeated 20 times to get 20 images of the same protein.
  %6
  \item The 20 images are projected into a single one by \hl{maximum intensity projection}, to improve the protein readouts.
\end{enumerate}

Figure \ref{fig:4i:1} illustrates the steps of the \gls{4i} protocol to captures the saturation and distribution of a specific protein. Keep in mind that even though the \gls{4i} protocol captures sever images of the tissue, it returns an uni-channel image (step 6). Figure \ref{fig:4i:2} shows the \gls{4i} protocol applied 40 times with different \gls{if} and over a 384-well plate, to capture the concentration and distribution of 40 different targeted proteins.

\begin{figure}[htb]
  \centering
  \begin{subfigure}[t]{.3\linewidth}
    \includegraphics[width=\linewidth]{4i_1.png}
    \caption{\Acrfull{4i} protocol.}
    \label{fig:4i:1}
  \end{subfigure}
  \hspace{4mm}
  \begin{subfigure}[t]{.45\linewidth}
    \includegraphics[width=\linewidth]{4i_2.png}
    \caption{\gls{4i} protocol applied over a specific well of plate and for 40 different \gls{if}.}
    \label{fig:4i:2}
  \end{subfigure}%
  \caption{Schematic representation of the \gls{4i} protocol for a single well and for 40 different fluorescent antibodies. Figure \subref{fig:4i:2} also shows the image analysis to identify single cells and its components (nucleus and cytoplasm). Images source: \cite{Guteaar7042}.}
  \label{fig:4i}
\end{figure}

By the time \cite{Guteaar7042} was published, the \gls{4i} protocol was able to capture cell culture images with up to 40 channels without degrading the tissue, which is why \gls{mpm} is called a \textit{nondegrading} protocol.

\subsubsection{Multiplexed single cell analysis}

Once the multichannel images were generated using the \gls{4i} protocol, a series of image preprocessing and image analysis methods (\cite{Carpenter2006} and \cite{snijder2012single}) are applied to generate segmentation masks to identify individual cells, as well as their cytoplasm and nucleus. Figure \ref{fig:4i:2} shows this segmentation at a cellular level, while figure \ref{fig:4i:segmentation} shows it also at a subcellular level. In both cases the boundaries are marked with white lines. This single cell analysis is also used to identify cells that do not satisfy certain quality controls (like cells in the border of the image or in mitosis stage). However, this will be addressed in detail on section \ref{sec:dataset:data_pp}.

\begin{figure}[htb]
  \centering
  \includegraphics[width=0.5\linewidth]{4i_segmentation.png}
  \caption{Visualization of the subcellular segmentation of a \gls{4i} protocol for 18 \gls{if} stains. The image was created by combining the readouts of 3 of this \gls{if} stains: PCNA (cyan), FBL (magenta) and TFRC (yellow). The number next to each staining label indicates their corresponding 4i acquisition cycle (\gls{4i} protocol step 5). The orange rectangle and the tile at its right shows a section of the nucleus and cytoplasm of a single cell. The other 3 tiles shows the \gls{4i} readout of each of the 3 proteins.}
  \label{fig:4i:segmentation}
\end{figure}

\subsubsection{Multiplexed single-pixel analysis framework}
Even though the cell cultures are now segmented into individual cells and nucleus, there is still one missing part that must be considered, and that is that cells are 3-dimensional objects. Recall that the \gls{4i} protocol saturates the cell culture with a liquid containing fluorescent antibodies. This means that the antibody can either bind to its corresponding protein inside or outside the cell nucleus. Therefore, even though that we segmented a cell into nucleus and cytoplasm, a readout assigned to the nucleus could come from a protein in the cytoplasm under or above the nucleus, and not from inside it. Fortunately, readouts from proteins inside the nucleus are much higher than those in the cytoplasm. Therefore, by means of a two steps clustering approach\footnote{To identify clusters in an unsupervised manner, \hl{Self Organizing Maps} algorithm and \hl{Phenograph} analysis were used over a very large number of pixels sampled from a large number of single cells \cite{Guteaar7042}.} pixels can be classified accordingly to their intensity profile (figures \ref{fig:mcu:1} and \ref{fig:mcu:2}), so the source of their readout can be identified. This pixel type classification is called \Acrfull{mcu} and is illustrated in figure \ref{fig:mcu:3}. After pixels clusters (intensity profiles) where identified, the pixels whose measurement comes from the cytoplasm and not from the nucleus are removed.

\begin{figure}[htb]
  \centering
  \begin{subfigure}[t]{.3\linewidth}
    \includegraphics[width=\linewidth]{mcu_1.png}
    \caption{Extraction of pixel intensities.}
    \label{fig:mcu:1}
  \end{subfigure}
  \hspace{4mm}
  \begin{subfigure}[t]{.3\linewidth}
    \includegraphics[width=\linewidth]{mcu_2.png}
    \caption{Pixel clustering by Self Organizing Maps and Phenograph.}
    \label{fig:mcu:2}
  \end{subfigure}
  \hspace{4mm}
  \begin{subfigure}[t]{.3\linewidth}
    \includegraphics[width=\linewidth]{mcu_3.png}
    \caption{Cell subdivision base on the \gls{mcu}.}
    \label{fig:mcu:3}
  \end{subfigure}
  \caption{Figure \subref{fig:mcu:1} shows the pixel intensity extraction for a single cell. The pixel intensity is a vector containing the readout of that 2D location for each protein, one specific protein readout per entrance. Figure \subref{fig:mcu:2} shows the clusters found by Self Organizing Maps algorithm and Phenograph analysis over the pixel intensities. Figure \subref{fig:mcu:3} shows a cell masked with the clusters found by the \gls{mcu} analysis. Images source: \cite{Guteaar7042}.}
  \label{fig:mcu}
\end{figure}

Finally, the nucleus of each cell is stored separately and identified with a unique id.\fxnote{After you finish writing the dataset section review if this sentence is accurate.}

\subsubsection{Cell cycle phase classification: $G_1,\ S,\ G_2$ and $M$ phase}

The \gls{mpm} protocol is not only capable to capture the concentration and distribution of molecules inside thousands of cells. It can also identify the phase each cell is in, which is tightly related with the abundances and distribution of molecules inside a cell \cite{Guteaar7042}.

Roughly speaking, cell cycle phase was determined by means of a \gls{svm} classifier and k-means clustering. First, a \gls{svm} classifier is trained to identify $M$ phase cells based on the nuclear information in one of the image channels (\hl{DAPI}\footnote{A brief description of this marker can be bound on section \ref{sec:appendix:if_markers}.}). Then, based on the nuclear information of channel \hl{PCNA}, a second \gls{svm} classifier is trained to identify cells in phase $S$. Finally, cells in phase $G_1$ and $G_2$ are classified using a k-means algorithm, using the pixel intensity profiles of the DAPI channels excluding the cells in $S$ and $M$ phase. A more detailed explanation of the cell cycle classification process can be found on the dataset paper \cite{Guteaar7042}.

\subsubsection{Pharmacological and metabolic perturbations}

To further explore the capabilities of the \gls{mpm} protocol, the creators of the dataset (Gabriele Gut et al. \cite{Guteaar7042}) applied the \gls{mpm} protocol to a cell populations that were to nine pharmacological and metabolic perturbations. The analysis reveled expected and unexpected changes in the concentration and distribution of molecules inside the cell. However, this work focused on cells without pharmacological and metabolic perturbations. This means that only cells marked as \hl{normal} (no perturbed cells) and \hl{DMSO}\footnote{Dimethyl sulfoxide, or DMSO, is an organic compound used to dissolve test compounds in in drug discovery and design \cite{cushnie2020bioprospecting}.} (control cells) were used.

\subsubsection{Dataset description}

The provided dataset contains the following features

\begin{itemize}
  \item How many single cells were given.
  \item Mention how many normal and DMSO.
  \item Number of cells per perturbation.
  \item Number of channels and name of the channels (only the dataset channel id and channel name. The description and implementation names are on the appendix.)
  \item Information given in the metadata (cell identifier: $mapobject\_id\_cell$, cell cycle, well, site, mitotic or not, etc.)
\end{itemize}

\fxnote{Add this info when you write the preprocessing section.}


\section{Data preprocessing}
\label{sec:dataset:data_pp}

Some technical details about \gls{4i}:
\begin{itemize}
  \item The images where obtained using a 40x objective and a \gls{scmos} camera, where the surface of each pixel is 165 nm by 165 nm.
  \item Each photo captures around 20,000 single cells.
  \item In \cite{Guteaar7042} (Fig 2), it is mentioned that use of 384-well plates, where wells were imaged at 40× magnification in a 7 × 6 tiled fashion for 21 4i cycles.
  \item The used cultured cells (tissue) where HeLa. HeLa is an immortal cell line used in scientific research.
\end{itemize}

The information for each single cell is not stored as a separated image as one may think, instead the information is first divided by Wells. Then, the information of the cells in a given well is stored in 7 files \fxnote{this is may not be included in the final work}:
\begin{itemize}
  \item \texttt{metadata.csv}: contains information of each cell in the well
  \item \texttt{channels.csv}: contains information of each protein (channel) photographed
  \item \texttt{labels.npy}: 1 dim array containing the cell label of each pixel
  \item \texttt{mpp.npy}: 2 dim array, where size of first dim is the same as number of measured pixels and size of second dim is the same as the number of channels. This file contains the observed measured values (intensities) of each pixel and for each protein. The values in \texttt{mpp.npy} vary from 0 to 65535 i.e. $2^{16}$ i.e. 2 bytes or 16 bits.
  \item \texttt{x.npy} and \texttt{y.npy}: 1 dim array containing the x/y coordinates of the measured pixels (pixels where a signal was detected by the \gls{scmos} camera. Accordingly with \cite{Guteaar7042}, the size of a single cell image (for each channel) is 2560x2160. Therefore, the values in \texttt{x.npy} vary between 1 and 2560 and form 1 to 2160 for \texttt{x.npy}
  \item \texttt{mapobject\_ids.npy}: 1 dim array where its size is the same as the number of rows in the \texttt{metadata.csv} file. Therefore, \texttt{mapobject\_ids.npy} maps the information given in the metadata file with each pixel in the well.
\end{itemize}

\textbf{QUESTION TO HANNAH:} In the paper it is mentioned that for each channel (protein), the tissue is stained with with 2 fluorescent antibodies (in that case TUBA1A and CTNNB1, NOT at the same time, TUBa1A-elution-CTNNB1), why? to create a background as reference for the protein that is being photographed?
Answer: To increase the measured signal, \ie to make the protein Shine more.

Since we are using a \gls{cnn} to predict the \gls{tr}, therefore we need the data as multichannel images. However, as we already explained in \ref{sec:Motivation_and_Background:Dataset}, the information is providad in a pixel by pixel format. Using the library \texttt{mpp\_data.py} written by Dr. Hannah Spitzer \fxnote{See how to refer properly to this library}, the raw data is transformed to multichannel images (one channel per protein measured). During this process, the bordered cells and cells in division process (Mitosis state) are excluded. The provided data set also contain the target variable, \ie the \gls{tr} (amount of \gls{mrna} molecules produced by the cell nucleus in the last 30 minutes). However, this information is coded also as a protein channel (00\_EU). Therefore, after converting the data set into multichannel images, the channel corresponding to the protein 00\_EU is extracted and converted into a number \fxnote{It is necessary to improve the description of how this 00\_EU channel is turned into a number. So far it is done by averaging all the entrances of the channel, however it still need to be decided if this will be done like this.}.


\section{Data Augmentation}
\label{sec:dataset:data_augmentation}

%% =============================================================================
%% Methodology chapter
%% =============================================================================
\chapter{Methodology}
\label{ch:methodology}
%% Magic command to compile root document
% !TEX root = ../../thesis.tex

%% Reset glossary to show long gls names
\glsresetall

We can interpret \gls{tr} as the amount of new RNA molecules inside a cell nucleus in a given period of time. By means of a fluorescent marker, it is possible to identify these new RNA molecules and thus approximate \gls{tr}. But, what about the morphology of other molecules and organelles within the cell nucleus? The distribution, shape and location of molecules, proteins and organelles within the nucleus could potentially encode relevant information for cellular expression. This has been the main motivation for this work. By means of a \gls{cnn}, we seek to predict \gls{tr} base mainly in spacial information encoded on images of cell nucleus.

In this section we introduce the process used to generate the data for this work, the \gls{mpm} protocol. In addition to this, we introduce the preprocessing and data augmentation techniques used. These techniques aim to improve the model's training performance, prevent overfitting and remove non-relevant information from the images. With this, we seek to encourage the model to base its prediction mainly on the spatial information encoded in the images of cell nucleus.


\section{Dataset Setup}
\label{sec:methodology:tfds}

\section{Models}
\label{sec:methodology:models}

\subsection{Linear Model}
\label{sec:methodology:lm}

\subsection{Baseline CNN}
\label{sec:methodology:BL_CNN}

\subsection{ResNet50V2}
\label{sec:methodology:RN50V2}

\subsection{Model Metrics}
\label{sec:methodology:metrics}

\section{Interpretability Methods}
\label{sec:methodology:interpretability_methods}
\glsresetall
\graphicspath{{./Sections/Methodology/Resources/}}

There are several hyper-parameters that need to be chosen in order to compute the score map for each cell image.

For the \gls{ig} attribution map, recall that in practice computing $\phi^{IG}$ could be unfeasible or computationally very expensive. However, we can approximate $\phi^{IG}$ by means of $\phi^{Approx\ IG}$ (see equation \ref{eq:ig:approx}). Therefore, we need to define the number of steps $m$ for the Riemann sum approximation. In section \ref{sec:basics:IG} we also mentioned the necessity to set a baseline image $x'$, which should contain no information about the image, in order to compute the \gls{ig}. There are several options that can be used, each one of them with different advantages and disadvantages. However, for this work we only implemented two of them: 1) a simple black image (image containing only zeros) and 2) an image filled with Gaussian noise ($\mu=0,\ \sigma=1$). A very good analysis on the choice of the baseline can be found in this reference \cite{sturmfels2020visualizing}.

In section \ref{sec:basics:VarGrad} we saw that for \gls{vg} we need to define 2 parameters, the number of noisy images $n$ (sample size) and the standard deviation $\sigma$ for the the noise distribution.

As a rule of thumbs, a sample should not be smaller than 30, so this could be a feasible option. However, since Smilkov et al. \cite{Smilkov_smoothgrad} showed empirically that no further improvemnt (less noise) in score maps is observed for sample sizes greater than 50, we chose this bound as sample size.

Table \ref{table:VGIG_exp_set:params} shows a summary of the parameters chosen to calculate the \gls{vgig} score maps.

\begin{table}[!ht]
  \centering
  \begin{tabular}{c|c|c}
    Method & Hyperparameter & Value \\
    \hline
    \multirow{2}{*}{\gls{ig}} & $m$ & 70 \\
    \cline{2-3}
     & $x'$ & black image \\
    \hline
    \multirow{2}{*}{\gls{vg}} & $n$ & 50 \\
    \cline{2-3}
     & $\sigma$ & 1 \\
    \hline
  \end{tabular}
  \caption{Parameters to compute score maps.}
  \label{table:VGIG_exp_set:params}
\end{table}

In section \ref{sec:basics:IG}, we mentioned that the \gls{ig} algorithm holds the \textit{Completeness Axiom}, which means that the sum of all the components of the \gls{ig} attribution map must be equal to the difference between the model's output evaluated at the image and the model's output evaluated at the baseline (see equation \ref{eq:ig_completeness}). This property allow us to check empirically if the number of steps $m$ selected for the Riemann sum approximation is sufficiently large. Figure \ref{fig:VGIG_exp_set:m_sanity} shows that for our baseline\fxnote{after finishing, check that the baseline model is still called baseline} model, a random image and $m=70$, the  completeness axiom is satisfied sufficiently well.

\begin{figure}[!ht]
  \centering
  \includegraphics[width=0.8\linewidth]{sanity_check_for_m.jpg}
  \caption{Sanity check for the number of steps $m$ in the Riemann sum to approximate $\phi^{IG}$. The red dotted line represent the difference $f(x)-f(x')$. The blue line represents the value of $\sum_i \phi^{Approx\ IG}_i(f, x, x', m)$ over $\alpha$.}
  \label{fig:VGIG_exp_set:m_sanity}
\end{figure}


%\subsection{Experimental Setup}
%\label{sec:methodology:VarGrad_IG_Experimental_Setup}
%\glsresetall
\graphicspath{{./Sections/Methodology/Resources/}}

There are several hyper-parameters that need to be chosen in order to compute the score map for each cell image.

For the \gls{ig} attribution map, recall that in practice computing $\phi^{IG}$ could be unfeasible or computationally very expensive. However, we can approximate $\phi^{IG}$ by means of $\phi^{Approx\ IG}$ (see equation \ref{eq:ig:approx}). Therefore, we need to define the number of steps $m$ for the Riemann sum approximation. In section \ref{sec:basics:IG} we also mentioned the necessity to set a baseline image $x'$, which should contain no information about the image, in order to compute the \gls{ig}. There are several options that can be used, each one of them with different advantages and disadvantages. However, for this work we only implemented two of them: 1) a simple black image (image containing only zeros) and 2) an image filled with Gaussian noise ($\mu=0,\ \sigma=1$). A very good analysis on the choice of the baseline can be found in this reference \cite{sturmfels2020visualizing}.

In section \ref{sec:basics:VarGrad} we saw that for \gls{vg} we need to define 2 parameters, the number of noisy images $n$ (sample size) and the standard deviation $\sigma$ for the the noise distribution.

As a rule of thumbs, a sample should not be smaller than 30, so this could be a feasible option. However, since Smilkov et al. \cite{Smilkov_smoothgrad} showed empirically that no further improvemnt (less noise) in score maps is observed for sample sizes greater than 50, we chose this bound as sample size.

Table \ref{table:VGIG_exp_set:params} shows a summary of the parameters chosen to calculate the \gls{vgig} score maps.

\begin{table}[!ht]
  \centering
  \begin{tabular}{c|c|c}
    Method & Hyperparameter & Value \\
    \hline
    \multirow{2}{*}{\gls{ig}} & $m$ & 70 \\
    \cline{2-3}
     & $x'$ & black image \\
    \hline
    \multirow{2}{*}{\gls{vg}} & $n$ & 50 \\
    \cline{2-3}
     & $\sigma$ & 1 \\
    \hline
  \end{tabular}
  \caption{Parameters to compute score maps.}
  \label{table:VGIG_exp_set:params}
\end{table}

In section \ref{sec:basics:IG}, we mentioned that the \gls{ig} algorithm holds the \textit{Completeness Axiom}, which means that the sum of all the components of the \gls{ig} attribution map must be equal to the difference between the model's output evaluated at the image and the model's output evaluated at the baseline (see equation \ref{eq:ig_completeness}). This property allow us to check empirically if the number of steps $m$ selected for the Riemann sum approximation is sufficiently large. Figure \ref{fig:VGIG_exp_set:m_sanity} shows that for our baseline\fxnote{after finishing, check that the baseline model is still called baseline} model, a random image and $m=70$, the  completeness axiom is satisfied sufficiently well.

\begin{figure}[!ht]
  \centering
  \includegraphics[width=0.8\linewidth]{sanity_check_for_m.jpg}
  \caption{Sanity check for the number of steps $m$ in the Riemann sum to approximate $\phi^{IG}$. The red dotted line represent the difference $f(x)-f(x')$. The blue line represents the value of $\sum_i \phi^{Approx\ IG}_i(f, x, x', m)$ over $\alpha$.}
  \label{fig:VGIG_exp_set:m_sanity}
\end{figure}


\section{Interpretability Methods Evaluation}
\label{sec:methodology:vgig_eval}

%% =============================================================================
%% Results chapter
%% =============================================================================
\chapter{Results}
\label{ch:results}
%% Magic command to compile root document
% !TEX root = ../../thesis.tex

%% Reset glossary to show long gls names
\glsresetall

We can interpret \gls{tr} as the amount of new RNA molecules inside a cell nucleus in a given period of time. By means of a fluorescent marker, it is possible to identify these new RNA molecules and thus approximate \gls{tr}. But, what about the morphology of other molecules and organelles within the cell nucleus? The distribution, shape and location of molecules, proteins and organelles within the nucleus could potentially encode relevant information for cellular expression. This has been the main motivation for this work. By means of a \gls{cnn}, we seek to predict \gls{tr} base mainly in spacial information encoded on images of cell nucleus.

In this section we introduce the process used to generate the data for this work, the \gls{mpm} protocol. In addition to this, we introduce the preprocessing and data augmentation techniques used. These techniques aim to improve the model's training performance, prevent overfitting and remove non-relevant information from the images. With this, we seek to encourage the model to base its prediction mainly on the spatial information encoded in the images of cell nucleus.


\section{Model Performance}
\label{sec:model_performance}

\subsection{Baseline values}
\label{sec:results:bl_values}

\subsection{Linear Model}
\label{sec:results:lm}

\subsection{Baseline CNN}
\label{sec:results:bl_cnn}

\subsection{ResNet50V2}
\label{sec:results:RN50V2}

\section{Model Interpretation}
\label{sec:results:model_interpretation}

\section{Model Interpretation Evaluation}
\label{sec:results:model_inter_eval}

\section{Discussion}

%% =============================================================================
%% conclusion chapter
%% =============================================================================
%% conclusion.tex
%%

%% ==================
\chapter{Conclusion}
\label{ch:Conclusion}
%% ==================

Nothing new here, only a short recap of the project, it's results, as well as possible future work.

%%% Local Variables:
%%% mode: latex
%%% TeX-master: "thesis"
%%% End:


%% =============================================================================
%% Appendix chapter
%% =============================================================================
\appendix
%% appendix.tex
%%

%% ==============================
%\chapter{Appendix}
%\label{ch:Appendix}
\chapter{Remarks on Implementation}
\label{Appendix-Implementation}
%% ==============================

This appendix contains notes about how to execute all the scripts and notebooks used in this work. It also contains information about the parameters that need to be specified for each program.
All the scripts and notebooks were written in \texttt{Python} and executed over \texttt{Anaconda}. You can find information about the environment setup, packages version, etc. \href{https://github.com/andresbecker/master_thesis}{here}.

The logic to execute any \texttt{Python} script is always the same
\begin{lstlisting}[language=Bash]
python python_script_name.py -p ./Parameters_file_name.json
\end{lstlisting}

For the \texttt{Jupyter Notebooks} you just have to open it and set the variable \texttt{PARAMETERS\_FILE} with the absolute path and name of the input parameters file
\begin{lstlisting}[language=Python]
PARAMETERS_FILE = "/path_to_file_dir/Parameters_file_name.json"
\end{lstlisting}

For each script/notebook all the needed parameters have to be specified inside its parameter file only. The format for the parameters file is always \texttt{JSON} and the parameters values are specified in a python-dictionary format.

\fxnote{If there is time, add a section containing common parameters}

\section{VarGrad IG Implementation Notes}
\label{sec:appendix:VarGrad_IG_Experimental_Setup}
%% Magic command to compile root document
% !TEX root = ../../thesis.tex

%% Reset glossary to show long gls names
\glsresetall

In order to generate the \acrlong{vg} \acrlong{ig} score maps, you must execute the python script \texttt{get\_VarGradIG\_from\_TFDS.py} specifying the parameters file
\begin{lstlisting}[language=Bash]
python get_VarGradIG_from_TFDS.py -p ./Parameters_file_name.json
\end{lstlisting}

Table \ref{table:imp_notes:VGIG_params} show all the parameters that need to be specified to execute \\
\noindent \texttt{get\_VarGradIG\_from\_TFDS.py} successfully.

% set table lengths
\setlength{\mylinewidth}{\linewidth-7pt}%
\setlength{\mylengtha}{0.18\mylinewidth-2\arraycolsep}%
\setlength{\mylengthb}{0.27\mylinewidth-2\arraycolsep}%
\setlength{\mylengthc}{0.55\mylinewidth-2\arraycolsep}%

\begin{table}[!ht]
  \centering
  \begin{tabular}{>{\centering\arraybackslash}m{\mylengtha}|>{\centering\arraybackslash}m{\mylengthb}|m{\mylengthc}} % m stands for middle (p:top, b:bottom), max 144 mm
    \hline
    Hyperparam & JSON variable name & Notes \\
    \hline
    $m$ & \texttt{IG\_m\_steps} & Number of steps to approximate \gls{ig} \\
    \hline
    $x'$ &  \texttt{IG\_baseline} & Baseline image for \gls{ig}. Available: "black" for a simple black image and "noise" for an image filled with Gaussian noise ($\mu=0,\ \sigma=1$) \\
    \hline
    $n$ & \texttt{VarGrad\_n\_samples} & Number of noisy images to compute \gls{vg} \\
    \hline
  \end{tabular}
  \caption{Parameters to compute score maps.}
  \label{table:imp_notes:VGIG_params}
\end{table}



%%% Local Variables:
%%% mode: latex
%%% TeX-master: "thesis"
%%% End:

% all appendices behind backmatter will go without numbers
\backmatter

%% =============================================================================
%% Lists, glossaries, etc.
%% =============================================================================

%List of Figures
\listoffigures

\vspace*{1.5cm}

%List of Tables
\listoftables


%List of algorithms
% only in conjunction with algorithm2e
%\phantomsection% for hyperref
%\addcontentsline{toc}{chapter}{Algorithmenverzeichnis}%
%\markboth{Algorithmenverzeichnis}{Algorithmenverzeichnis}
%\listofalgorithms


%Algorithmenverzeichnis
% nur in Verbindung mit algorithm2e
%\phantomsection% fuer hyperref
%\addcontentsline{toc}{chapter}{Algorithmenverzeichnis}%
%\markboth{Algorithmenverzeichnis}{Algorithmenverzeichnis}
%\listofalgorithms

% Index
% Add index to table of contents
\addcontentsline{toc}{chapter}{Index}
\printindex

%Print the glossary
%\printglossaries
\printglossary[type=\acronymtype]

% Bibliography
\printbibliography[heading=bibintoc]

%In the final version, this should be empty and needn't be commented out. Someone who works sloppily should at least remember to comment out the fixme list, so that unfinished places are
%not so obvious for the examiners.
\listoffixmes
%
\end{document}
% ==============================================================================
% End of document
% ==============================================================================
