% ==============================================================================
% thesis.tex
% Example file for tumthesis.csl
% Michael Ritter, 2012
% Licence:
% This work may be distributed and/or modified under the
% conditions of the LaTeX Project Public License, either version 1.3
% of this license or (at your option) any later version.
% The latest version of this license is in
% http://www.latex-project.org/lppl.txt
% and version 1.3 or later is part of all distributions of LaTeX
% version 2005/12/01 or later.
% ==============================================================================
\documentclass[biblatexBackend=bibtex]{tumthesis}

% ------------------------------------------------------------------------------
%FixMe-Status: final (no FixMe comments) or draft (comments visible)
\fxsetup{draft}
%\fxsetup{final}
% ------------------------------------------------------------------------------

% ------------------------------------------------------------------------------
%  Language selection for metadata and main text (can be changed at any point in
%  the main text.
%\selectlanguage{ngerman}
\selectlanguage{english}
% ------------------------------------------------------------------------------

% ------------------------------------------------------------------------------
% Data for the bibliography
\addbibresource{thesis.bib}
% ------------------------------------------------------------------------------

% ------------------------------------------------------------------------------
% Further packages and TikZ libraries can be incorporated here.
\usetikzlibrary{arrows}
% ------------------------------------------------------------------------------

% ------------------------------------------------------------------------------
% PDF-Metadaten
\hypersetup{
 pdfauthor={Andres Alberto Becker Sanabria},
 pdftitle={Predicting transcription rate from multiplexed protein maps using deep learning},
 pdfsubject={Predicting transcription rate from multiplexed protein maps using deep learning},
 pdfkeywords={Master's Thesis},
 colorlinks=true, %coloured links (for the PDF version)
% colorlinks=false, % no coloured links (for the print version)
}
% ------------------------------------------------------------------------------

% ------------------------------------------------------------------------------
% Data for the title page and declaration
\author{Andres Becker}
\title{Predicting transcription rate from multiplexed protein maps using deep learning}
%\subtitle{A Tutorial for Theses}
%\faculty{Fakultät für Mathematik}
\faculty{Department of Mathematics}
%\institute{Lehrstuhl für Mathematische Modelle biologischer Systeme}
\institute{Chair of Mathematical Modeling of Biological Systems}
\subject{master}
%\subject{bachelor}
%\subject{diploma}
%\subject{project}
%\subject{seminar}
%\subject{idp}
%\subject{Short Overview}
\professor{Prof. Dr. Fabian J. Theis} %Themensteller
\advisor{Dr. Hannah Spitzer} %Betreuer
\date{April 14, 2021} %Submission Date
\place{München} %Place where document is signed
% ------------------------------------------------------------------------------

% ==============================================================================
% Acronyms
% ==============================================================================

% Packages should be called in the preamble document (preamble.tex). However,
% glossaries package requires to be called after hyperref, babel, polyglossia,
% inputenc and fontenc. Therefore, calling glossaries package in preamble doc
% produce errors during compilation (and no generation of Acronyms section).

%\usepackage[style=long,nonumberlist, toc,acronym,nomain]{glossaries}
% style=long: center the acronyms in acronym section
% nonumberlist: remove the list all places in docu where the acron was called
% toc: Add Acronyms section to table of content
% acronym, nomain: necessary
\usepackage[acronym, toc, nomain]{glossaries}
\makeglossaries

%\newacronym[<options>]{<label>}{<short>}{<long>}
\newacronym{4i}{4i}{iterative indirect immunofluorescence imaging}
\newacronym{if}{IF}{Indirect immunofluorescence}
\newacronym{mpm}{MPM}{multiplexed protein map}
\newacronym{mcu}{MCU}{multiplexed cell unit}
\newacronym{tr}{TR}{transcription rate}
\newacronym{scmos}{sCMOS}{scientific complementary metal oxide semiconductor}
\newacronym{ml}{ML}{Machine Learning}
\newacronym{cnn}{CNN}{Convolutional Neural Network}
\newacronym{mlp}{MLP}{Multilayer Perceptron}
\newacronym{ann}{ANN}{Artificial Neural Network}
\newacronym{relu}{ReLU}{Rectified Linear Unit}
\newacronym{sgd}{SGD}{Stochastic Gradient Descent}
\newacronym{gd}{GD}{Gradient Descent}
\newacronym{adam}{Adam}{Adaptive Moment Estimation}
\newacronym{mrna}{mRNA}{messenger RNA}
\newacronym{pmrna}{pre-mRNA}{pre-messenger RNA}
\newacronym{ig}{IG}{Integrated Gradient}
\newacronym{vg}{VG}{VarGrad}
\newacronym{sg}{SG}{SmoothGrad}
\newacronym{vgig}{VGIG}{VarGrad Integrated Gradient}
\newacronym{tfds}{TFDS}{TensorFlow Dataset}
% Metrics
\newacronym{mae}{MAE}{Mean Absolute Error}
\newacronym{mse}{MSE}{Mean Squared Error}
% Models
\newacronym{dnn}{DNN}{Deep Neural Network}
\newacronym{svm}{SVM}{Support Vector Machine}
% ------------------------------------------------------------------------------

% ==============================================================================
% Costume commands
% ==============================================================================
% Defined to highlight new words
\newcommand{\hl}[1]{\textit{#1}}
% change the thickness of horizontal lines in tables
\newcommand{\ChangeRT}[1]{\noalign{\hrule height #1}}
% rename commad to make symbols bold in equation mode
\newcommand{\bs}[1]{\boldsymbol{#1}}
% ------------------------------------------------------------------------------

% ==============================================================================
% Utilities and curiosities
% ==============================================================================
% print page width (in pts)
%\the\textwidth
%Output: 408.2971
% in mm: 144.03814361, https://www.unitconverters.net/length/point-to-millimeter.html

% set some variables to store linewidths (to create tables that fits the page)
\newlength{\mylinewidth}
\newlength{\mylengtha}
\newlength{\mylengthb}
\newlength{\mylengthc}
\newlength{\mylengthd}
\newlength{\mylengthe}
\newlength{\mylengthf}
\newlength{\mylengthg}
\newlength{\mylengthh}

% ==============================================================================
% Main part of the document
% ==============================================================================
\makeindex[title=Index,options=-s myindex]
\begin{document}
\pagestyle{empty}
\frontmatter%
%\selectlanguage{ngerman}
\selectlanguage{english}
\maketitlepage%
%\maketitlepageDissertation % a more elegant one, but for PHd dissertation
%\makedeclaration%

%% ==============================
\chapter*{Declaration}
%% ==============================

\vfill

I hereby declare that this thesis is my own work and that no other sources have been used except those clearly indicated and referenced.

\vspace{3em}

\noindent Andres Alberto Becker Sanabria, München, 14.05.2021

%==================================================
% abstract.tex
% Beispieldatei für tumthesis.cls und thesis.tex
% Michael Ritter, 2012
% Lizenz:
% This work may be distributed and/or modified under the
% conditions of the LaTeX Project Public License, either version 1.3
% of this license or (at your option) any later version.
% The latest version of this license is in
% http://www.latex-project.org/lppl.txt
% and version 1.3 or later is part of all distributions of LaTeX
% version 2005/12/01 or later.
%==================================================

%% Magic command to compile root document
% !TEX root = ../thesis.tex

%% Reset glossary to show long gls names
\glsresetall

%\cleardoublepage

\selectlanguage{english}
\section*{Abstract}

By means of fluorescent antibodies it is possible to observe the amount of nascent RNA within the nucleus of a cell, and thus estimate its \gls{tr}. But what about the other molecules, proteins, organelles, etc. within the nucleus of the cell? Is it possible to estimate the \gls{tr} using only the shape and distribution of these subnuclear components? By means of multichannel images of single cell nucleus (obtained through the \gls{mpm} protocol \cite{Guteaar7042}) and \glspl{cnn}, we show that this is possible. 
Applying pre-processing and data augmentation techniques, we reduce the information contained in the intensity of the pixels and the correlation of these between the different channels. This allowed the \gls{cnn} to focus mainly on the information provided by the location, size and distribution of elements within the cell nucleus.
For this task different architectures were tried, from a simple \gls{cnn} (with only 160k parameters), to more complex architectures such as the ResNet50V2 or the Xception (with more than 20m parameters).
Furthermore, through the interpretability methods \gls{ig} and \gls{vg}, we could obtain score maps that allowed us to observe the pixels that the \gls{cnn} considered as relevant to predict the \gls{tr} for each cell nucleus input image. The analysis of these score maps reveals how as the \gls{tr} changes, the \gls{cnn} focuses on different proteins and areas of the nucleus. This shows that interpretability methods can help us to understand how a \gls{cnn} make its predictions and learn from it, which has the potential to provide guidance for new discoveries in the field of biology.

\selectlanguage{english}
\section*{Acknowledgments}

To my father, my partner in my wildest adventures, best friend in life and who taught me what are the important things in life. He may never read this, but let the world known he is a loved and admired man.


%%% Local Variables:
%%% mode: latex
%%% TeX-master: "thesis"
%%% End:


\tableofcontents%

\mainmatter%
\pagestyle{headings}

%% =============================================================================
%% Introduction chapter
%% =============================================================================
\chapter{Introduction}
\label{ch:introduction}
%% Magic command to compile root document
% !TEX root = ../../thesis.tex

%% Reset glossary to show long gls names
\glsresetall

We can interpret \gls{tr} as the amount of new RNA molecules inside a cell nucleus in a given period of time. By means of a fluorescent marker, it is possible to identify these new RNA molecules and thus approximate \gls{tr}. But, what about the morphology of other molecules and organelles within the cell nucleus? The distribution, shape and location of molecules, proteins and organelles within the nucleus could potentially encode relevant information for cellular expression. This has been the main motivation for this work. By means of a \gls{cnn}, we seek to predict \gls{tr} base mainly in spacial information encoded on images of cell nucleus.

In this section we introduce the process used to generate the data for this work, the \gls{mpm} protocol. In addition to this, we introduce the preprocessing and data augmentation techniques used. These techniques aim to improve the model's training performance, prevent overfitting and remove non-relevant information from the images. With this, we seek to encourage the model to base its prediction mainly on the spatial information encoded in the images of cell nucleus.


\section{Motivation}
\label{sec:intro:motivation}
%% Magic command to compile root document
% !TEX root = ../../thesis.tex

\glsresetall

Understanding how RNA concentration in eukaryotes cells is regulated is very important to understand gene expression. However, measuring the amount of RNA inside a cell, may not be enough to fully describe cellular function. Accordingly to Buxbaum et al. \cite{Buxbaum_2014} and Korolchuk et al. \cite{Korolchuk2011}, cellular function can heavily depend on the specific intracellular location and interaction with other molecules and intracellular structures.
According to Vogel et al. \cite{vogel2010sequence}, explaining gene expression using only nuclear protein abundance can be done only until a certain extent.

In recent years the development of new methods for capturing spatial organization and distribution of several subnuclear bodies and proteins/molecules in high resolution images \cite{Guteaar7042}, makes the use of \gls{cnn} models to analyze this information a natural choice. As \glspl{cnn} have proven to be powerful tools in the recognition of spatial patterns \cite{krizhevsky2012imagenet}.
The use of \glspl{cnn} not only would allow us to estimate gene expression. The continuous development of interpretability methods \cite{hooker2018benchmark}, would also allow us to understand how these models work and learn from them.

For these reasons, the use of models capable of focusing on subnuclear spatial information, can potentially help us to understand better cell expression.


\section{Goal of this Thesis}
\label{sec:intro:goals}
The objective of this work is to predict cell \gls{tr} through cell images that show specific proteins inside the nucleus and \gls{cnn}. This work also try to understand better the factors that influence cell \gls{tr} by applying interpretability techniques to the \gls{cnn} model.


\section{Literature review}
\label{sec:intro:literature_review}
%% Magic command to compile root document
% !TEX root = ../../thesis.tex

%% Set path to look for the images
\graphicspath{{./Sections/Introduction/Resources/}}

\glsresetall

Can we fully describe gene expression using only information about concentration of proteins and/or molecules like RNA inside the cell nucleus? Accordingly to Buxbaum et al. \cite{Buxbaum_2014}, the location of \gls{mrna} within the cell plays an important role in protein synthesis. In \cite{Korolchuk2011}, Korolchuk et al. show that cellular response to nutrient levels is a mechanism that needs to contemplate the position of Lysosomes (dynamic intracellular organelles) in order to be fully understood. However, the need for localization information to explain cellular mechanisms is not only limited to a subcellular level, but also at a subnuclear level. For instance, in \cite{van2019role} van Steensel et al. argue that the spatial organization of subnuclear components can have an important role in the regulation of gene expression. In \cite{vogel2010sequence}, Vogel et al. shows that in human cells the concentration of \gls{mrna} can only explain protein abundance to a certain extent, which could indicate the need to consider spatial information to predict protein expression.

In recent years, the implementation of new imaging technologies has made it possible to access subnuclear spatial information. In \cite{Guteaar7042}, Gut et al. introduce the \gls{4i} protocol, which is a process that allows to efficiently capture thousands of single cell multichannel images from a cell culture without degrading it. The \gls{4i} protocol is part of the \gls{mpm} protocol also introduced in \cite{Guteaar7042}, which allows the segmentation of the tissue images into single cell nucleus images (Multiplexed single cell analysis).
The \gls{mpm} protocol also introduces two other features that are not used in this work, but are still worth mentioning.
The first one is the \gls{mcu} analysis, which segments the cell nucleus image into regions.
These regions can be then used to identify subnuclear bodies or protein complexes. The segmentation is done through two unsupervised clustering algorithms\footnote{To identify clusters in an unsupervised manner, \hl{Self Organizing Maps} algorithm and \hl{Phenograph} analysis are used over a very large number of pixels sampled from all the single cells images.}, applied over the measured pixel intensities. The \gls{mcu} analysis is shown on figure \ref{fig:mcu}.

\begin{figure}[htb]
  \centering
  \begin{subfigure}[t]{.3\linewidth}
    \includegraphics[width=\linewidth]{mcu_1.png}
    \caption{Extraction of pixel intensities.}
    \label{fig:mcu:1}
  \end{subfigure}
  \hspace{4mm}
  \begin{subfigure}[t]{.3\linewidth}
    \includegraphics[width=\linewidth]{mcu_2.png}
    \caption{Pixel clustering by Self Organizing Maps and Phenograph.}
    \label{fig:mcu:2}
  \end{subfigure}
  \hspace{4mm}
  \begin{subfigure}[t]{.3\linewidth}
    \includegraphics[width=\linewidth]{mcu_3.png}
    \caption{Cell subdivision base on the \gls{mcu}.}
    \label{fig:mcu:3}
  \end{subfigure}
  \caption{Figure \subref{fig:mcu:1} shows the pixel intensity extraction for a single cell. The pixel intensity is a vector containing the readout of that 2D location for each protein, one specific protein readout per entrance. Figure \subref{fig:mcu:2} shows the clusters found by Self Organizing Maps algorithm and Phenograph analysis over the pixel intensities. Figure \subref{fig:mcu:3} shows a cell masked with the clusters found by the \gls{mcu} analysis. Images source \cite{Guteaar7042}.}
  \label{fig:mcu}
\end{figure}

The second feature of the \gls{mpm} protocol that is not discussed here, but could be used in future work, is the application of pharmacological and metabolic perturbations to some sections of the cell culture. The analysis shown in \cite{Guteaar7042} revealed expected and unexpected changes in the concentration and distribution of molecules inside the cell.

% why NN
\gls{ann} are very robust tools widely used in the field of \gls{ml} that can potentially approximate any function \cite{cybenko1989approximation}, \cite{hornik1989multilayer}, \cite{funahashi1989approximate}.
In the field of biology, \glspl{ann} have proven capable of solving very complex and high-impact problems.
One of the best examples in recent years is the three-dimensional prediction of the structure of a protein using amino acid sequences encoded in the genes \cite{AlphaFold}, which is a very important problem since the structure of a protein largely determines its function. In \cite{chen2016gene}, Chen et al. introduced a deep \gls{ann} model known as \hl{D-GEX}, which outperformed previous linear model approaches when trained using gene expression profiling data.

However, in \cite{krizhevsky2012imagenet} Krizhevsky et al. show that \glspl{cnn} are powerful tools in the recognition of spatial patterns, achieving outstanding results in ImageNet LSVRC-2010 contest.
This makes \glspl{cnn} suitable models to analyze spatial information embedded in images of single cell nucleus, like the ones provided by the \gls{mpm} protocol.

% what is the problem with NN
However, in many fields of study and industries, the interpretation of the models is essential. For example, in the medical field, \gls{cnn} architectures have achieved remarkable results in the segmentation of brain tumors \cite{saleem2021visual}. However, to successfully implement deep learning models in the diagnosis of patients, it is not enough only to know what the model predicts, but also how it does it.

% How to solve it? Interpretability methods
Many researchers have proposed different techniques to explain what happens inside black-box models like \glspl{cnn}.
The difference between these methods is basically whether they are applicable to any type of model (model-agnostic/model-independent) or only to a specific group (model-specific).
An example of a model-independent method is the \gls{lime}, which basically aims to approximate the underlying model $f$ (not interpretable) by means of an interpretable model $g$ (e.g. a linear model) for a specific region of the input \cite{ribeiro2016model}. As the name suggest, \gls{lime} provides a local and individual explanation of each input.
However, there are other methods that provide a general (global) explanation of the model. An example of a global method (and also model-specific) would be the visualization of the learned filters/kernels of a \gls{cnn}, which can indicate the features in the data that are important for the model prediction \cite{zeiler2014visualizing}.

% Gradient based methods
However, in this work we use \hl{attribution methods}, which are aimed to rank each input feature based on how much they contribute to the output of the model.
These methods create an importance (or score) map for each element of the input data. There are several ways to compute these score maps \cite{JMLR:v11:baehrens10a}, \cite{ShrikumarGSK16}. However, most of these methods base the importance assignment of each input feature on the gradient of the model output with respect to the input (gradient-based methods) \cite{SimonyanVZ13}, \cite{BinderMBMS16} and \cite{Springenberg}.

% why IG
Nevertheless, just using the gradient as a feature importance designation method is not enough. As a model learns the relationship between an input and its output, the gradient of the model's output with respect to the input features will approximate to 0 (saturation).
To alleviate this issue, Sundararajan et al. \cite{sundararajan2017axiomatic} proposed \gls{ig}, which accumulates the gradient of the output with respect to the input when it goes from an uninformative value to the original input.

% why VarGrad
However, in practice attribution methods like \gls{ig} often produce noisy and diffuse score maps, and in some cases they are not even better than a random designation of feature importance \cite{hooker2018benchmark}.
For this reason Smilkov et al. \cite{Smilkov_smoothgrad} proposed an ensemble interpretability method known as \gls{sg}, which in practice reduces noise in score maps and can be easily combined with other attribution methods such as  \gls{ig}.
In this work we use a slightly different version proposed by Adebayo et al. \cite{adebayo2018local} known as \gls{vg}, which is inspired by \gls{sg} and has been shown to empirically outperform such a random assignment of importance \cite{hooker2018benchmark}.


%% =============================================================================
%% Basics chapter
%% =============================================================================
\chapter{Basics}
\label{ch:basics}
%% Magic command to compile root document
% !TEX root = ../../thesis.tex

%% Reset glossary to show long gls names
\glsresetall

We can interpret \gls{tr} as the amount of new RNA molecules inside a cell nucleus in a given period of time. By means of a fluorescent marker, it is possible to identify these new RNA molecules and thus approximate \gls{tr}. But, what about the morphology of other molecules and organelles within the cell nucleus? The distribution, shape and location of molecules, proteins and organelles within the nucleus could potentially encode relevant information for cellular expression. This has been the main motivation for this work. By means of a \gls{cnn}, we seek to predict \gls{tr} base mainly in spacial information encoded on images of cell nucleus.

In this section we introduce the process used to generate the data for this work, the \gls{mpm} protocol. In addition to this, we introduce the preprocessing and data augmentation techniques used. These techniques aim to improve the model's training performance, prevent overfitting and remove non-relevant information from the images. With this, we seek to encourage the model to base its prediction mainly on the spatial information encoded in the images of cell nucleus.


\section{Biology Background}
\label{sec:basics:bio_back}
\glsresetall
% define where the images are
\graphicspath{{./Sections/Basics/Resources/}}

Cells are considered the smallest unit of life. There are two types of cells, \textit{prokaryotic} and \textit{eukaryotic}. The main difference between these, is that prokaryotic cells do not contain nucleus and that that prokaryotes are considered single-celled organisms, while eukaryotes organisms can be either single-celled or multicellular. For multicellular organisms, like plants or mammals, eukaryotic cells are the \textit{building-blocks} of life. This work focuses on a process of eukaryotic cells. Therefore, in the subsequent when we refer to cells, we will be referring to eukaryotic cells only.

Multicellular organisms (like us) have different cell types, where each one of them can have many or an specific function. For instance, red blood cells are responsible for carrying the oxygen in the body. In order to carry as much oxygen as possible they lack a nucleus, and therefore they are unable to undergo \textit{mitosis}\footnote{Mitosis is the process through which eukaryotic cells reproduce themselves and give rise to new organisms.}.

However, there are also cells aimed to produce (\textit{synthesize}) certain substances that regulate process in our body. For instance, \textit{Alpha cells} are pancreatic cells responsible for synthesizing the \textit{glucagon} hormone, which elevates the glucose levels in the blood \cite{1e48f81ce88f4602a25a4ebbcea3a6e7}. The process in which cells produce this substances is called \textit{cellular expression} or \textit{gene expression}. The reason why this process is also called gene expression, is because the instructions to synthesize every substance (or any functional product, like hormones or proteins) are encoded in a specific gene\footnote{A \textit{gene} is defined as a region of the \textit{DNA} that encodes a function. DNA is contained in \textit{chromosomes}, which are long DNA strands containing many genes.}.

There are two key steps involved in gene expression, \textit{transcription} and \textit{translation}. Roughly speaking, transcription is the process in which the instructions to synthesize a product (like proteins) are copied from a gene in the DNA, to a single strand molecule called \gls{mrna}. On the other hand, Translation is the process in which the instructions in the \gls{mrna} are interpreted to produce a functional product. Figure \ref{fig:BB:tt} shows a simple representation of this process.

\begin{figure}[htb]
  \centering
  \includegraphics[width=\linewidth]{central-dogma-large.png}
  \caption{Simple representation of the gene expression process \cite{transcript_translation_diagram}.}
  \label{fig:BB:tt}
\end{figure}

The transcription process happens inside the cell nucleus, while translation happens in the \textit{ribosome} (outside the nucleus). The reason why transcription is necessary, is because the instructions needed to build a product are encoded in the DNA, which is inside the nucleus. Since DNA is too big to pass the membrane that covers the nucleus (nuclear envelop) to travel to the ribosome (which is the organelle in charge of building the product), the necessary instructions in the DNA are copied into a smaller strand (\gls{mrna}), which is now able to escape the nucleus and travels to the ribosome to start the translation process. Figure \ref{fig:BB:euka} shows a diagram of an eukaryotic cell and some of its parts. This work focuses on the transcription process and the factors that seed up or slow down this process.

\begin{figure}[htb]
  \centering
  \includegraphics[width=0.7\linewidth]{Animal_cell_structure_en.png}
  \caption{Animal eukaryotic cell diagram \cite{eukacell}.}
  \label{fig:BB:euka}
\end{figure}

\subsection{Transcription Process}
\label{sec:basics:transcription_process}
\input{Sections/Basics/Transcription_process}

\section{Machine Learning}
%% Magic command to compile root document
% !TEX root = ../../thesis.tex

%% Reset glossary to show long gls names
\glsresetall

%% Set path to look for the images
\graphicspath{{./Sections/Basics/Resources/}}

\glspl{ann} are universal approximators widely used in the field of \gls{ml} and an important part of this work. This is a very broad subject and there are entire books that cover this in detail, like  \cite{Goodfellow-et-al-2016} or \cite{bishop2006pattern}. However, in this section we will give a small introduction to \glspl{ann} and \glspl{cnn}, which are a type of \gls{ann} that were specifically designed to deal with data in the form of images.

Before defining what exactly is a \gls{ann}, lest first recall the definition of machine learning. We refer as \gls{ml} to the group of algorithms that automatically improve (learn) through experience. Among this algorithms, we could say that there are three main classes (which depend on the kind of experience we provide):

\begin{itemize}
  \item \textbf{Supervised Learning}: The experience is given in the form of input and output examples, and the goal is to learn a general rule that maps inputs to outputs.
  \item \textbf{Unsupervised Learning}: The experience is given in the form of data (no outputs provided) and the goal is to discover hidden patterns in data.
  \item \textbf{Reinforcement Learning}: No experience (data) is given, instead a dynamic \hl{environment} is provided and an \hl{agent} must learn how to interact with it in order to achieve a goal.
\end{itemize}

An \gls{ann} can be used in any of the 3 kinds of learning algorithms listed above.

However, recall that the objective of this work is to approximate a function (in this case a \gls{cnn}), such that when it is fed with images of a cell nucleus (input data), it predicts the corresponding \gls{tr} (output data) Therefore, we are dealing with a \hl{supervised learning} task.

Before explaining what a \gls{cnn} is, let us first introduce and explain \gls{ann} in general.

\label{sec:basics:ML}
\subsection{Artificial Neural Networks}
%% Magic command to compile root document
% !TEX root = ../../thesis.tex

%% Reset glossary to show long gls names
\glsresetall

%% Set path to look for the images
\graphicspath{{./Sections/Basics/Resources/}}

% what is a ANN
Before explaining what a \gls{cnn} is, let us first introduce and explain \gls{ann} in general.

Roughly speaking, an \gls{ann} is simply a mathematical function $f:\mathbb{R}^D \leftarrow \mathbb{R}^L$, that maps in input vector $x\in\mathbb{R}^D$ with an output vector $y\in\mathbb{R}^L$. However, to be considered an \gls{ann}, $f$ needs to have a specific form

\begin{equation}
  f()
  \ref{eq:basics:slp}
\end{equation}

where $w\in\mathbb{R}^M$ and $\phi$ is a fixed non linear function.
The vector $w$ are often known as \hl{weights}, while the function $\phi$ as \hl{base function}. Thus an \gls{ann} is simply a nonlinear function that maps a set of input variables $x_i$ to a set of output variables $y_k$ controlled by a vector $w$ of adjustable parameters \cite{bishop2006pattern}.

In practice, the function $f$ shown in equation \ref{} is known as \hl{layer} (or single layer perceptron) and a deep neural network is a function $F$ composed of several layers $f$

\begin{equation}
  F()
  \ref{eq:basics:ann}
\end{equation}
where $h$ must a non linear function known as \hl{activation function}.

Note that in equation \ref{}, the base function $\phi$ is a composition of the activation function h1 and a linear function xw evaluated in the input vector x.

Graphically, a neural network can be observed in figure \ref{fig:basics:ann:ann}. The layers (without considering the output layer) shown in figure \ref{fig:basics:ann:ann} are known as \hl{hidden layers}, while the circles on this hidden layers are known as \hl{hidden units}.

\begin{figure}[!ht]
  \centering
  \includegraphics[width=0.8\linewidth]{Model_training_process.png}
  \caption{Graphical representation of an \gls{ann}}
  \label{fig:basics:ann:ann}
\end{figure}

% explain back prop
However, once we have defined a \gls{ann}, how can we approximate its parameters (weights and biases)?

%%%
% Tal vez para esta parte define antes el input as X y el output as Y
%%%
Recall that we are dealing with a superviced learning problem. That means, that we have the data with which the \gls{ann} has to be fed (input data i.e., images of cell nucleus), as well as the values that the network should return (output data, i.e. the \gls{tr}). This means that we can fed the \gls{ann} with the available data, and then measure its performance by comparain its output $\hat{y}$ against the true values $y$. For this comparation we must define a \hl{loss function}, which should return high values when the output of the \gls{ann} is far from the true value, and low values when the $\hat{y}$ and $y$ are close. Then, can minimize the value of the loss funtion, just by adjusting the values of $w$ and $b$ (en alguna parte di, por simplicidad los w's y b's se engloban en W).

%here mention why the negative gradient of the output (or loss) with respect parameter

% here introduce the update rule

% here say, now we know we need gradients, so how do ge get them? Back prop!
The answer to this is through an iterative evaluation an  process known as \hl{back propagation}, which is performed during the \hl{training process}.

% here put an algorithm for the update process (maybe)

% whay ann works?
In this work, every time we refer to an \gls{ann} we mean a \hl{fully connected feedforward neural network}. In some literature, this is also known as \gls{mlp}.

The properties of \glspl{ann} have been studied extensively before (\cite{cybenko1989approximation}, \cite{hornik1989multilayer}, \cite{funahashi1989approximate}) and established in the \hl{Universal approximation theorem}

\begin{theorem}[Universal approximation theorem]
  An \gls{mlp} with a linear output layer and one hidden layer can approximate any continuous function defined over a closed and bounded subset of $\mathbb{R}^D$, under mild assumptions on the activation function (\hl{squashing} activation function) and given the number of hidden units is large enough.
\end{theorem}

For this reason \gls{ann} are known as \hl{universal approximators}, since they are able to approximate any continuous function on a compact\footnote{A set $A$ in a metric space is said to be \hl{compact} if it is close (i.e., it contain all its limit points) and bounded (i.e., all its points lie within some fixed distance of each other) \cite{bartle2000introduction}.} input domain with an arbitrary accuracy \cite{bishop2006pattern}. These means that, as long as a \gls{ann} has a sufficiently large number of hidden units, the nn can be reduce the loss function as much as we want. However, nice property can also lead to a not desirable property know as \hl{overfitting}.

In practice, \hl{overfitting} means that the ann is able predict the data used to train the model with low error (low bias), but it has a high error when it is used on unseen data (high variance). This means that the ann \hl{memorized} the train data, and therefore it is unable to \hl{generalize} correctly.

On the other hand, \hl{underfitting} is the when the model always returns a high error, no matter what data is used to feed it (high bias and low variance).

Figure \ref{} shows an example non-linear regression problem with over (\subfig{}) and under fitting (\subfig{}).

\begin{figure}[!ht]
  \centering
  \includegraphics[width=0.8\linewidth]{}
  \caption{}
  \label{fig:basics:ann:over_under_fitting}
\end{figure}

In practice, we seek to approximate models that has low bias and low variance (i.e., good accuracy and good generalization).

% here say how to prevent overfitting (i.e. how to train models and chose hyper params)

As we saw in equation \ref{eq:basics:ann}, a \gls{ann} has sever hyperparameters that need to be specified. This hyperparameters are the \hl{number of layers} (also known as hidden layers), the \hl{number of units} (also known as \hl{neurons}) per layers and \hl{activation function} of each layer. In practice, these hyperparameters are chosen empirically by means of a validation set.

\begin{figure}[!ht]
  \centering
  \includegraphics[width=0.8\linewidth]{Model_training_process.png}
  \caption{Model building process.}
  \label{fig:basics:model_train_process}
\end{figure}

\label{sec:basics:ANN}

\subsection{Convolutional Neural Networks}
\label{sec:basics:CNN}
%% Magic command to compile root document
% !TEX root = ../../thesis.tex

%% Reset glossary to show long gls names
\glsresetall

%% Set path to look for the images
\graphicspath{{./Sections/Basics/Resources/}}

So far we have explained how \glspl{ann} works assuming that we feed them with vectors of fixed length. Even though we could take a multichannel image and transform it into a vector, in practice this would be computationally very expensive. For instance, assuming that we have a 3 channel image of size 224 by 224, this would result into an input vector of length $3 \cdot 224 \cdot 224=150'528$. Then, if the first layer of our network has 100 units, this would mean more than 15 millions of parameters only for the first layer. Furthermore, the transformation of our image into a vector would mean a loss of spatial information. This means that the \gls{ann} would not be able to capture or use the spatial relationship between pixels and shapes within the image.

A \gls{cnn} is a type of \gls{ann} widely used to analyze data in the form of images. The intuition behind a \gls{cnn} is that instead of just looking at an image and trying to predict the target value directly, first learn some \hl{features} within the image, and then make the predict base on this features.
To achieve this, \glspl{cnn} mainly use \hl{convolution} and \hl{pooling} layers.

\subsubsection{Convolution layer}

The only difference a

A convolution layer is very similar to a regular layer described in section \ref{sec:basics:ANN}. Basically, they only differ in the way the layer input is multiplied by the the layer weights.
Recall that in a regular layer, the input of a unit is the dot product between the layer input and its corresponding weight vector (i.e., $z=\bs{w}^T\bs{x}$).
This means that for each element in the input vector $\bs{x}$, there is a corresponding element in the weight vector $\bs{w}$. However, for a convolution layer this is not the case.
Convolution layers are based on the shared-weight architecture of the convolution \hl{kernels} or \hl{filters} that slide along the input and returns a translation known as \hl{feature maps} \cite{zhang1988shift}. This means that the \hl{kernels} weights will be used for multiple elements of the layer input. Figure \ref{fig:basics:conv_layer} shows the convolution process with a 2 by 2 kernel over a RGB image (3 channels) of size 4 by 4. Each entrance of the returned feature map $z_i$ is the dot product between the kernel weights $\bs{w}$ and the $\bs{x}_i-th$ chunk of the image.

\begin{figure}[!ht]
  \centering
  \includegraphics[width=\linewidth]{Diagrams/Conv_layer.png}
  \caption{Convolution process steps. In red, green and blue the input image, in orange the convolution kernel (size 2 by 2 and stride of 1) and in gray the convolution output (feature map).}
  \label{fig:basics:conv_layer}
\end{figure}

Mathematically this looks as follow

\begin{equation}
  z_i = \bs{w}^T \bs{x}_i + b
\end{equation}

where $\bs{w}\in\mathbb{R}^{2 \times 2 \times 3}$, $\bs{x_i}\in\mathbb{R}^{2 \times 2 \times 3}$ and $b\in\mathbb{R}$ is the bias (not shown in the images).

Like the kernel size, the number of pixels we shift the kernel each time along side the input (\hl{Stride}) is also a hyperparameter of convolution layers. IN figure \ref{fig:basics:conv_layer}, the stride size is 1.

Figure \ref{fig:basics:conv_layer} also shows that size (width and height) of the returned feature map is smaller than the input image. If we want to keep the input and output size the same (\hl{Same convolution}), then we must add zeros at the edges of the input features (zero-padding). This is shown on figure \ref{fig:basics:conv_layer_pad}.

\begin{figure}[!ht]
  \centering
  \includegraphics[width=0.7\linewidth]{Diagrams/Conv_layer_pad.png}
  \caption{Convolution with padding. In blue a single-channel input features, in orange the convolution kernel (size 3 by 3 and stride of 1) and in gray the convolution output (feature map).}
  \label{fig:basics:conv_layer_pad}
\end{figure}

So far we have seen that a convolution projects a multi-channel input feature (image) into a single-channel feature map. Therefore, if we want our output feature map to have $n$ channels, then our convolution must have $n$ different kernels.

Normally, a non-linear activation function is applied to the output of convolution layers (and normally also after batch normalization) to enable the \gls{cnn} to learn non-linear relations.

\subsubsection{Pooling layer}

Unlike convolution layers, the goal of Pooling layers is to reduce the feature image (height and width, but not depth) rather than learn features.
However, Pooling layers work in a similar way to convolution layers in the way that they also slide a kernel along the input. However, in this case the kernel works independently on each feature map (that is, each channel) and has no weights to learn.
This means that the pooling layers maintain the same number of input and output channels.
There are several ways to do this downsampling, but the most common are Max Polling and Average Polling. As the name suggests, Average pooling shrinks the feature image by averaging sections of it, while Max pooling takes the maximum value. Figure \ref{fig:basics:pooling} shows an example of a max and average pooling layer on a single-channel feature image using a 2 by 2 kernel and a stride of 2.

\begin{figure}[!ht]
  \centering
  \includegraphics[width=0.7\linewidth]{Diagrams/Pooling.png}
  \caption{Max and average pooling with a 2 by 2 kernel and stride 2. The color denotes the kernel position.}
  \label{fig:basics:pooling}
\end{figure}

Normally, Pooling layers are applied over the output of the activation functions.

\subsubsection{Global Average Pooling layer}

As we mentioned at the beginning of this section, the idea of a \gls{cnn} is to first learn the features within the input images and then make a prediction based on these features. To do this, the \hl{Global Average Pooling layer} transforms the channels of the last feature map into a vector (by averaging each of its channels), so that this can be used as input in a regular \gls{ann} to make the final prediction. Figure \ref{fig:basics:global_avg_pool} shows an example of this, when it is applied into a feature map with 7 channels.

\begin{figure}[!ht]
  \centering
  \includegraphics[width=0.7\linewidth]{Diagrams/Global_avg_pooling.png}
  \caption{Global Average Pooling layer.}
  \label{fig:basics:global_avg_pool}
\end{figure}

\subsubsection{Inception module}

Recall that a convolution layer is meant to learn features from a 3D object with 2 spatial dimensions (width and height) and a channel dimension. This means that each kernel in the convolution needs to learn simultaneously cross-channel and spatial correlations.
The intuition behind the \hl{Inception module} is to improve this process by separating this two tasks, so that the cross-channel correlations and the spatial correlations can be learned separately and independently \cite{chollet2017xception}.

A normal inception model looks at the cross-channel correlations first through a set of 3 or 4 \hl{pointwise convolutions}\footnote{A \hl{pointwise convolution} is a convolution with 1 by 1 kernels and stride 1.}, and then learns the spacial information in the downsampled feature image (in depth, not height and width), by means of regular convolution (usually with 3 by 3 or 5 by 5 kernels). Figure \ref{fig:basics:inception_module} shows a diagram of an Inveption V3 module.

\begin{figure}[!ht]
  \centering
  \includegraphics[width=0.6\linewidth]{Inception_module.png}
  \caption{A regular Inception module (Inveption V3). Image source \cite{chollet2017xception}.}
  \label{fig:basics:inception_module}
\end{figure}

François Chollet \cite{chollet2017xception}, used the inception module as reference to propose the \hl{depthwise separable convolution}, which is something between a normal convolution and a normal convolution combined/followed by a pointwise convolution.
Figure \ref{fig:basics:extreme_inception_module} shows an \hl{extreme} version of the inception module shown in figure \ref{fig:basics:inception_module}. The \hl{depthwise separable convolution} is very similar to the one shown in figure \ref{fig:basics:extreme_inception_module}, the only difference is that the pointwise convolution is applied before the 3 by 3 convolutions instead of after.

\begin{figure}[!ht]
  \centering
  \includegraphics[width=0.6\linewidth]{Extreme_Inception_module.png}
  \caption{An extreme version of our Inception module. Image source \cite{chollet2017xception}.}
  \label{fig:basics:extreme_inception_module}
\end{figure}

Even though the \hl{depthwise separable convolution} is a simplified version of the inception module, the idea and motivation behind it is the same. The \hl{depthwise separable convolution}, and the residual block, are the main components of the \hl{Xception} architecture \cite{chollet2017xception}.


\section{Interpretability Methods}
\label{sec:basics:interpretability_methods}
\glsresetall
% Motivation and problem
In recent years, \glspl{dnn} have been used to solve a wide variety of problems and gained popularity. Amazing results such as those achieved by Deep Mind's Alpha Fold team, have shown the great potential \gls{dnn} has to solve complex problems. However, the difficulty to interpret \glspl{dnn} has become one of the main obstacles to their acceptance in applications where the interpretability of the model is necessary.

% solution
To understand how the \glspl{dnn} predict the \gls{tr} of a cell, we use \textit{Attribution Methods}. This methods are meant to measure how much each component of the input image contributes to the model's prediction by creating a \textit{Score Map} (also known as \textit{Importance Map, Sensitivity Map} or \textit{Saliency Map}) of the same shape as the model's input. In particular, in this work we use a combination between \gls{ig} \cite{sundararajan2017axiomatic} and \gls{vg} \cite{adebayo2020sanity} as attribution method. In general we will denote attribution method as $\phi$.

% other advantages
Attribution methods are not only used to interpret black-box models like \gls{dnn}, the can also be used to debug models or as a sanity check to validate that the model base its prediction on the relevant features of the input.

% in our case
In our case, this interpretability techniques will show us which parts of the cell image are relevant for the prediction of the \gls{tr}. However, this will not just help us to interpret the results of the model, this also have the potential to help us understand unknown cellular processes.


\subsection{Integrated Gradients}
\label{sec:basics:IG}

% what it IG
%\glsresetall
\glsfirst{ig} is an interpretability technique (attribution method) proposed by Sundararajan et al. \cite{sundararajan2017axiomatic}, aimed to assign an importance to the input features (in our case pixels from a cell image) with respect to the model prediction. The attribution problem have been studied before in other papers \cite{JMLR:v11:baehrens10a}, \cite{SimonyanVZ13}, \cite{ShrikumarGSK16}, \cite{BinderMBMS16} and \cite{Springenberg}.

In our case, we seek to predict \gls{tr} given a cell image $x \in \mathbb{R}^{d \times d \times c}$, where $d$ is the height and width of the image and $c$ is the number of channels. Therefore, our \gls{dnn} would be a function $f:\mathbb{R}^{d \times d \times c} \rightarrow \mathbb{R}$ and an attribution method should be a function $\phi:\mathbb{R}^{d \times d \times c} \rightarrow \mathbb{R}^{d \times d \times c}$ having an input and output of the same shape as the model's input image.

Early interpretability methods only use gradients to assign importance to each input feature

\begin{equation}
  \begin{split}
    \phi(f,x) &:= \nabla f(x) \\
    &= \frac{\partial f}{\partial x}
  \end{split}
\end{equation}

Mathematically speaking, $\phi_i(f,x)$ assign an importance score to the pixel $i$ (out of the $d \times d \times c$ there are), representing how much it adds or subtract from the model output. However, this score maps have some drawback when they are used to interpret deep neural networks \cite{sturmfels2020visualizing}. Recall that the gradient with respect to the input indicate us the pixels that have the steepest local slope with respect to the model's output. This means that it only describes local changes in the input, and not the whole prediction model. Another mayor problem is saturation\footnote{In the context of artificial neural networks, a neuron is said to be saturated when the predominant output value of a neuron is close to the asymptotic ends of the bounded activation function. This behavior can potentially damage the learning capacity of a neural network.}.
As the model learns the relationship between an input image and its \gls{tr}, the gradient of the most important pixels will approximate to 0, i.e. the pixel's gradient saturates.

To overcome this problems, Sundararajan et al. proposed \gls{ig} as an attribution method, where the importance of the input feature $i$ is defined as follow
\begin{equation}
  \phi^{IG}_i(f, x, x') := (x_{i} - x'_{i})\int_{\alpha=0}^1\frac{\partial f(x'+\alpha (x - x'))}{\partial x_i}{d\alpha}
  \label{eq:ig:definition}
\end{equation}

Intuitively speaking, \gls{ig} accumulates the input gradient when it goes from a baseline $x'$, which should represents \textit{absence} of information, to the actual input image $x$. With this, we avoid losing information about relevant pixels for the model's prediction in the importance map, even if they saturate eventually.

For a better understanding, we can divide the \gls{ig} definition as follow
\begin{equation}
  \phi^{IG}_i(f, x, x') := \overbrace{(x_{i} - x'_{i})}^\text{Difference from baseline}
  \underbrace{\int_{\alpha=0}^1}_\text{From baseline to input...}
  \overbrace{\frac{\partial f(x'+\alpha (x - x'))}{\partial x_i}{d\alpha}}^\text{…accumulate local gradients}
  \label{eq:ig:explanation}
\end{equation}

The integral in equation \ref{eq:ig:explanation} accumulate the gradients for the interpolated images $x'+\alpha (x - x'))$ between the baseline $x'$ and the image $x$. On the other hand, the difference $(x_i - x_i')$ outside the integral comes from the chain rule and the fact that we are interested in integrating over the path between the baseline and the image.

%https://arxiv.org/pdf/1806.03000.pdf
\gls{ig} is very simple and easy to implement, since it does not require any modification to the model and it only require some calls to the gradient operator.

The \gls{ig} satisfy several properties and axioms that are addressed in detail in the paper. However, there is one axiom satisfied by \gls{ig} that is of special importance for us, \textit{completeness}. Completeness means that the value of the summed attributes will be equal to difference between the model's output when it is evaluated at the image and the model's output when it is evaluated at the baseline
\begin{equation}
  \sum_i \phi(f, x, x')^{IG} = f(x) - f(x')
  \label{eq:ig_completeness}
\end{equation}

In practice, computing the analytic expression for the integral in equation \ref{eq:ig:definition} would be complicated, and in some cases unfeasible.
However, luckily we can numerically approximate $\phi(f, x, x')^{IG}$ using a Riemann sum
\begin{equation}
  \phi^{Approx\ IG}_i(f, x, x', m) := (x_{i} - x'_{i})\sum_{k=1}^m\frac{\partial f(x'+\frac{k}{m} (x - x'))}{\partial x_i} \frac{1}{m}
  \label{eq:ig:approx}
\end{equation}

\noindent where $m$ is number of steps for the Riemann sum approximation.

This is when the completeness axiom comes into scene, which is a good value for the parameter $m$? 10, 100, 500? To answer this question, we can simply apply the completeness axiom as a sanity check for the election of $m$. If $m$ is good enough, then the value of $\sum_i \phi^{Approx\ IG}_i(f, x, x', m)$ should be close to $f(x)-f(x')$, or equivalently, the value of $|\sum_i \phi^{Approx\ IG}_i(f, x, x', m) - (f(x)-f(x'))|$ should be close to 0.

Figures \ref{fig:vg:img_gradients} and \ref{fig:vg:img_IG} show a comparative between the gradient of a model output with respect to a cell image, and the \gls{ig}. One can see that either for score maps computed using \gls{ig} or vanilla gradients, the output is noisy and diffuse.


\subsection{VarGrad}
\label{sec:basics:VarGrad}
% define where the images are
\graphicspath{{./Sections/Basics/Resources/}}
\glsresetall

As we can see in figure \ref{fig:vg:img_IG}, \gls{ig} attribution maps can be noisy and diffuse. To improve their empirical quality, Smilkov et al. \cite{Smilkov_smoothgrad} proposed \gls{sg}, which tends to reduce noise in practice and can be combined with other attribution map algorithms (like \gls{ig}). The idea behind \gls{sg} is pretty simple, given an input image $x$, you create a sample of similar images by adding noise, then compute the attribution map for each one of them using the algorithm you prefer (in our case \gls{ig}), and take the average of the attribution maps.
Although Smilkov et al. do not provide a mathematical proof of why \gls{sg} reduce noise in score maps, they provide a conjecture and empirical evidence.
For this work we use a slightly difference version called \gls{vg}, proposed by Adebayo et al. \cite{adebayo2018local} but inspired by \gls{sg}, which takes the variance of the attribution maps instead of the mean. The reason for this choice is that Seo et al. \cite{Seo_noise} analyzed theoretically \gls{vg}, and concluded that it is independent to the gradient and capture higher order partial derivatives.

In general, \gls{vg} is defined as follow

\begin{equation}
  \phi^{SG}(f, x) := Var(\phi(f, x + z_j))
\end{equation}

\noindent where $x \in \mathbb{R}^{d \times d \times c}$ is the input image, $f:\mathbb{R}^{d \times d \times c} \rightarrow \mathbb{R}$ a model, $\phi$ an attribution method to get preliminary score maps and $z_j \sim \mathcal{N}(0, \sigma^2)$, with $j\in\{1, \dots, n\}$, are i.i.d. noise images of same shape as the input image.

Since we use \gls{ig} to get preliminary score maps, in our case \gls{vg} (in the subsequent defined as \gls{vgig}) looks as follow

\begin{equation}
  \phi^{SG}(f, x) := Var(\phi^{IG}(f, x + z_j, x'))
\end{equation}

\noindent where $x' \in \mathbb{R}^{d \times d \times c}$ is a given baseline needed to compute the \gls{ig} score maps.

Figures \ref{fig:vg:img_IG} and \ref{fig:vg:img_VG_IG} show a comparative between \gls{ig} and \gls{vgig} score maps. One can see that \gls{vgig} produces less noisy score maps than vanilla \gls{ig}.

% this plots were created using the notebook ~/Documents/Master_Thesis/Project/workspace/Interpretability/Integrated_Gradient_Sanity_check.ipynb
\begin{figure}[!ht]
  \centering
  \begin{subfigure}[b]{.45\linewidth}
    \includegraphics[width=\linewidth]{Cell_Image.jpg}
    \caption{Original cell image.}
    \label{fig:vg:cell_img}
  \end{subfigure}
  \begin{subfigure}[b]{.45\linewidth}
    \includegraphics[width=\linewidth]{Image_Gradient.jpg}
    \caption{Gradient wrt the input image.}
    \label{fig:vg:img_gradients}
  \end{subfigure}%
  \vspace{3mm}
  \begin{subfigure}[b]{.45\linewidth}
    \includegraphics[width=\linewidth]{Integrated_Gradient.jpg}
    \caption{Integrated Gradient.}
    \label{fig:vg:img_IG}
  \end{subfigure}
  \begin{subfigure}[b]{.45\linewidth}
    \includegraphics[width=\linewidth]{VarGrad_Integrated_Gradient.jpg}
    \caption{VarGrad with Integrated Gradients.}
    \label{fig:vg:img_VG_IG}
  \end{subfigure}
  \caption{Comparative between a cell image and the different attribution methods. All the figures show the same 3 channels taken from a cell image. \subref{fig:vg:cell_img}) cell image, i.e. no attribution method. \subref{fig:vg:img_gradients}) score map using only the gradient of the model with respect to the input image. \subref{fig:vg:img_IG}) \acrlong{ig} score map. \subref{fig:vg:img_VG_IG}) \acrlong{vgig} score map.}
  \label{fig:vg:comparative}
\end{figure}


%\section{Interpretability Methods Evaluation}
%\label{sec:basics:vgig_eval}

%% =============================================================================
%% Dataset chapter
%% =============================================================================
\chapter{The Dataset}
\label{ch:dataset}
%% Magic command to compile root document
% !TEX root = ../../thesis.tex

%% Reset glossary to show long gls names
\glsresetall

We can interpret \gls{tr} as the amount of new RNA molecules inside a cell nucleus in a given period of time. By means of a fluorescent marker, it is possible to identify these new RNA molecules and thus approximate \gls{tr}. But, what about the morphology of other molecules and organelles within the cell nucleus? The distribution, shape and location of molecules, proteins and organelles within the nucleus could potentially encode relevant information for cellular expression. This has been the main motivation for this work. By means of a \gls{cnn}, we seek to predict \gls{tr} base mainly in spacial information encoded on images of cell nucleus.

In this section we introduce the process used to generate the data for this work, the \gls{mpm} protocol. In addition to this, we introduce the preprocessing and data augmentation techniques used. These techniques aim to improve the model's training performance, prevent overfitting and remove non-relevant information from the images. With this, we seek to encourage the model to base its prediction mainly on the spatial information encoded in the images of cell nucleus.


\section{Multiplexed Protein Maps}
\label{sec:dataset:multiplexed_protein_maps}
%% Magic command to compile root document
% !TEX root = ../../thesis.tex

%% Reset glossary to show long gls names
\glsresetall

%% Set path to look for the images
\graphicspath{{./Sections/Dataset/Resources/}}

% A small motivation to create Multiplexed Protein Maps
The amount of protein or \gls{mrna} inside a cell may not be enough to fully describe cellular function. Accordingly to Buxbaum et al. \cite{Buxbaum_2014} and Korolchuk et al. \cite{Korolchuk2011}, cellular function can heavily depends on the specific intracellular location and interaction with other molecules and intracellular structures. Therefore, cellular expression is determined by the functional state, abundance, morphology, and turnover of its intracellular organelles and cytoskeletal structures. This means that having the ability to look at the concentration and distribution of different molecules within a cell, is an important technological achievement that can significantly leverage scientific discoveries in biomedicine.
This is exactly what \gls{mpm} allows us to do (\cite{Guteaar7042}). \gls{mpm} are protein readouts from cell cultures, that simultaneously captures different properties of the cell, like its shape, cycle state, detailed morphology of organelles, nuclear subcompartments, etc. It also captures highly multiplexed subcellular protein maps, which can be used to identify functionally relevant single-cell states, like \gls{tr}. These maps can also identify new cellular states and allow quantitative comparisons of intracellular organization between single cells in different cell cycle states, microenvironments, and drug treatments \cite{Guteaar7042}.

So, let us explain more in deept what are these \gls{mpm}. Accordingly to Gabriele Gut et al. \cite{Guteaar7042}, \gls{mpm} is a nondegrading protocol that allows to capture efficiently thousands of single cell multichannel images, where each channel contains captures the distribution and concentration of a protein of interest inside each cell. To achieve this, the protocol is made up of different steps that will be briefly explained here.

% 4i explanation
\subsubsection{Iterative indirect immunofluorescence imaging}
The \gls{mpm} protocol starts with a process called \gls{4i} developed by the same group. The \gls{4i} is a complete protocol by itself, and it allows to capture the concentration and distribution of individual proteins in thousands of different cells in a tissue\footnote{The tissues are made from cell cultures, which were made using the HeLa Kyoto cell line. \hl{HeLa} is the oldest and most commonly used immortal human cell line used in scientific research. The story behind it quite interesting, so it's worth checking out.}.
Before applying the \gls{4i} protocol, the \hl{plate} where the cell culture is must to be divided into squared sections called \hl{wells}. Then, the \gls{4i} protocol is applied over each well and photographed in sections called \hl{sites}.

Roughly speaking, \gls{4i} works as follow
\begin{enumerate}
  % 1
  \item The selected well is prepared for the staining-elution process.
  %2
  \item A specific protein inside the cells is photographed by saturating a well with a liquid containing \hl{antibodies}\footnote{An antibody is a Y-shaped protein that can recognize and bind to a unique molecule (its antigen, e.g. a specific protein).} stained with a fluorescent ink (\gls{if}), which binds to the targeted protein.
  %3
  \item The well is exposed to a high-energy light and photographed using a light microscopy.
  %4
  \item The antibodies inside the tissue are washed-out using a chemical elution substrate.
  %5
  \item Steps 2 and 4 are repeated 20 times to get 20 images of the same protein.
  %6
  \item The 20 images are projected into a single one by \hl{maximum intensity projection}, to improve the protein readouts.
\end{enumerate}

Figure \ref{fig:4i:1} illustrates the steps of the \gls{4i} protocol to captures the saturation and distribution of a specific protein. Keep in mind that even though the \gls{4i} protocol captures sever images of the tissue, it returns an uni-channel image (step 6). Figure \ref{fig:4i:2} shows the \gls{4i} protocol applied 40 times with different \gls{if} and over a 384-well plate, to capture the concentration and distribution of 40 different targeted proteins.

\begin{figure}[htb]
  \centering
  \begin{subfigure}[t]{.3\linewidth}
    \includegraphics[width=\linewidth]{4i_1.png}
    \caption{\Acrfull{4i} protocol.}
    \label{fig:4i:1}
  \end{subfigure}
  \hspace{4mm}
  \begin{subfigure}[t]{.45\linewidth}
    \includegraphics[width=\linewidth]{4i_2.png}
    \caption{\gls{4i} protocol applied over a specific well of plate and for 40 different \gls{if}.}
    \label{fig:4i:2}
  \end{subfigure}%
  \caption{Schematic representation of the \gls{4i} protocol for a single well and for 40 different fluorescent antibodies. Figure \subref{fig:4i:2} also shows the image analysis to identify single cells and its components (nucleus and cytoplasm). Images source: \cite{Guteaar7042}.}
  \label{fig:4i}
\end{figure}

By the time \cite{Guteaar7042} was published, the \gls{4i} protocol was able to capture cell culture images with up to 40 channels without degrading the tissue, which is why \gls{mpm} is called a \textit{nondegrading} protocol.

\subsubsection{Multiplexed single cell analysis}

Once the multichannel images were generated using the \gls{4i} protocol, a series of image preprocessing and image analysis methods (\cite{Carpenter2006} and \cite{snijder2012single}) are applied to generate segmentation masks to identify individual cells, as well as their cytoplasm and nucleus. Figure \ref{fig:4i:2} shows this segmentation at a cellular level, while figure \ref{fig:4i:segmentation} shows it also at a subcellular level. In both cases the boundaries are marked with white lines. This single cell analysis is also used to identify cells that do not satisfy certain quality controls (like cells in the border of the image or in mitosis stage). However, this will be addressed in detail on section \ref{sec:dataset:data_pp}.

\begin{figure}[htb]
  \centering
  \includegraphics[width=0.5\linewidth]{4i_segmentation.png}
  \caption{Visualization of the subcellular segmentation of a \gls{4i} protocol for 18 \gls{if} stains. The image was created by combining the readouts of 3 of this \gls{if} stains: PCNA (cyan), FBL (magenta) and TFRC (yellow). The number next to each staining label indicates their corresponding 4i acquisition cycle (\gls{4i} protocol step 5). The orange rectangle and the tile at its right shows a section of the nucleus and cytoplasm of a single cell. The other 3 tiles shows the \gls{4i} readout of each of the 3 proteins.}
  \label{fig:4i:segmentation}
\end{figure}

\subsubsection{Multiplexed single-pixel analysis framework}
Even though the cell cultures are now segmented into individual cells and nucleus, there is still one missing part that must be considered, and that is that cells are 3-dimensional objects. Recall that the \gls{4i} protocol saturates the cell culture with a liquid containing fluorescent antibodies. This means that the antibody can either bind to its corresponding protein inside or outside the cell nucleus. Therefore, even though that we segmented a cell into nucleus and cytoplasm, a readout assigned to the nucleus could come from a protein in the cytoplasm under or above the nucleus, and not from inside it. Fortunately, readouts from proteins inside the nucleus are much higher than those in the cytoplasm. Therefore, by means of a two steps clustering approach\footnote{To identify clusters in an unsupervised manner, \hl{Self Organizing Maps} algorithm and \hl{Phenograph} analysis were used over a very large number of pixels sampled from a large number of single cells \cite{Guteaar7042}.} pixels can be classified accordingly to their intensity profile (figures \ref{fig:mcu:1} and \ref{fig:mcu:2}), so the source of their readout can be identified. This pixel type classification is called \Acrfull{mcu} and is illustrated in figure \ref{fig:mcu:3}. After pixels clusters (intensity profiles) where identified, the pixels whose measurement comes from the cytoplasm and not from the nucleus are removed.

\begin{figure}[htb]
  \centering
  \begin{subfigure}[t]{.3\linewidth}
    \includegraphics[width=\linewidth]{mcu_1.png}
    \caption{Extraction of pixel intensities.}
    \label{fig:mcu:1}
  \end{subfigure}
  \hspace{4mm}
  \begin{subfigure}[t]{.3\linewidth}
    \includegraphics[width=\linewidth]{mcu_2.png}
    \caption{Pixel clustering by Self Organizing Maps and Phenograph.}
    \label{fig:mcu:2}
  \end{subfigure}
  \hspace{4mm}
  \begin{subfigure}[t]{.3\linewidth}
    \includegraphics[width=\linewidth]{mcu_3.png}
    \caption{Cell subdivision base on the \gls{mcu}.}
    \label{fig:mcu:3}
  \end{subfigure}
  \caption{Figure \subref{fig:mcu:1} shows the pixel intensity extraction for a single cell. The pixel intensity is a vector containing the readout of that 2D location for each protein, one specific protein readout per entrance. Figure \subref{fig:mcu:2} shows the clusters found by Self Organizing Maps algorithm and Phenograph analysis over the pixel intensities. Figure \subref{fig:mcu:3} shows a cell masked with the clusters found by the \gls{mcu} analysis. Images source: \cite{Guteaar7042}.}
  \label{fig:mcu}
\end{figure}

Finally, the nucleus of each cell is stored separately and identified with a unique id.\fxnote{After you finish writing the dataset section review if this sentence is accurate.}

\subsubsection{Cell cycle phase classification: $G_1,\ S,\ G_2$ and $M$ phase}

The \gls{mpm} protocol is not only capable to capture the concentration and distribution of molecules inside thousands of cells. It can also identify the phase each cell is in, which is tightly related with the abundances and distribution of molecules inside a cell \cite{Guteaar7042}.

Roughly speaking, cell cycle phase was determined by means of a \gls{svm} classifier and k-means clustering. First, a \gls{svm} classifier is trained to identify $M$ phase cells based on the nuclear information in one of the image channels (\hl{DAPI}\footnote{A brief description of this marker can be bound on section \ref{sec:appendix:if_markers}.}). Then, based on the nuclear information of channel \hl{PCNA}, a second \gls{svm} classifier is trained to identify cells in phase $S$. Finally, cells in phase $G_1$ and $G_2$ are classified using a k-means algorithm, using the pixel intensity profiles of the DAPI channels excluding the cells in $S$ and $M$ phase. A more detailed explanation of the cell cycle classification process can be found on the dataset paper \cite{Guteaar7042}.

\subsubsection{Pharmacological and metabolic perturbations}

To further explore the capabilities of the \gls{mpm} protocol, the creators of the dataset (Gabriele Gut et al. \cite{Guteaar7042}) applied the \gls{mpm} protocol to a cell populations that were to nine pharmacological and metabolic perturbations. The analysis reveled expected and unexpected changes in the concentration and distribution of molecules inside the cell. However, this work focused on cells without pharmacological and metabolic perturbations. This means that only cells marked as \hl{normal} (no perturbed cells) and \hl{DMSO}\footnote{Dimethyl sulfoxide, or DMSO, is an organic compound used to dissolve test compounds in in drug discovery and design \cite{cushnie2020bioprospecting}.} (control cells) were used.

\subsubsection{Dataset description}

The provided dataset contains the following features

\begin{itemize}
  \item How many single cells were given.
  \item Mention how many normal and DMSO.
  \item Number of cells per perturbation.
  \item Number of channels and name of the channels (only the dataset channel id and channel name. The description and implementation names are on the appendix.)
  \item Information given in the metadata (cell identifier: $mapobject\_id\_cell$, cell cycle, well, site, mitotic or not, etc.)
\end{itemize}

\fxnote{Add this info when you write the preprocessing section.}


\section{Data preprocessing}
\label{sec:dataset:data_pp}

Some technical details about \gls{4i}:
\begin{itemize}
  \item The images where obtained using a 40x objective and a \gls{scmos} camera, where the surface of each pixel is 165 nm by 165 nm.
  \item Each photo captures around 20,000 single cells.
  \item In \cite{Guteaar7042} (Fig 2), it is mentioned that use of 384-well plates, where wells were imaged at 40× magnification in a 7 × 6 tiled fashion for 21 4i cycles.
  \item The used cultured cells (tissue) where HeLa. HeLa is an immortal cell line used in scientific research.
\end{itemize}

The information for each single cell is not stored as a separated image as one may think, instead the information is first divided by Wells. Then, the information of the cells in a given well is stored in 7 files \fxnote{this is may not be included in the final work}:
\begin{itemize}
  \item \texttt{metadata.csv}: contains information of each cell in the well
  \item \texttt{channels.csv}: contains information of each protein (channel) photographed
  \item \texttt{labels.npy}: 1 dim array containing the cell label of each pixel
  \item \texttt{mpp.npy}: 2 dim array, where size of first dim is the same as number of measured pixels and size of second dim is the same as the number of channels. This file contains the observed measured values (intensities) of each pixel and for each protein. The values in \texttt{mpp.npy} vary from 0 to 65535 i.e. $2^{16}$ i.e. 2 bytes or 16 bits.
  \item \texttt{x.npy} and \texttt{y.npy}: 1 dim array containing the x/y coordinates of the measured pixels (pixels where a signal was detected by the \gls{scmos} camera. Accordingly with \cite{Guteaar7042}, the size of a single cell image (for each channel) is 2560x2160. Therefore, the values in \texttt{x.npy} vary between 1 and 2560 and form 1 to 2160 for \texttt{x.npy}
  \item \texttt{mapobject\_ids.npy}: 1 dim array where its size is the same as the number of rows in the \texttt{metadata.csv} file. Therefore, \texttt{mapobject\_ids.npy} maps the information given in the metadata file with each pixel in the well.
\end{itemize}

\textbf{QUESTION TO HANNAH:} In the paper it is mentioned that for each channel (protein), the tissue is stained with with 2 fluorescent antibodies (in that case TUBA1A and CTNNB1, NOT at the same time, TUBa1A-elution-CTNNB1), why? to create a background as reference for the protein that is being photographed?
Answer: To increase the measured signal, \ie to make the protein Shine more.

Since we are using a \gls{cnn} to predict the \gls{tr}, therefore we need the data as multichannel images. However, as we already explained in \ref{sec:Motivation_and_Background:Dataset}, the information is providad in a pixel by pixel format. Using the library \texttt{mpp\_data.py} written by Dr. Hannah Spitzer \fxnote{See how to refer properly to this library}, the raw data is transformed to multichannel images (one channel per protein measured). During this process, the bordered cells and cells in division process (Mitosis state) are excluded. The provided data set also contain the target variable, \ie the \gls{tr} (amount of \gls{mrna} molecules produced by the cell nucleus in the last 30 minutes). However, this information is coded also as a protein channel (00\_EU). Therefore, after converting the data set into multichannel images, the channel corresponding to the protein 00\_EU is extracted and converted into a number \fxnote{It is necessary to improve the description of how this 00\_EU channel is turned into a number. So far it is done by averaging all the entrances of the channel, however it still need to be decided if this will be done like this.}.


\section{Data augmentation}
\label{sec:dataset:data_augmentation}
%% Magic command to compile root document
% !TEX root = ../../thesis.tex

\glsresetall
% define where the images are
\graphicspath{{./Sections/Dataset/Resources/}}


\section{Discussion}
\label{sec:dataset:discussion}
%% Magic command to compile root document
% !TEX root = ../../thesis.tex
% reset glossary to show full acrs
\glsresetall
% define where the images are
\graphicspath{{./Sections/Dataset/Resources/}}

Besides the preprocessing techniques introduced in section \ref{sec:dataset:data_pp} (clipping and standardization), the following approaches were also tried

\begin{itemize}
  \item Linear scaling using the $98\%$ percentile with and without clipping.
  \item Mean extraction and linear scaling using the $98\%$ percentile with clipping (like standardization, but with the $98\%$ percentile instead of the standard deviation).
  \item $49\%$ percentile extraction and linear scaling using the $98\%$ percentile (no clipping).
\end{itemize}

\noindent This approaches were tried at a per-channel level. However, clipping plus standardization where the prprocessing techniques that showed the best performance. Since we seek the model to predict \gls{tr} base on spacial information, rather than pixel intensity/color, good performance means low \gls{mae} for the \gls{cnn} models, but high \gls{mae} for the linear model (since the linear model is unable to use the spatial information). This indicates that the spatial information encoded in the images of the data set has more influence on the prediction of the model than the information encoded in the colors.

Another aspect of the dataset that is worth to mention, is that more than half cells are in phase $G_1$ (see table \ref{table:data_pp:dataset_dist_cc}), while cells in $S$ phase are less than $30\%$ and around $15\%$ for $G_2$ cells. This causes the model to focus more on correctly predicting the \gls{tr} of $G_1$ cells, than for cells in the other two phases. This happens because $G_1$ cells have more influence on the minimization of the objective function, since it is more likely that the model is fed with $G_1$ cells during training.

As it is shown on figure \ref{fig:dataset:discus:tr_dist}, \gls{tr} of $G_1$ cells is significantly lower than \gls{tr} of $S$ and $G_2$ cells. This, and that cells in different phases are not in the same proportion in the dataset, can cause the model to make a biased prediction when it is fed with a $S$ or $G_2$ cell. Two possible solutions to this problem are, either to add more cells in phases $S$ and $G_2$ to the dataset, or to sample with replacement over the available cells, so the proportion of cells in the three different phases is the same in the dataset. Another possible solution would be to do a weighted loss function based on the cell phase proportions, such that every phase has the same influence on it during training.\fxnote{Discuss with Hannah if this is the correct place to put this.}

\begin{figure}[htb]
  \centering
  \includegraphics[width=0.7\linewidth]{TR_dist.jpg}
  \caption{\gls{tr} distribution separated by cell phase.}
  \label{fig:dataset:discus:tr_dist}
\end{figure}



Talk here about the other augmentation techniques tried.

% TODO: https://www.baseclick.eu/product/5-ethynyl-uridine/ inicates: (EU) is a modified nucleoside that can be used for labeling of nascent RNA inside cells. ....experiments indicate that it will be first integrated in de novo RNA, but we have also observed incorporation into DNA after some incubation time.
% This means that the RNA readouts may be contaminated with DNA if the duration time (30 mins) is to long. Write in the future work that this should be repeated with the dataset where the duration time was 10 mins.


%% =============================================================================
%% Methodology chapter
%% =============================================================================
\chapter{Methodology}
\label{ch:methodology}
%% Magic command to compile root document
% !TEX root = ../../thesis.tex

%% Reset glossary to show long gls names
\glsresetall

We can interpret \gls{tr} as the amount of new RNA molecules inside a cell nucleus in a given period of time. By means of a fluorescent marker, it is possible to identify these new RNA molecules and thus approximate \gls{tr}. But, what about the morphology of other molecules and organelles within the cell nucleus? The distribution, shape and location of molecules, proteins and organelles within the nucleus could potentially encode relevant information for cellular expression. This has been the main motivation for this work. By means of a \gls{cnn}, we seek to predict \gls{tr} base mainly in spacial information encoded on images of cell nucleus.

In this section we introduce the process used to generate the data for this work, the \gls{mpm} protocol. In addition to this, we introduce the preprocessing and data augmentation techniques used. These techniques aim to improve the model's training performance, prevent overfitting and remove non-relevant information from the images. With this, we seek to encourage the model to base its prediction mainly on the spatial information encoded in the images of cell nucleus.


\section{Dataset Setup}
\label{sec:methodology:tfds}
%% Magic command to compile root document
% !TEX root = ../../thesis.tex

%% Reset glossary to show long gls names
\glsresetall
\graphicspath{{./Sections/Methodology/Resources/}}

In this section we specify all the hyperparameters needed to execute the process explained on chapter \ref{ch:dataset}. This contemplates the raw data processing, the quality control, the \gls{tfds} creation, the image preprocessing, as well as data augmentation.

\subsection{Data preprocessing}

As we explained in section \ref{sec:dataset:data_pp}, the data preprocessing consist of 4 main steps; 1) the raw data processing, 2) the quality control, 3) the creation of the dataset and 4) the image preprocessing.

A complementary explanation of the data preprocessing parameters, as well as implementation references, can be found in the appendices \ref{sec:appendix:raw_data} and \ref{sec:appendix:tfds}.

\subsubsection{Raw data processing}

As we explained in section \ref{sec:dataset:data_pp:dataset_creation}, to build the \gls{tfds} it is necessary to specify the perturbations that will be included in the dataset. For this reason, all the available wells were processed and transformed into images. This included wells exposed to pharmacological and metabolic perturbations, control wells and unperturbed wells. This allows the user to easily create new datasets without having to run the raw data processing first. Table \ref{table:methodology:dataset:raw_data} shows the processed wells separated by perturbation.

\begin{table}[!ht]
  \centering
  \begin{tabular}{>{\centering\arraybackslash}m{0.35\linewidth} | >{\centering\arraybackslash}m{0.2\linewidth} | >{\centering\arraybackslash}m{0.3\linewidth}}
    \hline
    Perturbation type & Perturbation name & Well names \\
    \hline
    \multirow{5}{*}{pharmacological/metabolic} & CX5461 & I18, J22, J09 \\
    \cline{2-3}
     & AZD4573 & I13, J21, J14, I17, J18 \\
    \cline{2-3}
     & meayamycin & I12, I20 \\
    \cline{2-3}
    & triptolide & I10, J15 \\
    \cline{2-3}
    & TSA & J20, I16, J13 \\
    \hline
    control & DMSO & J16, I14 \\
    \hline
    unperturbed & normal & J10, I09, I11, J18, J12 \\
    \hline
  \end{tabular}
  \caption{Well names divided by perturbation name and type.}
  \label{table:methodology:dataset:raw_data}
\end{table}

Another hyperparameter that needs to be specified during the raw data processing, is the size of the output images $I_s$. This size applies to both, the width and height of the image (square images). Since some prebuilt architectures use a standard image size of 224 by 224, we define $I_s$ as 224.

\subsubsection{Quality control}

As it is mentioned in section \ref{sec:dataset:data_pp:qc}, the quality control is meant to exclude cells with undesirable features. In our case we discriminate mitotic and border cells. The information used by the quality control is contained in the metadata of each well. Table \ref{table:methodology:dataset:qc} shows the metadata columns and the discriminated values. If a cell has any of these values, then it is excluded.

\begin{table}[!ht]
  \centering
  \begin{tabular}{>{\centering\arraybackslash}m{0.3\linewidth} | >{\centering\arraybackslash}m{0.3\linewidth} | >{\centering\arraybackslash}m{0.2\linewidth}}
    \hline
    Feature & Metadata column name & Discriminated value \\
    \hline
    \multirow{3}{*}{Cell in mitosis phace} & \texttt{is\_polynuclei\_HeLa} & 1 \\
    \cline{2-3}
     & \texttt{is\_polynuclei\_184A1} & 1 \\
    \cline{2-3}
     &  \texttt{cell\_cycle} & \texttt{NaN} \\
    \hline
    Border cell & \texttt{is\_border\_cell} & 1 \\
    \hline
  \end{tabular}
  \caption{Discrimination characteristics for quality control.}
  \label{table:methodology:dataset:qc}
\end{table}

\subsubsection{Dataset creation and image preprocessing}

As it is explained in section \ref{sec:dataset:data_pp:dataset_creation}, in this work we decided to use a custom \gls{tfds}. Table \ref{table:methodology:dataset:tfds} shows the parameters used to build the dataset employed in this work, together with the image preprocessing parameters.

% set table lengths
\setlength{\mylinewidth}{\linewidth-7pt}%
\setlength{\mylengtha}{0.25\mylinewidth-2\arraycolsep}%
\setlength{\mylengthb}{0.65\mylinewidth-2\arraycolsep}%

\begin{longtable}{>{\centering\arraybackslash}m{\mylengtha} | >{\centering\arraybackslash}m{\mylengthb}}
    \hline
    Parameter & Description \\
    \hline
    Perturbations to be included in the dataset & \hl{normal} and \hl{DMSO} \\
    \hline
    Cell phases to be included in the dataset & $G_1$, $S$, $G_2$ \\
    \hline
    Training set split fraction & 0.8 \\
    \hline
    Validation set split fraction & 0.1 \\
    \hline
    Seed & 123 (for reproducibility of the train, val and test split) \\
    \hline
    Percentile & 98 (for clipping / linear scaling / standardization) \\
    \hline
    Clipping flag & 1 \\
    \hline
    Mean extraction flag & 0  \\
    \hline
    Linear scaling flag & 0 \\
    \hline
    Standardization (z-score) flag & 1 \\
    \hline
    Model input channels & All of them except for channel \texttt{00\_EU} (see table \ref{table:tfds_in:channels})  \\
    \hline
    Channel used to compute target variable (output channel) & \texttt{00\_EU} (channel id 35, see table \ref{table:tfds_in:channels}) \\
    \hline
  \caption{Parameters used to biuld \gls{tfds} and image preprocessing.}
  \label{table:methodology:dataset:tfds}
\end{longtable}

The custom \gls{tfds} created with the parameters specified in table \ref{table:methodology:dataset:tfds} is called \\
\texttt{mpp\_ds\_normal\_dmso\_z\_score}.
The Python script that builds the custom \gls{tfds}, also returns a file with the image preprocessing parameters (\texttt{channels\_df.cvs}) (as this is applied at a per-channel level) and information about the channels (channel name, id, etc.). It also returns another file with the metadata of each cell included in the \gls{tfds} (\texttt{metadata\_df.csv}). These files are stored in the same directory as the \gls{tfds} files.

In table \ref{table:methodology:dataset:tfds} we also specify the channel used to compute the target variable (ground truth), which is the channel corresponding to the marker \hl{EU} (channel id 35, see tables \ref{table:tfds_in:channels} and \ref{table:apendix:if_markers}).
Recall that this channel contains nuclear readouts of nascent RNA (\gls{pmrna}) in a given period of time.
For the data provided, this time period was the same for all the cells (30 minutes) and is specified in the \hl{duration} columns of the metadata.
Since channel 35 is used to compute the target variable (ground truth), it is removed from the prediction/input channels.

\subsection{Data augmentation}
\label{sec:methodology:data:augm}

In this section we specify the data augmentation techniques (see section \ref{sec:dataset:data_augmentation}) and its hyperparameters used to train all the models of this work. Recall that the techniques are either aimed to remove non-relevant characteristics of the data (color shifting, central zoom in/out) or to improve model generalization (horizontal flipping, 90 degree rotations). Table \ref{table:methodology:dataset:augm} shows this techniques and its hyperparameters grouped by objective and technique. In practice, the augmentation techniques are applied as shown in table \ref{table:methodology:dataset:augm} from top to bottom.

% set table lengths
\setlength{\mylinewidth}{\linewidth-7pt}%
\setlength{\mylengtha}{0.2\mylinewidth-2\arraycolsep}%
\setlength{\mylengthb}{0.2\mylinewidth-2\arraycolsep}%
\setlength{\mylengthc}{0.25\mylinewidth-2\arraycolsep}%
\setlength{\mylengthd}{0.22\mylinewidth-2\arraycolsep}%

\begin{longtable}{>{\centering\arraybackslash}m{\mylengtha} | >{\centering\arraybackslash}m{\mylengthb} | >{\centering\arraybackslash}m{\mylengthc} | >{\centering\arraybackslash}m{\mylengthd}}
    \hline
    Objective & Technique & Hyperparameter & Description \\
    \hline
    \multirow{2}{\mylengtha}{\centering Remove non-relevant features} & random color shifting & distribution & $U(-3,3)$ \\
    \cline{2-4}
     & random central zoom in/out & distribution\footnotemark & $N(\mu=0.6, \sigma=0.1)$ \\
    \hline
     \multirow{2}{\mylengtha}{\centering Improve generalization} & random horizontal flipping & NA & NA \\
    \cline{2-4}
     & random 90 degrees rotations & NA & NA \\
    \hline
  \caption{Parameters used for data augmentation techniques. The NA means that there are no hyperparameters for this technique or that there is no further description.}
  \label{table:methodology:dataset:augm}
\end{longtable}

\footnotetext{This distribution is used to sample the \hl{cell nucleus size ratio} $S_{ratio}$ (see section \ref{sec:dataset:data_aug:zoom}) of each cell. However, the parameters for this distribution (mean and standard deviation) were not provided by us. Instead, they were estimated using the information in column \texttt{cell\_size\_ratio} of the \gls{tfds} metadata file. Therefore, the \texttt{return\_cell\_size\_ratio} flag must be set to 1 (True) during raw data processing, so this column is created (see section \ref{sec:dataset:data_pp:raw_data_p} and appendix \ref{sec:appendix:raw_data}).}

Even thought we specify the data augmentation hyperparameters here, in practice these are selected for each model and applied during training. However, all the models showed in this work were trained using the techniques and values shown in table \ref{table:methodology:dataset:augm}. A complementary explanation can be found in appendix \ref{sec:appendix:Model_training_IN}.

In section \ref{sec:dataset:data_augmentation} we mentioned that data augmentation techniques can be applied to both the training set and the validation set. However, we also mentioned that only horizontal flips and 90 degree rotations are applied for the validation set. Furthermore, for the training set these techniques are applied randomly, while for the validation set they are applied deterministically. Therefore, table \ref{table:methodology:dataset:augm} only applies to the training set.


\section{Models}
\label{sec:methodology:models}
%% Magic command to compile root document
% !TEX root = ../../thesis.tex

%% Reset glossary to show long gls names
\glsresetall
\graphicspath{{./Sections/Methodology/Resources/}}

In this section talk about the models in general. For example, here you should mention that a channel filter layer is added at the beginning of the model to remove unwanted layers, included the cell mask.

We are using the Adpative .... adam optimizer, mention the used parameters.

we are using the hubber loss function.


\subsection{Linear Model}
\label{sec:methodology:lm}
%% Magic command to compile root document
% !TEX root = ../../thesis.tex

\glsresetall
% define where the images are
\graphicspath{{./Sections/Results/Resources/}}

poner aqui los mismos plots que en la precentacion de scott pero para el validation!!! esto no debe de tomarte mas de 1 hora!!!

\begin{figure}[!ht]
  \centering
  \begin{subfigure}[b]{.9\linewidth}
    \includegraphics[width=\linewidth]{train_comp_c_and_s.jpg}
    \caption{test.}
    \label{fig:results:lm_performance:test}
  \end{subfigure}%
  \vspace{3mm}
  \begin{subfigure}[b]{.9\linewidth}
    \includegraphics[width=\linewidth]{train_comp_structure.jpg}
    \caption{test}
    \label{fig:results:lm_performance:test2}
  \end{subfigure}
  \caption{test.}
  \label{fig:results:lm_performance}
\end{figure}


\newpage
\subsection{Baseline CNN}
\label{sec:methodology:BL_CNN}
%% Magic command to compile root document
% !TEX root = ../../thesis.tex

%% Reset glossary to show long gls names
%\glsresetall

\graphicspath{{./Sections/Methodology/Resources/}}

% experiment:
%BL_RIV2_test4.json

The \hl{Baseline} \gls{cnn} architecture is specified in table \ref{table:metho:models:baseline}. The rows of the table represent each layer of the model, which are evaluated from top to bottom. This model has $160,129$ free (learnable) parameters in total.

% set table lengths
\setlength{\mylinewidth}{\linewidth-7pt}%
\setlength{\mylengtha}{0.35\mylinewidth-2\arraycolsep}%
\setlength{\mylengthb}{0.25\mylinewidth-2\arraycolsep}%
\setlength{\mylengthc}{0.18\mylinewidth-2\arraycolsep}%

\begin{longtable}{m{\mylengtha} | m{\mylengthb} | m{\mylengthc}}
    \hline
    Layer & Output Shape & Number of parameters \\
    \hline
    Input & $(bs, 224, 224, 38)$ & 0 \\
    \hline
    Channel filtering & $(bs, 224, 224, 33)$ & 0 \\
    \hline
    Convolution & $(bs, 224, 224, 64)$ & 19072 \\
    \hline
    Batch Normalization & $(bs, 224, 224, 64)$ & 256 \\
    \hline
    ReLU & $(bs, 224, 224, 64)$ & 0 \\
    \hline
    Max Pooling & $(bs, 112, 112, 64)$ & 0 \\
    \hline
    Convolution & $(bs, 112, 112, 128)$ & 73856 \\
    \hline
    Batch Normalization & $(bs, 112, 112, 128)$ & 512 \\
    \hline
    ReLU & $(bs, 112, 112, 128)$ & 0 \\
    \hline
    Max Pooling & $(bs, 56, 56, 128)$ & 0 \\
    \hline
    Global Average Pooling & $(bs, 128)$ & 0 \\
    \hline
    Dense & $(bs, 256)$ & 33024 \\
    \hline
    Batch Normalization & $(bs, 256)$ & 1024 \\
    \hline
    ReLU & $(bs, 256)$ & 0 \\
    \hline
    Dense & $(bs, 128)$ & 32896 \\
    \hline
    Batch Normalization & $(bs, 128)$ & 512 \\
    \hline
    ReLU & $(bs, 128)$ & 0 \\
    \hline
    Dense & $(bs, 1)$ & 129 \\
    \hline
  \caption{Baseline \gls{cnn} architecture. The rows represent each layer of the model. The flow of the model is from top to bottom. The $bs$ on the \hl{Output Shape} column stands for \hl{Batch size}.}
  \label{table:metho:models:baseline}
\end{longtable}

All convolution layers specified in table \ref{table:metho:models:baseline} used kernels of size 3 by 3 and stride of 1. Besides that, all pooling layers used a kernel of size 2 by 2 and stride of 2.

Last but not least, the learning rate used to train the baseline model was $0.0005$.


\subsection{ResNet50V2}
\label{sec:methodology:RN50V2}
%% Magic command to compile root document
% !TEX root = ../../thesis.tex

%% Reset glossary to show long gls names
%\glsresetall

\graphicspath{{./Sections/Methodology/Resources/}}

% experiment:
% RN_RIV2_test7.json

Besides the \hl{Simple} \gls{cnn}, we also tried the \hl{ResNet50V2} \gls{cnn}, which is a more complex (deeper) architecture. The ResNet50V2 consist basically of several residual blocks (see section \ref{sec:basics:ANN}), composed with convolution and pooling layers (see section \ref{sec:basics:CNN}), stacked one after another.
There is a lot of literature on the ResNet50V2 architecture (\cite{he2015deep}, \cite{he2016identity}), so we will not dive into details here. However, the model architecture is shown in table \ref{table:metho:models:RN50V2}.
The raw \hl{ResNet50V2 feature extraction}, represent the feature extraction layers (i.e., all the layers containing convolution and/or pooling layers) of the ResNet50V2\footnote{For this work, we did not implement the ResNet50V2 architecture from scratch, instead we used the pre-built model that is provided in the Keras library. For more information pleases refer to the \href{https://www.tensorflow.org/api_docs/python/tf/keras/applications/ResNet50V2}{official documentation}.}, while the remaining rows represent the layers intended to make the final prediction. The layers are evaluated from top to bottom. This model has $24,171,777$ free (learnable) parameters in total and it was trained with a learning rate of $0.0005$.

% set table lengths
\setlength{\mylinewidth}{\linewidth-7pt}%
\setlength{\mylengtha}{0.4\mylinewidth-2\arraycolsep}%
\setlength{\mylengthb}{0.25\mylinewidth-2\arraycolsep}%
\setlength{\mylengthc}{0.18\mylinewidth-2\arraycolsep}%

\begin{longtable}{m{\mylengtha} | m{\mylengthb} | m{\mylengthc}}
    \hline
    Layer & Output Shape & Number of parameters \\
    \hline
    Input & $(bs, 224, 224, 38)$ & 0 \\
    \hline
    Channel filtering & $(bs, 224, 224, 33)$ & 0 \\
    \hline
    ResNet50V2 feature extraction & $(bs, 7, 7, 2048)$ & 23,612,672 \\
    \hline
    Global Average Pooling & $(bs, 2048)$ & 0 \\
    \hline
    Dense & $(bs, 256)$ & 524544 \\
    \hline
    Batch Normalization & $(bs, 256)$ & 1024 \\
    \hline
    ReLU & $(bs, 256)$ & 0 \\
    \hline
    Dense & $(bs, 128)$ & 32896 \\
    \hline
    Batch Normalization & $(bs, 128)$ & 512 \\
    \hline
    ReLU & $(bs, 128)$ & 0 \\
    \hline
    Dense & $(bs, 1)$ & 129 \\
    \hline
  \caption{ResNet50V2 \gls{cnn} architecture. The rows represent each layer of the model. The flow of the model is from top to bottom. The $bs$ on the \hl{Output Shape} column stands for \hl{Batch size}.}
  \label{table:metho:models:RN50V2}
\end{longtable}


\subsection{Xception}
\label{sec:methodology:XC}
%% Magic command to compile root document
% !TEX root = ../../thesis.tex

%% Reset glossary to show long gls names
%\glsresetall

\graphicspath{{./Sections/Methodology/Resources/}}

% experiment:
% You need to train the model!!

As we saw in section \ref{sec:basics:ANN}, each kernel in a regular convolution layer needs to simultaneously learn spatial and cross-channel correlations. For this reason, we also tested an architecture capable of separating these two tasks, the \hl{Xception} \cite{chollet2017xception}.

The Xception architecture combines the idea behind the Inception module and the residual blocks (see section \ref{sec:basics:ANN}).
We will not dive into details about the Xception here. However, the model architecture is shown in table \ref{table:metho:models:XC}.
The raw \hl{Xception feature extraction}, represent the feature extraction layers (i.e., all the layers containing convolution and/or pooling layers) of the Xception\footnote{For this work, we did not implement the Xception architecture from scratch, instead we used the pre-built model that is provided in the Keras library. For more information pleases refer to the \href{https://www.tensorflow.org/api_docs/python/tf/keras/applications/xception}{official documentation}.}, while the remaining rows represent the layers intended to make the final prediction. The layers are evaluated from top to bottom. This model has $21,373,929$ free (learnable) parameters in total.

% set table lengths
\setlength{\mylinewidth}{\linewidth-7pt}%
\setlength{\mylengtha}{0.35\mylinewidth-2\arraycolsep}%
\setlength{\mylengthb}{0.25\mylinewidth-2\arraycolsep}%
\setlength{\mylengthc}{0.18\mylinewidth-2\arraycolsep}%

\begin{longtable}{m{\mylengtha} | m{\mylengthb} | m{\mylengthc}}
    \hline
    Layer & Output Shape & Number of parameters \\
    \hline
    Input & $(bs, 224, 224, 38)$ & 0 \\
    \hline
    Channel filtering & $(bs, 224, 224, 33)$ & 0 \\
    \hline
    Xception feature extraction & $(bs, 7, 7, 2048)$ & 20,814,824 \\
    \hline
    Global Average Pooling & $(bs, 2048)$ & 0 \\
    \hline
    Dense & $(bs, 256)$ & 524544 \\
    \hline
    Batch Normalization & $(bs, 256)$ & 1024 \\
    \hline
    ReLU & $(bs, 256)$ & 0 \\
    \hline
    Dense & $(bs, 128)$ & 32896 \\
    \hline
    Batch Normalization & $(bs, 128)$ & 512 \\
    \hline
    ReLU & $(bs, 128)$ & 0 \\
    \hline
    Dense & $(bs, 1)$ & 129 \\
    \hline
  \caption{Xception \gls{cnn} architecture. The rows represent each layer of the model. The flow of the model is from top to bottom. The $bs$ on the \hl{Output Shape} column stands for \hl{Batch size}.}
  \label{table:metho:models:XC}
\end{longtable}

The learning rate used to train the baseline model was $0.001$.


\subsection{Model Metrics}
\label{sec:methodology:metrics}
%% Magic command to compile root document
% !TEX root = ../../thesis.tex

%% Reset glossary to show long gls names
\glsresetall
\graphicspath{{./Sections/Methodology/Resources/}}

To evaluate and compare the performance of the models, besides the loss function (\hl{Huber loss}), we also used 2 other error measures

\begin{itemize}
  \item The \acrfull{mse}
    \begin{equation}
      E_{MSE}(Y,\hat{Y}) := \frac{1}{N} \sum_{n=1}^N (y_i-\hat{y}_i)^2
    \end{equation}
  \item The \acrfull{mae}
    \begin{equation}
      E_{MAE}(Y,\hat{Y}) := \frac{1}{N} \sum_{n=1}^N |y_i-\hat{y}_i|^2
    \end{equation}
\end{itemize}

\noindent where $Y, \hat{Y} \in \mathbb{R}^N$ are the true and predicted \gls{tr} values respectively.

Additionally, we also used the \hl{Coefficient of determination} $R^2$, which provides the proportion of the variance in the dependent variable $y$ that is explained by the model \cite{steel1960principles}

\begin{equation}
  %\begin{split}
    R^2 := 1 - \frac{SS_{res}}{SS_{tot}}
  %\end{split}
\end{equation}
\noindent where $SS_{res}:=\sum_{n=1}^N (y_i-\hat{y}_i)^2$ and $SS_{tot}:=\sum_{n=1}^N (y_i-\bar{y})^2$ are the \hl{Residual sum of squares} and the \hl{Total sum of squares} respectively.


\section{Interpretability Methods}
\label{sec:methodology:interpretability_methods}
\glsresetall
\graphicspath{{./Sections/Methodology/Resources/}}

There are several hyper-parameters that need to be chosen in order to compute the score map for each cell image.

For the \gls{ig} attribution map, recall that in practice computing $\phi^{IG}$ could be unfeasible or computationally very expensive. However, we can approximate $\phi^{IG}$ by means of $\phi^{Approx\ IG}$ (see equation \ref{eq:ig:approx}). Therefore, we need to define the number of steps $m$ for the Riemann sum approximation. In section \ref{sec:basics:IG} we also mentioned the necessity to set a baseline image $x'$, which should contain no information about the image, in order to compute the \gls{ig}. There are several options that can be used, each one of them with different advantages and disadvantages. However, for this work we only implemented two of them: 1) a simple black image (image containing only zeros) and 2) an image filled with Gaussian noise ($\mu=0,\ \sigma=1$). A very good analysis on the choice of the baseline can be found in this reference \cite{sturmfels2020visualizing}.

In section \ref{sec:basics:VarGrad} we saw that for \gls{vg} we need to define 2 parameters, the number of noisy images $n$ (sample size) and the standard deviation $\sigma$ for the the noise distribution.

As a rule of thumbs, a sample should not be smaller than 30, so this could be a feasible option. However, since Smilkov et al. \cite{Smilkov_smoothgrad} showed empirically that no further improvemnt (less noise) in score maps is observed for sample sizes greater than 50, we chose this bound as sample size.

Table \ref{table:VGIG_exp_set:params} shows a summary of the parameters chosen to calculate the \gls{vgig} score maps.

\begin{table}[!ht]
  \centering
  \begin{tabular}{c|c|c}
    Method & Hyperparameter & Value \\
    \hline
    \multirow{2}{*}{\gls{ig}} & $m$ & 70 \\
    \cline{2-3}
     & $x'$ & black image \\
    \hline
    \multirow{2}{*}{\gls{vg}} & $n$ & 50 \\
    \cline{2-3}
     & $\sigma$ & 1 \\
    \hline
  \end{tabular}
  \caption{Parameters to compute score maps.}
  \label{table:VGIG_exp_set:params}
\end{table}

In section \ref{sec:basics:IG}, we mentioned that the \gls{ig} algorithm holds the \textit{Completeness Axiom}, which means that the sum of all the components of the \gls{ig} attribution map must be equal to the difference between the model's output evaluated at the image and the model's output evaluated at the baseline (see equation \ref{eq:ig_completeness}). This property allow us to check empirically if the number of steps $m$ selected for the Riemann sum approximation is sufficiently large. Figure \ref{fig:VGIG_exp_set:m_sanity} shows that for our baseline\fxnote{after finishing, check that the baseline model is still called baseline} model, a random image and $m=70$, the  completeness axiom is satisfied sufficiently well.

\begin{figure}[!ht]
  \centering
  \includegraphics[width=0.8\linewidth]{sanity_check_for_m.jpg}
  \caption{Sanity check for the number of steps $m$ in the Riemann sum to approximate $\phi^{IG}$. The red dotted line represent the difference $f(x)-f(x')$. The blue line represents the value of $\sum_i \phi^{Approx\ IG}_i(f, x, x', m)$ over $\alpha$.}
  \label{fig:VGIG_exp_set:m_sanity}
\end{figure}


%\section{Interpretability Methods Evaluation}
%\label{sec:methodology:vgig_eval}

\subsection{Discussion}
\label{sec:methodology:discussion}
%% Magic command to compile root document
% !TEX root = ../../thesis.tex
% reset glossary to show full acrs
\glsresetall
% define where the images are
\graphicspath{{./Sections/Dataset/Resources/}}

Besides the preprocessing techniques introduced in section \ref{sec:dataset:data_pp} (clipping and standardization), the following approaches were also tried

\begin{itemize}
  \item Linear scaling using the $98\%$ percentile with and without clipping.
  \item Mean extraction and linear scaling using the $98\%$ percentile with clipping (like standardization, but with the $98\%$ percentile instead of the standard deviation).
  \item $49\%$ percentile extraction and linear scaling using the $98\%$ percentile (no clipping).
\end{itemize}

\noindent This approaches were tried at a per-channel level. However, clipping plus standardization where the prprocessing techniques that showed the best performance. Since we seek the model to predict \gls{tr} base on spacial information, rather than pixel intensity/color, good performance means low \gls{mae} for the \gls{cnn} models, but high \gls{mae} for the linear model (since the linear model is unable to use the spatial information). This indicates that the spatial information encoded in the images of the data set has more influence on the prediction of the model than the information encoded in the colors.

Another aspect of the dataset that is worth to mention, is that more than half cells are in phase $G_1$ (see table \ref{table:data_pp:dataset_dist_cc}), while cells in $S$ phase are less than $30\%$ and around $15\%$ for $G_2$ cells. This causes the model to focus more on correctly predicting the \gls{tr} of $G_1$ cells, than for cells in the other two phases. This happens because $G_1$ cells have more influence on the minimization of the objective function, since it is more likely that the model is fed with $G_1$ cells during training.

As it is shown on figure \ref{fig:dataset:discus:tr_dist}, \gls{tr} of $G_1$ cells is significantly lower than \gls{tr} of $S$ and $G_2$ cells. This, and that cells in different phases are not in the same proportion in the dataset, can cause the model to make a biased prediction when it is fed with a $S$ or $G_2$ cell. Two possible solutions to this problem are, either to add more cells in phases $S$ and $G_2$ to the dataset, or to sample with replacement over the available cells, so the proportion of cells in the three different phases is the same in the dataset. Another possible solution would be to do a weighted loss function based on the cell phase proportions, such that every phase has the same influence on it during training.\fxnote{Discuss with Hannah if this is the correct place to put this.}

\begin{figure}[htb]
  \centering
  \includegraphics[width=0.7\linewidth]{TR_dist.jpg}
  \caption{\gls{tr} distribution separated by cell phase.}
  \label{fig:dataset:discus:tr_dist}
\end{figure}



Talk here about the other augmentation techniques tried.

% TODO: https://www.baseclick.eu/product/5-ethynyl-uridine/ inicates: (EU) is a modified nucleoside that can be used for labeling of nascent RNA inside cells. ....experiments indicate that it will be first integrated in de novo RNA, but we have also observed incorporation into DNA after some incubation time.
% This means that the RNA readouts may be contaminated with DNA if the duration time (30 mins) is to long. Write in the future work that this should be repeated with the dataset where the duration time was 10 mins.


%% =============================================================================
%% Results chapter
%% =============================================================================
\chapter{Results}
\label{ch:results}
%% Magic command to compile root document
% !TEX root = ../../thesis.tex

%% Reset glossary to show long gls names
\glsresetall

We can interpret \gls{tr} as the amount of new RNA molecules inside a cell nucleus in a given period of time. By means of a fluorescent marker, it is possible to identify these new RNA molecules and thus approximate \gls{tr}. But, what about the morphology of other molecules and organelles within the cell nucleus? The distribution, shape and location of molecules, proteins and organelles within the nucleus could potentially encode relevant information for cellular expression. This has been the main motivation for this work. By means of a \gls{cnn}, we seek to predict \gls{tr} base mainly in spacial information encoded on images of cell nucleus.

In this section we introduce the process used to generate the data for this work, the \gls{mpm} protocol. In addition to this, we introduce the preprocessing and data augmentation techniques used. These techniques aim to improve the model's training performance, prevent overfitting and remove non-relevant information from the images. With this, we seek to encourage the model to base its prediction mainly on the spatial information encoded in the images of cell nucleus.


\section{Model Performance}
\label{sec:model_performance}
%%% Magic command to compile root document
% !TEX root = ../../thesis.tex

%% Reset glossary to show long gls names
\glsresetall

We can interpret \gls{tr} as the amount of new RNA molecules inside a cell nucleus in a given period of time. By means of a fluorescent marker, it is possible to identify these new RNA molecules and thus approximate \gls{tr}. But, what about the morphology of other molecules and organelles within the cell nucleus? The distribution, shape and location of molecules, proteins and organelles within the nucleus could potentially encode relevant information for cellular expression. This has been the main motivation for this work. By means of a \gls{cnn}, we seek to predict \gls{tr} base mainly in spacial information encoded on images of cell nucleus.

In this section we introduce the process used to generate the data for this work, the \gls{mpm} protocol. In addition to this, we introduce the preprocessing and data augmentation techniques used. These techniques aim to improve the model's training performance, prevent overfitting and remove non-relevant information from the images. With this, we seek to encourage the model to base its prediction mainly on the spatial information encoded in the images of cell nucleus.


\subsection{Metrics baseline values}
\label{sec:results:bl_values}
%% Magic command to compile root document
% !TEX root = ../../thesis.tex

%% Reset glossary to show long gls names
%\glsresetall

\graphicspath{{./Sections/Results/Resources/}}

If we assume that there is no information in the training data $\bs{X}_{train}$ that can be used to explain the independent variable $y$ (i.e. the \gls{tr}), then a model $f$ should not be able to make better predictions than the average of the target values in the training set, i,e,

\begin{equation}
  f(\bs{x}_i)=\bar{y}_{train}
  \label{eq:results:metric_bl:ybar}
\end{equation}

\noindent for all $\bs{x}_i \in \bs{X}_{train}$ and where $\bar{y}_{train}=\frac{1}{N_{train}}\sum_{i=1}^{N_{train}}y_i$.

This idea is similar to the \hl{Coefficient of determination} $R^2$, which represents how much of the variance in the target variable $y$ is explained by the model when compared to $\bar{y}$.
Thus, a model that always returns $\bar{y}$, will have a $R^2=0$. On the other hand, if the model gives better predictions than $\bar{y}$, then $0 < R^2 \leq 1$, otherwise $R^2 < 0$.

Table \ref{table:results:metric_bl_vals} shows the value of the metrics and $R^2$ (see section \ref{sec:methodology:metrics}), when they are evaluated in the training set assuming that equation \ref{eq:results:metric_bl:ybar} holds.

\begin{table}[!ht]
  \centering
  \begin{tabular}{c|c|c|c}
    \hline
    Huber & MSE & MAE & $R^2$ \\
    \hline
    46.31 & 3685.75 & 46.81 & 0.0 \\
    \hline
  \end{tabular}
  \caption{Metrics baseline values.}
  \label{table:results:metric_bl_vals}
\end{table}

As we already mentioned, the objective of this work is to explain cell expression using spatial information in multichannel images of cell nucleus.
Therefore, it is important to somehow estimate how much information related with the pixel intensities (color information) remains in the data after preprocessing and augmentation techniques.
Since the linear model cannot take advantage of spatial information, we can use the performance of the linear model and the information in table \ref{table:results:metric_bl_vals} to estimate this.
Then, if the linear model can reach a lower value in the loss function than the one shown in the table \ref{table:results:metric_bl_vals}, this means that there is still color information in the data that can be used to predict the \gls{tr}.

On the other hand, we can also use the information provided in table \ref{table:results:metric_bl_vals} to validate that the \glspl{cnn} are actually capable of predicting the \gls{tr} based only on spatial information.


\subsection{Linear Model}
\label{sec:results:lm}

\subsection{Baseline CNN}
\label{sec:results:bl_cnn}

\subsection{ResNet50V2}
\label{sec:results:RN50V2}

\subsection{Model Performance Comparative}
\label{sec:results:comparative}
%% Magic command to compile root document
% !TEX root = ../../thesis.tex

\glsresetall
% define where the images are
\graphicspath{{./Sections/Results/Resources/}}

Table \ref{table:results:model_performance_comparative} shows a comparison of the performance of each model in the validation set. Column \textit{Dataset Properties} specifies the information contained in the training data; \textit{structure} indicates only spatial information (which means that per-channel random color shifting was applied as augmentation technique to reduce pixel intensity information), while \textit{color and structure} indicate spatial and pixel intensity information (which means that no random color shifting was applied). The row $\bar{y}$ (in the \hl{Model} column) contains the baseline values for the performance metrics (see section \ref{sec:results:bl_values}). The numbers in bold indicate the models with the best overall performance (i.e., trained using data with and without pixel intensity information) per metric, while the shaded cells indicate the models with the best performance using only spatial data (structure).

% set table lengths
\setlength{\mylinewidth}{\linewidth-7pt}%
\setlength{\mylengtha}{0.17\mylinewidth-2\arraycolsep}%
\setlength{\mylengthb}{0.2\mylinewidth-2\arraycolsep}%
\setlength{\mylengthc}{0.1\mylinewidth-2\arraycolsep}%
\setlength{\mylengthd}{0.1\mylinewidth-2\arraycolsep}%
\setlength{\mylengthe}{0.08\mylinewidth-2\arraycolsep}%
\setlength{\mylengthf}{0.09\mylinewidth-2\arraycolsep}%
\setlength{\mylengthg}{0.1\mylinewidth-2\arraycolsep}%
\setlength{\mylengthh}{0.1\mylinewidth-2\arraycolsep}%


\begin{table}[!ht]
  \centering
  \begin{tabular}{m{\mylengtha} |
                  >{\centering\arraybackslash}m{\mylengthb} |
                  >{\centering\arraybackslash}m{\mylengthc} |
                  >{\centering\arraybackslash}m{\mylengthd} |
                  >{\centering\arraybackslash}m{\mylengthe} |
                  >{\centering\arraybackslash}m{\mylengthf} |
                  >{\centering\arraybackslash}m{\mylengthg} |
                  >{\centering\arraybackslash}m{\mylengthh}
                  }
    \hline
    \centering Model & Dataset Properties & $\bar{e}$ & $s(e)$ & R2 & MAE & MSE & Huber \\
    \ChangeRT{1pt}
    \centering $\bar{y}$ (baseline) & targets avg & -2.36 & 62.27 & 0 & 48.24 & 3883 & 47.74 \\
    \hline
    \multirow{2}{\mylengtha}{\centering Linear} & color-structure & \textbf{-0.63} & 44.68 & 0.49 & 33.08 & 1991 & 32.58 \\
    \cline{2-8}
     & structure & \cellcolor[HTML]{d9d9d9}-2.45 & 54.99 & 0.22 & 41.54 & 3022 & 41.04 \\
    \hline
    \multirow{2}{\mylengtha}{\centering Simple CNN} & color-structure & -1.44 & 42.25 & 0.54 & 31.81 & 1782 & 31.32 \\
    \cline{2-8}
     & structure & -2.72 & 44.84 & 0.48 & 32.84 & 2012 & 32.35 \\
    \hline
    \multirow{2}{\mylengtha}{\centering ResNet50V2} & color-structure & -2.41 & 42.34 & 0.53 & 31.62 & 1794 & 31.12 \\
    \cline{2-8}
     & structure & -4.82 & \cellcolor[HTML]{d9d9d9}44.39 & \cellcolor[HTML]{d9d9d9}0.48 & 33.18 & \cellcolor[HTML]{d9d9d9}1988 & 32.68 \\
    \hline
    \multirow{2}{\mylengtha}{\centering Xception} & color-structure & 2.35 & \textbf{39.98} &	\textbf{0.58} & \textbf{30.04} & \textbf{1600} & \textbf{29.54} \\
    \cline{2-8}
     & structure & 2.70 & 45.03 & 0.47 & \cellcolor[HTML]{d9d9d9}32.48 & 2030 & \cellcolor[HTML]{d9d9d9}31.98 \\
    \hline
  \end{tabular}
  \caption{Model performance comparison. Performance measures where taken from the validation set, with and without pixel intensity information (color-structure and structure respectively). Bold cells indicate the model-metric with best general performance. Shaded cells indicate the model-metric with best performance using only spatial (structure) data.}
  \label{table:results:model_performance_comparative}
\end{table}

As expected, the error measures, as well as the $R^2$ coefficient, are always lower when the model is trained with data containing spatial and color information (color-structure), than when it is trained only with spatial information (structure).

Table \ref{table:results:model_performance_comparative} also show that all the models performed better than $\bar{y}$ (baseline values), which means that the models were able to learn something meaningful from the data.
Note that the only metric in which the linear model surpassed the \gls{cnn} model in both datasets, was the model bias $\bar{e}$.
For the dataset with spatial information (structure) only, all \gls{cnn} models surpassed the linear model.
Moreover, for the other performance measures, ResNet50V2 and Xception were the architectures that performed the best in both datasets.
However, the performance of the simple \gls{cnn} model was similar to that of the more complex models.
This can also be seen in figures \ref{fig:results:train_per_com:cs} and \ref{fig:results:train_per_com:s}, which show the validation \gls{mae} of each model during training.
In these figures we can see that the simple model has visibly less variance than the other two \gls{cnn}, especially in figure \ref{fig:results:train_per_com:cs}.

\begin{figure}[!ht]
  \centering
  \begin{subfigure}[b]{.9\linewidth}
    \includegraphics[width=\linewidth]{train_comp_c_and_s.jpg}
    \caption{Validation \gls{mae} using data with color and structure.}
    \label{fig:results:train_per_com:cs}
  \end{subfigure}%
  \vspace{3mm}
  \begin{subfigure}[b]{.9\linewidth}
    \includegraphics[width=\linewidth]{train_comp_structure.jpg}
    \caption{Validation \gls{mae} using data only with structure.}
    \label{fig:results:train_per_com:s}
  \end{subfigure}
  \caption{Validation \gls{mae} during training using data with (figure \ref{fig:results:train_per_com:cs}) and without (figure \ref{fig:results:train_per_com:s}) pixel intensity information (color-structure and structure respectively). Each color represent a different model. The dot indicates the epoch in which the model reached its lowest validation \gls{mae}. The gray line indicates baseline \gls{mae} in the validation set. }
  \label{fig:results:train_per_com}
\end{figure}

The dots in figure \ref{fig:results:train_per_com} indicate the epochs with the best performance with respect to the validation \gls{mae} of each model.
The gray horizontal line corresponds to the \gls{mae} of the baseline evaluated in the validation set (see section \ref{sec:results:bl_values}). Due to early stopping, the number of epochs is not the same for all the models.

Figure \ref{fig:results:train_per_com:s} shows that the \gls{mae} of the linear model was generally higher than the \glspl{mae} of the \gls{cnn} models, which reinforces our hypothesis that to some extent it is possible to describe cell expression, using only spatial information within the cell nucleus.

The \hl{ResNet50V2} and \hl{Xception} models have more than 24 and 21 million of parameters respectively, while the simple \gls{cnn} model has only around 160 thousand. Therefore, the training of these two models require way more computational resources and time than the simple \gls{cnn} model.
However, table \ref{table:results:model_performance_comparative} and figure \ref{fig:results:train_per_com} show that the performance of the simple \gls{cnn} model is similar to the \hl{ResNet50V2} and \hl{Xception}.
Moreover, the importance maps of the simple \gls{cnn} model (shown in section \ref{sec:results:model_interpretation}), were less noisy and informative than those obtained with the more complicated models. For this reason, in subsequent sections we will only address the results corresponding to the simple \gls{cnn} model.


\section{Model Interpretation}
\label{sec:results:model_interpretation}

%\section{Model Interpretation Evaluation}
%\label{sec:results:model_inter_eval}

\section{Discussion}

%% =============================================================================
%% conclusion chapter
%% =============================================================================
\chapter{Conclusion}
\label{ch:Conclusion}
%% conclusion.tex
%%

%% ==================
\chapter{Conclusion}
\label{ch:Conclusion}
%% ==================

Nothing new here, only a short recap of the project, it's results, as well as possible future work.

%%% Local Variables:
%%% mode: latex
%%% TeX-master: "thesis"
%%% End:


%% =============================================================================
%% Appendix chapter
%% =============================================================================
\appendix

\chapter{Remarks on Implementation}
\label{Appendix-Implementation}
\input{Sections/Appendix/Remarks_on_Implementation}

\section{Raw data processing and QC implementation notes}
\label{sec:appendix:raw_data}
%% Magic command to compile root document
% !TEX root = ../../thesis.tex

%% Reset glossary to show long gls names
\glsresetall

This appendix contains implementation and technical notes about the data preprocessing process that needs to be performed before the construction of the dataset used to train the models. This process is performed by a single Pyhton script (Jupyter Notebook) and contemplate two main steps

\begin{enumerate}
  \item The reconstruction of single cell images from the raw data (text files).
  \item The discrimination of single cell images accordingly to a quality control.
\end{enumerate}

As it is explained in section \ref{sec:dataset:data_pp}, the protein readout of each well are contained in several files. Here we introduce those that are relevant for this work

\begin{itemize}
  \item \texttt{mpp.npy}: 2D NumPy array. Each row contains the protein readouts (intensities) of each pixel of the well (one column per protein). The values of this array vary from 0 to 65535 i.e. $2^{16}$ i.e. 2 bytes or 16 bits.
  \item \texttt{x.npy}/\texttt{y.npy}: 1D NumPy array. Each entrance contains the $x/y$ coordinate of a pixel of the well protein readouts (i.e. \texttt{x.npy} and \texttt{y.npy} map the protein readouts in \texttt{mpp.npy} with a 2D plane). Accordingly with \cite{Guteaar7042}, the size of a single channel well image is 2560x2160. Therefore, the values in \texttt{x.npy} vary between 1 and 2560 and form 1 to 2160 for \texttt{y.npy}
  \item \texttt{mapobject\_ids.npy}: 1D NumPy array. Each entrance contains an id that maps the protein readouts in \texttt{mpp.npy} with the nucleus of a cell in the well. Each cell nucleus in the well is identified by a unique id.
\end{itemize}

Since files \texttt{mpp.npy}, \texttt{x.npy}, \texttt{y.npy} and \texttt{mapobject\_ids.npy} contains different parts of the well protein readouts, the first dimension of the arrays contained in the files always has the same size.

Beside the files with protein readouts (\texttt{npy} files), each well also comes with two additional \texttt{csv} files\footnote{These \texttt{csv} files can be easily opened as a \hl{Pandas DataFrame}. For more information, please refer to the \href{https://pandas.pydata.org/pandas-docs/stable/reference/api/pandas.DataFrame.html}{official documentation}.} containing further information about each cell in the well

\begin{itemize}
  \item \texttt{metadata.csv}. Contains one raw per single cell nucleus in the well. The mapping between the metadata file and the protein readouts (\texttt{npy} files), is made through the column \hl{mapobject\_id}, which uniquely identify cells (but only within the well). On the other hand, column \hl{mapobject\_id\_cell} uniquely identify cells across all wells. Columns \hl{is\_polynuclei\_HeLa} and	\hl{is\_polynuclei\_184A1} indicate if a cell is in mitosis phase. This metadata file also contains information about the experimental setup, like plate name, well name, site position, etc..
  \item \texttt{channels.csv}. Contains only two columns that maps the immunofluorescence marker name (column \hl{channel\_name}) and the channel id (column \hl{id}) of the protein readouts.
\end{itemize}

The files introduced so far are specific to each well. However, we still need to introduce other 3 files which contains information about all the wells

\begin{itemize}
  \item \texttt{secondary\_only\_relative\_normalisation.csv}. Contains the experimental setup information related to the image capturing process. Among other information, it contains the \hl{background value} of each channel that has to be subtracted from the protein readouts during the reconstruction of the images.
  \item \texttt{cell\_cycle\_classification.csv}. Contains the phase of each cell.
  \item \texttt{wells\_metadata.csv}. Contains more information about the experimental setup. Among other information, it contains the pharmacological/metabolic perturbation applied to each well.
\end{itemize}

To execute the raw data processing, one has to open the Python Jupyter Notebook \\
\noindent\texttt{MPPData\_into\_images\_no\_split.ipynb} and replace the variable \texttt{PARAMETERS\_FILE} with the absolute path and name of the input parameters file before running the notebook. A sample parameter file (\texttt{MppData\_to\_imgs\_no\_split\_sample.json}) is provided along with this work. It contains the parameters used for the experiments shown in this work. Table \ref{table:row_data_in:params} provides an explanation of some of this parameters.

% set table lengths
\setlength{\mylinewidth}{\linewidth-7pt}%
\setlength{\mylengtha}{0.4\mylinewidth-2\arraycolsep}%
\setlength{\mylengthb}{0.6\mylinewidth-2\arraycolsep}%

\begin{longtable}{>{\centering\arraybackslash}m{\mylengtha} | m{\mylengthb}}
    \hline
    JSON variable name & Description \\
    \hline
    \texttt{raw\_data\_dir} & Path where the directories that contain the raw data files of each well are \\
    \hline
    \texttt{perturbations\_and\_wells} & Dictionary. The dictionary keys must be the directories for each perturbation, while the elements (a list) must contain the directory name of each well (one list entrance per well) \\
    \hline
    \texttt{output\_pp\_data\_path} & Path where the output folder of the notebook must be located \\
    \hline
    \texttt{output\_pp\_data\_dir\_name} & Folder name where the notebook output will be saved \\
    \hline
    \texttt{img\_saving\_mode} & Indicate the shape of the output images. To replicate the experiments of this work, this variable must be set to \texttt{original\_img\_and\_fixed\_size}, which means squared images of fixed size \\
    \hline
    \texttt{img\_size} & Integer. High and width of the output image (squared) \\
    \hline
    \texttt{return\_cell\_size\_ratio} & Binary. Indicate if cell size ratio (percentage of the image that is occupied by the cell nucleus measurements) must be added to the output metadata file. During the data augmentation, this information can be used to approximate the parameters of the distribution used to randomly vary the size of the cell nucleus \\
    \hline
    \texttt{background\_value} & Path and name (normally \texttt{secondary\_only\_relative\_normalisation.csv}) of the metadata file containing the per-channel background values \\
    \hline
    \texttt{subtract\_background} & Binary. Indicate if background color need to be subtracted from each channel \\
    \hline
    \texttt{cell\_cycle\_file} & Path and name (normally \texttt{cell\_cycle\_classification.csv}) of the metadata file containing the phase of each cell \\
    \hline
    \texttt{add\_cell\_cycle\_to\_metadata} & Binary. Indicate if cell phase must be add to the output metadata file \\
    \hline
    \texttt{well\_info\_file} & Path and name (normally \texttt{wells\_metadata.csv}) of the metadata file containing the information about well perturbation \\
    \hline
    \texttt{add\_well\_info\_to\_metadata} & Binary. Indicate if columns of \texttt{well\_info\_file} must be add to the output metadata file \\
    \hline
    \texttt{filter\_criteria} & List containing the metadata columns names that will be used in the quality control. For this work ["is\_border\_cell", "is\_polynuclei\_184A1", "is\_polynuclei\_HeLa", "cell\_cycle"] was used \\
    \hline
    \texttt{filter\_values} & List containing the filtered values for the columns indicated in \texttt{filter\_criteria}. For this work [1, 1, 1, "NaN"] was used \\
    \hline
    \texttt{aggregate\_output} & Indicate how to project each image channel into a number. Must be equal to "avg" (average) \\
    \hline
    \texttt{project\_into\_scalar} & Binary. Indicate if the channel scalar projection must be add to the output metadata file \\
    \hline
  \caption{Parameters to perform the raw data processing.}
  \label{table:row_data_in:params}
\end{longtable}

Roughly speaking, the notebook iterates over the specified wells sequentially. This means that for each well the notebook
\begin{enumerate}
  \item Reads the well metadata file \texttt{metadata.csv} and merge it with the general metadata files, \texttt{cell\_cycle\_classification.csv} and \texttt{wells\_metadata.csv}.
  \item Performs the quality control and select the ids (\hl{mapobject\_id\_cell}) that were approved.
  \item Converts\footnote{The library \texttt{mpp\_data\_V2.py} used to perform the raw data transformation, is almost entirely based on Dr. Hannah Spitzer library \texttt{mpp\_data.py}. I thank the Dr. Spitzer for providing me with her library for this work.} and saves the selected ids using the well protein readouts files \texttt{mpp.npy}, \texttt{x.npy}, \texttt{y.npy}, \texttt{mapobject\_ids.npy} and the general file \\ \texttt{secondary\_only\_relative\_normalisation.csv}.
\end{enumerate}

The notebook also saves at the end a general metadata file (\texttt{csv} file), containing the metadata of all the processed wells.


\section{TensorFlow Dataset and image preprocessing implementation notes}
\label{sec:appendix:tfds}
%% Magic command to compile root document
% !TEX root = ../../thesis.tex

%% Reset glossary to show long gls names
\glsresetall


\section{Model training implementation notes}
\label{sec:appendix:Model_training_IN}
%% Magic command to compile root document
% !TEX root = ../../thesis.tex

%% Reset glossary to show long gls names
\glsresetall

This appendix is intended to provide a brief explanation of how to run the Python script (Jupyter Notebook) responsible for training the models used in this work. In addition, here we also provide a short explanation of the parameter file that must be specified to train any model.

Since data augmentation techniques can be selected independently for each trained model, their corresponding hyperparameters are also explained here.

The Jupyter Notebook responsible for training the models is the one that requires the largest number of parameters. However, the function \texttt{set\_model\_default\_parameters} (in the \texttt{Utils.py} library) provides default values for all the parameters. Therefore, if some hyperparameter is not specified here or in section \ref{sec:methodology:models}, then the value used was the one specified in that function.

To train a model, one has to open the Python Jupyter Notebook \\
\noindent\texttt{Model\_training\_class.ipynb} and replace the variable \texttt{PARAMETERS\_FILE} with the absolute path and name of the input parameters file before running the notebook. A sample parameter file (\texttt{Train\_model\_sample.json}) is provided along with this work. It contains the parameters used to train the \hl{Simple} \gls{cnn} (see section \ref{sec:methodology:simple_CNN}), using the data augmentation techniques specified in section \ref{sec:methodology:data:augm}.
Table \ref{table:model_train_in:params} provides an explanation of some of the model training parameters, while table \ref{table:model_train_in:params_da} an explanation of some of the data augmentation parameters. Although the training and data augmentation parameters are specified in separate tables, they must be in the same \texttt{JSON} parameter file (and also as items in the same dictionary).

% set table lengths
\setlength{\mylinewidth}{\linewidth-7pt}%
\setlength{\mylengtha}{0.3\mylinewidth-2\arraycolsep}%
\setlength{\mylengthb}{0.7\mylinewidth-2\arraycolsep}%

\begin{longtable}{>{\centering\arraybackslash}m{\mylengtha} | m{\mylengthb}}
    \hline
    JSON variable name & Description \\
    \hline
    \texttt{model\_name} & Name of the architecture to be trained. Available: \texttt{simple\_CNN}, \texttt{ResNet50V2}, \texttt{Xception}, \texttt{Linear\_Regression} \\
    \hline
    \texttt{pre\_training} & Binary, whether or not use pretrained weights and biases as initial parameters. Only available for \texttt{ResNet50V2} or \texttt{Xception} architectures \\
    \hline
    \texttt{dense\_reg} & [$L_1$, $L_2$], where $L_1$ and $L_2$ are the regularization strengths for the dense layers weights\\
    \hline
    \texttt{conv\_reg} & [$L_1$, $L_2$], where $L_1$ and $L_2$ are the regularization strengths for the convolution layers weights \\
    \hline
    \texttt{bias\_l2\_reg} & $L_2$ regularization strengths for convolution and dense layers biases \\
    \hline
    \texttt{number\_of\_epochs} & Maximum number of epochs to train \\
    \hline
    \texttt{early\_stop\_patience} & For early stopping. Specify how many epochs at most the model can train without decreasing the loss function before stopping the training \\
    \hline
    \texttt{loss} & Loss function name. Available: \texttt{mse}, \texttt{huber}, \texttt{mean\_absolute\_error} \\
    \hline
    \texttt{learning\_rate} & Learning rate for Adam optimizer \\
    \hline
    \texttt{BATCH\_SIZE} & Batch size \\
    \hline
    \texttt{model\_path} & Path to save the models and checkpoints \\
    \hline
    \texttt{clean\_model\_dir} & Binary, whether or not to delete the content of the directory specified by \texttt{model\_path} \\
    \hline
    \texttt{tf\_ds\_name} & Name of the TFDS to be used during training\\
    \hline
    \texttt{local\_tf\_datasets} & Local path where the TFDSs are stored \\
    \hline
    \texttt{input\_channels} & List containing the name of the channels (elements of the column \hl{Marker identifier} of table \ref{table:tfds_in:channels}) to be included in the images contained in the \gls{tfds} \\
    \hline
    \texttt{shuffle\_files} & Binary, whether or not to shuffle the dataset at the beginning of each epoch \\
    \hline
    \texttt{seed} & Random seed to reproduce the shuffling of the TFDS \\
    \hline
  \caption{Model training parameters.}
  \label{table:model_train_in:params}
\end{longtable}

% set table lengths
\setlength{\mylinewidth}{\linewidth-7pt}%
\setlength{\mylengtha}{0.4\mylinewidth-2\arraycolsep}%
\setlength{\mylengthb}{0.6\mylinewidth-2\arraycolsep}%

\begin{longtable}{>{\centering\arraybackslash}m{\mylengtha} | m{\mylengthb}}
    \hline
    JSON variable name & Description \\
    \hline
    \texttt{random\_horizontal\_flipping} & Binary, whether or not to perform random horizontal flips on the training set \\
    \hline
    \texttt{random\_90deg\_rotations} & Binary, whether or not to perform random 90deg rotations on the training set \\
    \hline
    \texttt{CenterZoom} & Binary, whether or not to perform random center zoom-in/out on the training set \\
    \hline
    \texttt{CenterZoom\_mode} & Zoom proportion R.V. distribution. Available: \texttt{random\_normal}, \texttt{random\_uniform} \\
    \hline
    \texttt{Random\_channel\_intencity} & Binary, whether or not to perform per-channel random color shifting on the training set \\
    \hline
    \texttt{RCI\_dist} & Distribution of random color shifts. Available: \texttt{uniform}, \texttt{normal}. If \texttt{uniform} distribution selected ($U(-a, a)$), then $a=\mu+3\sigma$ \\
    \hline
    \texttt{RCI\_mean} & Mean $\mu$ for the distribution specified by \texttt{RCI\_dist} \\
    \hline
    \texttt{RCI\_stddev} & Standard deviation $\sigma$ for the distribution specified by \texttt{RCI\_dist} \\
    \hline
    \texttt{Random\_noise} & Binary, whether or not to add random normal noise ($N(0, \sigma)$) on the training set \\
    \hline
    \texttt{Random\_noise\_stddev} & Standard deviation corresponding to the normal distribution of random noise \\
    \hline
  \caption{Data augmentation parameters.}
  \label{table:model_train_in:params_da}
\end{longtable}


\section{VarGrad IG implementation notes}
\label{sec:appendix:VarGrad_IG_Experimental_Setup}
%% Magic command to compile root document
% !TEX root = ../../thesis.tex

%% Reset glossary to show long gls names
\glsresetall

In order to generate the \acrlong{vg} \acrlong{ig} score maps, you must execute the python script \texttt{get\_VarGradIG\_from\_TFDS.py} specifying the parameters file
\begin{lstlisting}[language=Bash]
python get_VarGradIG_from_TFDS.py -p ./Parameters_file_name.json
\end{lstlisting}

Table \ref{table:imp_notes:VGIG_params} show all the parameters that need to be specified to execute \\
\noindent \texttt{get\_VarGradIG\_from\_TFDS.py} successfully.

% set table lengths
\setlength{\mylinewidth}{\linewidth-7pt}%
\setlength{\mylengtha}{0.18\mylinewidth-2\arraycolsep}%
\setlength{\mylengthb}{0.27\mylinewidth-2\arraycolsep}%
\setlength{\mylengthc}{0.55\mylinewidth-2\arraycolsep}%

\begin{table}[!ht]
  \centering
  \begin{tabular}{>{\centering\arraybackslash}m{\mylengtha}|>{\centering\arraybackslash}m{\mylengthb}|m{\mylengthc}} % m stands for middle (p:top, b:bottom), max 144 mm
    \hline
    Hyperparam & JSON variable name & Notes \\
    \hline
    $m$ & \texttt{IG\_m\_steps} & Number of steps to approximate \gls{ig} \\
    \hline
    $x'$ &  \texttt{IG\_baseline} & Baseline image for \gls{ig}. Available: "black" for a simple black image and "noise" for an image filled with Gaussian noise ($\mu=0,\ \sigma=1$) \\
    \hline
    $n$ & \texttt{VarGrad\_n\_samples} & Number of noisy images to compute \gls{vg} \\
    \hline
  \end{tabular}
  \caption{Parameters to compute score maps.}
  \label{table:imp_notes:VGIG_params}
\end{table}


\chapter{General Remarks}
\label{Appendix-general-remarks}
%% Magic command to compile root document
% !TEX root = ../../thesis.tex

This appendix contains general remarks relevant for this work, like a small explanation of the fluorescent markers used for the \gls{mpm} protocol.


\section{Indirect Immunofluorescence markers description}
\label{sec:appendix:if_markers}
%% Magic command to compile root document
% !TEX root = ../../thesis.tex

%% Reset glossary to show long gls names
\glsresetall

In order to capture the distribution and amount of proteins inside a cell nucleus, the \gls{mpm} protocol use a set of fluorescent markers called \gls{if}. Table \ref{table:apendix:if_markers} shows a description of the most relevant markers. The identifiers and ids corresponding to the markers described in table \ref{table:apendix:if_markers} can be consulted in table \ref{table:tfds_in:channels}.

Some of the markers used in the \gls{mpm} protocol are strongly related with the \hl{RNA polymerase} enzyme\footnote{An enzyme is a proteins that act as biological catalysts to accelerate chemical reactions.}. As it was explained in section \ref{sec:basics:bio_back} (see figure \ref{fig:BB:premrna_synth}), the RNA polymerase it the enzyme responsible for starting the transcription process of genes (i.e., copping a sequence from a section of the DNA into a \gls{pmrna} strand).

% set table lengths
\setlength{\mylinewidth}{\linewidth-7pt}%
\setlength{\mylengtha}{0.24\mylinewidth-2\arraycolsep}%
\setlength{\mylengthb}{0.76\mylinewidth-2\arraycolsep}%

%\begin{table}[!ht]
%  \centering
%  \begin{tabular}{>{\centering\arraybackslash}m{\mylengtha}|m{\mylengthb}} % m stands for middle (p:top, b:bottom)
\begin{longtable}{>{\centering\arraybackslash}m{\mylengtha} | m{\mylengthb}}
    \hline
    Marker name & Description \\
    \hline
    DAPI & \hl{4',6-Diamidino-2-Phenylindole}, or DAPI, is a fluorescent stain that binds strongly to adenine–thymine-rich regions in DNA \cite{kapuscinski1995dapi} \\
    \hline
    GTF2B & \hl{Transcription factor II B}, or TFIIB (also known as GTF2B), is an antibody that binds to the general transcription factor involved in the formation of the RNA polymerase II preinitiation complex \cite{lewin2004genes} \\
    \hline
    SRRM2 & \hl{Serine/arginine repetitive matrix protein 2}, or SRRM2, is an antibody that binds to the protein that in humans is encoded by the SRRM2 gene and which is required for pre-mRNA splicing as component of the spliceosome. Along with the protein SON, SRRM2 is essential for \gls{ns}\footnotemark formation \cite{ilik2020and} \\
    \hline
    SON & SON is protein that in humans is encoded by the SON gene. The protein binds to RNA and promotes pre-mRNA splicing, particularly of transcripts with poor splice sites. Along with the protein SRRM2, SON is essential for \gls{ns} formation \cite{ilik2020and} \\
    \hline
    SP100 & \hl{SP100 nuclear antigen\footnotemark}, or SP100, is a gene that encodes a subnuclear organelle and major component of the PML (promyelocytic leukemia)-SP100 nuclear bodies \cite{sp100} \\
    \hline
    PML & \hl{Promyelocytic Leukemia}, or PML, is a protein encoded by the PML gene. PML is a nuclear body involved in oncogenesis (tumor suppressor) and viral infection. This subnuclear domain has been reported to be rich in RNA and a site of nascent RNA synthesis, implicating its direct involvement in the regulation of gene expression \cite{boisvert2000promyelocytic} \\
    \hline
    PCNA & \hl{Proliferating Cell Nuclear Antigen}, or PCNA, is a DNA clamp that acts as a processivity factor for \hl{DNA polymerase} $\delta$\footnotemark in eukaryotic cells and is essential for replication \cite{kisielewska2005gfp} \\
    \hline
    NCL & \hl{Nucleolin}, or NCL, is an antibody that binds to a protein that in humans is encoded by the NCL gene. The protein is involved in the synthesis and maturation of ribosomes. It is located mainly in dense fibrillar regions of the nucleolus \cite{erard1988major} \\
    \hline
    POL2RA\_pS2 & \hl{RNA Polymerase II Phosphospecific (Ser2)}, or POL2RA\_pS2, is an antibody that binds to the largest subunit of the RNA polymerase II (which is the enzyme responsible for transcribing DNA into \gls{pmrna}) \cite{POLR2ApS2} \\
    \hline
    CDK9 & \hl{Cyclin-dependent kinase 9}, or CDK9, is a protein encoded by the CDK9 gene and is involved in the regulation of transcription. CDK9 is a member of the cyclin-dependent kinase (CDK) family, which includes two main subgroups of kinases, those that mainly regulate cell cycle progression (including CDK1, CDK2, and CDK4/6) and those that control transcriptional processes (including CDK7, CDK8, CDK9, CDK12, and CDK13) \cite{cassandri2020cdk9} \\
    \hline
    CDK9\_pT186 & \hl{Cyclin Dependent Kinase 9 Phospho-Thr186 Antibody}, or CDK9\_pT186, is a molecule derived from human CDK9 around the phosphorylation site of T186 \cite{CDK9pT186} \\
    \hline
    RB1\_pS807\_S811 & \hl{Retinoblastoma Protein pS807/pS811 Antibody}, or RB1\_pS807\_S811. Retinoblastoma Protein (RB1 or just RB) is a tumor suppressor protein, which prevents excessive cell growth by inhibiting cell cycle progression until the cell is ready to divide.  \cite{murphree1984retinoblastoma} \\
    \hline
    PABPN1 & \hl{Polyadenylate-Binding Nuclear Protein 1}, or PABPN1 (also known as PABP-2), is a protein encoded by the PABPN1 gene, which is involved in the addition of a Poly-A tail to the \gls{pmrna} during the splicing process (see figure \ref{fig:BB:splicing} on section \ref{sec:basics:transcription_process}) \cite{muniz2015poly} \\
    \hline
    SETD1A & \hl{SET Domain Containing 1A, Histone Lysine Methyltransferase}, or SETD1A. The protein encoded by this gene is a component of a histone methyltransferase (HMT) complex that produces mono-, di-, and trimethylated histone H3 at Lys4. Trimethylation of histone H3 at lysine 4 (H3K4me3) is a chromatin modification known to generally mark the transcription start sites of active genes \cite{SETD1A} \\
    \hline
    COIL & \hl{Coilin}, or COIL. The protein encoded by this gene is an integral component of Cajal bodies, which are nuclear suborganelles involved in the post-transcriptional modification of small nuclear and small nucleolar RNAs \cite{COIL} \\
    \hline
    EU & \hl{5-Ethynyl Uridine}, or EU, is a molecule that binds to newly transcribed RNA \cite{jao2008exploring}. This means that EU can be used to detect RNA synthesis in cells and/or predict \gls{tr} \\
    \hline
%  \end{tabular}
  \caption{\Acrlong{if} markers description. The first column shows the markers name, the second the identifier used on the implementation (parameters file) and the third a brief description of it.}
  \label{table:apendix:if_markers}
%\end{table}
\end{longtable}

\footnotetext[2]{The \gls{ns} (also known as \hl{Splicing speckles}) are structures inside the cell nucleus in which the \gls{pmrna} is transformed into a mature \gls{mrna} (see section \ref{sec:basics:transcription_process}) \cite{spector2011nuclear}.}

\footnotetext[3]{An \hl{antigen} is a molecule that triggers the formation of antibodies (by bounding to its specific antibody or B-cell antigen receptor) and can cause an immune response.}

\footnotetext[4]{DNA polymerase delta, or DNA Pol $\delta$, is an enzyme complex found in eukaryotes that is involved in DNA replication and repair.}


% all appendices behind backmatter will go without numbers
\backmatter

%% =============================================================================
%% Lists, glossaries, etc.
%% =============================================================================

%List of Figures
\listoffigures

\vspace*{1.5cm}

%List of Tables
\listoftables


%List of algorithms
% only in conjunction with algorithm2e
%\phantomsection% for hyperref
%\addcontentsline{toc}{chapter}{Algorithmenverzeichnis}%
%\markboth{Algorithmenverzeichnis}{Algorithmenverzeichnis}
%\listofalgorithms


%Algorithmenverzeichnis
% nur in Verbindung mit algorithm2e
%\phantomsection% fuer hyperref
%\addcontentsline{toc}{chapter}{Algorithmenverzeichnis}%
%\markboth{Algorithmenverzeichnis}{Algorithmenverzeichnis}
%\listofalgorithms

% Index
% Add index to table of contents
\addcontentsline{toc}{chapter}{Index}
\printindex

%Print the glossary
%\printglossaries
\printglossary[type=\acronymtype]

% Bibliography
\printbibliography[heading=bibintoc]

%In the final version, this should be empty and needn't be commented out. Someone who works sloppily should at least remember to comment out the fixme list, so that unfinished places are
%not so obvious for the examiners.
\listoffixmes
%
\end{document}
% ==============================================================================
% End of document
% ==============================================================================
