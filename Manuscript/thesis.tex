% ==============================================================================
% thesis.tex
% Example file for tumthesis.csl
% Michael Ritter, 2012
% Licence:
% This work may be distributed and/or modified under the
% conditions of the LaTeX Project Public License, either version 1.3
% of this license or (at your option) any later version.
% The latest version of this license is in
% http://www.latex-project.org/lppl.txt
% and version 1.3 or later is part of all distributions of LaTeX
% version 2005/12/01 or later.
% ==============================================================================
\documentclass[biblatexBackend=bibtex]{tumthesis}

% ------------------------------------------------------------------------------
%FixMe-Status: final (no FixMe comments) or draft (comments visible)
\fxsetup{draft}
%\fxsetup{final}
% ------------------------------------------------------------------------------

% ------------------------------------------------------------------------------
%  Language selection for metadata and main text (can be changed at any point in
%  the main text.
%\selectlanguage{ngerman}
\selectlanguage{english}
% ------------------------------------------------------------------------------

% ------------------------------------------------------------------------------
% Data for the bibliography
\addbibresource{thesis.bib}
% ------------------------------------------------------------------------------

% ------------------------------------------------------------------------------
% Further packages and TikZ libraries can be incorporated here.
\usetikzlibrary{arrows}
% ------------------------------------------------------------------------------

% ------------------------------------------------------------------------------
% Using the include-command: The content of the work will be incorporated using
% the include-command. So that you can test things out without always needing
% the full thesis, you can define here which sections will be incorporated and
% which will not. Of course, for the final version you should be sure to
% incorporate everything.
\includeonly{%
%titlepage,%
Sections/Declaration,%
Sections/Abstract,%
Sections/Acknowledgments,
Sections/Introduction,%
Sections/Basics,
Sections/Dataset,
Sections/Methodology,
%Sections/Implementation,
Sections/Results,
Sections/conclusion,%
Sections/Appendix%
}%
% ------------------------------------------------------------------------------

% ------------------------------------------------------------------------------
% PDF-Metadaten
\hypersetup{
 pdfauthor={Andres Alberto Becker Sanabria},
 pdftitle={Predicting transcription rate from multiplexed protein maps using deep learning},
 pdfsubject={Predicting transcription rate from multiplexed protein maps using deep learning},
 pdfkeywords={Master's Thesis},
 colorlinks=true, %coloured links (for the PDF version)
% colorlinks=false, % no coloured links (for the print version)
}
% ------------------------------------------------------------------------------

% ------------------------------------------------------------------------------
% Data for the title page and declaration
\author{Andres Becker}
\title{Predicting transcription rate from multiplexed protein maps using deep learning}
%\subtitle{A Tutorial for Theses}
%\faculty{Fakultät für Mathematik}
\faculty{Department of Mathematics}
%\institute{Lehrstuhl für Mathematische Modelle biologischer Systeme}
\institute{Chair of Mathematical Modeling of Biological Systems}
\subject{master}
%\subject{bachelor}
%\subject{diploma}
%\subject{project}
%\subject{seminar}
%\subject{idp}
%\subject{Short Overview}
\professor{Prof. Dr. Fabian J. Theis} %Themensteller
\advisor{Dr. Hannah Spitzer} %Betreuer
\date{April 14, 2021} %Submission Date
\place{München} %Place where document is signed
% ------------------------------------------------------------------------------

% ==============================================================================
% Acronyms
% ==============================================================================

% Packages should be called in the preamble document (preamble.tex). However,
% glossaries package requires to be called after hyperref, babel, polyglossia,
% inputenc and fontenc. Therefore, calling glossaries package in preamble doc
% produce errors during compilation (and no generation of Acronyms section).

%\usepackage[style=long,nonumberlist, toc,acronym,nomain]{glossaries}
% style=long: center the acronyms in acronym section
% nonumberlist: remove the list all places in docu where the acron was called
% toc: Add Acronyms section to table of content
% acronym, nomain: necessary
\usepackage[acronym, toc, nomain]{glossaries}
\makeglossaries

%\newacronym[<options>]{<label>}{<short>}{<long>}
\newacronym{4i}{4i}{iterative indirect immunofluorescence imaging}
\newacronym{tr}{TR}{transcription rate}
\newacronym{scmos}{sCMOS}{scientific complementary metal oxide semiconductor}
\newacronym{cnn}{CNN}{Convolutional Neural Network}
\newacronym{mrna}{mRNA}{messenger RNA}
\newacronym{ig}{IG}{Integrated Gradient}
\newacronym{vg}{VG}{VarGrad}
\newacronym{sg}{SG}{SmoothGrad}
\newacronym{vgig}{VGIG}{VarGrad Integrated Gradient}
\newacronym{dnn}{DNN}{Deep Neural Network}
% ------------------------------------------------------------------------------

% ==============================================================================
% Costume commands
% ==============================================================================
\newcommand{\hl}[1]{\textbf{#1}} % Defined to highlight new words

% ------------------------------------------------------------------------------

% ==============================================================================
% Main part of the document
% ==============================================================================
\makeindex[title=Index,options=-s myindex]
\begin{document}
\pagestyle{empty}
\frontmatter%
%\selectlanguage{ngerman}
\selectlanguage{english}
\maketitlepage%
%\maketitlepageDissertation % a more elegant one, but for PHd dissertation
%\makedeclaration%

%% ==============================
\chapter*{Declaration}
%% ==============================

\vfill

I hereby declare that this thesis is my own work and that no other sources have been used except those clearly indicated and referenced.

\vspace{3em}

\noindent Andres Alberto Becker Sanabria, München, 14.05.2021


%==================================================
% abstract.tex
% Beispieldatei für tumthesis.cls und thesis.tex
% Michael Ritter, 2012
% Lizenz:
% This work may be distributed and/or modified under the
% conditions of the LaTeX Project Public License, either version 1.3
% of this license or (at your option) any later version.
% The latest version of this license is in
% http://www.latex-project.org/lppl.txt
% and version 1.3 or later is part of all distributions of LaTeX
% version 2005/12/01 or later.
%==================================================

%% Magic command to compile root document
% !TEX root = ../thesis.tex

%% Reset glossary to show long gls names
\glsresetall

%\cleardoublepage

\selectlanguage{english}
\section*{Abstract}

By means of fluorescent antibodies it is possible to observe the amount of nascent RNA within the nucleus of a cell, and thus estimate its \gls{tr}. But what about the other molecules, proteins, organelles, etc. within the nucleus of the cell? Is it possible to estimate the \gls{tr} using only the shape and distribution of these subnuclear components? By means of multichannel images of single cell nucleus (obtained through the \gls{mpm} protocol \cite{Guteaar7042}) and \glspl{cnn}, we show that this is possible. 
Applying pre-processing and data augmentation techniques, we reduce the information contained in the intensity of the pixels and the correlation of these between the different channels. This allowed the \gls{cnn} to focus mainly on the information provided by the location, size and distribution of elements within the cell nucleus.
For this task different architectures were tried, from a simple \gls{cnn} (with only 160k parameters), to more complex architectures such as the ResNet50V2 or the Xception (with more than 20m parameters).
Furthermore, through the interpretability methods \gls{ig} and \gls{vg}, we could obtain score maps that allowed us to observe the pixels that the \gls{cnn} considered as relevant to predict the \gls{tr} for each cell nucleus input image. The analysis of these score maps reveals how as the \gls{tr} changes, the \gls{cnn} focuses on different proteins and areas of the nucleus. This shows that interpretability methods can help us to understand how a \gls{cnn} make its predictions and learn from it, which has the potential to provide guidance for new discoveries in the field of biology.


%% Magic command to compile root document
% !TEX root = ../thesis.tex

\selectlanguage{english}
\section*{Acknowledgments}

To my father, my partner in my wildest adventures, best friend and who taught me what are the important things in life. He may never read this, but let the world known he is a loved and admired man.


%%% Local Variables:
%%% mode: latex
%%% TeX-master: "thesis"
%%% End:


\tableofcontents%

\mainmatter%
\pagestyle{headings}

%% ==============================
\chapter{Introduction}
\label{ch:introduction}
%% ==============================

A small overview. Explain how CNN can be used to predict Transcription Rate and how we can use interpretability methods to learn from this models.

\section{Motivation}
\label{sec:intro:motivation}

\section{Literature review}
\label{sec:intro:literature_review}

\section{Background}
\label{sec:intro:background}

\subsection{Cellular Expression}
\label{sec:intro:cellular_expression}

\subsection{Multiplexed Protein Maps}
\label{sec:intro:multiplexed_protein_maps}
%% Magic command to compile root document
% !TEX root = ../../thesis.tex

%% Reset glossary to show long gls names
\glsresetall

%% Set path to look for the images
\graphicspath{{./Sections/Dataset/Resources/}}

% A small motivation to create Multiplexed Protein Maps
The amount of protein or \gls{mrna} inside a cell may not be enough to fully describe cellular function. Accordingly to Buxbaum et al. \cite{Buxbaum_2014} and Korolchuk et al. \cite{Korolchuk2011}, cellular function can heavily depends on the specific intracellular location and interaction with other molecules and intracellular structures. Therefore, cellular expression is determined by the functional state, abundance, morphology, and turnover of its intracellular organelles and cytoskeletal structures. This means that having the ability to look at the concentration and distribution of different molecules within a cell, is an important technological achievement that can significantly leverage scientific discoveries in biomedicine.
This is exactly what \gls{mpm} allows us to do (\cite{Guteaar7042}). \gls{mpm} are protein readouts from cell cultures, that simultaneously captures different properties of the cell, like its shape, cycle state, detailed morphology of organelles, nuclear subcompartments, etc. It also captures highly multiplexed subcellular protein maps, which can be used to identify functionally relevant single-cell states, like \gls{tr}. These maps can also identify new cellular states and allow quantitative comparisons of intracellular organization between single cells in different cell cycle states, microenvironments, and drug treatments \cite{Guteaar7042}.

So, let us explain more in deept what are these \gls{mpm}. Accordingly to Gabriele Gut et al. \cite{Guteaar7042}, \gls{mpm} is a nondegrading protocol that allows to capture efficiently thousands of single cell multichannel images, where each channel contains the distribution and concentration of a protein of interest inside each cell. To achieve this, the protocol is made up of different steps that will be briefly explained here.

% 4i explanation
\subsubsection{Iterative indirect immunofluorescence imaging}
The \gls{mpm} protocol starts with a process called \gls{4i} developed by the same group. The \gls{4i} is a complete protocol by itself, and it allows to capture the concentration and distribution of individual proteins in thousands of different cells in a tissue\footnote{The tissues were made from cell cultures using the \hl{HeLa Kyoto} \hl{184A1} cell line. HeLa is the oldest and most commonly used immortal human cell line in scientific research. The story behind it is quite interesting, so it's worth checking out.}.
Before applying the \gls{4i} protocol, the \hl{plate} where the cell culture is must be divided into squared sections called \hl{wells}. Then, the \gls{4i} protocol is applied over each well and photographed in sections called \hl{sites}.

Roughly speaking, \gls{4i} works as follow
\begin{enumerate}
  % 1
  \item The selected well is prepared for the staining-elution process.
  %2
  \item The well is saturated with a liquid containing \hl{antibodies}\footnote{An antibody is a Y-shaped protein that can recognize and bind to a unique molecule (its antigen, e.g. another protein).} stained with a fluorescent ink (\gls{if}), which binds to a target protein.
  %3
  \item The well is exposed to a high-energy light and photographed using a light microscopy (which produces a single channel image).
  %4
  \item The antibodies inside the tissue are washed-out using a chemical elution substrate.
  %5
  \item Steps 2 to 4 are repeated 20 times to get 20 images of the same protein.
  %6
  \item To improve the protein readouts, the 20 single channel images are projected into one by \hl{maximum intensity projection}.
\end{enumerate}

Figure \ref{fig:4i:1} illustrates the steps of the \gls{4i} protocol that capture the saturation and distribution of a specific protein. Keep in mind that even though the \gls{4i} protocol captures sever images of the tissue, it returns an uni-channel image (step 6). Figure \ref{fig:4i:2} shows the \gls{4i} protocol applied 40 times with different \gls{if} and over a 384-well plate, which captures the concentration and distribution of 40 different specific proteins.

\begin{figure}[htb]
  \centering
  \begin{subfigure}[t]{.3\linewidth}
    \includegraphics[width=\linewidth]{4i_1.png}
    \caption{\Acrfull{4i} protocol.}
    \label{fig:4i:1}
  \end{subfigure}
  \hspace{4mm}
  \begin{subfigure}[t]{.45\linewidth}
    \includegraphics[width=\linewidth]{4i_2.png}
    \caption{\gls{4i} protocol applied over a specific well of a plate and for 40 different \gls{if}.}
    \label{fig:4i:2}
  \end{subfigure}%
  \caption{Schematic representation of the \gls{4i} protocol for a single well and for 40 different fluorescent antibodies. Figure \subref{fig:4i:2} also shows the image analysis to identify single cells and its components (nucleus and cytoplasm). Images source: \cite{Guteaar7042}.}
  \label{fig:4i}
\end{figure}

By the time \cite{Guteaar7042} was published, the \gls{4i} protocol was able to capture cell culture images with up to 40 channels without degrading the tissue, which is why \gls{mpm} is called a \textit{nondegrading} protocol.

\subsubsection{Multiplexed single cell analysis}

Once the multichannel images were generated using the \gls{4i} protocol, a series of image preprocessing and image analysis methods (\cite{Carpenter2006} and \cite{snijder2012single}) are applied to generate segmentation masks to identify individual cells, as well as their cytoplasm and nucleus. Figure \ref{fig:4i:2} shows this segmentation at a cellular level, while figure \ref{fig:4i:segmentation} shows it also at a subcellular level. In both cases the boundaries are marked with a white contour. This single cell analysis is also used to identify cells that do not satisfy certain quality controls (like cells in the border of the image or in mitosis stage). However, this will be addressed in detail on section \ref{sec:dataset:data_pp}.

\begin{figure}[htb]
  \centering
  \includegraphics[width=0.5\linewidth]{4i_segmentation.png}
  \caption{Visualization of the subcellular segmentation of a \gls{4i} protocol for 18 \gls{if} stains. The image was created by combining the readouts of 3 of this \gls{if} stains: PCNA (cyan), FBL (magenta) and TFRC (yellow). The number next to each staining label indicates their corresponding 4i acquisition cycle (\gls{4i} protocol step 5). The orange rectangle and the tile at its right shows a section of the nucleus and cytoplasm of a single cell. The other 3 tiles shows the \gls{4i} readout of each of the 3 proteins. Images source: \cite{Guteaar7042}.}
  \label{fig:4i:segmentation}
\end{figure}

\subsubsection{Multiplexed single-pixel analysis framework}
Even though the cell cultures are now segmented into individual cells and nucleus, there is still one missing part that must be considered, and that is that cells are 3-dimensional objects. Recall that the \gls{4i} protocol saturates the cell culture with a liquid containing fluorescent antibodies. This means that the antibody can either bind to its corresponding protein inside or outside the cell nucleus. Therefore, even though that we segmented a cell into nucleus and cytoplasm, a readout assigned to the nucleus could come from a protein in the cytoplasm under or above the nucleus, and not from inside it. Fortunately, intensity readouts from proteins inside the nucleus are much higher than those in the cytoplasm. Therefore, by means of a two steps clustering approach\footnote{To identify clusters in an unsupervised manner, \hl{Self Organizing Maps} algorithm and \hl{Phenograph} analysis were used over a very large number of pixels sampled from a large number of single cells \cite{Guteaar7042}.}, pixels can be classified accordingly to their intensity profile (figures \ref{fig:mcu:1} and \ref{fig:mcu:2}), so the source of their readout can be identified. This pixel type classification is called \Acrfull{mcu} and is illustrated in figure \ref{fig:mcu:3}. After pixels clusters (intensity profiles) where identified, the pixels whose measurement comes from the cytoplasm and not from the nucleus are removed.

\begin{figure}[htb]
  \centering
  \begin{subfigure}[t]{.3\linewidth}
    \includegraphics[width=\linewidth]{mcu_1.png}
    \caption{Extraction of pixel intensities.}
    \label{fig:mcu:1}
  \end{subfigure}
  \hspace{4mm}
  \begin{subfigure}[t]{.3\linewidth}
    \includegraphics[width=\linewidth]{mcu_2.png}
    \caption{Pixel clustering by Self Organizing Maps and Phenograph.}
    \label{fig:mcu:2}
  \end{subfigure}
  \hspace{4mm}
  \begin{subfigure}[t]{.3\linewidth}
    \includegraphics[width=\linewidth]{mcu_3.png}
    \caption{Cell subdivision base on the \gls{mcu}.}
    \label{fig:mcu:3}
  \end{subfigure}
  \caption{Figure \subref{fig:mcu:1} shows the pixel intensity extraction for a single cell. The pixel intensity is a vector containing the readout of that 2D location for each protein, one specific protein readout per entrance. Figure \subref{fig:mcu:2} shows the clusters found by Self Organizing Maps algorithm and Phenograph analysis over the pixel intensities. Figure \subref{fig:mcu:3} shows a cell masked with the clusters found by the \gls{mcu} analysis. Images source: \cite{Guteaar7042}.}
  \label{fig:mcu}
\end{figure}

Finally, the nucleus of each cell is stored separately and identified with a unique id.\fxnote{After you finish writing the dataset section review if this sentence is accurate.}

\subsubsection{Cell cycle phase classification: $G_1,\ S,\ G_2$ and $M$ phase}

The \gls{mpm} protocol is not only capable to capture the concentration and distribution of molecules inside thousands of cells. It can also identify the phase each cell is in, which is tightly related with the abundances and distribution of molecules inside a cell \cite{Guteaar7042}.

Roughly speaking, cell cycle phase was determined by means of a \gls{svm} classifier and k-means clustering. First, a \gls{svm} classifier is trained to identify $M$ phase cells based on the nuclear information in one of the image channels (\hl{DAPI}\footnote{A brief description of this marker can be found on section \ref{sec:appendix:if_markers}.}). Then, based on the nuclear information of channel \hl{PCNA}, a second \gls{svm} classifier is trained to identify cells in phase $S$. Finally, cells in phase $G_1$ and $G_2$ are classified using a k-means algorithm, using the pixel intensity profiles of the DAPI channels excluding the cells in $S$ and $M$ phase. A more detailed explanation of the cell cycle classification process can be found on the dataset paper \cite{Guteaar7042}.

\subsubsection{Pharmacological and metabolic perturbations}

To further explore the capabilities of the \gls{mpm} protocol, the creators of the dataset (Gabriele Gut et al. \cite{Guteaar7042}) applied the \gls{mpm} protocol to a cell populations that were to nine pharmacological and metabolic perturbations. The analysis reveled expected and unexpected changes in the concentration and distribution of molecules inside the cell. However, this work focused on cells without pharmacological and metabolic perturbations. This means that only cells marked as \hl{normal} (no perturbed cells) and \hl{DMSO}\footnote{Dimethyl sulfoxide, or DMSO, is an organic compound used to dissolve test compounds in in drug discovery and design \cite{cushnie2020bioprospecting}.} (control cells) were used.



\subsection{Artificial Neural Networks}
\label{sec:intro:Cellular_Expression}


\subsubsection{Convolutional Neural Networks}
\label{sec:intro:CNN}


\subsection{Interpretability Methods}
\label{sec:intro:interpretability_methods}


\subsubsection{Integrated Gradients}
\label{sec:intro:IG}


\subsubsection{VarGrad}
\label{sec:intro:VarGrad}


\subsection{RemOve And Retrain}
\label{sec:intro:ROAR}

%%% Local Variables:
%%% mode: latex
%%% TeX-master: "thesis"
%%% End:


%% ==============================
\chapter{Basics}
\label{ch:basics}
%% ==============================

In this chapter we will...
A small overview. Explain how CNN can be used to predict Transcription Rate and how we can use interpretability methods to learn from this models.

\section{Biology Background}
\label{sec:basics:bio_back}

\section{Artificial Neural Networks}
\label{sec:basics:Cellular_Expression}

\subsection{Convolutional Neural Networks}
\label{sec:basics:CNN}

\section{Interpretability Methods}
\label{sec:basics:interpretability_methods}
\glsresetall
% Motivation and problem
In recent years, \glspl{dnn} have been used to solve a wide variety of problems and gained popularity. Amazing results such as those achieved by Deep Mind's Alpha Fold team, have shown the great potential \gls{dnn} has to solve complex problems. However, the difficulty to interpret \glspl{dnn} has become one of the main obstacles to their acceptance in applications where the interpretability of the model is necessary.

% solution
To understand how the \glspl{dnn} predict the \gls{tr} of a cell, we use \textit{Attribution Methods}. This methods are meant to measure how much each component of the input image contributes to the model's prediction by creating a \textit{Score Map} (also known as \textit{Importance Map, Sensitivity Map} or \textit{Saliency Map}) of the same shape as the model's input. In particular, in this work we use a combination between \gls{ig} \cite{sundararajan2017axiomatic} and \gls{vg} \cite{adebayo2020sanity} as attribution method. In general we will denote attribution method as $\phi$.

% other advantages
Attribution methods are not only used to interpret black-box models like \gls{dnn}, the can also be used to debug models or as a sanity check to validate that the model base its prediction on the relevant features of the input.

% in our case
In our case, this interpretability techniques will show us which parts of the cell image are relevant for the prediction of the \gls{tr}. However, this will not just help us to interpret the results of the model, this also have the potential to help us understand unknown cellular processes.


\subsection{Integrated Gradients}
\label{sec:basics:IG}
%% Magic command to compile root document
% !TEX root = ../../thesis.tex

%% Reset glossary to show long gls names
\glsresetall

%% Set path to look for the images
\graphicspath{{./Sections/Basics/Resources/}}

\glsfirst{ig} is an interpretability technique (attribution method) proposed by Sundararajan et al. \cite{sundararajan2017axiomatic}, aimed to assign an importance to the input features (in our case pixels from a cell image) with respect to the model prediction. The attribution problem have been studied before in other papers \cite{JMLR:v11:baehrens10a}, \cite{SimonyanVZ13}, \cite{ShrikumarGSK16}, \cite{BinderMBMS16} and \cite{Springenberg}.

In our case, we seek to predict \gls{tr} given a cell image $x \in \mathbb{R}^{d \times d \times c}$, where $d$ is the height and width of the image and $c$ is the number of channels.
Therefore, our \gls{dnn} would be a function $f:\mathbb{R}^{d \times d \times c} \rightarrow \mathbb{R}$ and an attribution method should be a function $\phi:\mathbb{R}^{d \times d \times c} \rightarrow \mathbb{R}^{d \times d \times c}$ having an input and output of the same shape as the model's input image.

Early interpretability methods only use gradients to assign importance to each input feature

\begin{equation}
  \begin{split}
    \phi(f,x) &:= \nabla f(x) \\
    &= \frac{\partial f}{\partial x}
  \end{split}
\end{equation}

Mathematically speaking, $\phi_i(f,x)$ assign an importance score to the pixel $i$ (out of the $d \times d \times c$ there are), representing how much it adds or subtract from the model output.
However, this score maps have some drawback when they are used to interpret deep neural networks \cite{sturmfels2020visualizing}. Recall that the gradient with respect to the input indicate us the pixels that have the steepest local slope with respect to the model's output.
This means that it only describes local changes in the input, and not the whole prediction model. Another mayor problem is saturation\footnote{In the context of artificial neural networks, a neuron is said to be saturated when the predominant output value of a neuron is close to the asymptotic ends of the bounded activation function. This behavior can potentially damage the learning capacity of a neural network.}.
As the model learns the relationship between an input image and its \gls{tr}, the gradient of the most important pixels will approximate to 0, i.e. the pixel's gradient saturates.

To overcome this problems, Sundararajan et al. proposed \gls{ig} as an attribution method, where the importance of the input feature $i$ is defined as follow
\begin{equation}
  \phi^{IG}_i(f, x, x') := (x_{i} - x'_{i})\int_{\alpha=0}^1\frac{\partial f(x'+\alpha (x - x'))}{\partial x_i}{d\alpha}
  \label{eq:ig:definition}
\end{equation}

Intuitively speaking, \gls{ig} accumulates the input gradient when it goes from a baseline $x'$, which should represents \textit{absence} of information, to the actual input image $x$. With this, we avoid losing information about relevant pixels for the model's prediction in the importance map, even if they saturate eventually. Figure \ref{fig:basics:IG_image_prog} shows an example of the image progression fed into IG. Note that the amount of information in the images is parameterized by $\alpha \in [0,1]$, and that the \hl{absence} of information is interpreted as a black image.

\begin{figure}[!ht]
  \centering
  \includegraphics[width=\linewidth]{IG_alpha.png}
  \caption{Progression from an image with no information (back image) to a normal one parameterized by $\alpha$.}
  \label{fig:basics:IG_image_prog}
\end{figure}

For a better understanding, we can divide the \gls{ig} definition as follow
\begin{equation}
  \phi^{IG}_i(f, x, x') := \overbrace{(x_{i} - x'_{i})}^\text{Difference from baseline}
  \underbrace{\int_{\alpha=0}^1}_\text{From baseline to input...}
  \overbrace{\frac{\partial f(x'+\alpha (x - x'))}{\partial x_i}{d\alpha}}^\text{…accumulate local gradients}
  \label{eq:ig:explanation}
\end{equation}

The integral in equation \ref{eq:ig:explanation} accumulate the gradients for the interpolated images $x'+\alpha (x - x'))$ between the baseline $x'$ and the image $x$. On the other hand, the difference $(x_i - x_i')$ outside the integral comes from the chain rule and the fact that we are interested in integrating over the path between the baseline and the image.

%https://arxiv.org/pdf/1806.03000.pdf
\gls{ig} is very simple and easy to implement, since it does not require any modification to the model and it only require some calls to the gradient operator.

The \gls{ig} satisfy several properties and axioms that are addressed in detail in the paper. However, there is one axiom satisfied by \gls{ig} that is of special importance for us, \textit{completeness}. Completeness means that the value of the summed attributes will be equal to difference between the model's output when it is evaluated at the image and the model's output when it is evaluated at the baseline
\begin{equation}
  \sum_i \phi(f, x, x')^{IG} = f(x) - f(x')
  \label{eq:ig_completeness}
\end{equation}

In practice, computing the analytic expression for the integral in equation \ref{eq:ig:definition} would be complicated, and in some cases unfeasible.
However, luckily we can numerically approximate $\phi(f, x, x')^{IG}$ using a Riemann sum
\begin{equation}
  \phi^{Approx\ IG}_i(f, x, x', m) := (x_{i} - x'_{i})\sum_{k=1}^m\frac{\partial f(x'+\frac{k}{m} (x - x'))}{\partial x_i} \frac{1}{m}
  \label{eq:ig:approx}
\end{equation}

\noindent where $m$ is number of steps for the Riemann sum approximation.

This is when the completeness axiom comes into scene, which is a good value for the parameter $m$? 10, 100, 500? To answer this question, we can simply apply the completeness axiom as a sanity check for the election of $m$. If $m$ is good enough, then the value of $\sum_i \phi^{Approx\ IG}_i(f, x, x', m)$ should be close to $f(x)-f(x')$, or equivalently, the value of $|\sum_i \phi^{Approx\ IG}_i(f, x, x', m) - (f(x)-f(x'))|$ should be close to 0.

Figures \ref{fig:vg:img_gradients} and \ref{fig:vg:img_IG} show a comparison between the gradient of a model output with respect to a cell image, and the \gls{ig}. One can see that either for score maps computed using \gls{ig} or vanilla gradients, the output is noisy and diffuse.



\subsection{VarGrad}
\label{sec:basics:VarGrad}
%% Magic command to compile root document
% !TEX root = ../../thesis.tex

% define where the images are
\graphicspath{{./Sections/Basics/Resources/}}
\glsresetall

As we can see in figure \ref{fig:vg:img_IG}, \gls{ig} attribution maps can be noisy and diffuse. To improve their empirical quality, Smilkov et al. \cite{Smilkov_smoothgrad} proposed \gls{sg}, which tends to reduce noise in practice and can be combined with other attribution map algorithms (like \gls{ig}). The idea behind \gls{sg} is pretty simple, given an input image $x$, you create a sample of similar images by adding noise, then compute the attribution map for each one of them using the algorithm you prefer (in our case \gls{ig}), and take the average of the attribution maps.
Although Smilkov et al. do not provide a mathematical proof of why \gls{sg} reduce noise in score maps, they provide a conjecture and empirical evidence.
For this work we use a slightly different version called \gls{vg}, proposed by Adebayo et al. \cite{adebayo2018local} but inspired by \gls{sg}, which takes the variance of the attribution maps instead of the mean. The reason for this choice is that Seo et al. \cite{Seo_noise} analyzed theoretically \gls{vg}, and concluded that it is independent to the gradient and capture higher order partial derivatives.

In general, \gls{vg} is defined as follow

\begin{equation}
  \phi^{SG}(f, x) := Var(\phi(f, x + z_j))
\end{equation}

\noindent where $x \in \mathbb{R}^{d \times d \times c}$ is the input image, $f:\mathbb{R}^{d \times d \times c} \rightarrow \mathbb{R}$ a model, $\phi$ an attribution method to get preliminary score maps and $z_j \sim \mathcal{N}(0, \sigma^2)$, with $j\in\{1, \dots, n\}$, are i.i.d. noise images of same shape as the input image.

Since we use \gls{ig} to get preliminary score maps, in our case \gls{vg} (in the subsequent defined as \gls{vgig}) looks as follow

\begin{equation}
  \phi^{SG}(f, x) := Var(\phi^{IG}(f, x + z_j, x'))
\end{equation}

\noindent where $x' \in \mathbb{R}^{d \times d \times c}$ is a given baseline needed to compute the \gls{ig} score maps.

Figures \ref{fig:vg:img_IG} and \ref{fig:vg:img_VG_IG} show a comparison between \gls{ig} and \gls{vgig} score maps. One can see that \gls{vgig} produces less noisy score maps than vanilla \gls{ig}.

% this plots were created using the notebook ~/Documents/Master_Thesis/Project/workspace/Interpretability/Integrated_Gradient_Sanity_check.ipynb
\begin{figure}[htb]
  \centering
  \begin{subfigure}[b]{.45\linewidth}
    \includegraphics[width=\linewidth]{Cell_Image.jpg}
    \caption{Original cell image.}
    \label{fig:vg:cell_img}
  \end{subfigure}
  \begin{subfigure}[b]{.45\linewidth}
    \includegraphics[width=\linewidth]{Image_Gradient.jpg}
    \caption{Gradient wrt the input image.}
    \label{fig:vg:img_gradients}
  \end{subfigure}%
  \vspace{3mm}
  \begin{subfigure}[b]{.45\linewidth}
    \includegraphics[width=\linewidth]{Integrated_Gradient.jpg}
    \caption{Integrated Gradient.}
    \label{fig:vg:img_IG}
  \end{subfigure}
  \begin{subfigure}[b]{.45\linewidth}
    \includegraphics[width=\linewidth]{VarGrad_Integrated_Gradient.jpg}
    \caption{VarGrad with Integrated Gradients.}
    \label{fig:vg:img_VG_IG}
  \end{subfigure}
  \caption{Comparison between a cell image and the different attribution methods. All the figures show the same 3 channels taken from a cell image. \subref{fig:vg:cell_img}) cell image, i.e. no attribution method. \subref{fig:vg:img_gradients}) score map using only the gradient of the model with respect to the input image. \subref{fig:vg:img_IG}) \acrlong{ig} score map. \subref{fig:vg:img_VG_IG}) \acrlong{vgig} score map.}
  \label{fig:vg:comparative}
\end{figure}



\section{Interpretability Methods Evaluation}
\label{sec:basics:vgig_eval}

%%% Local Variables:
%%% mode: latex
%%% TeX-master: "thesis"
%%% End:


%% ==============================
\chapter{The Dataset}
\label{ch:dataset}
%% ==============================

A small overview. Explain roughly the pipline: Preprocessing -> TDFS -> CNN

\section{Multiplexed Protein Maps}
\label{sec:dataset:multiplexed_protein_maps}
%% Magic command to compile root document
% !TEX root = ../../thesis.tex

%% Reset glossary to show long gls names
\glsresetall

%% Set path to look for the images
\graphicspath{{./Sections/Dataset/Resources/}}

% A small motivation to create Multiplexed Protein Maps
The amount of protein or \gls{mrna} inside a cell may not be enough to fully describe cellular function. Accordingly to Buxbaum et al. \cite{Buxbaum_2014} and Korolchuk et al. \cite{Korolchuk2011}, cellular function can heavily depends on the specific intracellular location and interaction with other molecules and intracellular structures. Therefore, cellular expression is determined by the functional state, abundance, morphology, and turnover of its intracellular organelles and cytoskeletal structures. This means that having the ability to look at the concentration and distribution of different molecules within a cell, is an important technological achievement that can significantly leverage scientific discoveries in biomedicine.
This is exactly what \gls{mpm} allows us to do (\cite{Guteaar7042}). \gls{mpm} are protein readouts from cell cultures, that simultaneously captures different properties of the cell, like its shape, cycle state, detailed morphology of organelles, nuclear subcompartments, etc. It also captures highly multiplexed subcellular protein maps, which can be used to identify functionally relevant single-cell states, like \gls{tr}. These maps can also identify new cellular states and allow quantitative comparisons of intracellular organization between single cells in different cell cycle states, microenvironments, and drug treatments \cite{Guteaar7042}.

So, let us explain more in deept what are these \gls{mpm}. Accordingly to Gabriele Gut et al. \cite{Guteaar7042}, \gls{mpm} is a nondegrading protocol that allows to capture efficiently thousands of single cell multichannel images, where each channel contains the distribution and concentration of a protein of interest inside each cell. To achieve this, the protocol is made up of different steps that will be briefly explained here.

% 4i explanation
\subsubsection{Iterative indirect immunofluorescence imaging}
The \gls{mpm} protocol starts with a process called \gls{4i} developed by the same group. The \gls{4i} is a complete protocol by itself, and it allows to capture the concentration and distribution of individual proteins in thousands of different cells in a tissue\footnote{The tissues were made from cell cultures using the \hl{HeLa Kyoto} \hl{184A1} cell line. HeLa is the oldest and most commonly used immortal human cell line in scientific research. The story behind it is quite interesting, so it's worth checking out.}.
Before applying the \gls{4i} protocol, the \hl{plate} where the cell culture is must be divided into squared sections called \hl{wells}. Then, the \gls{4i} protocol is applied over each well and photographed in sections called \hl{sites}.

Roughly speaking, \gls{4i} works as follow
\begin{enumerate}
  % 1
  \item The selected well is prepared for the staining-elution process.
  %2
  \item The well is saturated with a liquid containing \hl{antibodies}\footnote{An antibody is a Y-shaped protein that can recognize and bind to a unique molecule (its antigen, e.g. another protein).} stained with a fluorescent ink (\gls{if}), which binds to a target protein.
  %3
  \item The well is exposed to a high-energy light and photographed using a light microscopy (which produces a single channel image).
  %4
  \item The antibodies inside the tissue are washed-out using a chemical elution substrate.
  %5
  \item Steps 2 to 4 are repeated 20 times to get 20 images of the same protein.
  %6
  \item To improve the protein readouts, the 20 single channel images are projected into one by \hl{maximum intensity projection}.
\end{enumerate}

Figure \ref{fig:4i:1} illustrates the steps of the \gls{4i} protocol that capture the saturation and distribution of a specific protein. Keep in mind that even though the \gls{4i} protocol captures sever images of the tissue, it returns an uni-channel image (step 6). Figure \ref{fig:4i:2} shows the \gls{4i} protocol applied 40 times with different \gls{if} and over a 384-well plate, which captures the concentration and distribution of 40 different specific proteins.

\begin{figure}[htb]
  \centering
  \begin{subfigure}[t]{.3\linewidth}
    \includegraphics[width=\linewidth]{4i_1.png}
    \caption{\Acrfull{4i} protocol.}
    \label{fig:4i:1}
  \end{subfigure}
  \hspace{4mm}
  \begin{subfigure}[t]{.45\linewidth}
    \includegraphics[width=\linewidth]{4i_2.png}
    \caption{\gls{4i} protocol applied over a specific well of a plate and for 40 different \gls{if}.}
    \label{fig:4i:2}
  \end{subfigure}%
  \caption{Schematic representation of the \gls{4i} protocol for a single well and for 40 different fluorescent antibodies. Figure \subref{fig:4i:2} also shows the image analysis to identify single cells and its components (nucleus and cytoplasm). Images source: \cite{Guteaar7042}.}
  \label{fig:4i}
\end{figure}

By the time \cite{Guteaar7042} was published, the \gls{4i} protocol was able to capture cell culture images with up to 40 channels without degrading the tissue, which is why \gls{mpm} is called a \textit{nondegrading} protocol.

\subsubsection{Multiplexed single cell analysis}

Once the multichannel images were generated using the \gls{4i} protocol, a series of image preprocessing and image analysis methods (\cite{Carpenter2006} and \cite{snijder2012single}) are applied to generate segmentation masks to identify individual cells, as well as their cytoplasm and nucleus. Figure \ref{fig:4i:2} shows this segmentation at a cellular level, while figure \ref{fig:4i:segmentation} shows it also at a subcellular level. In both cases the boundaries are marked with a white contour. This single cell analysis is also used to identify cells that do not satisfy certain quality controls (like cells in the border of the image or in mitosis stage). However, this will be addressed in detail on section \ref{sec:dataset:data_pp}.

\begin{figure}[htb]
  \centering
  \includegraphics[width=0.5\linewidth]{4i_segmentation.png}
  \caption{Visualization of the subcellular segmentation of a \gls{4i} protocol for 18 \gls{if} stains. The image was created by combining the readouts of 3 of this \gls{if} stains: PCNA (cyan), FBL (magenta) and TFRC (yellow). The number next to each staining label indicates their corresponding 4i acquisition cycle (\gls{4i} protocol step 5). The orange rectangle and the tile at its right shows a section of the nucleus and cytoplasm of a single cell. The other 3 tiles shows the \gls{4i} readout of each of the 3 proteins. Images source: \cite{Guteaar7042}.}
  \label{fig:4i:segmentation}
\end{figure}

\subsubsection{Multiplexed single-pixel analysis framework}
Even though the cell cultures are now segmented into individual cells and nucleus, there is still one missing part that must be considered, and that is that cells are 3-dimensional objects. Recall that the \gls{4i} protocol saturates the cell culture with a liquid containing fluorescent antibodies. This means that the antibody can either bind to its corresponding protein inside or outside the cell nucleus. Therefore, even though that we segmented a cell into nucleus and cytoplasm, a readout assigned to the nucleus could come from a protein in the cytoplasm under or above the nucleus, and not from inside it. Fortunately, intensity readouts from proteins inside the nucleus are much higher than those in the cytoplasm. Therefore, by means of a two steps clustering approach\footnote{To identify clusters in an unsupervised manner, \hl{Self Organizing Maps} algorithm and \hl{Phenograph} analysis were used over a very large number of pixels sampled from a large number of single cells \cite{Guteaar7042}.}, pixels can be classified accordingly to their intensity profile (figures \ref{fig:mcu:1} and \ref{fig:mcu:2}), so the source of their readout can be identified. This pixel type classification is called \Acrfull{mcu} and is illustrated in figure \ref{fig:mcu:3}. After pixels clusters (intensity profiles) where identified, the pixels whose measurement comes from the cytoplasm and not from the nucleus are removed.

\begin{figure}[htb]
  \centering
  \begin{subfigure}[t]{.3\linewidth}
    \includegraphics[width=\linewidth]{mcu_1.png}
    \caption{Extraction of pixel intensities.}
    \label{fig:mcu:1}
  \end{subfigure}
  \hspace{4mm}
  \begin{subfigure}[t]{.3\linewidth}
    \includegraphics[width=\linewidth]{mcu_2.png}
    \caption{Pixel clustering by Self Organizing Maps and Phenograph.}
    \label{fig:mcu:2}
  \end{subfigure}
  \hspace{4mm}
  \begin{subfigure}[t]{.3\linewidth}
    \includegraphics[width=\linewidth]{mcu_3.png}
    \caption{Cell subdivision base on the \gls{mcu}.}
    \label{fig:mcu:3}
  \end{subfigure}
  \caption{Figure \subref{fig:mcu:1} shows the pixel intensity extraction for a single cell. The pixel intensity is a vector containing the readout of that 2D location for each protein, one specific protein readout per entrance. Figure \subref{fig:mcu:2} shows the clusters found by Self Organizing Maps algorithm and Phenograph analysis over the pixel intensities. Figure \subref{fig:mcu:3} shows a cell masked with the clusters found by the \gls{mcu} analysis. Images source: \cite{Guteaar7042}.}
  \label{fig:mcu}
\end{figure}

Finally, the nucleus of each cell is stored separately and identified with a unique id.\fxnote{After you finish writing the dataset section review if this sentence is accurate.}

\subsubsection{Cell cycle phase classification: $G_1,\ S,\ G_2$ and $M$ phase}

The \gls{mpm} protocol is not only capable to capture the concentration and distribution of molecules inside thousands of cells. It can also identify the phase each cell is in, which is tightly related with the abundances and distribution of molecules inside a cell \cite{Guteaar7042}.

Roughly speaking, cell cycle phase was determined by means of a \gls{svm} classifier and k-means clustering. First, a \gls{svm} classifier is trained to identify $M$ phase cells based on the nuclear information in one of the image channels (\hl{DAPI}\footnote{A brief description of this marker can be found on section \ref{sec:appendix:if_markers}.}). Then, based on the nuclear information of channel \hl{PCNA}, a second \gls{svm} classifier is trained to identify cells in phase $S$. Finally, cells in phase $G_1$ and $G_2$ are classified using a k-means algorithm, using the pixel intensity profiles of the DAPI channels excluding the cells in $S$ and $M$ phase. A more detailed explanation of the cell cycle classification process can be found on the dataset paper \cite{Guteaar7042}.

\subsubsection{Pharmacological and metabolic perturbations}

To further explore the capabilities of the \gls{mpm} protocol, the creators of the dataset (Gabriele Gut et al. \cite{Guteaar7042}) applied the \gls{mpm} protocol to a cell populations that were to nine pharmacological and metabolic perturbations. The analysis reveled expected and unexpected changes in the concentration and distribution of molecules inside the cell. However, this work focused on cells without pharmacological and metabolic perturbations. This means that only cells marked as \hl{normal} (no perturbed cells) and \hl{DMSO}\footnote{Dimethyl sulfoxide, or DMSO, is an organic compound used to dissolve test compounds in in drug discovery and design \cite{cushnie2020bioprospecting}.} (control cells) were used.


\section{Data preprocessing}
\label{sec:dataset:data_pp}
%% Magic command to compile root document
% !TEX root = ../../thesis.tex

\glsresetall
% define where the images are
\graphicspath{{./Sections/Dataset/Resources/}}

\noindent The data preprocessing consist of 4 main steps

\begin{enumerate}
  \item The raw data processing, where raw files are converted into images.
  \item The quality control, where cells that are not useful for analysis are discarded.
  \item The creation of the dataset, where data is spitted into \hl{Train, validation} and \hl{Test} sets and stored in a way that can be used for model training efficiently.
  \item The image preprocessing, where the images are prepared before training the model (clipping and standardization).
\end{enumerate}

In this section we explain these 4 steps. However, the implementation is discussed in the sections \ref{sec:appendix:raw_data} (for steps 1 and 2) and \ref{sec:appendix:tfds} (for steps 3 and 4).

\subsection{Raw data processing}
\label{sec:dataset:data_pp:raw_data_p}

As we mentioned in section \ref{sec:dataset:multiplexed_protein_maps}, the \gls{mpm} protocol is applied over section of cell cultures called \hl{wells}. The \gls{mpm} protocol will return several files for each well, containing the nuclear protein readouts of single cells, information from the subsequent analysis made to the intensities of the protein readouts, as well as information about the \gls{mpm} protocol experimental setup. We do not go into details about this files and how to transform them into multichannel images of single cell nucleus. However, a brief explanation of this can be found in the appendix \ref{sec:appendix:raw_data}. Appendix \ref{sec:appendix:raw_data} also show how to run the Python script that transforms the raw data into images, along with an explanation of the required parameters.

The Python script introduced on appendix \ref{sec:appendix:raw_data} extract the protein readouts from the raw data files, and use them to build multichannel images containing the nucleus of a single cell (see figure \ref{fig:data_pp:sample_cell:nucleus}). This means that during the reconstruction of the images, it is necessary to add black pixels (zeros) in the places where no measures were taken (like in the low corner of figure \ref{fig:data_pp:sample_cell:nucleus}). However, as we saw on section \ref{sec:basics:CNN}, in order to train a \gls{cnn} model, all the cell images need to have a fixed size, which is denoted as $I_s$. For this reason, after the image is reconstructed, it is necessary to add zeros to the images borders (zero-padding) in order to make it squared and of a fixed size (see figure \ref{fig:data_pp:sample_cell:nucleus_pad}). Finally, for each single cell nucleus, a \hl{cell mask} is created to keep track of the measured and non-measured pixels (see figure \ref{fig:data_pp:sample_cell:cell_mask}). As we can see in figure \ref{fig:data_pp:sample_cell}, the cell nucleus is always located in the center of the image.

\begin{figure}[htb]
  \centering
  \begin{subfigure}[t]{.211\linewidth}
    \includegraphics[width=\linewidth]{cell_nucleus.jpg}
    \caption{Single cell nucleus.}
    \label{fig:data_pp:sample_cell:nucleus}
  \end{subfigure}
  \hspace{4mm}
  \begin{subfigure}[t]{.3\linewidth}
    \includegraphics[width=\linewidth]{cell_nucleus_w_pad.jpg}
    \caption{Single cell nucleus with zero-padding.}
    \label{fig:data_pp:sample_cell:nucleus_pad}
  \end{subfigure}
  \hspace{4mm}
  \begin{subfigure}[t]{.3\linewidth}
    \includegraphics[width=\linewidth]{cell_mask.jpg}
    \caption{Single cell nucleus mask.}
    \label{fig:data_pp:sample_cell:cell_mask}
  \end{subfigure}
  \caption{Figure \subref{fig:data_pp:sample_cell:nucleus} shows channels 10, 11 and 15 of the nucleus of a single cell multichannel image reconstructed form the raw data. Figure \subref{fig:data_pp:sample_cell:nucleus_pad} shows image \subref{fig:data_pp:sample_cell:nucleus} after adding zero to the borders (zero-padding) to make it of size 224 by 224 pixels. Figure \subref{fig:data_pp:sample_cell:cell_mask} shows the cell mask, i.e. measured pixels (in white) during the \gls{mpm} protocol.}
  \label{fig:data_pp:sample_cell}
\end{figure}

The raw data processing script saves in a specified directory files containing 3 compressed NumPy arrays; 1) the multichannel image (figure \ref{fig:data_pp:sample_cell:nucleus_pad}), a 3D array contains the protein readouts of the nucleus of a single cell 2) the cell mask (figure \ref{fig:data_pp:sample_cell:cell_mask}), a 2D array that indicates the measured pixels by the \gls{mpm} protocol (ones on the measured $x$ and $y$ coordinates and zeros otherwise) and 3) the channels average, a 1D array containing the average of the measured pixels per channel/protein. Each file is named using the unique id assigned to each single cell nucleus (\texttt{mapobject\_id\_cell}). The script also returns a \texttt{csv} file\footnote{This \texttt{csv} file can be easily opened as a \hl{Pandas DataFrame}. For more information, please refer to the \href{https://pandas.pydata.org/pandas-docs/stable/reference/api/pandas.DataFrame.html}{official documentation}.} containing the metadata of each single cell from every processed well (one row per cell and one column per cell feature). Table \ref{table:dataset:metadata} shows the metadata columns that were relevant for this work.

% set table lengths
\setlength{\mylinewidth}{\linewidth-7pt}%
\setlength{\mylengtha}{0.3\mylinewidth-2\arraycolsep}%
\setlength{\mylengthb}{0.7\mylinewidth-2\arraycolsep}%

\begin{table}[!ht]
  \centering
  \begin{tabular}{>{\centering\arraybackslash}m{\mylengtha}|m{\mylengthb}} % m stands for middle (p:top, b:bottom), max 144 mm
    \hline
    Column name & Description \\
    \hline
    \texttt{mapobject\_id\_cell} & ID to uniquely identify each cell among all wells \\
    \hline
    \texttt{mapobject\_id} & ID to uniquely identify each cell on its well \\
    \hline
    \texttt{is\_border\_cell} & Binary flag, 1 if the cell is on the plate, well or site border; 0 if not \\
    \hline
    \texttt{cell\_cycle} & String, \texttt{G1} if cell is in $G_1$ phase, \texttt{S} if cell is in synthesis phase, \texttt{G2} if cell is in $G_2$ phase. If \texttt{NaN}, then the cell is in mitosis phase \\
    \hline
    \texttt{is\_polynuclei\_184A1} & Binary flag for \hl{184A1} cells, 1 if the cell was identified to have more than one nucleus (i.e. it is in mitosis phase); 0 if not\\
    \hline
    \texttt{is\_polynuclei\_HeLa} & Binary flag for \hl{HeLa} cells, 1 if the cell was identified to have more than one nucleus (i.e. it is in mitosis phase); 0 if not\\
    \hline
    \texttt{perturbation} & String indicating the pharmacological/metabolic perturbation \\
    \hline
  \end{tabular}
  \caption{Relevant metadata columns.}
  \label{table:dataset:metadata}
\end{table}

\subsection{Quality Control}
\label{sec:dataset:data_pp:qc}

During the transformation from raw data into images, cells that does not pass a quality control are discriminated. This quality control consist on avoiding cells that holds at least one of the following conditions
\begin{enumerate}
  \item The cell is in mitotic phase (i.e. on metadata, either \texttt{is\_polynuclei\_HeLa} or \texttt{is\_polynuclei\_184A1} is equal to 1 or \texttt{cell\_cycle} is \texttt{NaN}).
  \item The cell is in the border of the plate, well or site (i.e. on metadata, \texttt{is\_border\_cell} is equal to 1).
\end{enumerate}

The quality control is performed by the same script that transforms the raw data into multichannel images. Its implementation and execution, as well as an explanation of the required parameters, can be found on appendix \ref{sec:appendix:raw_data}.

\subsection{Dataset creation}
\label{sec:dataset:data_pp:dataset_creation}

After the raw data from all wells were processed, and mitotic and/or border cells were eliminated (quality control), we are able to build a dataset\footnote{For this work we decided to use (and build) a custom \acrfull{tfds}, which is a subclass of \texttt{tensorflow\_datasets.core.DatasetBuilder} and allows to create a pipeline that can easily feed data into a machine learning model built using TensorFlow. For more information, please refer to the \href{https://www.tensorflow.org/datasets/add_dataset}{official documentation}.} that can be used efficiently to train models. We will not explain here how to create this dataset. However, a brief explanation of this can be found in the appendix \ref{sec:appendix:tfds}. Appendix \ref{sec:appendix:tfds} also show how to run the Python script that builds this dataset, along with an explanation of the required parameters.

Even though this script can bu used to build a dataset containing all available single cell images, for this work we created a dataset containing cells without pharmacological or metabolic perturbations (i.e. cells such that in the metadata \texttt{perturbation} is ether equal to \hl{normal} or \hl{DMSO}). Further more, during the creation of the dataset, it is possible to filter the image channels and select the target value from the channels average vector (which is stored along with each single cell image). In this case we kept all the input channels\footnote{The unnecessary/unwanted channels are removed during the model training/evaluation (see section \ref{sec:methodology:models}). The reason why this filtering is not made during the dataset creation, is to make the dataset set more robust (i.e. to avoid the need to create a new dataset each time the input channels of the image changed).}, except for the channel used to calculate the target value. This means that channel 35 was excluded (\texttt{00\_EU}\footnote{A brief description of this marker can be found on section \ref{sec:appendix:if_markers}.}), and entrance 35 from the channel average vector (interpreted as \gls{tr}) was selected as target value.

Last but not least, for each cell, its mask is added at the end as an extra channel to keep track of the measured pixels. The reason why the cell mask is stored as a channel, is because it will be needed by other process latter in the pipeline (some of the data augmentation techniques, see section \ref{sec:dataset:data_augmentation}). However, this (and other channels) are removed before the image is used to feed the model (during and after the training process, see section \ref{sec:methodology:models}).

Table \ref{table:tfds_in:channels} (on appendix \ref{sec:appendix:tfds}) shows the image channels in the \gls{tfds}, including the name (column \hl{Channel name}) and identifier of each immunofluorescence markers (column \hl{Marker identifier}). Table \ref{table:tfds_in:channels} also shows the ids corresponding to the markers in the raw data (column \hl{Raw data id}) and in the \gls{tfds} (column \hl{TFDS id}). \hl{NA} means that the channel is not used/available either on the raw data or the \gls{tfds}.

% Data extracted form notebook Preprocessing_resources.ipynb
\begin{table}[!ht]
  \centering
  \begin{tabular}{c|c|c}
    Set & Size & Percentage \\
    \ChangeRT{1.7pt}
    Train & 2962 & $80\%$ \\
    \hline
    Validation & 371 & $10\%$ \\
    \hline
    Test & 370 & $10\%$ \\
    \ChangeRT{1.7pt}
    Total & 3703 & $100\%$ \\
  \end{tabular}
  \caption{Distribution of the dataset partitions.}
  \label{table:data_pp:dataset_dist}
\end{table}

During the creation of the dataset, the images are also spitted into 3 sets, \hl{Train, Validation} and \hl{Test}, using the proportions $80\%$, $10\%$ and $10\%$ respectively. Table \ref{table:data_pp:dataset_dist} shows the size of these 3 sets.

% Data extracted form notebook Preprocessing_resources.ipynb
\begin{table}[!ht]
  \centering
  \begin{tabular}{c|c|c|c}
    Set & Cell Cycle & Size & Percentage \\
    \ChangeRT{1.7pt}
    \multirow{3}{*}{Train} & $G_1$ & 1652 & $55.77\%$ \\
    \cline{2-4}
    & $S$ & 864 & $29.17\%$ \\
    \cline{2-4}
    & $G_2$ & 446 & $15.06\%$ \\
    \hline
    \multirow{3}{*}{Validation} & $G_1$ & 205 & $55.41\%$ \\
    \cline{2-4}
    & $S$ & 103 & $27.84\%$ \\
    \cline{2-4}
    & $G_2$ & 62 & $16.76\%$ \\
    \hline
    \multirow{3}{*}{Test} & $G_1$ & 213 & $57.41\%$ \\
    \cline{2-4}
    & $S$ & 103 & $27.76\%$ \\
    \cline{2-4}
    & $G_2$ & 55 & $14.82\%$ \\
    \ChangeRT{1.7pt}
    \multirow{3}{*}{Total} & $G_1$ & 2070 & $55.90\%$ \\
    \cline{2-4}
    & $S$ & 1070 & $28.90\%$ \\
    \cline{2-4}
    & $G_2$ & 563 & $15.20\%$ \\
  \end{tabular}
  \caption{Distribution of the dataset partitions by cell phase (cell cycle).}
  \label{table:data_pp:dataset_dist_cc}
\end{table}

% Data extracted form notebook Preprocessing_resources.ipynb
\begin{table}[!ht]
  \centering
  \begin{tabular}{c|c|c|c}
    Set & Perturbation & Size & Percentage \\
    \ChangeRT{1.7pt}
    \multirow{2}{*}{Train} & Normal & 2040 & $68.87\%$ \\
    \cline{2-4}
    & DMSO & 922 & $31.13\%$ \\
    \hline
    \multirow{2}{*}{Validation} & Normal & 257 & $69.46\%$ \\
    \cline{2-4}
    & DMSO & 113 & $30.54\%$ \\
    \hline
    \multirow{2}{*}{Test} & Normal & 260 & $70.08\%$ \\
    \cline{2-4}
    & DMSO & 111 & $29.92\%$ \\
    \ChangeRT{1.7pt}
    \multirow{2}{*}{Total} & Normal & 2557 & $69.05\%$ \\
    \cline{2-4}
    & DMSO & 1146 & $30.95\%$ \\
  \end{tabular}
  \caption{Distribution of the dataset partitions by perturbation.}
  \label{table:data_pp:dataset_dist_per}
\end{table}

Since we are dealing with cells in different phases (cell cycles), it is important that the distribution of the 3 phases is kept  in the train, validation and test sets\fxnote{Should I mention than half of the cells are in $G_1$ phase, which means that cell in $G_1$ phase apply more pressure on the optimization of the model parameters during training, while the cells in the $G_2$ phase will not? And/Or should I mention this in the conclusions as a future work?}. The same must happen with the proportion of cells without pharmacological/metabolic perturbation (\hl{Normal} cells) and control cells (\hl{DMSO} cells). Tables \ref{table:data_pp:dataset_dist_cc} and \ref{table:data_pp:dataset_dist_per} show respectively that these proportions are hold across the 3 sets.

\subsection{Image preprocessing}

In this work we use \glspl{cnn} and images of cell nucleus to predict \gls{tr}. This means that there are two main features of the images that came into account when the model learns and predicts the \gls{tr}, the spatial distribution of the elements in the image and the intensity of the colors.
However, this work aims to explain and predict transcription based on the information encoded in the spatial distribution of proteins and organelles within the nucleus. Therefore, the image preprocessing techniques applied here should help mitigate the influence of color during training and prediction, so that the model can focus only on spatial information. For this reason, two preprocessing techniques are applied to each cell image, clipping and standardization. The clipping, as well as the standardization, are performed during the construction of the \gls{tfds}, which can be consulted in appendix \ref{sec:appendix:tfds}.

\subsubsection{Clipping}

The idea of clipping is to avoid extreme outliers to influence or leverage the model parameters during training. Figure \ref{fig:data_pp:outlier} gives an example of this. The blue line shows a model fitted including the outliers (the two dots on the right upper corner), while the orange line a model fitted without them.

% Figure created with notebook Preprocessing_resources.ipynb
\begin{figure}[htb]
  \centering
  \includegraphics[width=0.5\linewidth]{outlier.jpg}
  \caption{Comparison between two linear regression models, fitted with (blue line) and without (orange line) outliers.}
  \label{fig:data_pp:outlier}
\end{figure}

To prevent high intense pixels to influence the model, we truncate/limit the value of pixels that are above a certain threshold. This threshold is different for each image channel and is determined using the cell images belonging to the training set. For each channel, the train images are loaded and the threshold is set as the $98\%$ percentile of the measured pixel intensities belonging to the channel. Then, using this threshold vector (one entrance per channel) all the images in the dataset (train, validation and test) are clipped. This is done before the data standardization. Finally, the clipping parameter (threshold) of each channel is stored in a metadata file, provided along with the \gls{tfds}. Figures \ref{fig:data_pp:pixel_dist:ori} and \ref{fig:data_pp:pixel_dist:clip} show the pixel intensity distribution of channel HDAC3 before and after clipping respectively.

\subsubsection{Standardization}

As we mentioned at the beginning of this section, to predict cell \gls{tr} we seek the model to rely on spatial information, rather than the intensity of the pixels. Therefore, to reduce pixel intensity influence, we apply per-channel standardization, which is just a shift and rescaling (a linear transformation) of the original data. Standardization is also called \hl{Z-score}, since the data is transformed using the mean $\mu$ and standard deviation $\sigma$ (normal distribution parameters) of a sample, as a shift and rescaling parameters respectively. As it is done in clipping, the standardization parameters are different for each channel and are computed using the images belonging to the training set. For all the measured pixels intensities in the \gls{tfds} (i.e. for train, validation and test sets), the standardization of pixel $i$ belonging to channel $c$ (i.e. $z_{i,c}$), is done as follow

\begin{equation}
  z_{i,c} = \frac{x_{i,c} - \mu_c}{\sigma_c}
  \label{eq:data_pp:z-score}
\end{equation}

\noindent where $x_{i,c}$ is the corresponding readout $i$ from channel $c$, and $\mu_c$, $\sigma_c$ are the mean and standard deviation (respectively) of channel $c$ computed using the training images.

The standardization centers the measured pixels of each channel around 0 (see figures \ref{fig:data_pp:pixel_dist:clip} and \ref{fig:data_pp:pixel_dist:clip_z}), reducing the color correlation between channels, which also reduce pixel intensity influence over the model.

% Plots made using notebook Data_PP_clipping_and_standard.ipynb
\begin{figure}[htb]
  \centering
  \begin{subfigure}[t]{.29\linewidth}
    \includegraphics[width=\linewidth]{Pixel_dist.png}
    \caption{Pixel intensity distribution.}
    \label{fig:data_pp:pixel_dist:ori}
  \end{subfigure}
  \hspace{4mm}
  \begin{subfigure}[t]{.3\linewidth}
    \includegraphics[width=\linewidth]{Pixel_dist_clip.png}
    \caption{Pixel intensity distribution after clipping.}
    \label{fig:data_pp:pixel_dist:clip}
  \end{subfigure}
  \hspace{4mm}
  \begin{subfigure}[t]{.28\linewidth}
    \includegraphics[width=\linewidth]{Pixel_dist_clip_z.png}
    \caption{Pixel intensity distribution after clipping and standardization.}
    \label{fig:data_pp:pixel_dist:clip_z}
  \end{subfigure}
  \caption{Intensity distribution of measured pixels for channel HDAC3. The channel readouts were taken from the training set. Figure \subref{fig:data_pp:pixel_dist:ori}) shows the distribution without any modification. Figure \subref{fig:data_pp:pixel_dist:clip}) shows the distribution after applying $98\%$ percentile clipping, while figure \subref{fig:data_pp:pixel_dist:clip_z}) shows the distribution after applying same clipping and standardization.}
  \label{fig:data_pp:pixel_dist}
\end{figure}

Figure \ref{fig:data_pp:cell_sample} shows 3 different cell nucleus sampled from the resulting \gls{tfds}. Each nucleus is in a different cell phase ($G_1$, $S$ and $G_2$ respectively), and shows a different group of 3 markers (channels).

% Figure created with notebook Preprocessing_resources.ipynb
\begin{figure}[htb]
  \centering
  \includegraphics[width=\linewidth]{ds_sample.jpg}
  \caption{Cell nucleus in phases $G_1$, $S$ and $G_2$ respectively. Each nucleus shows a different group of 3 markers.}
  \label{fig:data_pp:cell_sample}
\end{figure}


\section{TensorFlow Dataset}
\label{sec:dataset:tfds}

\section{Data Augmentation}
\label{sec:dataset:data_augmentation}

%%% Local Variables:
%%% mode: latex
%%% TeX-master: "thesis"
%%% End:


%% ==============================
\chapter{Methodology}
\label{ch:methodology}
%% ==============================

In this chapter we will describe the models used to predict transcription rate, as well as their experimental setups. We will also provide the experimental setups for the interpretability methods\fxnote{Consider adding here also the justification for the number the steps $m$ in the Riemann sum.}.

\section{Models}
\label{sec:methodology:models}

\subsection{Linear Model}
\label{sec:methodology:lm}

\subsection{Baseline CNN}
\label{sec:methodology:BL_CNN}

\subsection{ResNet50V2}
\label{sec:methodology:RN50V2}

\subsection{Model Metrics}
\label{sec:methodology:metrics}

\section{Interpretability Methods}
\label{sec:methodology:interpretability_methods}

%\subsection{Experimental Setup}
%\label{sec:methodology:VarGrad_IG_Experimental_Setup}
\glsresetall
\graphicspath{{./Sections/Methodology/Resources/}}

There are several hyper-parameters that need to be chosen in order to compute the score map for each cell image.

For the \gls{ig} attribution map, recall that in practice computing $\phi^{IG}$ could be unfeasible or computationally very expensive. However, we can approximate $\phi^{IG}$ by means of $\phi^{Approx\ IG}$ (see equation \ref{eq:ig:approx}). Therefore, we need to define the number of steps $m$ for the Riemann sum approximation. In section \ref{sec:basics:IG} we also mentioned the necessity to set a baseline image $x'$, which should contain no information about the image, in order to compute the \gls{ig}. There are several options that can be used, each one of them with different advantages and disadvantages. However, for this work we only implemented two of them: 1) a simple black image (image containing only zeros) and 2) an image filled with Gaussian noise ($\mu=0,\ \sigma=1$). A very good analysis on the choice of the baseline can be found in this reference \cite{sturmfels2020visualizing}.

In section \ref{sec:basics:VarGrad} we saw that for \gls{vg} we need to define 2 parameters, the number of noisy images $n$ (sample size) and the standard deviation $\sigma$ for the the noise distribution.

As a rule of thumbs, a sample should not be smaller than 30, so this could be a feasible option. However, since Smilkov et al. \cite{Smilkov_smoothgrad} showed empirically that no further improvemnt (less noise) in score maps is observed for sample sizes greater than 50, we chose this bound as sample size.

Table \ref{table:VGIG_exp_set:params} shows a summary of the parameters chosen to calculate the \gls{vgig} score maps.

\begin{table}[!ht]
  \centering
  \begin{tabular}{c|c|c}
    Method & Hyperparameter & Value \\
    \hline
    \multirow{2}{*}{\gls{ig}} & $m$ & 70 \\
    \cline{2-3}
     & $x'$ & black image \\
    \hline
    \multirow{2}{*}{\gls{vg}} & $n$ & 50 \\
    \cline{2-3}
     & $\sigma$ & 1 \\
    \hline
  \end{tabular}
  \caption{Parameters to compute score maps.}
  \label{table:VGIG_exp_set:params}
\end{table}

In section \ref{sec:basics:IG}, we mentioned that the \gls{ig} algorithm holds the \textit{Completeness Axiom}, which means that the sum of all the components of the \gls{ig} attribution map must be equal to the difference between the model's output evaluated at the image and the model's output evaluated at the baseline (see equation \ref{eq:ig_completeness}). This property allow us to check empirically if the number of steps $m$ selected for the Riemann sum approximation is sufficiently large. Figure \ref{fig:VGIG_exp_set:m_sanity} shows that for our baseline\fxnote{after finishing, check that the baseline model is still called baseline} model, a random image and $m=70$, the  completeness axiom is satisfied sufficiently well.

\begin{figure}[!ht]
  \centering
  \includegraphics[width=0.8\linewidth]{sanity_check_for_m.jpg}
  \caption{Sanity check for the number of steps $m$ in the Riemann sum to approximate $\phi^{IG}$. The red dotted line represent the difference $f(x)-f(x')$. The blue line represents the value of $\sum_i \phi^{Approx\ IG}_i(f, x, x', m)$ over $\alpha$.}
  \label{fig:VGIG_exp_set:m_sanity}
\end{figure}


\section{Interpretability Methods Evaluation}
\label{sec:methodology:vgig_eval}

%%% Local Variables:
%%% mode: latex
%%% TeX-master: "thesis"
%%% End:


%% ==============================
\chapter{Results}
\label{ch:results}
%% ==============================

A small overview. Explain roughly the pipline: Preprocessing -> TDFS -> CNN -> VatGrad IG -> ROAR

\section{Metrics}
\label{sec:results:metrics}

\subsection{Baseline values}
\label{sec:results:bl_values}

\section{Linear Model}
\label{sec:results:lm}

\section{Baseline CNN}
\label{sec:results:bl_cnn}

\section{ResNet50V2}
\label{sec:results:RN50V2}

\section{VarGrad Integrated Gradients}
\label{sec:results:VarGrad_IG}

\section{RemOve And Retrain}
\label{sec:results:ROAR}

%%% Local Variables:
%%% mode: latex
%%% TeX-master: "thesis"
%%% End:


%% conclusion.tex
%%

%% ==================
\chapter{Conclusion}
\label{ch:Conclusion}
%% ==================

Nothing new here, only a short recap of the project, it's results, as well as possible future work.

%Future work
As future work, it would be interesting to explore other interpretability  methods or go more in-depth with the current one by, for instance, exploring different baseline for \acrlong{ig} and analyzing their impact on the importance maps.

%%% Local Variables:
%%% mode: latex
%%% TeX-master: "thesis"
%%% End:


% appendices (if appropriate)
\appendix
%% appendix.tex
%%

%% ==============================
%\chapter{Appendix}
%\label{ch:Appendix}
\chapter{Remarks on Implementation}
\label{Appendix-Implementation}
%% ==============================

This appendix contains notes about how to execute all the scripts and notebooks used in this work. It also contains information about the parameters that need to be specified for each program.
All the scripts and notebooks were written in \texttt{Python} and executed over \texttt{Anaconda}. You can find information about the environment setup, packages version, etc. \href{https://github.com/andresbecker/master_thesis}{here}.

The logic to execute any \texttt{Python} script is always the same
\begin{lstlisting}[language=Bash]
python python_script_name.py -p ./Parameters_file_name.json
\end{lstlisting}

For the \texttt{Jupyter Notebooks} you just have to open it and set the variable \texttt{PARAMETERS\_FILE} with the absolute path and name of the input parameters file
\begin{lstlisting}[language=Python]
PARAMETERS_FILE = "/path_to_file_dir/Parameters_file_name.json"
\end{lstlisting}

For each script/notebook all the needed parameters have to be specified inside its parameter file only. The format for the parameters file is always \texttt{JSON} and the parameters values are specified in a python-dictionary format.

\fxnote{If there is time, add a section containing common parameters}

\section{VarGrad IG Implementation Notes}
\label{sec:appendix:VarGrad_IG_Experimental_Setup}

In order to generate the \acrlong{vg} \acrlong{ig} score maps, you must execute the python script \texttt{get\_VarGradIG\_from\_TFDS.py} specifying the parameters file
\begin{lstlisting}[language=Bash]
python get_VarGradIG_from_TFDS.py -p ./Parameters_file_name.json
\end{lstlisting}

Table \ref{table:imp_notes:VGIG_params} show all the parameters that need to be specified to execute \texttt{get\_VarGradIG\_from\_TFDS.py} successfully.

\begin{table}[!ht]
  \centering
  \begin{tabular}{c|c|c}
    Hyperparam & JSON variable name & Notes \\
    \hline
    $m$ & \texttt{IG\_m\_steps} & Number of steps to approximate \gls{ig} \\
    \hline
     &  & Baseline image for \gls{ig}. Available: "black" \\
    $x'$ &  \texttt{IG\_baseline} & for a simple black image and "noise" for an image \\
     &  &  filled with Gaussian noise ($\mu=0,\ \sigma=1$) \\
    \hline
    $n$ & \texttt{VarGrad\_n\_samples} & Number of noisy images to compute \gls{vg} \\
    \hline
  \end{tabular}
  \caption{Parameters to compute score maps.}
  \label{table:imp_notes:VGIG_params}
\end{table}



%%% Local Variables:
%%% mode: latex
%%% TeX-master: "thesis"
%%% End:

% all appendices behind backmatter will go without numbers
\backmatter

%List of Figures
\listoffigures

\vspace*{1.5cm}

%List of Tables
\listoftables


%List of algorithms
% only in conjunction with algorithm2e
%\phantomsection% for hyperref
%\addcontentsline{toc}{chapter}{Algorithmenverzeichnis}%
%\markboth{Algorithmenverzeichnis}{Algorithmenverzeichnis}
%\listofalgorithms


%Algorithmenverzeichnis
% nur in Verbindung mit algorithm2e
%\phantomsection% fuer hyperref
%\addcontentsline{toc}{chapter}{Algorithmenverzeichnis}%
%\markboth{Algorithmenverzeichnis}{Algorithmenverzeichnis}
%\listofalgorithms

% Index
% Add index to table of contents
\addcontentsline{toc}{chapter}{Index}
\printindex

%Print the glossary
%\printglossaries
\printglossary[type=\acronymtype]

% Bibliography
\printbibliography[heading=bibintoc]

%In the final version, this should be empty and needn't be commented out. Someone who works sloppily should at least remember to comment out the fixme list, so that unfinished places are
%not so obvious for the examiners.
\listoffixmes
%
\end{document}
% ==============================================================================
% End of document
% ==============================================================================
