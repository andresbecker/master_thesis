% ==============================================================================
% thesis.tex
% Example file for tumthesis.csl
% Michael Ritter, 2012
% Licence:
% This work may be distributed and/or modified under the
% conditions of the LaTeX Project Public License, either version 1.3
% of this license or (at your option) any later version.
% The latest version of this license is in
% http://www.latex-project.org/lppl.txt
% and version 1.3 or later is part of all distributions of LaTeX
% version 2005/12/01 or later.
% ==============================================================================
\documentclass[biblatexBackend=bibtex]{tumthesis}

% ------------------------------------------------------------------------------
%FixMe-Status: final (no FixMe comments) or draft (comments visible)
\fxsetup{draft}
%\fxsetup{final}
% ------------------------------------------------------------------------------

% ------------------------------------------------------------------------------
%  Language selection for metadata and main text (can be changed at any point in
%  the main text.
%\selectlanguage{ngerman}
\selectlanguage{english}
% ------------------------------------------------------------------------------

% ------------------------------------------------------------------------------
% Data for the bibliography
\addbibresource{thesis.bib}
% ------------------------------------------------------------------------------

% ------------------------------------------------------------------------------
% Further packages and TikZ libraries can be incorporated here.
\usetikzlibrary{arrows}
% ------------------------------------------------------------------------------

% ------------------------------------------------------------------------------
% PDF-Metadaten
\hypersetup{
 pdfauthor={Andres Alberto Becker Sanabria},
 pdftitle={Predicting transcription rate from multiplexed protein maps using deep learning},
 pdfsubject={Predicting transcription rate from multiplexed protein maps using deep learning},
 pdfkeywords={Master's Thesis},
 colorlinks=true, %coloured links (for the PDF version)
% colorlinks=false, % no coloured links (for the print version)
}
% ------------------------------------------------------------------------------

% ------------------------------------------------------------------------------
% Data for the title page and declaration
\author{Andres Becker}
\title{Predicting transcription rate from multiplexed protein maps using deep learning}
%\subtitle{A Tutorial for Theses}
%\faculty{Fakultät für Mathematik}
\faculty{Department of Mathematics}
%\institute{Lehrstuhl für Mathematische Modelle biologischer Systeme}
\institute{Chair of Mathematical Modeling of Biological Systems}
\subject{master}
%\subject{bachelor}
%\subject{diploma}
%\subject{project}
%\subject{seminar}
%\subject{idp}
%\subject{Short Overview}
\professor{Prof. Dr. Fabian J. Theis} %Themensteller
\advisor{Dr. Hannah Spitzer} %Betreuer
\date{April 14, 2021} %Submission Date
\place{München} %Place where document is signed
% ------------------------------------------------------------------------------

% ==============================================================================
% Acronyms
% ==============================================================================

% Packages should be called in the preamble document (preamble.tex). However,
% glossaries package requires to be called after hyperref, babel, polyglossia,
% inputenc and fontenc. Therefore, calling glossaries package in preamble doc
% produce errors during compilation (and no generation of Acronyms section).

%\usepackage[style=long,nonumberlist, toc,acronym,nomain]{glossaries}
% style=long: center the acronyms in acronym section
% nonumberlist: remove the list all places in docu where the acron was called
% toc: Add Acronyms section to table of content
% acronym, nomain: necessary
\usepackage[acronym, toc, nomain]{glossaries}
\makeglossaries

%\newacronym[<options>]{<label>}{<short>}{<long>}
\newacronym{4i}{4i}{iterative indirect immunofluorescence imaging}
\newacronym{if}{IF}{Indirect immunofluorescence}
\newacronym{mpm}{MPM}{multiplexed protein map}
\newacronym{mcu}{MCU}{multiplexed cell unit}
\newacronym{tr}{TR}{transcription rate}
\newacronym{scmos}{sCMOS}{scientific complementary metal oxide semiconductor}
\newacronym{ml}{ML}{Machine Learning}
\newacronym{cnn}{CNN}{Convolutional Neural Network}
\newacronym{mlp}{MLP}{Multilayer Perceptron}
\newacronym{ann}{ANN}{Artificial Neural Network}
\newacronym{relu}{ReLU}{Rectified Linear Unit}
\newacronym{sgd}{SGD}{Stochastic Gradient Descent}
\newacronym{gd}{GD}{Gradient Descent}
\newacronym{adam}{Adam}{Adaptive Moment Estimation}
\newacronym{mrna}{mRNA}{messenger RNA}
\newacronym{pmrna}{pre-mRNA}{pre-messenger RNA}
\newacronym{ig}{IG}{Integrated Gradient}
\newacronym{vg}{VG}{VarGrad}
\newacronym{sg}{SG}{SmoothGrad}
\newacronym{vgig}{VGIG}{VarGrad Integrated Gradient}
\newacronym{tfds}{TFDS}{TensorFlow Dataset}
% Metrics
\newacronym{mae}{MAE}{mean absolute error}
% Models
\newacronym{dnn}{DNN}{Deep Neural Network}
\newacronym{svm}{SVM}{Support Vector Machine}
% ------------------------------------------------------------------------------

% ==============================================================================
% Costume commands
% ==============================================================================
% Defined to highlight new words
\newcommand{\hl}[1]{\textit{#1}}
% change the thickness of horizontal lines in tables
\newcommand{\ChangeRT}[1]{\noalign{\hrule height #1}}
% rename commad to make symbols bold in equation mode
\newcommand{\bs}[1]{\boldsymbol{#1}}
% ------------------------------------------------------------------------------

% ==============================================================================
% Utilities and curiosities
% ==============================================================================
% print page width (in pts)
%\the\textwidth
%Output: 408.2971
% in mm: 144.03814361, https://www.unitconverters.net/length/point-to-millimeter.html

% set some variables to store linewidths (to create tables that fits the page)
\newlength{\mylinewidth}
\newlength{\mylengtha}
\newlength{\mylengthb}
\newlength{\mylengthc}
\newlength{\mylengthd}
\newlength{\mylengthe}

% ==============================================================================
% Main part of the document
% ==============================================================================
\makeindex[title=Index,options=-s myindex]
\begin{document}
\pagestyle{empty}
\frontmatter%
%\selectlanguage{ngerman}
\selectlanguage{english}
\maketitlepage%
%\maketitlepageDissertation % a more elegant one, but for PHd dissertation
%\makedeclaration%

%% ==============================
\chapter*{Declaration}
%% ==============================

\vfill

I hereby declare that this thesis is my own work and that no other sources have been used except those clearly indicated and referenced.

\vspace{3em}

\noindent Andres Alberto Becker Sanabria, München, 14.05.2021

%==================================================
% abstract.tex
% Beispieldatei für tumthesis.cls und thesis.tex
% Michael Ritter, 2012
% Lizenz:
% This work may be distributed and/or modified under the
% conditions of the LaTeX Project Public License, either version 1.3
% of this license or (at your option) any later version.
% The latest version of this license is in
% http://www.latex-project.org/lppl.txt
% and version 1.3 or later is part of all distributions of LaTeX
% version 2005/12/01 or later.
%==================================================

%% Magic command to compile root document
% !TEX root = ../thesis.tex

%% Reset glossary to show long gls names
\glsresetall

%\cleardoublepage

\selectlanguage{english}
\section*{Abstract}

By means of fluorescent antibodies it is possible to observe the amount of nascent RNA within the nucleus of a cell, and thus estimate its \gls{tr}. But what about the other molecules, proteins, organelles, etc. within the nucleus of the cell? Is it possible to estimate the \gls{tr} using only the shape and distribution of these subnuclear components? By means of multichannel images of single cell nucleus (obtained through the \gls{mpm} protocol \cite{Guteaar7042}) and \glspl{cnn}, we show that this is possible. 
Applying pre-processing and data augmentation techniques, we reduce the information contained in the intensity of the pixels and the correlation of these between the different channels. This allowed the \gls{cnn} to focus mainly on the information provided by the location, size and distribution of elements within the cell nucleus.
For this task different architectures were tried, from a simple \gls{cnn} (with only 160k parameters), to more complex architectures such as the ResNet50V2 or the Xception (with more than 20m parameters).
Furthermore, through the interpretability methods \gls{ig} and \gls{vg}, we could obtain score maps that allowed us to observe the pixels that the \gls{cnn} considered as relevant to predict the \gls{tr} for each cell nucleus input image. The analysis of these score maps reveals how as the \gls{tr} changes, the \gls{cnn} focuses on different proteins and areas of the nucleus. This shows that interpretability methods can help us to understand how a \gls{cnn} make its predictions and learn from it, which has the potential to provide guidance for new discoveries in the field of biology.

%% Magic command to compile root document
% !TEX root = ../thesis.tex

\selectlanguage{english}
\section*{Acknowledgments}

To my father, my partner in my wildest adventures, best friend and who taught me what are the important things in life. He may never read this, but let the world known he is a loved and admired man.


%%% Local Variables:
%%% mode: latex
%%% TeX-master: "thesis"
%%% End:


\tableofcontents%

\mainmatter%
\pagestyle{headings}

%% =============================================================================
%% Introduction chapter
%% =============================================================================
\chapter{Introduction}
\label{ch:introduction}
%% Magic command to compile root document
% !TEX root = ../../thesis.tex

\glsresetall
% define where the images are
\graphicspath{{./Sections/Results/Resources/}}

This chapter is mainly divided into two parts

\begin{enumerate}
    \item The model performance analysis
    \item The model interpretation analysis
\end{enumerate}

In the first part, we show the results of the models introduced in section \ref{sec:methodology:models}, and compare the performance of each model against the others based on the metrics introduced in section \ref{sec:methodology:metrics}. Besides, the performance of each model is compared to the reference values of the metrics, which validate that the models are capable of learning something meaningful from the data. This section ends by analyzing more in-depth the performance of the linear model against one of the \glspl{cnn} models, and shows that the latter is still capable of predicting fairly well the \gls{tr}, even after significantly reducing the pixel intensity information of each channel and the correlation between them (by means of the data augmentation techniques introduced in section \ref{sec:dataset:data_augmentation} and \ref{sec:methodology:tfds}).

On the other hand, the Model interpretation section focuses on analyzing the results obtained using the interpretability methods introduced in section \ref{sec:basics:interpretability_methods}. The analysis begins with the division of the cells into three levels of transcription (low, medium and high), and ends by analyzing how the model changes the areas of interest in the input image as the \gls{tr} changes. 
Furthermore, the analysis shows that as the \gls{tr} increases, the model relies more on regions of the nucleus that are directly related to the genesis of mature \gls{mrna}, which shows that interpretation methods can be used as tools to discover unknown biological relationships when applied to black-box models like \glspl{cnn}.

The results of both sections also show that it is possible to predict (to some extent) the \gls{tr} of a cell, based mainly on spatial information within the nucleus.

\section{Motivation}
\label{sec:intro:motivation}
% why we want to predict/understand TR
Understanding how \gls{mrna} concentration in eukaryotes cells (\acrlong{tr}) is regulated is very important to understand gene expression. These concentrations can vary depending on the synthesis and/or degradation rates and the rapidity of the transition to a new concentration depends on the regulation of \gls{mrna} stability \cite{PEREZORTIN2007250}. \fxnote{Fix/rewrite this after finishing the results and bio back.}

% why we use images to predict transcription rate?
However, the protein or \gls{mrna} amount inside a cell, may not be enough to fully describe cellular function. Accordingly to Buxbaum et al. \cite{Buxbaum_2014} and Korolchuk et al. \cite{Korolchuk2011}, cellular function can heavily depends on the specific intracellular location and interaction with other molecules and intracellular structures. Therefore, cellular expression is determined by the functional state, abundance, morphology, and turnover of its intracellular organelles and cytoskeletal structures. This means that, models that predict \gls{tr} by means of single cell images, may be able to leverage scientific discoveries in biomedicine.


\section{Goal of this Thesis}
\label{sec:intro:goals}
The objective of this work is to predict cell \gls{tr} through cell images that show specific proteins inside the nucleus and \gls{cnn}. This work also try to understand better the factors that influence cell \gls{tr} by applying interpretability techniques to the \gls{cnn} model.


\section{Literature review}
\label{sec:intro:literature_review}
In recent years neural networks...

% Interpretability part
However, once we have successfully trained an artificial neural network to make predictions, one may also want to understand how the predictions was
made. Several approaches have being used to tackle this. In this work we focus in the ones that highlight features in an input that are most responsible for the model's output. We use a composition of two explainable AI techniques, \gls{ig} \cite{sundararajan2017axiomatic} to highlight relevant input components (saliency map) and \gls{vg} \cite{adebayo2020sanity} to enhance the empirical quality the saliency map.


%% =============================================================================
%% Basics chapter
%% =============================================================================
\chapter{Basics}
\label{ch:basics}
%% Magic command to compile root document
% !TEX root = ../../thesis.tex

\glsresetall
% define where the images are
\graphicspath{{./Sections/Results/Resources/}}

This chapter is mainly divided into two parts

\begin{enumerate}
    \item The model performance analysis
    \item The model interpretation analysis
\end{enumerate}

In the first part, we show the results of the models introduced in section \ref{sec:methodology:models}, and compare the performance of each model against the others based on the metrics introduced in section \ref{sec:methodology:metrics}. Besides, the performance of each model is compared to the reference values of the metrics, which validate that the models are capable of learning something meaningful from the data. This section ends by analyzing more in-depth the performance of the linear model against one of the \glspl{cnn} models, and shows that the latter is still capable of predicting fairly well the \gls{tr}, even after significantly reducing the pixel intensity information of each channel and the correlation between them (by means of the data augmentation techniques introduced in section \ref{sec:dataset:data_augmentation} and \ref{sec:methodology:tfds}).

On the other hand, the Model interpretation section focuses on analyzing the results obtained using the interpretability methods introduced in section \ref{sec:basics:interpretability_methods}. The analysis begins with the division of the cells into three levels of transcription (low, medium and high), and ends by analyzing how the model changes the areas of interest in the input image as the \gls{tr} changes. 
Furthermore, the analysis shows that as the \gls{tr} increases, the model relies more on regions of the nucleus that are directly related to the genesis of mature \gls{mrna}, which shows that interpretation methods can be used as tools to discover unknown biological relationships when applied to black-box models like \glspl{cnn}.

The results of both sections also show that it is possible to predict (to some extent) the \gls{tr} of a cell, based mainly on spatial information within the nucleus.

\section{Biology Background}
\label{sec:basics:bio_back}
\glsresetall
% define where the images are
\graphicspath{{./Sections/Basics/Resources/}}

Cells are considered the smallest unit of life. There are two types of cells, \textit{prokaryotic} and \textit{eukaryotic}. The main difference between these, is that prokaryotic cells do not contain nucleus and that that prokaryotes are considered single-celled organisms, while eukaryotes organisms can be either single-celled or multicellular. For multicellular organisms, like plants or mammals, eukaryotic cells are the \textit{building-blocks} of life. This work focuses on a process of eukaryotic cells. Therefore, in the subsequent when we refer to cells, we will be referring to eukaryotic cells only.

Multicellular organisms (like us) have different cell types, where each one of them can have many or an specific function. For instance, red blood cells are responsible for carrying the oxygen in the body. In order to carry as much oxygen as possible they lack a nucleus, and therefore they are unable to undergo \textit{mitosis}\footnote{Mitosis is the process through which eukaryotic cells reproduce themselves and give rise to new organisms.}.

However, there are also cells aimed to produce (\textit{synthesize}) certain substances that regulate process in our body. For instance, \textit{Alpha cells} are pancreatic cells responsible for synthesizing the \textit{glucagon} hormone, which elevates the glucose levels in the blood \cite{1e48f81ce88f4602a25a4ebbcea3a6e7}. The process in which cells produce this substances is called \textit{cellular expression} or \textit{gene expression}. The reason why this process is also called gene expression, is because the instructions to synthesize every substance (or any functional product, like hormones or proteins) are encoded in a specific gene\footnote{A \textit{gene} is defined as a region of the \textit{DNA} that encodes a function. DNA is contained in \textit{chromosomes}, which are long DNA strands containing many genes.}.

There are two key steps involved in gene expression, \textit{transcription} and \textit{translation}. Roughly speaking, transcription is the process in which the instructions to synthesize a product (like proteins) are copied from a gene in the DNA, to a single strand molecule called \gls{mrna}. On the other hand, Translation is the process in which the instructions in the \gls{mrna} are interpreted to produce a functional product. Figure \ref{fig:BB:tt} shows a simple representation of this process.

\begin{figure}[htb]
  \centering
  \includegraphics[width=\linewidth]{central-dogma-large.png}
  \caption{Simple representation of the gene expression process. Image source \cite{transcript_translation_diagram}.}
  \label{fig:BB:tt}
\end{figure}

The transcription process happens inside the cell nucleus, while translation happens in the \textit{ribosome} (outside the nucleus). The reason why transcription is necessary, is because the instructions needed to build a product are encoded in the DNA, which is inside the nucleus. Since DNA is too big to pass the membrane that covers the nucleus (nuclear envelop) to travel to the ribosome (which is the organelle in charge of building the product), the necessary instructions in the DNA are copied into a smaller strand (\gls{mrna}), which is now able to escape the nucleus and travels to the ribosome to start the translation process. Figure \ref{fig:BB:euka} shows a diagram of an eukaryotic cell and some of its parts. This work focuses on the transcription process and the factors that seed up or slow down this process.

\begin{figure}[htb]
  \centering
  \includegraphics[width=0.7\linewidth]{Animal_cell_structure_en.png}
  \caption{Animal eukaryotic cell diagram. Image source \cite{eukacell}.}
  \label{fig:BB:euka}
\end{figure}

\subsection{Transcription Process}
\label{sec:basics:transcription_process}
\input{Sections/Basics/Transcription_process}

\section{Machine Learning}
%% Magic command to compile root document
% !TEX root = ../../thesis.tex

%% Reset glossary to show long gls names
\glsresetall

%% Set path to look for the images
\graphicspath{{./Sections/Basics/Resources/}}

Road map. Here we only explain what is a NN and maybe mention the over and under fitting.
\begin{enumerate}
  \item What is a ANN
  \item Why this work (universal approx th.)
  \item what is a CNN
  \item training logic: Data -> training process -> prediction
\end{enumerate}

\glspl{ann} are universal approximators widely used in the field of \gls{ml} and an important part of this work. This is a very broad subject and there are entire books that cover this in detail, like  \cite{Goodfellow-et-al-2016} or \cite{bishop2006pattern}. However, in this section we will give a small introduction to \glspl{ann}, specially to a specific kind of \gls{ann} known as \glspl{cnn}.

Before defining what exactly is a \gls{ann}, lest first recall the definition of machine learning. We refer as \gls{ml} to the group of algorithms that automatically improve (learn) through experience. Among this algorithms, we could say that there are three main classes (which depend on the kind of experience we provide):

\begin{itemize}
  \item \textbf{Supervised Learning}: The experience is given in the form of input and output examples, and the goal is to learn a general rule that maps inputs to outputs.
  \item \textbf{Unsupervised Learning}: The experience is given in the form of data (no outputs provided) and the goal is to discover hidden patterns in data.
  \item \textbf{Reinforcement Learning}: No experience (data) is given, instead a dynamic \hl{environment} is provided and an \hl{agent} must learn how to interact with it in order to achieve a goal.
\end{itemize}

\glspl{ann} can be applied in any of the 3 kinds of learning algorithms listed above. However, \hl{supervised learning} is the one that best illustrates \glspl{ann} and it is also the kind of challenge we're dealing with. Recall that we seek to approximate a function (a \gls{cnn}), such that when it is fed with an images of a cell nucleus (input data), it can approximate the \gls{tr} (output data) of the corresponding cell.

\label{sec:basics:ANN}
\subsection{Artificial Neural Networks}
%% Magic command to compile root document
% !TEX root = ../../thesis.tex

%% Reset glossary to show long gls names
\glsresetall

%% Set path to look for the images
\graphicspath{{./Sections/Basics/Resources/}}

% what is a ANN

Roughly speaking, an \gls{ann} is a non-linear function $f:\mathbb{R}^D \rightarrow \mathbb{R}^L$, that maps an input $\bs{x}\in\mathbb{R}^D$ with an output $\bs{y}\in\mathbb{R}^L$. Of course, to consider $f$ as a \gls{ann}, $f$ must have a specific form that will be address later. However, for the sake of this explanation, let us start by defining a simple function as follow

\begin{equation}
  \begin{split}
    f(\bs{x},\bs{w}) &:= h \left(w_0 + \sum_{j=1}^{M-1}w_j\phi_j(\bs{x}) \right) \\
    &= h(\bs{w}^T\bs{\phi}(\bs{x})) \\
    &:= h(z)
  \end{split}
  \label{eq:basics:slp}
\end{equation}

\noindent where $\bs{\phi}:\mathbb{R}^{D+1} \rightarrow \mathbb{R}^M$ is an element-wise function, with $\phi_0:=1$, know as \hl{basis function}, $h:\mathbb{R} \rightarrow \mathbb{R}$ is a function know as \hl{activation function} and $\bs{w}\in\mathbb{R}^M$ is the parameter vector. The parameters $w_j$, with $j\in\{1,\dots M-1\}$ are known as \hl{weights}, while the parameter $w_0$ is know as \hl{bias}.

Then, an \gls{ann} is composition of functions of the same form as \ref{eq:basics:slp}, with non-linear \hl{activation functions}, and where the basis functions are also of the same form as \ref{eq:basics:slp} \cite{bishop2006pattern}

\begin{equation}
  F(\bs{x}, \bs{W}) :=
  h_K(\bs{w}^T_K h_{K-1}(\bs{w}^T_{K-1} \dots h_{0}(\bs{w}^T_0 \bs{x}) \dots ))
  \label{eq:basics:ann}
\end{equation}

The subscript in the parameter vectors $\bs{w}_k$ and the activation functions $h_k$, with $k\in\{0, \dots, K\}$, of \ref{eq:basics:ann} represents the depth of the layers. Note that unlike the other layers, the base function of the \hl{input layer} (k = 0) is the identity function. Furthermore, the activation function of the \hl{output layer} $h_K$ does not necessarily have to be non-linear. Instead, it is chosen based on the type of function we want to approximate. In our case, since we have a regression problem (predicting \gls{tr}), $h_K$ is chosen as the identity function.

There are different non-linear activation functions that can be chosen for the hidden units. However, all the models showed in this work use the \gls{relu}

\begin{equation}
  ReLU := max\{0, x\}
\end{equation}

Figure \ref{fig:basics:ann:relu} shows the \gls{relu} activation function.

% Figure made in notebook Preprocessing_resources.ipynb
\begin{figure}[!ht]
  \centering
  \includegraphics[width=0.6\linewidth]{ReLU.jpg}
  \caption{\gls{relu} activation function.}
  \label{fig:basics:ann:relu}
\end{figure}

Figure \ref{fig:basics:ann:ann} shows a graphical representation of a \gls{ann}. The circles represent the activation function applied to what is inside it. Black colored circles represent the identity function, red colored circles the non-linear activation function for the hidden layers, while green any function for the output layer that suits the problem we want to solve. Note that values inside the circles of the hidden and output layers $z^k_i$, for $k\in\{0, \dots, K\}$ and $i$ representing one of the units of the $k$ layer, are the output of a function of the same form as \ref{eq:basics:slp}. The lines connecting the circles represent the weights and biases corresponding to each layer $\bs{W}_k$, for $k\in\{0, \dots, K\}$. The circles in the \hl{hidden layers} are known as \hl{hidden units}.

\begin{figure}[!ht]
  \centering
  \includegraphics[width=0.8\linewidth]{Diagrams/ANN.jpg}
  \caption{Graphical representation of an \gls{ann}. The color of the circles represents the type of activation function. Black means the identity, red a non-linear function for the hidden layers and green any function for the output layer.}
  \label{fig:basics:ann:ann}
\end{figure}

Strictly speaking, equation \ref{eq:basics:ann} and figure \ref{fig:basics:ann:ann} represent a \hl{fully connected feedforward neural network}. However, in this work we will refer to it just as \gls{ann}, which in some literature is also known as \gls{mlp}. Also, hidden layers are also known as \hl{Dense layers}.

\subsubsection{Update rule}

Sow far we have introduced the general form an \gls{ann} must have. Moreover, equation \ref{eq:basics:ann} shows that an \gls{ann} is simply a non-linear function controlled by a set of adjustable parameters $\bs{W}$. Therefor the question is, how can we approximate this parameters?

Recall that we are dealing with a supervised learning problem, which means that we can use both the input data (images of cell nucleus, $\bs{X}$) and the output data (the \glspl{tr}, $\bs{Y}$) to approximate $\bs{W}$. Therefore, we can fed the \gls{ann} with $\bs{X}$, and then measure its performance by comparing its output $\hat{\bs{Y}}$ against the true values $\bs{Y}$.

This comparison is made by means of a \hl{loss function} $\mathcal{L}$ that must be chosen beforehand. The choice of $\mathcal{L}$ depends mainly on the type of problem you are solving (regression, classification, etc.). However, even for each type, there are many different options. For now, let us just say that $\mathcal{L}$ should return high values when $\hat{\bs{Y}}$ is far from the true values $\bs{Y}$, and low when they are close.

Then, we can fit the values of $\bs{W}$, by minimizing the loss function $\mathcal{L}$ each time the model is fed with an input value $\bs{x}$. Since the gradient of $\mathcal{L}$ with respect to $\bs{W}$ (i.e., $ \nabla_{\bs{W}} \mathcal{L}$) returns the direction in which the loss function grows the fastest, then we choose $- \nabla_{\bs{W}} \mathcal{L}$ as the direction of our update rule

\begin{equation}
  \bs{W}_{new} = \bs{W}_{old} - \alpha \nabla_{\bs{W}} \mathcal{L}(\bs{W}_{old})
  \label{eq:basics:ann:learn_rule}
\end{equation}

\noindent where $\alpha \in \mathbb{R}^+$ (known as \hl{learning rate}) controls how much we move in the direction of $-\nabla_{\bs{W}} \mathcal{L}(\bs{W}_{old})$ on every step.

The iterative method in which \ref{eq:basics:ann:learn_rule} is applied over elements of $\bs{X}$ to optimize $\bs{W}$ is known as \gls{gd} \cite{bishop2006pattern}.
However, in practice \ref{eq:basics:ann:learn_rule} is not applied for a single element of $\bs{X}$ every time, but to a random subset of $\bs{X}$ (known as a \hl{Batch}) instead.
The number of elements in batch is fixed over all the iteration (training), and is an hyperparameter known as \hl{Batch Size} $bs$\footnote{Normally the training data is separated in disjoint batches, which means that it could happen that last batch to be smaller than the selected $bs$.}.
As a rule of thumb, $bs$ should be no less than 30 (for the selected sample to be representative of $\bs{X}$). In practice $bs$ is usually chosen as a power of 2.
This process is known as \gls{sgd} and computationally is less  expensive than \gls{gd}.

However, \gls{gd} (\gls{sgd}) has a downside, the choice of its hyperparameter $\alpha$ (learning rate). In practice, it has been shown that the correct choice of $\alpha$ is essential to train an \gls{ann} successfully. Therefore, other algorithms (\hl{optimizers}) have been proposed to mitigate this problem. The revision of these optimizers is out of the scope to this work. However, all of them follow the same idea proposed by \gls{gd}. For example, instead of having a fixed learning rate $\alpha$ as in \gls{gd}, the \gls{adam} optimizer adapts its learning rate dynamically during training depending on the mean and variance of the loss function \cite{kingma2014adam}.

\subsubsection{Back propagation}

Nevertheless, there is still one question that needs to be answered, which is how to efficiently calculate the derivative of the loss function ($\nabla_{\bs{W}} \mathcal{L}$) with respect to all the parameters of the \gls{ann}. The answer to this is through an algorithm called \hl{backpropagation}, which is performed during the \hl{training process}. Again, there is a lot of literature that explains this in depth(for instance \cite{Goodfellow-et-al-2016} or \cite{bishop2006pattern}). Therefor, here we will just provide the intuition behind it.

Recall that $\mathcal{L}$ is a function of the true values $y$ and $\hat{y}$ i.e., $\mathcal{L}(y, \hat{y})$. Also from equation \ref{eq:basics:ann} and figure \ref{fig:basics:ann:ann} note that

\begin{equation}
  \begin{split}
    y &:= F(\bs{x}, \bs{W}) \\
    &= h_K(\bs{z}^K) \\
    &= h_K(\bs{W}_K^T h_{K-1}(\bs{z}^{K-1}))
  \end{split}
  \label{eq:basics:ann:backprop_1}
\end{equation}

and therefore

\begin{equation}
  \begin{split}
    \nabla_{\bs{W}_K} \mathcal{L} &= \frac{\partial \mathcal{L}}{\partial \bs{W}_K} \\
    &= \frac{\partial \mathcal{L}}{\partial \hat{y}}
    \frac{\partial \hat{y}}{\partial \bs{z^K}}
    \frac{\partial \bs{z^K}}{\partial \bs{W}_K} \\
  \end{split}
  \label{eq:basics:ann:backprop_2}
\end{equation}

\noindent which is just the product of the derivative of the loss function w.r.t. $\hat{y}$ (i.e., $\frac{\partial \mathcal{L}}{\partial \hat{y}}$), the derivative of the activation function of the output layer w.r.t the argument of the last layer (i.e., $\frac{\partial \hat{y}}{\partial \bs{z^K}}$) and the output of the layer $K-1$ (i.e., $\frac{\partial \bs{z^K}}{\partial \bs{W}_K}=h_{K-1}(\bs{z^{K-1}})$).

Note that we can easily compute the gradient of $\mathcal{L}$ w.r.t deeper parameters $\bs{W}_k$ (for $k\in\{0, \dots, K-1\}$), just by extending \ref{eq:basics:ann:backprop_1} and \ref{eq:basics:ann:backprop_2}.

This shows how by means of the \hl{chain rule}\footnote{$(f \circ g)'=(f \circ g) \cdot g'$, or equivalently $h'(x)=f'(g(x))g'(x)$, for $h(x):=f(g(x))$.}, the backpropagation algorithm can compute the gradient of the loss function w.r.t. a specific parameter, just by multiplying the derivative of the loss function, the derivative of the activation functions and some values computed during the evaluation of the \gls{ann}.

\subsubsection{Model development}

The properties of \glspl{ann} have been studied extensively before (\cite{cybenko1989approximation}, \cite{hornik1989multilayer}, \cite{funahashi1989approximate}) and established in the \hl{Universal approximation theorem}

\begin{theorem}[Universal approximation theorem]
  An \gls{mlp} with a linear output layer and one hidden layer can approximate any continuous function defined over a closed and bounded subset of $\mathbb{R}^D$, under mild assumptions on the activation function (\hl{squashing} activation function) and given the number of hidden units is large enough.
\end{theorem}

For this reason \gls{ann} are known as \hl{universal approximators}, since they are able to approximate any continuous function on a compact\footnote{A set $A$ in a metric space is said to be \hl{compact} if it is close (i.e., it contain all its limit points) and bounded (i.e., all its points lie within some fixed distance of each other) \cite{bartle2000introduction}.} input domain with an arbitrary accuracy \cite{bishop2006pattern}.

These means that, as long as a \gls{ann} has a sufficiently large number of hidden units, the loss function can be reduced as much as desired. However, this nice property can also lead to an unwanted one known as \hl{overfitting}.
Intuitively this means that the \gls{ann} \hl{memorize} the data used to train it (low error/bias), and therefore it is not able to perform (or \hl{generalize}) well when it is fed with new data (high error/bias and variance). This happens mainly when the \gls{ann} is optimized/fed too many times with the same data.

On the other hand, \hl{underfitting} means that the \gls{ann} performs poorly on both new data and data used to train the network (high bias and low variance). This usually happens when the training time is insufficient or the \gls{ann} is not complex enough (too few hidden units and/or layers).

Figure \ref{fig:basics:ann:fitting} shows synthetic data (blue circles), generated from a sine function (green line) and random noise sampled from a normal distribution.
The red line in figure \ref{fig:basics:ann:fitting:under} represents a fitted model with high bias and low variance (underfitting), while in figure \ref{fig:basics:ann:fitting:over} a model with low bias and high variance (overfitting). The red line in figure \ref{fig:basics:ann:fitting:good}, represents a model with low bias and variance (good fit and good generalization).

% this plots were extracted from page 7 (Bishop) and adapted in the file overfitting.odg
\begin{figure}[htb]
  \centering
  \begin{subfigure}[t]{.3\linewidth}
    \includegraphics[width=\linewidth]{Diagrams/underfitting.jpg}
    \caption{Underfitted model.}
    \label{fig:basics:ann:fitting:under}
  \end{subfigure}
  \vspace{3mm}
  \begin{subfigure}[t]{.3\linewidth}
    \includegraphics[width=\linewidth]{Diagrams/goodfit.jpg}
    \caption{Model with good fit and generalization.}
    \label{fig:basics:ann:fitting:good}
  \end{subfigure}
  \vspace{3mm}
  \begin{subfigure}[t]{.3\linewidth}
    \includegraphics[width=\linewidth]{Diagrams/overfitting.jpg}
    \caption{Overfitted model.}
    \label{fig:basics:ann:fitting:over}
  \end{subfigure}
  \caption{Representation of a model (red line) with underfitting \subref{fig:basics:ann:fitting:under}), good fit \subref{fig:basics:ann:fitting:good}) and overffiting \subref{fig:basics:ann:fitting:over}), trained over synthetic data (blue small circles). The synthetic data was generating by adding random noise to a sine function (green line) on the interval $[0,1]$. Image source \cite{bishop2006pattern}.}
  \label{fig:basics:ann:fitting}
\end{figure}

In practice, we seek to fit models that has low bias and low variance (i.e., good accuracy and good generalization). Therefore, to prevent overfitting we split the data into 3 different sets; \hl{training}, \hl{validation} and \hl{test}, and train the model using only the first set. Then, during model training, we measure how well the model is generalizing by comparing the value of the loss function when it is evaluated in the training and validation set \footnote{This is usually known as the \hl{bias–variance tradeoff}.}.
During the model development, the \hl{test} set is never evaluated and is only used at the end, to report the model performance. This methodology is shown in figure \ref{fig:basics:model_train_process}

Figure \ref{fig:basics:bias_variance} shows this \hl{bias–variance tradeoff} between training and validation set. In practice, multiple versions of the model are saved during training and then the one with the lowest validation error is chosen (red dot on figure \ref{fig:basics:bias_variance}).

% figure taken from BL_070121_0942.ipynb
\begin{figure}[!ht]
  \centering
  \includegraphics[width=0.8\linewidth]{bias_variance.png}
  \caption{Bias–variance tradeoff. In orange (respectively blue) the loss function curve when it is evaluated in the validation (respectively training) set. The red dot shows the lowest loss for the validation set.}
  \label{fig:basics:bias_variance}
\end{figure}

\begin{figure}[!ht]
  \centering
  \includegraphics[width=0.8\linewidth]{Diagrams/Model_methodologt.jpg}
  \caption{Model development methodology.}
  \label{fig:basics:model_train_process}
\end{figure}

The methodology shown in figure \ref{fig:basics:model_train_process} is also used to optimize the hyperparameters of the model, like the number hidden units/layers or the activation function of the hidden layers.

\subsubsection{Batch Normalization}

However, overfitting is not the only problem we may encounter when training an \gls{ann}. Training \gls{ann} with several layers can be complicates, since the distribution of the data can change from layer to layer. This means that the input and output distribution of a layer will not necessarily be the same. It has been empirically proven that this can affect the training performance, since it require the use of lower learning rates \cite{ioffe2015batch}. This can also lead to \hl{saturation}\footnote{\hl{Saturation} is a commonly used term to refer to the situation when the evaluation of a "squashing" function returns values close to some of its horizontal asymptotes most of the time. Remember that these "squash" functions (like \hl{Sigmoid} or \hl{tanh}) compress the real line $(-\inf, \inf)$ into an interval of finite length $(a, b)$.} of the activation functions, so a more careful initialization of the \gls{ann} parameters is required. To address this problem Ioffe et al. \cite{ioffe2015batch} proposed to normalize the layer inputs.

Roughly speaking, batch normalization consist of two main steps; 1) the standardization of the layer input and 2) the normalization of the standardized data. For the first step the layer input is standardized using parameters extracted from the \hl{batch}

\begin{equation}
    \bs{z}'_k := \frac{\bs{z}_k-\bs{\mu}_k}{\sqrt{\bs{\sigma}_k^2-\epsilon}}
\end{equation}
\noindent with
\begin{equation}
  \begin{split}
    \bs{\mu}_k &= \frac{1}{M}\sum_{m=1}^M \bs{z}_k \\
    \bs{\sigma}_k^2 &= \frac{1}{M}\sum_{m=1}^M (\bs{z}_k - \bs{\mu}_k)^2 \\
  \end{split}
\end{equation}

\noindent where $M$ is the \hl{Batch} size and $k$, with $k \in \{0 \dots K\}$, denotes the layer.

Note that for each layer $k$ we have different normalization parameters $\bs{\mu}_k$ and $\bs{\sigma}_k$. Moreover, this normalization parameters are vectors of the same shape as the layer size (i.e., one pair of normalization parameters per unit/neuron).

The second step in batch normalization consist on normalizing the standardized data $\bs{z}'_k$ using parameters $\bs{\gamma}_k$ and $\bs{\beta}_k$ learned during training

\begin{equation}
    \overset{\sim}{\bs{z}}_k := \bs{\gamma}_k \odot \bs{z}'_k + \bs{\beta}_k
\end{equation}

\noindent where $\odot$ denotes \hl{element-wise} multiplication. At the beginning of the training $\bs{\gamma}_k=1$ and $\bs{\beta}_k=0$ are used for all the layers and units.

During training, the normalization parameters of each epoch are stored, so the average ($\bar{\bs{\gamma}}_k$ and $\bar{\bs{\beta}}_k$) can be used during evaluation (when the model is not training).

\subsubsection{Residual Block V2}

As already mentioned, the \hl{Universal approximation theorem} guarantees that the training error can be reduced by adding more layer to an \gls{ann}. However, in practice it is not that simple. As we add layers to an \gls{ann}, the training becomes more unstable and difficult as we can face vanishing or exploding gradients (when the value of the gradients become very close to 0 or $\inf$ respectively during back propagation). To overcome this problem, He et al. (\cite{he2015deep} and \cite{he2016identity}) proposed the \hl{residual blocks}, which have been empirically shown to make deep \gls{ann} training more stable.
The core idea of residual blocks is to reformulate the layers as \hl{learning residual functions} with reference to the layer inputs, by adding an \hl{identity connection}. Then, if a layer is not longer beneficial to the \gls{ann} (e.g. in case of gradient vanishing), the \gls{ann} can just "skip" it. Figure \ref{fig:basics:residual_block} shows a diagram of the second version of a residual block \cite{he2016identity}.

\begin{figure}[!ht]
  \centering
  \includegraphics[width=\linewidth]{Diagrams/Residual_block_v2.jpg}
  \caption{Residual block V2.}
  \label{fig:basics:residual_block}
\end{figure}


\subsection{Convolutional Neural Networks}
\label{sec:basics:CNN}
%% Magic command to compile root document
% !TEX root = ../../thesis.tex

%% Reset glossary to show long gls names
\glsresetall

%% Set path to look for the images
\graphicspath{{./Sections/Basics/Resources/}}

So far we have explained how \glspl{ann} works assuming that we feed them with vectors of fixed length. Even though we could take a multichannel image and transform it into a vector, in practice this would be computationally very expensive. For instance, assuming that we have a 3 channel image of size 224 by 224, this would result into an input vector of length $3 \cdot 224 \cdot 224=150'528$. Then, if the first layer of our network has 100 units, this would mean more than 15 millions of parameters only for the first layer. Furthermore, the transformation of our image into a vector would mean a loss of spatial information. This means that the \gls{ann} would not be able to capture or use the spatial relationship between pixels and shapes within the image.

A \gls{cnn} is a type of \gls{ann} widely used to analyze data in the form of images. The intuition behind a \gls{cnn} is that instead of just looking at an image and trying to predict the target value directly, first learn some \hl{features} within the image, and then make the predict base on this features.
To achieve this, \glspl{cnn} mainly use \hl{convolution} and \hl{pooling} layers.

\subsubsection{Convolution layer}

The only difference a

A convolution layer is very similar to a regular layer described in section \ref{sec:basics:ANN}. Basically, they only differ in the way the layer input is multiplied by the the layer weights.
Recall that in a regular layer, the input of a unit is the dot product between the layer input and its corresponding weight vector (i.e., $z=\bs{w}^T\bs{x}$).
This means that for each element in the input vector $\bs{x}$, there is a corresponding element in the weight vector $\bs{w}$. However, for a convolution layer this is not the case.
Convolution layers are based on the shared-weight architecture of the convolution \hl{kernels} or \hl{filters} that slide along the input and returns a translation known as \hl{feature maps} \cite{zhang1988shift}. This means that the \hl{kernels} weights will be used for multiple elements of the layer input. Figure \ref{fig:basics:conv_layer} shows the convolution process with a 2 by 2 kernel over a RGB image (3 channels) of size 4 by 4. Each entrance of the returned feature map $z_i$ is the dot product between the kernel weights $\bs{w}$ and the $\bs{x}_i-th$ chunk of the image.

\begin{figure}[!ht]
  \centering
  \includegraphics[width=\linewidth]{Diagrams/Conv_layer.png}
  \caption{Convolution process steps. In red, green and blue the input image, in orange the convolution kernel (size 2 by 2 and stride of 1) and in gray the convolution output (feature map).}
  \label{fig:basics:conv_layer}
\end{figure}

Mathematically this looks as follow

\begin{equation}
  z_i = \bs{w}^T \bs{x}_i + b
\end{equation}

where $\bs{w}\in\mathbb{R}^{2 \times 2 \times 3}$, $\bs{x_i}\in\mathbb{R}^{2 \times 2 \times 3}$ and $b\in\mathbb{R}$ is the bias (not shown in the images).

Like the kernel size, the number of pixels we shift the kernel each time along side the input (\hl{Stride}) is also a hyperparameter of convolution layers. IN figure \ref{fig:basics:conv_layer}, the stride size is 1.

Figure \ref{fig:basics:conv_layer} also shows that size (width and height) of the returned feature map is smaller than the input image. If we want to keep the input and output size the same (\hl{Same convolution}), then we must add zeros at the edges of the input features (zero-padding). This is shown on figure \ref{fig:basics:conv_layer_pad}.

\begin{figure}[!ht]
  \centering
  \includegraphics[width=0.7\linewidth]{Diagrams/Conv_layer_pad.png}
  \caption{Convolution with padding. In blue a single-channel input features, in orange the convolution kernel (size 3 by 3 and stride of 1) and in gray the convolution output (feature map).}
  \label{fig:basics:conv_layer_pad}
\end{figure}

So far we have seen that a convolution projects a multi-channel input feature (image) into a single-channel feature map. Therefore, if we want our output feature map to have $n$ channels, then our convolution must have $n$ different kernels.

Normally, a non-linear activation function is applied to the output of convolution layers (and normally also after batch normalization) to enable the \gls{cnn} to learn non-linear relations.

\subsubsection{Pooling layer}

Unlike convolution layers, the goal of Pooling layers is to reduce the feature image (height and width, but not depth) rather than learn features.
However, Pooling layers work in a similar way to convolution layers in the way that they also slide a kernel along the input. However, in this case the kernel works independently on each feature map (that is, each channel) and has no weights to learn.
This means that the pooling layers maintain the same number of input and output channels.
There are several ways to do this downsampling, but the most common are Max Polling and Average Polling. As the name suggests, Average pooling shrinks the feature image by averaging sections of it, while Max pooling takes the maximum value. Figure \ref{fig:basics:pooling} shows an example of a max and average pooling layer on a single-channel feature image using a 2 by 2 kernel and a stride of 2.

\begin{figure}[!ht]
  \centering
  \includegraphics[width=0.7\linewidth]{Diagrams/Pooling.png}
  \caption{Max and average pooling with a 2 by 2 kernel and stride 2. The color denotes the kernel position.}
  \label{fig:basics:pooling}
\end{figure}

Normally, Pooling layers are applied over the output of the activation functions.

\subsubsection{Global Average Pooling layer}

As we mentioned at the beginning of this section, the idea of a \gls{cnn} is to first learn the features within the input images and then make a prediction based on these features. To do this, the \hl{Global Average Pooling layer} transforms the channels of the last feature map into a vector (by averaging each of its channels), so that this can be used as input in a regular \gls{ann} to make the final prediction. Figure \ref{fig:basics:global_avg_pool} shows an example of this, when it is applied into a feature map with 7 channels.

\begin{figure}[!ht]
  \centering
  \includegraphics[width=0.7\linewidth]{Diagrams/Global_avg_pooling.png}
  \caption{Global Average Pooling layer.}
  \label{fig:basics:global_avg_pool}
\end{figure}

\subsubsection{Inception module}

Recall that a convolution layer is meant to learn features from a 3D object with 2 spatial dimensions (width and height) and a channel dimension. This means that each kernel in the convolution needs to learn simultaneously cross-channel and spatial correlations.
The intuition behind the \hl{Inception module} is to improve this process by separating this two tasks, so that the cross-channel correlations and the spatial correlations can be learned separately and independently \cite{chollet2017xception}.

A normal inception model looks at the cross-channel correlations first through a set of 3 or 4 \hl{pointwise convolutions}\footnote{A \hl{pointwise convolution} is a convolution with 1 by 1 kernels and stride 1.}, and then learns the spacial information in the downsampled feature image (in depth, not height and width), by means of regular convolution (usually with 3 by 3 or 5 by 5 kernels). Figure \ref{fig:basics:inception_module} shows a diagram of an Inveption V3 module.

\begin{figure}[!ht]
  \centering
  \includegraphics[width=0.6\linewidth]{Inception_module.png}
  \caption{A regular Inception module (Inveption V3). Image source \cite{chollet2017xception}.}
  \label{fig:basics:inception_module}
\end{figure}

François Chollet \cite{chollet2017xception}, used the inception module as reference to propose the \hl{depthwise separable convolution}, which is something between a normal convolution and a normal convolution combined/followed by a pointwise convolution.
Figure \ref{fig:basics:extreme_inception_module} shows an \hl{extreme} version of the inception module shown in figure \ref{fig:basics:inception_module}. The \hl{depthwise separable convolution} is very similar to the one shown in figure \ref{fig:basics:extreme_inception_module}, the only difference is that the pointwise convolution is applied before the 3 by 3 convolutions instead of after.

\begin{figure}[!ht]
  \centering
  \includegraphics[width=0.6\linewidth]{Extreme_Inception_module.png}
  \caption{An extreme version of our Inception module. Image source \cite{chollet2017xception}.}
  \label{fig:basics:extreme_inception_module}
\end{figure}

Even though the \hl{depthwise separable convolution} is a simplified version of the inception module, the idea and motivation behind it is the same. The \hl{depthwise separable convolution}, and the residual block, are the main components of the \hl{Xception} architecture \cite{chollet2017xception}.


\section{Interpretability Methods}
\label{sec:basics:interpretability_methods}
\glsresetall
% Motivation and problem
In recent years, \glspl{dnn} have been used to solve a wide variety of problems and gained popularity. Amazing results such as those achieved by Deep Mind's Alpha Fold team, have shown the great potential \gls{dnn} has to solve complex problems. However, the difficulty to interpret \glspl{dnn} has become one of the main obstacles to their acceptance in applications where the interpretability of the model is necessary.

% solution
To understand how the \glspl{dnn} predict the \gls{tr} of a cell, we use \textit{Attribution Methods}. This methods are meant to measure how much each component of the input image contributes to the model's prediction by creating a \textit{Score Map} (also known as \textit{Importance Map, Sensitivity Map} or \textit{Saliency Map}) of the same shape as the model's input. In particular, in this work we use a combination between \gls{ig} \cite{sundararajan2017axiomatic} and \gls{vg} \cite{adebayo2020sanity} as attribution method. In general we will denote attribution method as $\phi$.

% other advantages
Attribution methods are not only used to interpret black-box models like \gls{dnn}, the can also be used to debug models or as a sanity check to validate that the model base its prediction on the relevant features of the input.

% in our case
In our case, this interpretability techniques will show us which parts of the cell image are relevant for the prediction of the \gls{tr}. However, this will not just help us to interpret the results of the model, this also have the potential to help us understand unknown cellular processes.


\subsection{Integrated Gradients}
\label{sec:basics:IG}
%% Magic command to compile root document
% !TEX root = ../../thesis.tex

%% Reset glossary to show long gls names
\glsresetall

%% Set path to look for the images
\graphicspath{{./Sections/Basics/Resources/}}

\glsfirst{ig} is an interpretability technique (attribution method) proposed by Sundararajan et al. \cite{sundararajan2017axiomatic}, aimed to assign an importance to the input features (in our case pixels from a cell image) with respect to the model prediction. The attribution problem have been studied before in other papers \cite{JMLR:v11:baehrens10a}, \cite{SimonyanVZ13}, \cite{ShrikumarGSK16}, \cite{BinderMBMS16} and \cite{Springenberg}.

In our case, we seek to predict \gls{tr} given a cell image $x \in \mathbb{R}^{d \times d \times c}$, where $d$ is the height and width of the image and $c$ is the number of channels.
Therefore, our \gls{dnn} would be a function $f:\mathbb{R}^{d \times d \times c} \rightarrow \mathbb{R}$ and an attribution method should be a function $\phi:\mathbb{R}^{d \times d \times c} \rightarrow \mathbb{R}^{d \times d \times c}$ having an input and output of the same shape as the model's input image.

Early interpretability methods only use gradients to assign importance to each input feature

\begin{equation}
  \begin{split}
    \phi(f,x) &:= \nabla f(x) \\
    &= \frac{\partial f}{\partial x}
  \end{split}
\end{equation}

Mathematically speaking, $\phi_i(f,x)$ assign an importance score to the pixel $i$ (out of the $d \times d \times c$ there are), representing how much it adds or subtract from the model output.
However, this score maps have some drawback when they are used to interpret deep neural networks \cite{sturmfels2020visualizing}. Recall that the gradient with respect to the input indicate us the pixels that have the steepest local slope with respect to the model's output.
This means that it only describes local changes in the input, and not the whole prediction model. Another mayor problem is saturation\footnote{In the context of artificial neural networks, a neuron is said to be saturated when the predominant output value of a neuron is close to the asymptotic ends of the bounded activation function. This behavior can potentially damage the learning capacity of a neural network.}.
As the model learns the relationship between an input image and its \gls{tr}, the gradient of the most important pixels will approximate to 0, i.e. the pixel's gradient saturates.

To overcome this problems, Sundararajan et al. proposed \gls{ig} as an attribution method, where the importance of the input feature $i$ is defined as follow
\begin{equation}
  \phi^{IG}_i(f, x, x') := (x_{i} - x'_{i})\int_{\alpha=0}^1\frac{\partial f(x'+\alpha (x - x'))}{\partial x_i}{d\alpha}
  \label{eq:ig:definition}
\end{equation}

Intuitively speaking, \gls{ig} accumulates the input gradient when it goes from a baseline $x'$, which should represents \textit{absence} of information, to the actual input image $x$. With this, we avoid losing information about relevant pixels for the model's prediction in the importance map, even if they saturate eventually. Figure \ref{fig:basics:IG_image_prog} shows an example of the image progression fed into IG. Note that the amount of information in the images is parameterized by $\alpha \in [0,1]$, and that the \hl{absence} of information is interpreted as a black image.

\begin{figure}[!ht]
  \centering
  \includegraphics[width=\linewidth]{IG_alpha.png}
  \caption{Progression from an image with no information (back image) to a normal one parameterized by $\alpha$.}
  \label{fig:basics:IG_image_prog}
\end{figure}

For a better understanding, we can divide the \gls{ig} definition as follow
\begin{equation}
  \phi^{IG}_i(f, x, x') := \overbrace{(x_{i} - x'_{i})}^\text{Difference from baseline}
  \underbrace{\int_{\alpha=0}^1}_\text{From baseline to input...}
  \overbrace{\frac{\partial f(x'+\alpha (x - x'))}{\partial x_i}{d\alpha}}^\text{…accumulate local gradients}
  \label{eq:ig:explanation}
\end{equation}

The integral in equation \ref{eq:ig:explanation} accumulate the gradients for the interpolated images $x'+\alpha (x - x'))$ between the baseline $x'$ and the image $x$. On the other hand, the difference $(x_i - x_i')$ outside the integral comes from the chain rule and the fact that we are interested in integrating over the path between the baseline and the image.

%https://arxiv.org/pdf/1806.03000.pdf
\gls{ig} is very simple and easy to implement, since it does not require any modification to the model and it only require some calls to the gradient operator.

The \gls{ig} satisfy several properties and axioms that are addressed in detail in the paper. However, there is one axiom satisfied by \gls{ig} that is of special importance for us, \textit{completeness}. Completeness means that the value of the summed attributes will be equal to difference between the model's output when it is evaluated at the image and the model's output when it is evaluated at the baseline
\begin{equation}
  \sum_i \phi(f, x, x')^{IG} = f(x) - f(x')
  \label{eq:ig_completeness}
\end{equation}

In practice, computing the analytic expression for the integral in equation \ref{eq:ig:definition} would be complicated, and in some cases unfeasible.
However, luckily we can numerically approximate $\phi(f, x, x')^{IG}$ using a Riemann sum
\begin{equation}
  \phi^{Approx\ IG}_i(f, x, x', m) := (x_{i} - x'_{i})\sum_{k=1}^m\frac{\partial f(x'+\frac{k}{m} (x - x'))}{\partial x_i} \frac{1}{m}
  \label{eq:ig:approx}
\end{equation}

\noindent where $m$ is number of steps for the Riemann sum approximation.

This is when the completeness axiom comes into scene, which is a good value for the parameter $m$? 10, 100, 500? To answer this question, we can simply apply the completeness axiom as a sanity check for the election of $m$. If $m$ is good enough, then the value of $\sum_i \phi^{Approx\ IG}_i(f, x, x', m)$ should be close to $f(x)-f(x')$, or equivalently, the value of $|\sum_i \phi^{Approx\ IG}_i(f, x, x', m) - (f(x)-f(x'))|$ should be close to 0.

Figures \ref{fig:vg:img_gradients} and \ref{fig:vg:img_IG} show a comparison between the gradient of a model output with respect to a cell image, and the \gls{ig}. One can see that either for score maps computed using \gls{ig} or vanilla gradients, the output is noisy and diffuse.


\subsection{VarGrad}
\label{sec:basics:VarGrad}
%% Magic command to compile root document
% !TEX root = ../../thesis.tex

% define where the images are
\graphicspath{{./Sections/Basics/Resources/}}
\glsresetall

As we can see in figure \ref{fig:vg:img_IG}, \gls{ig} attribution maps can be noisy and diffuse. To improve their empirical quality, Smilkov et al. \cite{Smilkov_smoothgrad} proposed \gls{sg}, which tends to reduce noise in practice and can be combined with other attribution map algorithms (like \gls{ig}). The idea behind \gls{sg} is pretty simple, given an input image $x$, you create a sample of similar images by adding noise, then compute the attribution map for each one of them using the algorithm you prefer (in our case \gls{ig}), and take the average of the attribution maps.
Although Smilkov et al. do not provide a mathematical proof of why \gls{sg} reduce noise in score maps, they provide a conjecture and empirical evidence.
For this work we use a slightly different version called \gls{vg}, proposed by Adebayo et al. \cite{adebayo2018local} but inspired by \gls{sg}, which takes the variance of the attribution maps instead of the mean. The reason for this choice is that Seo et al. \cite{Seo_noise} analyzed theoretically \gls{vg}, and concluded that it is independent to the gradient and capture higher order partial derivatives.

In general, \gls{vg} is defined as follow

\begin{equation}
  \phi^{SG}(f, x) := Var(\phi(f, x + z_j))
\end{equation}

\noindent where $x \in \mathbb{R}^{d \times d \times c}$ is the input image, $f:\mathbb{R}^{d \times d \times c} \rightarrow \mathbb{R}$ a model, $\phi$ an attribution method to get preliminary score maps and $z_j \sim \mathcal{N}(0, \sigma^2)$, with $j\in\{1, \dots, n\}$, are i.i.d. noise images of same shape as the input image.

Since we use \gls{ig} to get preliminary score maps, in our case \gls{vg} (in the subsequent defined as \gls{vgig}) looks as follow

\begin{equation}
  \phi^{SG}(f, x) := Var(\phi^{IG}(f, x + z_j, x'))
\end{equation}

\noindent where $x' \in \mathbb{R}^{d \times d \times c}$ is a given baseline needed to compute the \gls{ig} score maps.

Figures \ref{fig:vg:img_IG} and \ref{fig:vg:img_VG_IG} show a comparison between \gls{ig} and \gls{vgig} score maps. One can see that \gls{vgig} produces less noisy score maps than vanilla \gls{ig}.

% this plots were created using the notebook ~/Documents/Master_Thesis/Project/workspace/Interpretability/Integrated_Gradient_Sanity_check.ipynb
\begin{figure}[htb]
  \centering
  \begin{subfigure}[b]{.45\linewidth}
    \includegraphics[width=\linewidth]{Cell_Image.jpg}
    \caption{Original cell image.}
    \label{fig:vg:cell_img}
  \end{subfigure}
  \begin{subfigure}[b]{.45\linewidth}
    \includegraphics[width=\linewidth]{Image_Gradient.jpg}
    \caption{Gradient wrt the input image.}
    \label{fig:vg:img_gradients}
  \end{subfigure}%
  \vspace{3mm}
  \begin{subfigure}[b]{.45\linewidth}
    \includegraphics[width=\linewidth]{Integrated_Gradient.jpg}
    \caption{Integrated Gradient.}
    \label{fig:vg:img_IG}
  \end{subfigure}
  \begin{subfigure}[b]{.45\linewidth}
    \includegraphics[width=\linewidth]{VarGrad_Integrated_Gradient.jpg}
    \caption{VarGrad with Integrated Gradients.}
    \label{fig:vg:img_VG_IG}
  \end{subfigure}
  \caption{Comparison between a cell image and the different attribution methods. All the figures show the same 3 channels taken from a cell image. \subref{fig:vg:cell_img}) cell image, i.e. no attribution method. \subref{fig:vg:img_gradients}) score map using only the gradient of the model with respect to the input image. \subref{fig:vg:img_IG}) \acrlong{ig} score map. \subref{fig:vg:img_VG_IG}) \acrlong{vgig} score map.}
  \label{fig:vg:comparative}
\end{figure}


\section{Interpretability Methods Evaluation}
\label{sec:basics:vgig_eval}

%% =============================================================================
%% Dataset chapter
%% =============================================================================
\chapter{The Dataset}
\label{ch:dataset}
%% Magic command to compile root document
% !TEX root = ../../thesis.tex

\glsresetall
% define where the images are
\graphicspath{{./Sections/Results/Resources/}}

This chapter is mainly divided into two parts

\begin{enumerate}
    \item The model performance analysis
    \item The model interpretation analysis
\end{enumerate}

In the first part, we show the results of the models introduced in section \ref{sec:methodology:models}, and compare the performance of each model against the others based on the metrics introduced in section \ref{sec:methodology:metrics}. Besides, the performance of each model is compared to the reference values of the metrics, which validate that the models are capable of learning something meaningful from the data. This section ends by analyzing more in-depth the performance of the linear model against one of the \glspl{cnn} models, and shows that the latter is still capable of predicting fairly well the \gls{tr}, even after significantly reducing the pixel intensity information of each channel and the correlation between them (by means of the data augmentation techniques introduced in section \ref{sec:dataset:data_augmentation} and \ref{sec:methodology:tfds}).

On the other hand, the Model interpretation section focuses on analyzing the results obtained using the interpretability methods introduced in section \ref{sec:basics:interpretability_methods}. The analysis begins with the division of the cells into three levels of transcription (low, medium and high), and ends by analyzing how the model changes the areas of interest in the input image as the \gls{tr} changes. 
Furthermore, the analysis shows that as the \gls{tr} increases, the model relies more on regions of the nucleus that are directly related to the genesis of mature \gls{mrna}, which shows that interpretation methods can be used as tools to discover unknown biological relationships when applied to black-box models like \glspl{cnn}.

The results of both sections also show that it is possible to predict (to some extent) the \gls{tr} of a cell, based mainly on spatial information within the nucleus.

\section{Multiplexed Protein Maps}
\label{sec:dataset:multiplexed_protein_maps}
%% Magic command to compile root document
% !TEX root = ../../thesis.tex

%% Reset glossary to show long gls names
\glsresetall

%% Set path to look for the images
\graphicspath{{./Sections/Dataset/Resources/}}

% A small motivation to create Multiplexed Protein Maps
The amount of protein or \gls{mrna} inside a cell may not be enough to fully describe cellular function. Accordingly to Buxbaum et al. \cite{Buxbaum_2014} and Korolchuk et al. \cite{Korolchuk2011}, cellular function can heavily depends on the specific intracellular location and interaction with other molecules and intracellular structures. Therefore, cellular expression is determined by the functional state, abundance, morphology, and turnover of its intracellular organelles and cytoskeletal structures. This means that having the ability to look at the concentration and distribution of different molecules within a cell, is an important technological achievement that can significantly leverage scientific discoveries in biomedicine.
This is exactly what \gls{mpm} allows us to do (\cite{Guteaar7042}). \gls{mpm} are protein readouts from cell cultures, that simultaneously captures different properties of the cell, like its shape, cycle state, detailed morphology of organelles, nuclear subcompartments, etc. It also captures highly multiplexed subcellular protein maps, which can be used to identify functionally relevant single-cell states, like \gls{tr}. These maps can also identify new cellular states and allow quantitative comparisons of intracellular organization between single cells in different cell cycle states, microenvironments, and drug treatments \cite{Guteaar7042}.

So, let us explain more in deept what are these \gls{mpm}. Accordingly to Gabriele Gut et al. \cite{Guteaar7042}, \gls{mpm} is a nondegrading protocol that allows to capture efficiently thousands of single cell multichannel images, where each channel contains the distribution and concentration of a protein of interest inside each cell. To achieve this, the protocol is made up of different steps that will be briefly explained here.

% 4i explanation
\subsubsection{Iterative indirect immunofluorescence imaging}
The \gls{mpm} protocol starts with a process called \gls{4i} developed by the same group. The \gls{4i} is a complete protocol by itself, and it allows to capture the concentration and distribution of individual proteins in thousands of different cells in a tissue\footnote{The tissues were made from cell cultures using the \hl{HeLa Kyoto} \hl{184A1} cell line. HeLa is the oldest and most commonly used immortal human cell line in scientific research. The story behind it is quite interesting, so it's worth checking out.}.
Before applying the \gls{4i} protocol, the \hl{plate} where the cell culture is must be divided into squared sections called \hl{wells}. Then, the \gls{4i} protocol is applied over each well and photographed in sections called \hl{sites}.

Roughly speaking, \gls{4i} works as follow
\begin{enumerate}
  % 1
  \item The selected well is prepared for the staining-elution process.
  %2
  \item The well is saturated with a liquid containing \hl{antibodies}\footnote{An antibody is a Y-shaped protein that can recognize and bind to a unique molecule (its antigen, e.g. another protein).} stained with a fluorescent ink (\gls{if}), which binds to a target protein.
  %3
  \item The well is exposed to a high-energy light and photographed using a light microscopy (which produces a single channel image).
  %4
  \item The antibodies inside the tissue are washed-out using a chemical elution substrate.
  %5
  \item Steps 2 to 4 are repeated 20 times to get 20 images of the same protein.
  %6
  \item To improve the protein readouts, the 20 single channel images are projected into one by \hl{maximum intensity projection}.
\end{enumerate}

Figure \ref{fig:4i:1} illustrates the steps of the \gls{4i} protocol that capture the saturation and distribution of a specific protein. Keep in mind that even though the \gls{4i} protocol captures sever images of the tissue, it returns an uni-channel image (step 6). Figure \ref{fig:4i:2} shows the \gls{4i} protocol applied 40 times with different \gls{if} and over a 384-well plate, which captures the concentration and distribution of 40 different specific proteins.

\begin{figure}[htb]
  \centering
  \begin{subfigure}[t]{.3\linewidth}
    \includegraphics[width=\linewidth]{4i_1.png}
    \caption{\Acrfull{4i} protocol.}
    \label{fig:4i:1}
  \end{subfigure}
  \hspace{4mm}
  \begin{subfigure}[t]{.45\linewidth}
    \includegraphics[width=\linewidth]{4i_2.png}
    \caption{\gls{4i} protocol applied over a specific well of a plate and for 40 different \gls{if}.}
    \label{fig:4i:2}
  \end{subfigure}%
  \caption{Schematic representation of the \gls{4i} protocol for a single well and for 40 different fluorescent antibodies. Figure \subref{fig:4i:2} also shows the image analysis to identify single cells and its components (nucleus and cytoplasm). Images source: \cite{Guteaar7042}.}
  \label{fig:4i}
\end{figure}

By the time \cite{Guteaar7042} was published, the \gls{4i} protocol was able to capture cell culture images with up to 40 channels without degrading the tissue, which is why \gls{mpm} is called a \textit{nondegrading} protocol.

\subsubsection{Multiplexed single cell analysis}

Once the multichannel images were generated using the \gls{4i} protocol, a series of image preprocessing and image analysis methods (\cite{Carpenter2006} and \cite{snijder2012single}) are applied to generate segmentation masks to identify individual cells, as well as their cytoplasm and nucleus. Figure \ref{fig:4i:2} shows this segmentation at a cellular level, while figure \ref{fig:4i:segmentation} shows it also at a subcellular level. In both cases the boundaries are marked with a white contour. This single cell analysis is also used to identify cells that do not satisfy certain quality controls (like cells in the border of the image or in mitosis stage). However, this will be addressed in detail on section \ref{sec:dataset:data_pp}.

\begin{figure}[htb]
  \centering
  \includegraphics[width=0.5\linewidth]{4i_segmentation.png}
  \caption{Visualization of the subcellular segmentation of a \gls{4i} protocol for 18 \gls{if} stains. The image was created by combining the readouts of 3 of this \gls{if} stains: PCNA (cyan), FBL (magenta) and TFRC (yellow). The number next to each staining label indicates their corresponding 4i acquisition cycle (\gls{4i} protocol step 5). The orange rectangle and the tile at its right shows a section of the nucleus and cytoplasm of a single cell. The other 3 tiles shows the \gls{4i} readout of each of the 3 proteins. Images source: \cite{Guteaar7042}.}
  \label{fig:4i:segmentation}
\end{figure}

\subsubsection{Multiplexed single-pixel analysis framework}
Even though the cell cultures are now segmented into individual cells and nucleus, there is still one missing part that must be considered, and that is that cells are 3-dimensional objects. Recall that the \gls{4i} protocol saturates the cell culture with a liquid containing fluorescent antibodies. This means that the antibody can either bind to its corresponding protein inside or outside the cell nucleus. Therefore, even though that we segmented a cell into nucleus and cytoplasm, a readout assigned to the nucleus could come from a protein in the cytoplasm under or above the nucleus, and not from inside it. Fortunately, intensity readouts from proteins inside the nucleus are much higher than those in the cytoplasm. Therefore, by means of a two steps clustering approach\footnote{To identify clusters in an unsupervised manner, \hl{Self Organizing Maps} algorithm and \hl{Phenograph} analysis were used over a very large number of pixels sampled from a large number of single cells \cite{Guteaar7042}.}, pixels can be classified accordingly to their intensity profile (figures \ref{fig:mcu:1} and \ref{fig:mcu:2}), so the source of their readout can be identified. This pixel type classification is called \Acrfull{mcu} and is illustrated in figure \ref{fig:mcu:3}. After pixels clusters (intensity profiles) where identified, the pixels whose measurement comes from the cytoplasm and not from the nucleus are removed.

\begin{figure}[htb]
  \centering
  \begin{subfigure}[t]{.3\linewidth}
    \includegraphics[width=\linewidth]{mcu_1.png}
    \caption{Extraction of pixel intensities.}
    \label{fig:mcu:1}
  \end{subfigure}
  \hspace{4mm}
  \begin{subfigure}[t]{.3\linewidth}
    \includegraphics[width=\linewidth]{mcu_2.png}
    \caption{Pixel clustering by Self Organizing Maps and Phenograph.}
    \label{fig:mcu:2}
  \end{subfigure}
  \hspace{4mm}
  \begin{subfigure}[t]{.3\linewidth}
    \includegraphics[width=\linewidth]{mcu_3.png}
    \caption{Cell subdivision base on the \gls{mcu}.}
    \label{fig:mcu:3}
  \end{subfigure}
  \caption{Figure \subref{fig:mcu:1} shows the pixel intensity extraction for a single cell. The pixel intensity is a vector containing the readout of that 2D location for each protein, one specific protein readout per entrance. Figure \subref{fig:mcu:2} shows the clusters found by Self Organizing Maps algorithm and Phenograph analysis over the pixel intensities. Figure \subref{fig:mcu:3} shows a cell masked with the clusters found by the \gls{mcu} analysis. Images source: \cite{Guteaar7042}.}
  \label{fig:mcu}
\end{figure}

Finally, the nucleus of each cell is stored separately and identified with a unique id.\fxnote{After you finish writing the dataset section review if this sentence is accurate.}

\subsubsection{Cell cycle phase classification: $G_1,\ S,\ G_2$ and $M$ phase}

The \gls{mpm} protocol is not only capable to capture the concentration and distribution of molecules inside thousands of cells. It can also identify the phase each cell is in, which is tightly related with the abundances and distribution of molecules inside a cell \cite{Guteaar7042}.

Roughly speaking, cell cycle phase was determined by means of a \gls{svm} classifier and k-means clustering. First, a \gls{svm} classifier is trained to identify $M$ phase cells based on the nuclear information in one of the image channels (\hl{DAPI}\footnote{A brief description of this marker can be found on section \ref{sec:appendix:if_markers}.}). Then, based on the nuclear information of channel \hl{PCNA}, a second \gls{svm} classifier is trained to identify cells in phase $S$. Finally, cells in phase $G_1$ and $G_2$ are classified using a k-means algorithm, using the pixel intensity profiles of the DAPI channels excluding the cells in $S$ and $M$ phase. A more detailed explanation of the cell cycle classification process can be found on the dataset paper \cite{Guteaar7042}.

\subsubsection{Pharmacological and metabolic perturbations}

To further explore the capabilities of the \gls{mpm} protocol, the creators of the dataset (Gabriele Gut et al. \cite{Guteaar7042}) applied the \gls{mpm} protocol to a cell populations that were to nine pharmacological and metabolic perturbations. The analysis reveled expected and unexpected changes in the concentration and distribution of molecules inside the cell. However, this work focused on cells without pharmacological and metabolic perturbations. This means that only cells marked as \hl{normal} (no perturbed cells) and \hl{DMSO}\footnote{Dimethyl sulfoxide, or DMSO, is an organic compound used to dissolve test compounds in in drug discovery and design \cite{cushnie2020bioprospecting}.} (control cells) were used.


\section{Data preprocessing}
\label{sec:dataset:data_pp}
%% Magic command to compile root document
% !TEX root = ../../thesis.tex

\glsresetall
% define where the images are
\graphicspath{{./Sections/Dataset/Resources/}}

\noindent The data preprocessing consist of 4 main steps

\begin{enumerate}
  \item The raw data processing, where raw files are converted into images.
  \item The quality control, where cells that are not useful for analysis are discarded.
  \item The creation of the dataset, where data is spitted into \hl{Train, validation} and \hl{Test} sets and stored in a way that can be used for model training efficiently.
  \item The image preprocessing, where the images are prepared before training the model (clipping and standardization).
\end{enumerate}

In this section we explain these 4 steps. However, the implementation is discussed in the sections \ref{sec:appendix:raw_data} (for steps 1 and 2) and \ref{sec:appendix:tfds} (for steps 3 and 4).

\subsection{Raw data processing}
\label{sec:dataset:data_pp:raw_data_p}

As we mentioned in section \ref{sec:dataset:multiplexed_protein_maps}, the \gls{mpm} protocol is applied over section of cell cultures called \hl{wells}. The \gls{mpm} protocol will return several files for each well, containing the nuclear protein readouts of single cells, information from the subsequent analysis made to the intensities of the protein readouts, as well as information about the \gls{mpm} protocol experimental setup. We do not go into details about this files and how to transform them into multichannel images of single cell nucleus. However, a brief explanation of this can be found in the appendix \ref{sec:appendix:raw_data}. Appendix \ref{sec:appendix:raw_data} also show how to run the Python script that transforms the raw data into images, along with an explanation of the required parameters.

The Python script introduced on appendix \ref{sec:appendix:raw_data} extract the protein readouts from the raw data files, and use them to build multichannel images containing the nucleus of a single cell (see figure \ref{fig:data_pp:sample_cell:nucleus}). This means that during the reconstruction of the images, it is necessary to add black pixels (zeros) in the places where no measures were taken (like in the low corner of figure \ref{fig:data_pp:sample_cell:nucleus}). However, as we saw on section \ref{sec:basics:CNN}, in order to train a \gls{cnn} model, all the cell images need to have a fixed size, which is denoted as $I_s$. For this reason, after the image is reconstructed, it is necessary to add zeros to the images borders (zero-padding) in order to make it squared and of a fixed size (see figure \ref{fig:data_pp:sample_cell:nucleus_pad}). Finally, for each single cell nucleus, a \hl{cell mask} is created to keep track of the measured and non-measured pixels (see figure \ref{fig:data_pp:sample_cell:cell_mask}). As we can see in figure \ref{fig:data_pp:sample_cell}, the cell nucleus is always located in the center of the image.

\begin{figure}[htb]
  \centering
  \begin{subfigure}[t]{.211\linewidth}
    \includegraphics[width=\linewidth]{cell_nucleus.jpg}
    \caption{Single cell nucleus.}
    \label{fig:data_pp:sample_cell:nucleus}
  \end{subfigure}
  \hspace{4mm}
  \begin{subfigure}[t]{.3\linewidth}
    \includegraphics[width=\linewidth]{cell_nucleus_w_pad.jpg}
    \caption{Single cell nucleus with zero-padding.}
    \label{fig:data_pp:sample_cell:nucleus_pad}
  \end{subfigure}
  \hspace{4mm}
  \begin{subfigure}[t]{.3\linewidth}
    \includegraphics[width=\linewidth]{cell_mask.jpg}
    \caption{Single cell nucleus mask.}
    \label{fig:data_pp:sample_cell:cell_mask}
  \end{subfigure}
  \caption{Figure \subref{fig:data_pp:sample_cell:nucleus} shows channels 10, 11 and 15 of the nucleus of a single cell multichannel image reconstructed form the raw data. Figure \subref{fig:data_pp:sample_cell:nucleus_pad} shows image \subref{fig:data_pp:sample_cell:nucleus} after adding zero to the borders (zero-padding) to make it of size 224 by 224 pixels. Figure \subref{fig:data_pp:sample_cell:cell_mask} shows the cell mask, i.e. measured pixels (in white) during the \gls{mpm} protocol.}
  \label{fig:data_pp:sample_cell}
\end{figure}

The raw data processing script saves in a specified directory files containing 3 compressed NumPy arrays; 1) the multichannel image (figure \ref{fig:data_pp:sample_cell:nucleus_pad}), a 3D array contains the protein readouts of the nucleus of a single cell 2) the cell mask (figure \ref{fig:data_pp:sample_cell:cell_mask}), a 2D array that indicates the measured pixels by the \gls{mpm} protocol (ones on the measured $x$ and $y$ coordinates and zeros otherwise) and 3) the channels average, a 1D array containing the average of the measured pixels per channel/protein. Each file is named using the unique id assigned to each single cell nucleus (\texttt{mapobject\_id\_cell}). The script also returns a \texttt{csv} file\footnote{This \texttt{csv} file can be easily opened as a \hl{Pandas DataFrame}. For more information, please refer to the \href{https://pandas.pydata.org/pandas-docs/stable/reference/api/pandas.DataFrame.html}{official documentation}.} containing the metadata of each single cell from every processed well (one row per cell and one column per cell feature). Table \ref{table:dataset:metadata} shows the metadata columns that were relevant for this work.

% set table lengths
\setlength{\mylinewidth}{\linewidth-7pt}%
\setlength{\mylengtha}{0.3\mylinewidth-2\arraycolsep}%
\setlength{\mylengthb}{0.7\mylinewidth-2\arraycolsep}%

\begin{table}[!ht]
  \centering
  \begin{tabular}{>{\centering\arraybackslash}m{\mylengtha}|m{\mylengthb}} % m stands for middle (p:top, b:bottom), max 144 mm
    \hline
    Column name & Description \\
    \hline
    \texttt{mapobject\_id\_cell} & ID to uniquely identify each cell among all wells \\
    \hline
    \texttt{mapobject\_id} & ID to uniquely identify each cell on its well \\
    \hline
    \texttt{is\_border\_cell} & Binary flag, 1 if the cell is on the plate, well or site border; 0 if not \\
    \hline
    \texttt{cell\_cycle} & String, \texttt{G1} if cell is in $G_1$ phase, \texttt{S} if cell is in synthesis phase, \texttt{G2} if cell is in $G_2$ phase. If \texttt{NaN}, then the cell is in mitosis phase \\
    \hline
    \texttt{is\_polynuclei\_184A1} & Binary flag for \hl{184A1} cells, 1 if the cell was identified to have more than one nucleus (i.e. it is in mitosis phase); 0 if not\\
    \hline
    \texttt{is\_polynuclei\_HeLa} & Binary flag for \hl{HeLa} cells, 1 if the cell was identified to have more than one nucleus (i.e. it is in mitosis phase); 0 if not\\
    \hline
    \texttt{perturbation} & String indicating the pharmacological/metabolic perturbation \\
    \hline
  \end{tabular}
  \caption{Relevant metadata columns.}
  \label{table:dataset:metadata}
\end{table}

\subsection{Quality Control}
\label{sec:dataset:data_pp:qc}

During the transformation from raw data into images, cells that does not pass a quality control are discriminated. This quality control consist on avoiding cells that holds at least one of the following conditions
\begin{enumerate}
  \item The cell is in mitotic phase (i.e. on metadata, either \texttt{is\_polynuclei\_HeLa} or \texttt{is\_polynuclei\_184A1} is equal to 1 or \texttt{cell\_cycle} is \texttt{NaN}).
  \item The cell is in the border of the plate, well or site (i.e. on metadata, \texttt{is\_border\_cell} is equal to 1).
\end{enumerate}

The quality control is performed by the same script that transforms the raw data into multichannel images. Its implementation and execution, as well as an explanation of the required parameters, can be found on appendix \ref{sec:appendix:raw_data}.

\subsection{Dataset creation}
\label{sec:dataset:data_pp:dataset_creation}

After the raw data from all wells were processed, and mitotic and/or border cells were eliminated (quality control), we are able to build a dataset\footnote{For this work we decided to use (and build) a custom \acrfull{tfds}, which is a subclass of \texttt{tensorflow\_datasets.core.DatasetBuilder} and allows to create a pipeline that can easily feed data into a machine learning model built using TensorFlow. For more information, please refer to the \href{https://www.tensorflow.org/datasets/add_dataset}{official documentation}.} that can be used efficiently to train models. We will not explain here how to create this dataset. However, a brief explanation of this can be found in the appendix \ref{sec:appendix:tfds}. Appendix \ref{sec:appendix:tfds} also show how to run the Python script that builds this dataset, along with an explanation of the required parameters.

Even though this script can bu used to build a dataset containing all available single cell images, for this work we created a dataset containing cells without pharmacological or metabolic perturbations (i.e. cells such that in the metadata \texttt{perturbation} is ether equal to \hl{normal} or \hl{DMSO}). Further more, during the creation of the dataset, it is possible to filter the image channels and select the target value from the channels average vector (which is stored along with each single cell image). In this case we kept all the input channels\footnote{The unnecessary/unwanted channels are removed during the model training/evaluation (see section \ref{sec:methodology:models}). The reason why this filtering is not made during the dataset creation, is to make the dataset set more robust (i.e. to avoid the need to create a new dataset each time the input channels of the image changed).}, except for the channel used to calculate the target value. This means that channel 35 was excluded (\texttt{00\_EU}\footnote{A brief description of this marker can be found on section \ref{sec:appendix:if_markers}.}), and entrance 35 from the channel average vector (interpreted as \gls{tr}) was selected as target value.

Last but not least, for each cell, its mask is added at the end as an extra channel to keep track of the measured pixels. The reason why the cell mask is stored as a channel, is because it will be needed by other process latter in the pipeline (some of the data augmentation techniques, see section \ref{sec:dataset:data_augmentation}). However, this (and other channels) are removed before the image is used to feed the model (during and after the training process, see section \ref{sec:methodology:models}).

Table \ref{table:tfds_in:channels} (on appendix \ref{sec:appendix:tfds}) shows the image channels in the \gls{tfds}, including the name (column \hl{Channel name}) and identifier of each immunofluorescence markers (column \hl{Marker identifier}). Table \ref{table:tfds_in:channels} also shows the ids corresponding to the markers in the raw data (column \hl{Raw data id}) and in the \gls{tfds} (column \hl{TFDS id}). \hl{NA} means that the channel is not used/available either on the raw data or the \gls{tfds}.

% Data extracted form notebook Preprocessing_resources.ipynb
\begin{table}[!ht]
  \centering
  \begin{tabular}{c|c|c}
    Set & Size & Percentage \\
    \ChangeRT{1.7pt}
    Train & 2962 & $80\%$ \\
    \hline
    Validation & 371 & $10\%$ \\
    \hline
    Test & 370 & $10\%$ \\
    \ChangeRT{1.7pt}
    Total & 3703 & $100\%$ \\
  \end{tabular}
  \caption{Distribution of the dataset partitions.}
  \label{table:data_pp:dataset_dist}
\end{table}

During the creation of the dataset, the images are also spitted into 3 sets, \hl{Train, Validation} and \hl{Test}, using the proportions $80\%$, $10\%$ and $10\%$ respectively. Table \ref{table:data_pp:dataset_dist} shows the size of these 3 sets.

% Data extracted form notebook Preprocessing_resources.ipynb
\begin{table}[!ht]
  \centering
  \begin{tabular}{c|c|c|c}
    Set & Cell Cycle & Size & Percentage \\
    \ChangeRT{1.7pt}
    \multirow{3}{*}{Train} & $G_1$ & 1652 & $55.77\%$ \\
    \cline{2-4}
    & $S$ & 864 & $29.17\%$ \\
    \cline{2-4}
    & $G_2$ & 446 & $15.06\%$ \\
    \hline
    \multirow{3}{*}{Validation} & $G_1$ & 205 & $55.41\%$ \\
    \cline{2-4}
    & $S$ & 103 & $27.84\%$ \\
    \cline{2-4}
    & $G_2$ & 62 & $16.76\%$ \\
    \hline
    \multirow{3}{*}{Test} & $G_1$ & 213 & $57.41\%$ \\
    \cline{2-4}
    & $S$ & 103 & $27.76\%$ \\
    \cline{2-4}
    & $G_2$ & 55 & $14.82\%$ \\
    \ChangeRT{1.7pt}
    \multirow{3}{*}{Total} & $G_1$ & 2070 & $55.90\%$ \\
    \cline{2-4}
    & $S$ & 1070 & $28.90\%$ \\
    \cline{2-4}
    & $G_2$ & 563 & $15.20\%$ \\
  \end{tabular}
  \caption{Distribution of the dataset partitions by cell phase (cell cycle).}
  \label{table:data_pp:dataset_dist_cc}
\end{table}

% Data extracted form notebook Preprocessing_resources.ipynb
\begin{table}[!ht]
  \centering
  \begin{tabular}{c|c|c|c}
    Set & Perturbation & Size & Percentage \\
    \ChangeRT{1.7pt}
    \multirow{2}{*}{Train} & Normal & 2040 & $68.87\%$ \\
    \cline{2-4}
    & DMSO & 922 & $31.13\%$ \\
    \hline
    \multirow{2}{*}{Validation} & Normal & 257 & $69.46\%$ \\
    \cline{2-4}
    & DMSO & 113 & $30.54\%$ \\
    \hline
    \multirow{2}{*}{Test} & Normal & 260 & $70.08\%$ \\
    \cline{2-4}
    & DMSO & 111 & $29.92\%$ \\
    \ChangeRT{1.7pt}
    \multirow{2}{*}{Total} & Normal & 2557 & $69.05\%$ \\
    \cline{2-4}
    & DMSO & 1146 & $30.95\%$ \\
  \end{tabular}
  \caption{Distribution of the dataset partitions by perturbation.}
  \label{table:data_pp:dataset_dist_per}
\end{table}

Since we are dealing with cells in different phases (cell cycles), it is important that the distribution of the 3 phases is kept  in the train, validation and test sets\fxnote{Should I mention than half of the cells are in $G_1$ phase, which means that cell in $G_1$ phase apply more pressure on the optimization of the model parameters during training, while the cells in the $G_2$ phase will not? And/Or should I mention this in the conclusions as a future work?}. The same must happen with the proportion of cells without pharmacological/metabolic perturbation (\hl{Normal} cells) and control cells (\hl{DMSO} cells). Tables \ref{table:data_pp:dataset_dist_cc} and \ref{table:data_pp:dataset_dist_per} show respectively that these proportions are hold across the 3 sets.

\subsection{Image preprocessing}

In this work we use \glspl{cnn} and images of cell nucleus to predict \gls{tr}. This means that there are two main features of the images that came into account when the model learns and predicts the \gls{tr}, the spatial distribution of the elements in the image and the intensity of the colors.
However, this work aims to explain and predict transcription based on the information encoded in the spatial distribution of proteins and organelles within the nucleus. Therefore, the image preprocessing techniques applied here should help mitigate the influence of color during training and prediction, so that the model can focus only on spatial information. For this reason, two preprocessing techniques are applied to each cell image, clipping and standardization. The clipping, as well as the standardization, are performed during the construction of the \gls{tfds}, which can be consulted in appendix \ref{sec:appendix:tfds}.

\subsubsection{Clipping}

The idea of clipping is to avoid extreme outliers to influence or leverage the model parameters during training. Figure \ref{fig:data_pp:outlier} gives an example of this. The blue line shows a model fitted including the outliers (the two dots on the right upper corner), while the orange line a model fitted without them.

% Figure created with notebook Preprocessing_resources.ipynb
\begin{figure}[htb]
  \centering
  \includegraphics[width=0.5\linewidth]{outlier.jpg}
  \caption{Comparison between two linear regression models, fitted with (blue line) and without (orange line) outliers.}
  \label{fig:data_pp:outlier}
\end{figure}

To prevent high intense pixels to influence the model, we truncate/limit the value of pixels that are above a certain threshold. This threshold is different for each image channel and is determined using the cell images belonging to the training set. For each channel, the train images are loaded and the threshold is set as the $98\%$ percentile of the measured pixel intensities belonging to the channel. Then, using this threshold vector (one entrance per channel) all the images in the dataset (train, validation and test) are clipped. This is done before the data standardization. Finally, the clipping parameter (threshold) of each channel is stored in a metadata file, provided along with the \gls{tfds}. Figures \ref{fig:data_pp:pixel_dist:ori} and \ref{fig:data_pp:pixel_dist:clip} show the pixel intensity distribution of channel HDAC3 before and after clipping respectively.

\subsubsection{Standardization}

As we mentioned at the beginning of this section, to predict cell \gls{tr} we seek the model to rely on spatial information, rather than the intensity of the pixels. Therefore, to reduce pixel intensity influence, we apply per-channel standardization, which is just a shift and rescaling (a linear transformation) of the original data. Standardization is also called \hl{Z-score}, since the data is transformed using the mean $\mu$ and standard deviation $\sigma$ (normal distribution parameters) of a sample, as a shift and rescaling parameters respectively. As it is done in clipping, the standardization parameters are different for each channel and are computed using the images belonging to the training set. For all the measured pixels intensities in the \gls{tfds} (i.e. for train, validation and test sets), the standardization of pixel $i$ belonging to channel $c$ (i.e. $z_{i,c}$), is done as follow

\begin{equation}
  z_{i,c} = \frac{x_{i,c} - \mu_c}{\sigma_c}
  \label{eq:data_pp:z-score}
\end{equation}

\noindent where $x_{i,c}$ is the corresponding readout $i$ from channel $c$, and $\mu_c$, $\sigma_c$ are the mean and standard deviation (respectively) of channel $c$ computed using the training images.

The standardization centers the measured pixels of each channel around 0 (see figures \ref{fig:data_pp:pixel_dist:clip} and \ref{fig:data_pp:pixel_dist:clip_z}), reducing the color correlation between channels, which also reduce pixel intensity influence over the model.

% Plots made using notebook Data_PP_clipping_and_standard.ipynb
\begin{figure}[htb]
  \centering
  \begin{subfigure}[t]{.29\linewidth}
    \includegraphics[width=\linewidth]{Pixel_dist.png}
    \caption{Pixel intensity distribution.}
    \label{fig:data_pp:pixel_dist:ori}
  \end{subfigure}
  \hspace{4mm}
  \begin{subfigure}[t]{.3\linewidth}
    \includegraphics[width=\linewidth]{Pixel_dist_clip.png}
    \caption{Pixel intensity distribution after clipping.}
    \label{fig:data_pp:pixel_dist:clip}
  \end{subfigure}
  \hspace{4mm}
  \begin{subfigure}[t]{.28\linewidth}
    \includegraphics[width=\linewidth]{Pixel_dist_clip_z.png}
    \caption{Pixel intensity distribution after clipping and standardization.}
    \label{fig:data_pp:pixel_dist:clip_z}
  \end{subfigure}
  \caption{Intensity distribution of measured pixels for channel HDAC3. The channel readouts were taken from the training set. Figure \subref{fig:data_pp:pixel_dist:ori}) shows the distribution without any modification. Figure \subref{fig:data_pp:pixel_dist:clip}) shows the distribution after applying $98\%$ percentile clipping, while figure \subref{fig:data_pp:pixel_dist:clip_z}) shows the distribution after applying same clipping and standardization.}
  \label{fig:data_pp:pixel_dist}
\end{figure}

Figure \ref{fig:data_pp:cell_sample} shows 3 different cell nucleus sampled from the resulting \gls{tfds}. Each nucleus is in a different cell phase ($G_1$, $S$ and $G_2$ respectively), and shows a different group of 3 markers (channels).

% Figure created with notebook Preprocessing_resources.ipynb
\begin{figure}[htb]
  \centering
  \includegraphics[width=\linewidth]{ds_sample.jpg}
  \caption{Cell nucleus in phases $G_1$, $S$ and $G_2$ respectively. Each nucleus shows a different group of 3 markers.}
  \label{fig:data_pp:cell_sample}
\end{figure}


\section{Data augmentation}
\label{sec:dataset:data_augmentation}
%% Magic command to compile root document
% !TEX root = ../../thesis.tex

\glsresetall
% define where the images are
\graphicspath{{./Sections/Dataset/Resources/}}


\section{Discussion}
\label{sec:dataset:discussion}
%% Magic command to compile root document
% !TEX root = ../../thesis.tex

%% Reset glossary to show long gls names
\glsresetall
\graphicspath{{./Sections/Results/Resources/}}

Since the models were trained on computer cluster\footnote{reference to the HMGU} where the resources of each node are shared among all users (GPUs, CPUs, storage), the training time depended on the node assigned by the job scheduling system (SLURM) and the concurrence of users in the cluster. For this reason, it is not possible to provide a a fair comparison for training time between models.

Because of a lack of time, it was not possible to repeat the training sever times and see if the results are consistent.


%% =============================================================================
%% Methodology chapter
%% =============================================================================
\chapter{Methodology}
\label{ch:methodology}
%% Magic command to compile root document
% !TEX root = ../../thesis.tex

\glsresetall
% define where the images are
\graphicspath{{./Sections/Results/Resources/}}

This chapter is mainly divided into two parts

\begin{enumerate}
    \item The model performance analysis
    \item The model interpretation analysis
\end{enumerate}

In the first part, we show the results of the models introduced in section \ref{sec:methodology:models}, and compare the performance of each model against the others based on the metrics introduced in section \ref{sec:methodology:metrics}. Besides, the performance of each model is compared to the reference values of the metrics, which validate that the models are capable of learning something meaningful from the data. This section ends by analyzing more in-depth the performance of the linear model against one of the \glspl{cnn} models, and shows that the latter is still capable of predicting fairly well the \gls{tr}, even after significantly reducing the pixel intensity information of each channel and the correlation between them (by means of the data augmentation techniques introduced in section \ref{sec:dataset:data_augmentation} and \ref{sec:methodology:tfds}).

On the other hand, the Model interpretation section focuses on analyzing the results obtained using the interpretability methods introduced in section \ref{sec:basics:interpretability_methods}. The analysis begins with the division of the cells into three levels of transcription (low, medium and high), and ends by analyzing how the model changes the areas of interest in the input image as the \gls{tr} changes. 
Furthermore, the analysis shows that as the \gls{tr} increases, the model relies more on regions of the nucleus that are directly related to the genesis of mature \gls{mrna}, which shows that interpretation methods can be used as tools to discover unknown biological relationships when applied to black-box models like \glspl{cnn}.

The results of both sections also show that it is possible to predict (to some extent) the \gls{tr} of a cell, based mainly on spatial information within the nucleus.

\section{Dataset Setup}
\label{sec:methodology:tfds}
%% Magic command to compile root document
% !TEX root = ../../thesis.tex

%% Reset glossary to show long gls names
\glsresetall
\graphicspath{{./Sections/Methodology/Resources/}}

In this section we specify all the hyperparameters needed to execute the process explained on chapter \ref{ch:dataset}. This contemplates the raw data processing, the quality control, the \gls{tfds} creation, the image preprocessing, as well as data augmentation.

\subsection{Data preprocessing}

As we explained in section \ref{sec:dataset:data_pp}, the data preprocessing consist of 4 main steps; 1) the raw data processing, 2) the quality control, 3) the creation of the dataset and 4) the image preprocessing.

A complementary explanation of the data preprocessing parameters, as well as implementation references, can be found in the appendices \ref{sec:appendix:raw_data} and \ref{sec:appendix:tfds}.

\subsubsection{Raw data processing}

As we explained in section \ref{sec:dataset:data_pp:dataset_creation}, to build the \gls{tfds} it is necessary to specify the perturbations that will be included in the dataset. For this reason, all the available wells were processed and transformed into images. This included wells exposed to pharmacological and metabolic perturbations, control wells and unperturbed wells. This allows the user to easily create new datasets without having to run the raw data processing first. Table \ref{table:methodology:dataset:raw_data} shows the processed wells separated by perturbation.

\begin{table}[!ht]
  \centering
  \begin{tabular}{>{\centering\arraybackslash}m{0.35\linewidth} | >{\centering\arraybackslash}m{0.2\linewidth} | >{\centering\arraybackslash}m{0.3\linewidth}}
    \hline
    Perturbation type & Perturbation name & Well names \\
    \hline
    \multirow{5}{*}{pharmacological/metabolic} & CX5461 & I18, J22, J09 \\
    \cline{2-3}
     & AZD4573 & I13, J21, J14, I17, J18 \\
    \cline{2-3}
     & meayamycin & I12, I20 \\
    \cline{2-3}
    & triptolide & I10, J15 \\
    \cline{2-3}
    & TSA & J20, I16, J13 \\
    \hline
    control & DMSO & J16, I14 \\
    \hline
    unperturbed & normal & J10, I09, I11, J18, J12 \\
    \hline
  \end{tabular}
  \caption{Well names divided by perturbation name and type.}
  \label{table:methodology:dataset:raw_data}
\end{table}

Another hyperparameter that needs to be specified during the raw data processing, is the size of the output images $I_s$. This size applies to both, the width and height of the image (square images). Since some prebuilt architectures use a standard image size of 224 by 224, we define $I_s$ as 224.

\subsubsection{Quality control}

As it is mentioned in section \ref{sec:dataset:data_pp:qc}, the quality control is meant to exclude cells with undesirable features. In our case we discriminate mitotic and border cells. The information used by the quality control is contained in the metadata of each well. Table \ref{table:methodology:dataset:qc} shows the metadata columns and the discriminated values. If a cell has any of these values, then it is excluded.

\begin{table}[!ht]
  \centering
  \begin{tabular}{>{\centering\arraybackslash}m{0.3\linewidth} | >{\centering\arraybackslash}m{0.3\linewidth} | >{\centering\arraybackslash}m{0.2\linewidth}}
    \hline
    Feature & Metadata column name & Discriminated value \\
    \hline
    \multirow{3}{*}{Cell in mitosis phace} & \texttt{is\_polynuclei\_HeLa} & 1 \\
    \cline{2-3}
     & \texttt{is\_polynuclei\_184A1} & 1 \\
    \cline{2-3}
     &  \texttt{cell\_cycle} & \texttt{NaN} \\
    \hline
    Border cell & \texttt{is\_border\_cell} & 1 \\
    \hline
  \end{tabular}
  \caption{Discrimination characteristics for quality control.}
  \label{table:methodology:dataset:qc}
\end{table}

\subsubsection{Dataset creation and image preprocessing}

As it is explained in section \ref{sec:dataset:data_pp:dataset_creation}, in this work we decided to use a custom \gls{tfds}. Table \ref{table:methodology:dataset:tfds} shows the parameters used to build the dataset employed in this work, together with the image preprocessing parameters.

% set table lengths
\setlength{\mylinewidth}{\linewidth-7pt}%
\setlength{\mylengtha}{0.25\mylinewidth-2\arraycolsep}%
\setlength{\mylengthb}{0.65\mylinewidth-2\arraycolsep}%

\begin{longtable}{>{\centering\arraybackslash}m{\mylengtha} | >{\centering\arraybackslash}m{\mylengthb}}
    \hline
    Parameter & Description \\
    \hline
    Perturbations to be included in the dataset & \hl{normal} and \hl{DMSO} \\
    \hline
    Cell phases to be included in the dataset & $G_1$, $S$, $G_2$ \\
    \hline
    Training set split fraction & 0.8 \\
    \hline
    Validation set split fraction & 0.1 \\
    \hline
    Seed & 123 (for reproducibility of the train, val and test split) \\
    \hline
    Percentile & 98 (for clipping / linear scaling / standardization) \\
    \hline
    Clipping flag & 1 \\
    \hline
    Mean extraction flag & 0  \\
    \hline
    Linear scaling flag & 0 \\
    \hline
    Standardization (z-score) flag & 1 \\
    \hline
    Model input channels & All of them except for channel \texttt{00\_EU} (see table \ref{table:tfds_in:channels})  \\
    \hline
    Channel used to compute target variable (output channel) & \texttt{00\_EU} (channel id 35, see table \ref{table:tfds_in:channels}) \\
    \hline
  \caption{Parameters used to biuld \gls{tfds} and image preprocessing.}
  \label{table:methodology:dataset:tfds}
\end{longtable}

The custom \gls{tfds} created with the parameters specified in table \ref{table:methodology:dataset:tfds} is called \\
\texttt{mpp\_ds\_normal\_dmso\_z\_score}.

\newpage
The Python script that builds the custom \gls{tfds}, also returns a file with the image preprocessing parameters (\texttt{channels\_df.cvs}) (as this is applied at a per-channel level) and information about the channels (channel name, id, etc.). It also returns another file with the metadata of each cell included in the \gls{tfds} (\texttt{metadata\_df.csv}). These files are stored in the same directory as the \gls{tfds} files.

In table \ref{table:methodology:dataset:tfds} we also specify the channel used to compute the target variable (ground truth), which is the channel corresponding to the marker \hl{EU} (channel id 35, see tables \ref{table:tfds_in:channels} and \ref{table:apendix:if_markers}).
Recall that this channel contains nuclear readouts of nascent RNA (\gls{pmrna}) in a given period of time.
For the data provided, this time period was the same for all the cells (30 minutes) and is specified in the \hl{duration} columns of the metadata.
Since channel 35 is used to compute the target variable (ground truth), it is removed from the prediction/input channels.

\subsection{Data augmentation}
\label{sec:methodology:data:augm}

In this section we specify the data augmentation techniques (see section \ref{sec:dataset:data_augmentation}) and its hyperparameters used to train all the models of this work. Recall that the techniques are either aimed to remove non-relevant characteristics of the data (color shifting, central zoom in/out) or to improve model generalization (horizontal flipping, 90 degree rotations). Table \ref{table:methodology:dataset:augm} shows this techniques and its hyperparameters grouped by objective and technique. In practice, the augmentation techniques are applied as shown in table \ref{table:methodology:dataset:augm} from top to bottom.

% set table lengths
\setlength{\mylinewidth}{\linewidth-7pt}%
\setlength{\mylengtha}{0.2\mylinewidth-2\arraycolsep}%
\setlength{\mylengthb}{0.2\mylinewidth-2\arraycolsep}%
\setlength{\mylengthc}{0.25\mylinewidth-2\arraycolsep}%
\setlength{\mylengthd}{0.22\mylinewidth-2\arraycolsep}%

\begin{longtable}{>{\centering\arraybackslash}m{\mylengtha} | >{\centering\arraybackslash}m{\mylengthb} | >{\centering\arraybackslash}m{\mylengthc} | >{\centering\arraybackslash}m{\mylengthd}}
    \hline
    Objective & Technique & Hyperparameter & Description \\
    \hline
    \multirow{2}{\mylengtha}{\centering Remove non-relevant features} & random color shifting & distribution & $U(-3,3)$ \\
    \cline{2-4}
     & random central zoom in/out & distribution\footnotemark & $N(\mu=0.6, \sigma=0.1)$ \\
    \hline
     \multirow{2}{\mylengtha}{\centering Improve generalization} & random horizontal flipping & NA & NA \\
    \cline{2-4}
     & random 90 degrees rotations & NA & NA \\
    \hline
  \caption{Parameters used for data augmentation techniques. The NA means that there are no hyperparameters for this technique or that there is no further description.}
  \label{table:methodology:dataset:augm}
\end{longtable}

\footnotetext{This distribution is used to sample the \hl{cell nucleus size ratio} $S_{ratio}$ (see section \ref{sec:dataset:data_aug:zoom}) of each cell. However, the parameters for this distribution (mean and standard deviation) were not provided by us. Instead, they were estimated using the information in column \texttt{cell\_size\_ratio} of the \gls{tfds} metadata file. Therefore, the \texttt{return\_cell\_size\_ratio} flag must be set to 1 (True) during raw data processing, so this column is created (see section \ref{sec:dataset:data_pp:raw_data_p} and appendix \ref{sec:appendix:raw_data}).}

Even thought we specify the data augmentation hyperparameters here, in practice these are selected for each model and applied during training. However, all the models showed in this work were trained using the techniques and values shown in table \ref{table:methodology:dataset:augm}. A complementary explanation can be found in appendix \ref{sec:appendix:Model_training_IN}.

In section \ref{sec:dataset:data_augmentation} we mentioned that data augmentation techniques can be applied to both the training set and the validation set. However, we also mentioned that only horizontal flips and 90 degree rotations are applied for the validation set. Furthermore, for the training set these techniques are applied randomly, while for the validation set they are applied deterministically. Therefore, table \ref{table:methodology:dataset:augm} only applies to the training set.


\section{Models}
\label{sec:methodology:models}
%% Magic command to compile root document
% !TEX root = ../../thesis.tex

%% Reset glossary to show long gls names
\glsresetall
\graphicspath{{./Sections/Methodology/Resources/}}

In this section we introduce the models, and its architecture, used in this work. All the models were implemented in TensorFlow 2.2.0. We also specify all the used hyperparameters. Besides this, the appendix \ref{sec:appendix:Model_training_IN} provides a brief explanation of how to train and evaluate all the models introduced here.

In general, all the models where trained using $ReLU$ as activation function for the hidden layers. Also, the identity was used as activation function for the last layer. Table \ref{table:methodology:model:general_hyper}, shows the other general hyperparameters.

% set table lengths
\setlength{\mylinewidth}{\linewidth-7pt}%
\setlength{\mylengtha}{0.35\mylinewidth-2\arraycolsep}%
\setlength{\mylengthb}{0.55\mylinewidth-2\arraycolsep}%

\begin{longtable}{>{\centering\arraybackslash}m{\mylengtha} | >{\centering\arraybackslash}m{\mylengthb}}
    \hline
    Parameter & Description \\
    \hline
    Number of epochs & 800 \\
    \hline
    Early stopping patience & 100 \\
    \hline
    Batch size & 64 \\
    \hline
    TFDS name & \texttt{mpp\_ds\_normal\_dmso\_z\_score} \\
    \hline
    Random seed & 123 \\
    \hline
    Input Channels & all channels such that its TFDS id is in $\{0, \cdots, 32\}$ (see appendix \ref{sec:appendix:tfds}, table \ref{table:tfds_in:channels}) \\
    \hline
  \caption{Hyperparameters used in the training of all the models.}
  \label{table:methodology:model:general_hyper}
\end{longtable}

Even though that the number of epochs is specified in table \ref{table:methodology:model:general_hyper}, if the loss function does not improve (decrease) for more than 100 (i.e. \hl{Early stopping patience}) epochs during training, then the training stops.

Table \ref{table:methodology:model:general_hyper} also indicate the input channels to by used by the model as predictors.
In section \ref{sec:dataset:data_pp:dataset_creation} we mentioned that all the image channels (with the exception of channel \texttt{00\_EU}) were kept during the creation of the \gls{tfds}.
Moreover, since the data augmentation techniques are only applied to the measured pixels of the cell images, the cell mask was added to the image as the last channel.
For this reason the channel filtering process is made inside the model.
This means that after the input layer, the models have a \hl{channel filtering layer}, which basically remove the non-selected channels, by projecting the input image from a space of shape $(bs, 224, 224, 38)$ into a lower one of shape $(bs, 224, 224, 33)$.
This is done just by performing a matrix multiplication between the input batch $B \in \mathbb{R}^{bs \times 224 \times 224 \times 38}$ and a projection matrix $P \in \{0,1\}^{38 \times 33}$ (a zero matrix with ones on the diagonal elements corresponding with to the input channels), i.e. $B_{filterd}=BP$.

All the model in this work were trained using the \hl{Huber} loss function

\begin{equation}
  \mathcal{L}_{\delta}(y,f(x)) =
    \begin{cases}
      \frac{1}{2}(y - f(x))^2, & \text{for } |y-f(x)|\leq \delta \\
      \delta |y-f(x)| - \frac{1}{2} \delta^2, & \text{otherwise}
    \end{cases}
\end{equation}

\noindent where $\delta>0$ is the value where the Huber loss function changes from a quadratic to linear. The hyperparameter $\delta$ was set to 1 for all the models.

Huber loss function is quadratic when the error $a=|y - f(x)|$ is below the threshold $\delta$ (like the \gls{mse}), but linear when it is above it. This makes Huber loss less susceptible to outliers. Figure \ref{fig:meth:huber_plot} shows a comparison between the Huber (in green) and the \gls{mse} (in blue) loss functions.

\begin{figure}[!ht]
  \centering
  \includegraphics[width=0.5\linewidth]{Diagrams/Huber_loss.png}
  \caption{Huber (green) and the \gls{mse} (blue) loss functions. Image source \cite{huberplot}.}
  \label{fig:meth:huber_plot}
\end{figure}

In section \ref{sec:basics:ANN} we mentioned that we choose the \gls{adam} optimizer to fit the model parameters. With the exception of the \hl{learning rate}, the used parameters were $\beta_1=0.9$, $\beta_2=0.999$ and $\epsilon=1e-07$, which are the default TensorFlow hyperparameters\footnote{For more information pleases refer to the TensorFlow \href{https://www.tensorflow.org/api_docs/python/tf/keras/optimizers/Adam}{official documentation}.}. The learning rate is specified in the section corresponding to each model.


\subsection{Linear Model}
\label{sec:methodology:lm}

\subsection{Baseline CNN}
\label{sec:methodology:BL_CNN}

\subsection{ResNet50V2}
\label{sec:methodology:RN50V2}

\subsection{Model Metrics}
\label{sec:methodology:metrics}

\section{Interpretability Methods}
\label{sec:methodology:interpretability_methods}
\glsresetall
\graphicspath{{./Sections/Methodology/Resources/}}

There are several hyper-parameters that need to be chosen in order to compute the score map for each cell image.

For the \gls{ig} attribution map, recall that in practice computing $\phi^{IG}$ could be unfeasible or computationally very expensive. However, we can approximate $\phi^{IG}$ by means of $\phi^{Approx\ IG}$ (see equation \ref{eq:ig:approx}). Therefore, we need to define the number of steps $m$ for the Riemann sum approximation. In section \ref{sec:basics:IG} we also mentioned the necessity to set a baseline image $x'$, which should contain no information about the image, in order to compute the \gls{ig}. There are several options that can be used, each one of them with different advantages and disadvantages. However, for this work we only implemented two of them: 1) a simple black image (image containing only zeros) and 2) an image filled with Gaussian noise ($\mu=0,\ \sigma=1$). A very good analysis on the choice of the baseline can be found in this reference \cite{sturmfels2020visualizing}.

In section \ref{sec:basics:VarGrad} we saw that for \gls{vg} we need to define 2 parameters, the number of noisy images $n$ (sample size) and the standard deviation $\sigma$ for the the noise distribution.

As a rule of thumbs, a sample should not be smaller than 30, so this could be a feasible option. However, since Smilkov et al. \cite{Smilkov_smoothgrad} showed empirically that no further improvemnt (less noise) in score maps is observed for sample sizes greater than 50, we chose this bound as sample size.

Table \ref{table:VGIG_exp_set:params} shows a summary of the parameters chosen to calculate the \gls{vgig} score maps.

\begin{table}[!ht]
  \centering
  \begin{tabular}{c|c|c}
    Method & Hyperparameter & Value \\
    \hline
    \multirow{2}{*}{\gls{ig}} & $m$ & 70 \\
    \cline{2-3}
     & $x'$ & black image \\
    \hline
    \multirow{2}{*}{\gls{vg}} & $n$ & 50 \\
    \cline{2-3}
     & $\sigma$ & 1 \\
    \hline
  \end{tabular}
  \caption{Parameters to compute score maps.}
  \label{table:VGIG_exp_set:params}
\end{table}

In section \ref{sec:basics:IG}, we mentioned that the \gls{ig} algorithm holds the \textit{Completeness Axiom}, which means that the sum of all the components of the \gls{ig} attribution map must be equal to the difference between the model's output evaluated at the image and the model's output evaluated at the baseline (see equation \ref{eq:ig_completeness}). This property allow us to check empirically if the number of steps $m$ selected for the Riemann sum approximation is sufficiently large. Figure \ref{fig:VGIG_exp_set:m_sanity} shows that for our baseline\fxnote{after finishing, check that the baseline model is still called baseline} model, a random image and $m=70$, the  completeness axiom is satisfied sufficiently well.

\begin{figure}[!ht]
  \centering
  \includegraphics[width=0.8\linewidth]{sanity_check_for_m.jpg}
  \caption{Sanity check for the number of steps $m$ in the Riemann sum to approximate $\phi^{IG}$. The red dotted line represent the difference $f(x)-f(x')$. The blue line represents the value of $\sum_i \phi^{Approx\ IG}_i(f, x, x', m)$ over $\alpha$.}
  \label{fig:VGIG_exp_set:m_sanity}
\end{figure}


%\subsection{Experimental Setup}
%\label{sec:methodology:VarGrad_IG_Experimental_Setup}
%\glsresetall
\graphicspath{{./Sections/Methodology/Resources/}}

There are several hyper-parameters that need to be chosen in order to compute the score map for each cell image.

For the \gls{ig} attribution map, recall that in practice computing $\phi^{IG}$ could be unfeasible or computationally very expensive. However, we can approximate $\phi^{IG}$ by means of $\phi^{Approx\ IG}$ (see equation \ref{eq:ig:approx}). Therefore, we need to define the number of steps $m$ for the Riemann sum approximation. In section \ref{sec:basics:IG} we also mentioned the necessity to set a baseline image $x'$, which should contain no information about the image, in order to compute the \gls{ig}. There are several options that can be used, each one of them with different advantages and disadvantages. However, for this work we only implemented two of them: 1) a simple black image (image containing only zeros) and 2) an image filled with Gaussian noise ($\mu=0,\ \sigma=1$). A very good analysis on the choice of the baseline can be found in this reference \cite{sturmfels2020visualizing}.

In section \ref{sec:basics:VarGrad} we saw that for \gls{vg} we need to define 2 parameters, the number of noisy images $n$ (sample size) and the standard deviation $\sigma$ for the the noise distribution.

As a rule of thumbs, a sample should not be smaller than 30, so this could be a feasible option. However, since Smilkov et al. \cite{Smilkov_smoothgrad} showed empirically that no further improvemnt (less noise) in score maps is observed for sample sizes greater than 50, we chose this bound as sample size.

Table \ref{table:VGIG_exp_set:params} shows a summary of the parameters chosen to calculate the \gls{vgig} score maps.

\begin{table}[!ht]
  \centering
  \begin{tabular}{c|c|c}
    Method & Hyperparameter & Value \\
    \hline
    \multirow{2}{*}{\gls{ig}} & $m$ & 70 \\
    \cline{2-3}
     & $x'$ & black image \\
    \hline
    \multirow{2}{*}{\gls{vg}} & $n$ & 50 \\
    \cline{2-3}
     & $\sigma$ & 1 \\
    \hline
  \end{tabular}
  \caption{Parameters to compute score maps.}
  \label{table:VGIG_exp_set:params}
\end{table}

In section \ref{sec:basics:IG}, we mentioned that the \gls{ig} algorithm holds the \textit{Completeness Axiom}, which means that the sum of all the components of the \gls{ig} attribution map must be equal to the difference between the model's output evaluated at the image and the model's output evaluated at the baseline (see equation \ref{eq:ig_completeness}). This property allow us to check empirically if the number of steps $m$ selected for the Riemann sum approximation is sufficiently large. Figure \ref{fig:VGIG_exp_set:m_sanity} shows that for our baseline\fxnote{after finishing, check that the baseline model is still called baseline} model, a random image and $m=70$, the  completeness axiom is satisfied sufficiently well.

\begin{figure}[!ht]
  \centering
  \includegraphics[width=0.8\linewidth]{sanity_check_for_m.jpg}
  \caption{Sanity check for the number of steps $m$ in the Riemann sum to approximate $\phi^{IG}$. The red dotted line represent the difference $f(x)-f(x')$. The blue line represents the value of $\sum_i \phi^{Approx\ IG}_i(f, x, x', m)$ over $\alpha$.}
  \label{fig:VGIG_exp_set:m_sanity}
\end{figure}


\section{Interpretability Methods Evaluation}
\label{sec:methodology:vgig_eval}

%% =============================================================================
%% Results chapter
%% =============================================================================
\chapter{Results}
\label{ch:results}
%% Magic command to compile root document
% !TEX root = ../../thesis.tex

\glsresetall
% define where the images are
\graphicspath{{./Sections/Results/Resources/}}

This chapter is mainly divided into two parts

\begin{enumerate}
    \item The model performance analysis
    \item The model interpretation analysis
\end{enumerate}

In the first part, we show the results of the models introduced in section \ref{sec:methodology:models}, and compare the performance of each model against the others based on the metrics introduced in section \ref{sec:methodology:metrics}. Besides, the performance of each model is compared to the reference values of the metrics, which validate that the models are capable of learning something meaningful from the data. This section ends by analyzing more in-depth the performance of the linear model against one of the \glspl{cnn} models, and shows that the latter is still capable of predicting fairly well the \gls{tr}, even after significantly reducing the pixel intensity information of each channel and the correlation between them (by means of the data augmentation techniques introduced in section \ref{sec:dataset:data_augmentation} and \ref{sec:methodology:tfds}).

On the other hand, the Model interpretation section focuses on analyzing the results obtained using the interpretability methods introduced in section \ref{sec:basics:interpretability_methods}. The analysis begins with the division of the cells into three levels of transcription (low, medium and high), and ends by analyzing how the model changes the areas of interest in the input image as the \gls{tr} changes. 
Furthermore, the analysis shows that as the \gls{tr} increases, the model relies more on regions of the nucleus that are directly related to the genesis of mature \gls{mrna}, which shows that interpretation methods can be used as tools to discover unknown biological relationships when applied to black-box models like \glspl{cnn}.

The results of both sections also show that it is possible to predict (to some extent) the \gls{tr} of a cell, based mainly on spatial information within the nucleus.

\section{Model Performance}
\label{sec:model_performance}

\subsection{Baseline values}
\label{sec:results:bl_values}

\subsection{Linear Model}
\label{sec:results:lm}

\subsection{Baseline CNN}
\label{sec:results:bl_cnn}

\subsection{ResNet50V2}
\label{sec:results:RN50V2}

\subsection{Model Performance Comparative}
\label{sec:results:comparative}
%% Magic command to compile root document
% !TEX root = ../../thesis.tex

\glsresetall
% define where the images are
\graphicspath{{./Sections/Results/Resources/}}

Table \ref{table:results:model_performance_comparative} shows a comparison of the performance of each model in the test set. Column \textit{Dataset Properties} specifies the information contained in the training data; \textit{structure} indicates only spatial information (which means that per-channel random color shifting was applied as augmentation technique to reduce pixel intensity information), while \textit{color and structure} indicate spatial and pixel intensity information (which means that no random color shifting was applied). The row $\bar{y}$ (in the \hl{Model} column) contains the baseline values for the performance metrics (see section \ref{sec:results:bl_values}). The numbers in bold indicate the models with the best overall performance (i.e., trained using data with and without pixel intensity information) per metric, while the shaded cells indicate the models with the best performance using only spatial data (structure).

% set table lengths
\setlength{\mylinewidth}{\linewidth-7pt}%
\setlength{\mylengtha}{0.17\mylinewidth-2\arraycolsep}%
\setlength{\mylengthb}{0.2\mylinewidth-2\arraycolsep}%
\setlength{\mylengthc}{0.1\mylinewidth-2\arraycolsep}%
\setlength{\mylengthd}{0.1\mylinewidth-2\arraycolsep}%
\setlength{\mylengthe}{0.1\mylinewidth-2\arraycolsep}%
\setlength{\mylengthf}{0.09\mylinewidth-2\arraycolsep}%
\setlength{\mylengthg}{0.1\mylinewidth-2\arraycolsep}%
\setlength{\mylengthh}{0.1\mylinewidth-2\arraycolsep}%


\begin{table}[!ht]
  \centering
  \begin{tabular}{m{\mylengtha} |
                  >{\centering\arraybackslash}m{\mylengthb} |
                  >{\centering\arraybackslash}m{\mylengthc} |
                  >{\centering\arraybackslash}m{\mylengthd} |
                  >{\centering\arraybackslash}m{\mylengthe} |
                  >{\centering\arraybackslash}m{\mylengthf} |
                  >{\centering\arraybackslash}m{\mylengthg} |
                  >{\centering\arraybackslash}m{\mylengthh}
                  }
    \hline
    \centering Model & Dataset Properties & $\bar{e}$ & $s(e)$ & R2 & MAE & MSE & Huber \\
    \ChangeRT{1pt}
    \centering $\bar{y}$ (baseline) & targets avg & 4.86 & 59.99 & -0.01 & 45.56 & 3622 & 45.07 \\
    \hline
    \multirow{2}{\mylengtha}{\centering Linear} & color-structure & 4.06 & 46.83 & 0.38 & 35.26 & 2203 & 34.77 \\
    \cline{2-8}
     & structure & 4.03 & 54.15 & 0.18 & 40.52 & 2941 & 40.02 \\
     \hline
    \multirow{2}{\mylengtha}{\centering Simple CNN} & color-structure & 3.00 & \textbf{41.27} & \textbf{0.52} & \textbf{30.68} & \textbf{1708} & \textbf{30.18} \\
    \cline{2-8}
     & structure & 0.77 & 43.94 & 0.46 & 33.08 & 1926 & 32.59 \\
     \hline
    \multirow{2}{\mylengtha}{\centering ResNet50V2} & color-structure & -1.49 & 42.81 & 0.49 & 32.73 & 1830 & 32.24 \\
    \cline{2-8}
     & structure & \cellcolor[HTML]{d9d9d9}\textbf{0.45} & \cellcolor[HTML]{d9d9d9}43.38 & \cellcolor[HTML]{d9d9d9}0.47 & \cellcolor[HTML]{d9d9d9}31.83 & \cellcolor[HTML]{d9d9d9}1877 & \cellcolor[HTML]{d9d9d9}31.33 \\
     \hline
    \multirow{2}{\mylengtha}{\centering Xception} & color-structure & 6.69 & 41.57 & 0.50 & 31.66 & 1768 & 31.16 \\
    \cline{2-8}
     & structure & 7.23 & 45.50 & 0.41 & 33.92 & 2117 & 33.42 \\
     \hline
  \end{tabular}
  \caption{Model performance comparison. Performance measures where taken from the test set, with and without pixel intensity information (color-structure and structure respectively). Bold cells indicate the model-metric with best general performance. Shaded cells indicate the model-metric with best performance using only spatial (structure) data.}
  \label{table:results:model_performance_comparative}
\end{table}

Table \ref{table:results:model_performance_comparative} shows that, as expected, for both types of training data (color-structure and structure) all the \gls{cnn} models performed better than the linear model in all the performance measures, except for average error $\bar{e}$.
Also, for all the error measures and the $R2$ coefficient, both the linear model and the \gls{cnn} models performed better than $\bar{y}$ (baseline values), which means that the models were able to learn something meaningful from both types of data. Surprisingly, the \hl{ResNet50V2} model was the only one that had a better performance in the structure data.

In general the \hl{simple CNN} was the model with the best performance, while the \hl{ResNet50V2} was the model with the best performance in the structure data. However, it is worth mentioning that the \hl{simple CNN} model stayed behind the \hl{ResNet50V2} model in the structure data, surpassing the \hl{Xception} mode. Nevertheless, the performance of the \hl{simple CNN} model was similar to that of the more complex models during training.
This can be seen in figures \ref{fig:results:train_per_com:cs} and \ref{fig:results:train_per_com:s}, which show the validation \gls{mae} of each model during training.
In these figures we can see that the simple model has visibly less variance than the other two \gls{cnn} models, especially in figure \ref{fig:results:train_per_com:cs}.

\begin{figure}[!ht]
  \centering
  \begin{subfigure}[b]{.9\linewidth}
    \includegraphics[width=\linewidth]{train_comp_c_and_s.jpg}
    \caption{Validation \gls{mae} using data with color and structure.}
    \label{fig:results:train_per_com:cs}
  \end{subfigure}%
  \vspace{3mm}
  \begin{subfigure}[b]{.9\linewidth}
    \includegraphics[width=\linewidth]{train_comp_structure.jpg}
    \caption{Validation \gls{mae} using data only with structure.}
    \label{fig:results:train_per_com:s}
  \end{subfigure}
  \caption{Validation \gls{mae} during training using data with (figure \ref{fig:results:train_per_com:cs}) and without (figure \ref{fig:results:train_per_com:s}) pixel intensity information (color-structure and structure respectively). Each color represent a different model. The dot indicates the epoch in which the model reached its lowest validation \gls{mae}. The gray line indicates the baseline \gls{mae} in the validation set. }
  \label{fig:results:train_per_com}
\end{figure}

The dots in figure \ref{fig:results:train_per_com} indicate the epochs with the best performance with respect to the validation \gls{mae} of each model.
The gray horizontal line corresponds to the \gls{mae} of the baseline evaluated in the validation set (see section \ref{sec:results:bl_values}). Due to early stopping, the number of epochs is not the same for all the models.

Figure \ref{fig:results:train_per_com:s} shows that the \gls{mae} of the linear model was generally higher than the \glspl{mae} of the \gls{cnn} models, which reinforces our hypothesis that to some extent it is possible to describe cell expression, using only spatial information within the cell nucleus.

The \hl{ResNet50V2} and \hl{Xception} models have more than 24 and 21 million of parameters respectively, while the \hl{simple CNN} model has only around 160 thousand. Therefore, the training of these two models require way more computational resources and time than the \hl{simple CNN} model.
However, table \ref{table:results:model_performance_comparative} and figure \ref{fig:results:train_per_com} show that the performance of the \hl{simple CNN} model is similar to the \hl{ResNet50V2} and \hl{Xception}.
Moreover, we observe that the importance maps of the \hl{simple CNN} model (shown in section \ref{sec:results:model_interpretation}), were less noisy and informative than those obtained with the more complex models. For this reason, in subsequent sections we will focus on the \hl{simple CNN} model only.


\section{Model Interpretation}
\label{sec:results:model_interpretation}

\section{Model Interpretation Evaluation}
\label{sec:results:model_inter_eval}

\section{Discussion}

%% =============================================================================
%% conclusion chapter
%% =============================================================================
\chapter{Conclusion}
\label{ch:Conclusion}
%% conclusion.tex
%%

%% ==================
\chapter{Conclusion}
\label{ch:Conclusion}
%% ==================

Nothing new here, only a short recap of the project, it's results, as well as possible future work.

%Future work
As future work, it would be interesting to explore other interpretability  methods or go more in-depth with the current one by, for instance, exploring different baseline for \acrlong{ig} and analyzing their impact on the importance maps.

%%% Local Variables:
%%% mode: latex
%%% TeX-master: "thesis"
%%% End:


%% =============================================================================
%% Appendix chapter
%% =============================================================================
\appendix

\chapter{Remarks on Implementation}
\label{Appendix-Implementation}
This appendix contains notes about how to execute all the scripts and notebooks used in this work. It also contains information about the parameters that need to be specified for each program.
All the scripts and notebooks were written in \texttt{Python} and executed over \texttt{Anaconda}. You can find information about the environment setup, packages version, etc. \href{https://github.com/andresbecker/master_thesis}{here}.

The logic to execute any \texttt{Python} script is always the same
\begin{lstlisting}[language=Bash]
python python_script_name.py -p ./Parameters_file_name.json
\end{lstlisting}

For the \texttt{Jupyter Notebooks} you just have to open it and set the variable \\
\noindent\texttt{PARAMETERS\_FILE} with the absolute path and name of the input parameters file
\begin{lstlisting}[language=Python]
PARAMETERS_FILE = "/path_to_file_dir/Parameters_file_name.json"
\end{lstlisting}

For each script/notebook all the needed parameters have to be specified inside its parameter file only. The format for the parameters file is always \texttt{JSON} and the parameters values are specified in a python-dictionary format.

\fxnote{If there is time, add a section containing common parameters}


\section{Raw data processing and QC implementation notes}
\label{sec:appendix:raw_data}
%% Magic command to compile root document
% !TEX root = ../../thesis.tex

%% Reset glossary to show long gls names
\glsresetall

This appendix contains implementation and technical notes about the data preprocessing process that needs to be performed before the construction of the dataset used to train the models. This process is performed by a single Pyhton script (Jupyter Notebook) and contemplate two main steps

\begin{enumerate}
  \item The reconstruction of single cell images from the raw data (text files).
  \item The discrimination of single cell images accordingly to a quality control.
\end{enumerate}

As it is explained in section \ref{sec:dataset:data_pp}, the protein readout of each well are contained in several files. Here we introduce those that are relevant for this work

\begin{itemize}
  \item \texttt{mpp.npy}: 2D NumPy array. Each row contains the protein readouts (intensities) of each pixel of the well (one column per protein). The values of this array vary from 0 to 65535 i.e. $2^{16}$ i.e. 2 bytes or 16 bits.
  \item \texttt{x.npy}/\texttt{y.npy}: 1D NumPy array. Each entrance contains the $x/y$ coordinate of a pixel of the well protein readouts (i.e. \texttt{x.npy} and \texttt{y.npy} map the protein readouts in \texttt{mpp.npy} with a 2D plane). Accordingly with \cite{Guteaar7042}, the size of a single channel well image is 2560x2160. Therefore, the values in \texttt{x.npy} vary between 1 and 2560 and form 1 to 2160 for \texttt{y.npy}
  \item \texttt{mapobject\_ids.npy}: 1D NumPy array. Each entrance contains an id that maps the protein readouts in \texttt{mpp.npy} with the nucleus of a cell in the well. Each cell nucleus in the well is identified by a unique id.
\end{itemize}

Since files \texttt{mpp.npy}, \texttt{x.npy}, \texttt{y.npy} and \texttt{mapobject\_ids.npy} contains different parts of the well protein readouts, the first dimension of the arrays contained in the files always has the same size.

Beside the files with protein readouts (\texttt{npy} files), each well also comes with two additional \texttt{csv} files\footnote{These \texttt{csv} files can be easily opened as a \hl{Pandas DataFrame}. For more information, please refer to the \href{https://pandas.pydata.org/pandas-docs/stable/reference/api/pandas.DataFrame.html}{official documentation}.} containing further information about each cell in the well

\begin{itemize}
  \item \texttt{metadata.csv}. Contains one raw per single cell nucleus in the well. The mapping between the metadata file and the protein readouts (\texttt{npy} files), is made through the column \hl{mapobject\_id}, which uniquely identify cells (but only within the well). On the other hand, column \hl{mapobject\_id\_cell} uniquely identify cells across all wells. Columns \hl{is\_polynuclei\_HeLa} and	\hl{is\_polynuclei\_184A1} indicate if a cell is in mitosis phase. This metadata file also contains information about the experimental setup, like plate name, well name, site position, etc..
  \item \texttt{channels.csv}. Contains only two columns that maps the immunofluorescence marker name (column \hl{channel\_name}) and the channel id (column \hl{id}) of the protein readouts.
\end{itemize}

The files introduced so far are specific to each well. However, we still need to introduce other 3 files which contains information about all the wells

\begin{itemize}
  \item \texttt{secondary\_only\_relative\_normalisation.csv}. Contains the experimental setup information related to the image capturing process. Among other information, it contains the \hl{background value} of each channel that has to be subtracted from the protein readouts during the reconstruction of the images.
  \item \texttt{cell\_cycle\_classification.csv}. Contains the phase of each cell.
  \item \texttt{wells\_metadata.csv}. Contains more information about the experimental setup. Among other information, it contains the pharmacological/metabolic perturbation applied to each well.
\end{itemize}

To execute the raw data processing, one has to open the Python Jupyter Notebook \\
\noindent\texttt{MPPData\_into\_images\_no\_split.ipynb} and replace the variable \texttt{PARAMETERS\_FILE} with the absolute path and name of the input parameters file before running the notebook. A sample parameter file (\texttt{MppData\_to\_imgs\_no\_split\_sample.json}) is provided along with this work. It contains the parameters used for the experiments shown in this work. Table \ref{table:row_data_in:params} provides an explanation of some of this parameters.

% set table lengths
\setlength{\mylinewidth}{\linewidth-7pt}%
\setlength{\mylengtha}{0.4\mylinewidth-2\arraycolsep}%
\setlength{\mylengthb}{0.6\mylinewidth-2\arraycolsep}%

\begin{longtable}{>{\centering\arraybackslash}m{\mylengtha} | m{\mylengthb}}
    \hline
    JSON variable name & Description \\
    \hline
    \texttt{raw\_data\_dir} & Path where the directories that contain the raw data files of each well are \\
    \hline
    \texttt{perturbations\_and\_wells} & Dictionary. The dictionary keys must be the directories for each perturbation, while the elements (a list) must contain the directory name of each well (one list entrance per well) \\
    \hline
    \texttt{output\_pp\_data\_path} & Path where the output folder of the notebook must be located \\
    \hline
    \texttt{output\_pp\_data\_dir\_name} & Folder name where the notebook output will be saved \\
    \hline
    \texttt{img\_saving\_mode} & Indicate the shape of the output images. To replicate the experiments of this work, this variable must be set to \texttt{original\_img\_and\_fixed\_size}, which means squared images of fixed size \\
    \hline
    \texttt{img\_size} & Integer. High and width of the output image (squared) \\
    \hline
    \texttt{return\_cell\_size\_ratio} & Binary. Indicate if cell size ratio (percentage of the image that is occupied by the cell nucleus measurements) must be added to the output metadata file. During the data augmentation, this information can be used to approximate the parameters of the distribution used to randomly vary the size of the cell nucleus \\
    \hline
    \texttt{background\_value} & Path and name (normally \texttt{secondary\_only\_relative\_normalisation.csv}) of the metadata file containing the per-channel background values \\
    \hline
    \texttt{subtract\_background} & Binary. Indicate if background color need to be subtracted from each channel \\
    \hline
    \texttt{cell\_cycle\_file} & Path and name (normally \texttt{cell\_cycle\_classification.csv}) of the metadata file containing the phase of each cell \\
    \hline
    \texttt{add\_cell\_cycle\_to\_metadata} & Binary. Indicate if cell phase must be add to the output metadata file \\
    \hline
    \texttt{well\_info\_file} & Path and name (normally \texttt{wells\_metadata.csv}) of the metadata file containing the information about well perturbation \\
    \hline
    \texttt{add\_well\_info\_to\_metadata} & Binary. Indicate if columns of \texttt{well\_info\_file} must be add to the output metadata file \\
    \hline
    \texttt{filter\_criteria} & List containing the metadata columns names that will be used in the quality control. For this work ["is\_border\_cell", "is\_polynuclei\_184A1", "is\_polynuclei\_HeLa", "cell\_cycle"] was used \\
    \hline
    \texttt{filter\_values} & List containing the filtered values for the columns indicated in \texttt{filter\_criteria}. For this work [1, 1, 1, "NaN"] was used \\
    \hline
    \texttt{aggregate\_output} & Indicate how to project each image channel into a number. Must be equal to "avg" (average) \\
    \hline
    \texttt{project\_into\_scalar} & Binary. Indicate if the channel scalar projection must be add to the output metadata file \\
    \hline
  \caption{Parameters to perform the raw data processing.}
  \label{table:row_data_in:params}
\end{longtable}

Roughly speaking, the notebook iterates over the specified wells sequentially. This means that for each well the notebook
\begin{enumerate}
  \item Reads the well metadata file \texttt{metadata.csv} and merge it with the general metadata files, \texttt{cell\_cycle\_classification.csv} and \texttt{wells\_metadata.csv}.
  \item Performs the quality control and select the ids (\hl{mapobject\_id\_cell}) that were approved.
  \item Converts\footnote{The library \texttt{mpp\_data\_V2.py} used to perform the raw data transformation, is almost entirely based on Dr. Hannah Spitzer library \texttt{mpp\_data.py}. I thank the Dr. Spitzer for providing me with her library for this work.} and saves the selected ids using the well protein readouts files \texttt{mpp.npy}, \texttt{x.npy}, \texttt{y.npy}, \texttt{mapobject\_ids.npy} and the general file \\ \texttt{secondary\_only\_relative\_normalisation.csv}.
\end{enumerate}

The notebook also saves at the end a general metadata file (\texttt{csv} file), containing the metadata of all the processed wells.


\section{TensorFlow Dataset and image preprocessing implementation notes}
\label{sec:appendix:tfds}
%% Magic command to compile root document
% !TEX root = ../../thesis.tex

%% Reset glossary to show long gls names
\glsresetall

After the raw data was processed and converted into images of single cell nucleus (see section \ref{ch:dataset} and appendix \ref{sec:appendix:raw_data}), it is possible to build a \gls{tfds} data can easily end efficiently feed data into a model built in TensorFlow. A \acrlong{tfds} is build by writing a subclass of the \texttt{tensorflow\_datasets.core.DatasetBuilder} class (for more information, please refer to the \href{https://www.tensorflow.org/datasets/add_dataset}{official documentation}), and there are some steps that need to be followed to do so. The easiest way to build a \gls{tfds}, is by running the bash script \texttt{Create\_tf\_dataset.sh}, which executes this steps. The script needs to be executed (and located) in the same directory where the folder containing the Python code to build the dataset is

\begin{lstlisting}[language=Bash]
  ./Create_tf_dataset.sh -o /Path_to_save_TFDS -n Folder_name_containing_the_TFDS_builder_code -p /Path_to_parameters_files/parameters_file.json -e my_conda_env_name
\end{lstlisting}

\noindent where the flag \texttt{-o} indicates the path where \gls{tfds} will be located after it is built, \texttt{-n} the name of the directory (not the path, the folder name in the same directory as the script) containing the python (builder) code for required dataset, \texttt{-p} the absolute path and name of the input parameters file\footnote{This file needs to be \texttt{JSON} format and located in a directory named \texttt{Parameters}, which needs to be located inside the directory specified by the \texttt{-n} flag.} and \texttt{-e} the name of the Anaconda environment used to build the \gls{tfds}.
The specified Anaconda environment is necessary not just to build the \gls{tfds}, but also to register it in the environment. If a \gls{tfds} is not registers in an Anaconda environment, the \texttt{tensorflow\_datasets}\footnote{See the documentation \href{https://www.tensorflow.org/datasets}{here}.} library will not find it, and the user will not be able to call it and use it. Therefore, to register a custom \gls{tfds} in another environment, one just have to execute the \texttt{Create\_tf\_dataset.sh} script specifying the new environment using the \texttt{-e} flag. If the \gls{tfds} was already built by another environment, python will just register the dataset under the new environment and it will not build it again.

Table \ref{table:tfds_in:params} provides an explanation of the variables contained in the parameters file.

% set table lengths
\setlength{\mylinewidth}{\linewidth-7pt}%
\setlength{\mylengtha}{0.35\mylinewidth-2\arraycolsep}%
\setlength{\mylengthb}{0.65\mylinewidth-2\arraycolsep}%

\begin{longtable}{>{\centering\arraybackslash}m{\mylengtha} | m{\mylengthb}}
    \hline
    JSON variable name & Description \\
    \hline
    \texttt{data\_source\_parameters} & Path where the parameters file used in the raw data processing is (see appendix \ref{sec:appendix:raw_data}). Several parameters from this file are used to build the \gls{tfds} \\
    \hline
    \texttt{perturbations} & A list containing the names of the perturbations to be included in the \gls{tfds}. For instance, ["normal", "DMSO"] \\
    \hline
    \texttt{cell\_cycles} & A list containing the names of the cell phases to be included in the \gls{tfds}. For instance, ["G1", "S", "G2"] \\
    \hline
    \texttt{train\_frac} & Scalar between 0 and 1. Proportion of the data to include in the train set \\
    \hline
    \texttt{val\_frac} & Scalar between 0 and 1. Proportion of the data to include in the validation set. Proportion of the data to include in the test set is 1 - \texttt{train\_frac} - \texttt{val\_frac} \\
    \hline
    \texttt{seed} & Scalar. For reproducibility of the train, val and test split \\
    \hline
    \texttt{percentile} & Scalar between 0 and 100. Percentile used in clipping and/or linear scaling and/or standardization \\
    \hline
    \texttt{apply\_clipping} & Binary. If 1, per-channel clipping is applied using the channel percentile \\
    \hline
    \texttt{apply\_mean\_extraction} & Binary. If 1, per-channel mean shift is applied using the channel mean  \\
    \hline
    \texttt{apply\_linear\_scaling} & Binary. If 1, per-channel scaling is applied using the channel percentile \\
    \hline
    \texttt{apply\_z\_score} & Binary. If 1, per-channel standardization is applied using the channel parameters \\
    \hline
    \texttt{input\_channels} & List containing the name of the channels (elements of the column \hl{Marker identifier} of table \ref{table:tfds_in:channels}) to be included in the images contained in the \gls{tfds} \\
    \hline
    \texttt{output\_channels} & List of only ONE element containing the name of the channel to be used as the target variable (its protection, i.e. the channel average) \\
    \hline
  \caption{Parameters to perform the raw data processing.}
  \label{table:tfds_in:params}
\end{longtable}

As it is shown in table \ref{table:tfds_in:params}, the parameter \texttt{input\_channels} specifies the channels that will be included in the \gls{tfds} images (see table \ref{table:tfds_in:channels}). However, to avoid building a new dataset every time we change the input channels, all the channels are included here and then filtered in the model (see section \ref{sec:dataset:data_pp:dataset_creation} for a more detailed explanation).

A sample parameter file (\texttt{tf\_dataset\_parameters\_sample.json}) is provided along with this work. It contains the parameters used in the Python script \\ \texttt{MPP\_DS\_normal\_DMSO\_z\_score.py}, to build the dataset \\ \texttt{mpp\_ds\_normal\_dmso\_z\_score} used to train the models in this work.

\begin{table}[!ht]
  \centering
  \resizebox{0.7\linewidth}{!}{%
  \begin{tabular}{c|c|c|c}
    Channel name & Marker identifier & Raw data id & TFDS id \\
    \hline
    DAPI & \texttt{00\_DAPI} & 0 & 0 \\
    \hline
    H2B & \texttt{07\_H2B} & 1 & 1 \\
    \hline
    CDK9\_pT186 & \texttt{01\_CDK9\_pT186} & 2 & 2 \\
    \hline
    CDK9 & \texttt{03\_CDK9} & 3 & 3 \\
    \hline
    GTF2B & \texttt{05\_GTF2B} & 4 & 4 \\
    \hline
    SETD1A & \texttt{07\_SETD1A} & 5 & 5 \\
    \hline
    H3K4me3 & \texttt{08\_H3K4me3} & 6 & 6 \\
    \hline
    SRRM2 & \texttt{09\_SRRM2} & 7 & 7 \\
    \hline
    H3K27ac & \texttt{10\_H3K27ac} & 8 & 8 \\
    \hline
    KPNA2\_MAX & \texttt{11\_KPNA2\_MAX} & 9 & 9 \\
    \hline
    RB1\_pS807\_S811 & \texttt{12\_RB1\_pS807\_S811} & 10 & 10 \\
    \hline
    PABPN1 & \texttt{13\_PABPN1} & 11 & 11 \\
    \hline
    PCNA & \texttt{14\_PCNA} & 12 & 12 \\
    \hline
    SON & \texttt{15\_SON} & 13 & 13 \\
    \hline
    H3 & \texttt{16\_H3} & 14 & 14 \\
    \hline
    HDAC3 & \texttt{17\_HDAC3} & 15 & 15 \\
    \hline
    KPNA1\_MAX & \texttt{19\_KPNA1\_MAX} & 16 & 16 \\
    \hline
    SP100 & \texttt{20\_SP100} & 17 & 17 \\
    \hline
    NCL & \texttt{21\_NCL} & 18 & 18 \\
    \hline
    PABPC1 & \texttt{01\_PABPC1} & 19 & 19 \\
    \hline
    CDK7 & \texttt{02\_CDK7} & 20 & 20 \\
    \hline
    RPS6 & \texttt{03\_RPS6} & 21 & 21 \\
    \hline
    Sm & \texttt{05\_Sm} & 22 & 22 \\
    \hline
    POLR2A & \texttt{07\_POLR2A} & 23 & 23 \\
    \hline
    CCNT1 & \texttt{09\_CCNT1} & 24 & 24 \\
    \hline
    POL2RA\_pS2 & \texttt{10\_POL2RA\_pS2} & 25 & 25 \\
    \hline
    PML & \texttt{11\_PML} & 26 & 26 \\
    \hline
    YAP1 & \texttt{12\_YAP1} & 27 & 27 \\
    \hline
    POL2RA\_pS5 & \texttt{13\_POL2RA\_pS5} & 28 & 28 \\
    \hline
    U2SNRNPB & \texttt{15\_U2SNRNPB} & 29 & 29 \\
    \hline
    NONO & \texttt{18\_NONO} & 30 & 30 \\
    \hline
    ALYREF & \texttt{20\_ALYREF} & 31 & 31 \\
    \hline
    COIL & \texttt{21\_COIL} & 32 & 32 \\
    \hline
    BG488 & \texttt{00\_BG488} & 33 & 33 \\
    \hline
    BG568 & \texttt{00\_BG568} & 34 & 34 \\
    \hline
    EU & \texttt{00\_EU} & 35 & NA \\
    \hline
    SRRM2\_ILASTIK & \texttt{09\_SRRM2\_ILASTIK} & 36 & 35 \\
    \hline
    SON\_ILASTIK & \texttt{15\_SON\_ILASTIK} & 37 & 36 \\
    \hline
    Cell mask & NA & NA & 37 \\
    \hline
  \end{tabular}%
  }
  \caption{Image channels. Column \hl{Raw data id} shows the channel id used in the raw data, while column \hl{TFDS id} shows the channel id used in the TensorFlow dataset.}
  \label{table:tfds_in:channels}
\end{table}


\section{Model training implementation notes}
\label{sec:appendix:Model_training_IN}
%%% Magic command to compile root document
% !TEX root = ../../thesis.tex

%% Reset glossary to show long gls names
\glsresetall

This appendix is intended to provide a brief explanation of how to run the Python script (Jupyter Notebook) responsible for training the models used in this work. In addition, here we also provide a short explanation of the parameter file that must be specified to train any model.

Since data augmentation techniques can be selected independently for each trained model, their corresponding hyperparameters are also explained here.

The Jupyter Notebook responsible for training the models is the one that requires the largest number of parameters. However, the function \texttt{set\_model\_default\_parameters} (in the \texttt{Utils.py} library) provides default values for all the parameters. Therefore, if some hyperparameter is not specified here or in section \ref{sec:methodology:models}, then the value used was the one specified in that function.

To train a model, one has to open the Python Jupyter Notebook \\
\noindent\texttt{Model\_training\_class.ipynb} and replace the variable \texttt{PARAMETERS\_FILE} with the absolute path and name of the input parameters file before running the notebook. A sample parameter file (\texttt{Train\_model\_sample.json}) is provided along with this work. It contains the parameters used to train the \hl{Simple} \gls{cnn} (see section \ref{sec:methodology:simple_CNN}), using the data augmentation techniques specified in section \ref{sec:methodology:data:augm}.
Table \ref{table:model_train_in:params} provides an explanation of some of the model training parameters, while table \ref{table:model_train_in:params_da} an explanation of some of the data augmentation parameters. Although the training and data augmentation parameters are specified in separate tables, they must be in the same \texttt{JSON} parameter file (and also as items in the same dictionary).

% set table lengths
\setlength{\mylinewidth}{\linewidth-7pt}%
\setlength{\mylengtha}{0.3\mylinewidth-2\arraycolsep}%
\setlength{\mylengthb}{0.7\mylinewidth-2\arraycolsep}%

\begin{longtable}{>{\centering\arraybackslash}m{\mylengtha} | m{\mylengthb}}
    \hline
    JSON variable name & Description \\
    \hline
    \texttt{model\_name} & Name of the architecture to be trained. Available: \texttt{simple\_CNN}, \texttt{ResNet50V2}, \texttt{Xception}, \texttt{Linear\_Regression} \\
    \hline
    \texttt{pre\_training} & Binary, whether or not use pretrained weights and biases as initial parameters. Only available for \texttt{ResNet50V2} or \texttt{Xception} architectures \\
    \hline
    \texttt{dense\_reg} & [$L_1$, $L_2$], where $L_1$ and $L_2$ are the regularization strengths for the dense layers weights\\
    \hline
    \texttt{conv\_reg} & [$L_1$, $L_2$], where $L_1$ and $L_2$ are the regularization strengths for the convolution layers weights \\
    \hline
    \texttt{bias\_l2\_reg} & $L_2$ regularization strengths for convolution and dense layers biases \\
    \hline
    \texttt{number\_of\_epochs} & Maximum number of epochs to train \\
    \hline
    \texttt{early\_stop\_patience} & For early stopping. Specify how many epochs at most the model can train without decreasing the loss function before stopping the training \\
    \hline
    \texttt{loss} & Loss function name. Available: \texttt{mse}, \texttt{huber}, \texttt{mean\_absolute\_error} \\
    \hline
    \texttt{learning\_rate} & Learning rate for Adam optimizer \\
    \hline
    \texttt{BATCH\_SIZE} & Batch size \\
    \hline
    \texttt{model\_path} & Path to save the models and checkpoints \\
    \hline
    \texttt{clean\_model\_dir} & Binary, whether or not to delete the content of the directory specified by \texttt{model\_path} \\
    \hline
    \texttt{tf\_ds\_name} & Name of the TFDS to be used during training\\
    \hline
    \texttt{local\_tf\_datasets} & Local path where the TFDSs are stored \\
    \hline
    \texttt{input\_channels} & List containing the name of the channels (elements of the column \hl{Marker identifier} of table \ref{table:tfds_in:channels}) to be included in the images contained in the \gls{tfds} \\
    \hline
    \texttt{shuffle\_files} & Binary, whether or not to shuffle the dataset at the beginning of each epoch \\
    \hline
    \texttt{seed} & Random seed to reproduce the shuffling of the TFDS \\
    \hline
  \caption{Model training parameters.}
  \label{table:model_train_in:params}
\end{longtable}

% set table lengths
\setlength{\mylinewidth}{\linewidth-7pt}%
\setlength{\mylengtha}{0.4\mylinewidth-2\arraycolsep}%
\setlength{\mylengthb}{0.6\mylinewidth-2\arraycolsep}%

\begin{longtable}{>{\centering\arraybackslash}m{\mylengtha} | m{\mylengthb}}
    \hline
    JSON variable name & Description \\
    \hline
    \texttt{random\_horizontal\_flipping} & Binary, whether or not to perform random horizontal flips on the training set \\
    \hline
    \texttt{random\_90deg\_rotations} & Binary, whether or not to perform random 90deg rotations on the training set \\
    \hline
    \texttt{CenterZoom} & Binary, whether or not to perform random center zoom-in/out on the training set \\
    \hline
    \texttt{CenterZoom\_mode} & Zoom proportion R.V. distribution. Available: \texttt{random\_normal}, \texttt{random\_uniform} \\
    \hline
    \texttt{Random\_channel\_intencity} & Binary, whether or not to perform per-channel random color shifting on the training set \\
    \hline
    \texttt{RCI\_dist} & Distribution of random color shifts. Available: \texttt{uniform}, \texttt{normal}. If \texttt{uniform} distribution selected ($U(-a, a)$), then $a=\mu+3\sigma$ \\
    \hline
    \texttt{RCI\_mean} & Mean $\mu$ for the distribution specified by \texttt{RCI\_dist} \\
    \hline
    \texttt{RCI\_stddev} & Standard deviation $\sigma$ for the distribution specified by \texttt{RCI\_dist} \\
    \hline
    \texttt{Random\_noise} & Binary, whether or not to add random normal noise ($N(0, \sigma)$) on the training set \\
    \hline
    \texttt{Random\_noise\_stddev} & Standard deviation corresponding to the normal distribution of random noise \\
    \hline
  \caption{Data augmentation parameters.}
  \label{table:model_train_in:params_da}
\end{longtable}


\section{VarGrad IG implementation notes}
\label{sec:appendix:VarGrad_IG_Experimental_Setup}

In order to generate the \acrlong{vg} \acrlong{ig} score maps, you must execute the python script \texttt{get\_VarGradIG\_from\_TFDS.py} specifying the parameters file
\begin{lstlisting}[language=Bash]
python get_VarGradIG_from_TFDS.py -p ./Parameters_file_name.json
\end{lstlisting}

Table \ref{table:imp_notes:VGIG_params} show all the parameters that need to be specified to execute \texttt{get\_VarGradIG\_from\_TFDS.py} successfully.

\begin{table}[!ht]
  \centering
  \begin{tabular}{c|c|c}
    Hyperparam & JSON variable name & Notes \\
    \hline
    $m$ & \texttt{IG\_m\_steps} & Number of steps to approximate \gls{ig} \\
    \hline
     &  & Baseline image for \gls{ig}. Available: "black" \\
    $x'$ &  \texttt{IG\_baseline} & for a simple black image and "noise" for an image \\
     &  &  filled with Gaussian noise ($\mu=0,\ \sigma=1$) \\
    \hline
    $n$ & \texttt{VarGrad\_n\_samples} & Number of noisy images to compute \gls{vg} \\
    \hline
  \end{tabular}
  \caption{Parameters to compute score maps.}
  \label{table:imp_notes:VGIG_params}
\end{table}


\chapter{General Remarks}
\label{Appendix-general-remarks}
%% Magic command to compile root document
% !TEX root = ../../thesis.tex

This appendix contains general remarks relevant for this work, like a small explanation of the fluorescent markers used for the \gls{mpm} protocol.


\section{Indirect Immunofluorescence markers description}
\label{sec:appendix:if_markers}
%% Magic command to compile root document
% !TEX root = ../../thesis.tex

%% Reset glossary to show long gls names
\glsresetall

In order to capture the distribution and amount of proteins inside a cell nucleus, the \gls{mpm} protocol use a set of fluorescent markers called \gls{if}. Table \ref{table:apendix:if_markers} shows a description of the most relevant markers. The identifiers and ids corresponding to the markers described in table \ref{table:apendix:if_markers} can be consulted in table \ref{table:tfds_in:channels}.

Some of the markers used in the \gls{mpm} protocol are strongly related with the \hl{RNA polymerase} enzyme\footnote{An enzyme is a proteins that act as biological catalysts to accelerate chemical reactions.}. As it was explained in section \ref{sec:basics:bio_back} (see figure \ref{fig:BB:premrna_synth}), the RNA polymerase it the enzyme responsible for starting the transcription process of genes (i.e., copping a sequence from a section of the DNA into a \gls{pmrna} strand).

% set table lengths
\setlength{\mylinewidth}{\linewidth-7pt}%
\setlength{\mylengtha}{0.24\mylinewidth-2\arraycolsep}%
\setlength{\mylengthb}{0.76\mylinewidth-2\arraycolsep}%

%\begin{table}[!ht]
%  \centering
%  \begin{tabular}{>{\centering\arraybackslash}m{\mylengtha}|m{\mylengthb}} % m stands for middle (p:top, b:bottom)
\begin{longtable}{>{\centering\arraybackslash}m{\mylengtha} | m{\mylengthb}}
    \hline
    Marker name & Description \\
    \hline
    DAPI & \hl{4',6-Diamidino-2-Phenylindole}, or DAPI, is a fluorescent stain that binds strongly to adenine–thymine-rich regions in DNA \cite{kapuscinski1995dapi} \\
    \hline
    GTF2B & \hl{Transcription factor II B}, or TFIIB (also known as GTF2B), is an antibody that binds to the general transcription factor involved in the formation of the RNA polymerase II preinitiation complex \cite{lewin2004genes} \\
    \hline
    SRRM2 & \hl{Serine/arginine repetitive matrix protein 2}, or SRRM2, is an antibody that binds to the protein that in humans is encoded by the SRRM2 gene and which is required for pre-mRNA splicing as component of the spliceosome. Along with the protein SON, SRRM2 is essential for \gls{ns}\footnotemark formation \cite{ilik2020and} \\
    \hline
    SON & SON is protein that in humans is encoded by the SON gene. The protein binds to RNA and promotes pre-mRNA splicing, particularly of transcripts with poor splice sites. Along with the protein SRRM2, SON is essential for \gls{ns} formation \cite{ilik2020and} \\
    \hline
    SP100 & \hl{SP100 nuclear antigen\footnotemark}, or SP100, is a gene that encodes a subnuclear organelle and major component of the PML (promyelocytic leukemia)-SP100 nuclear bodies \cite{sp100} \\
    \hline
    PML & \hl{Promyelocytic Leukemia}, or PML, is a protein encoded by the PML gene. PML is a nuclear body involved in oncogenesis (tumor suppressor) and viral infection. This subnuclear domain has been reported to be rich in RNA and a site of nascent RNA synthesis, implicating its direct involvement in the regulation of gene expression \cite{boisvert2000promyelocytic} \\
    \hline
    PCNA & \hl{Proliferating Cell Nuclear Antigen}, or PCNA, is a DNA clamp that acts as a processivity factor for \hl{DNA polymerase} $\delta$\footnotemark in eukaryotic cells and is essential for replication \cite{kisielewska2005gfp} \\
    \hline
    NCL & \hl{Nucleolin}, or NCL, is an antibody that binds to a protein that in humans is encoded by the NCL gene. The protein is involved in the synthesis and maturation of ribosomes. It is located mainly in dense fibrillar regions of the nucleolus \cite{erard1988major} \\
    \hline
    POL2RA\_pS2 & \hl{RNA Polymerase II Phosphospecific (Ser2)}, or POL2RA\_pS2, is an antibody that binds to the largest subunit of the RNA polymerase II (which is the enzyme responsible for transcribing DNA into \gls{pmrna}) \cite{POLR2ApS2} \\
    \hline
    CDK9 & \hl{Cyclin-dependent kinase 9}, or CDK9, is a protein encoded by the CDK9 gene and is involved in the regulation of transcription. CDK9 is a member of the cyclin-dependent kinase (CDK) family, which includes two main subgroups of kinases, those that mainly regulate cell cycle progression (including CDK1, CDK2, and CDK4/6) and those that control transcriptional processes (including CDK7, CDK8, CDK9, CDK12, and CDK13) \cite{cassandri2020cdk9} \\
    \hline
    CDK9\_pT186 & \hl{Cyclin Dependent Kinase 9 Phospho-Thr186 Antibody}, or CDK9\_pT186, is a molecule derived from human CDK9 around the phosphorylation site of T186 \cite{CDK9pT186} \\
    \hline
    RB1\_pS807\_S811 & \hl{Retinoblastoma Protein pS807/pS811 Antibody}, or RB1\_pS807\_S811. Retinoblastoma Protein (RB1 or just RB) is a tumor suppressor protein, which prevents excessive cell growth by inhibiting cell cycle progression until the cell is ready to divide.  \cite{murphree1984retinoblastoma} \\
    \hline
    PABPN1 & \hl{Polyadenylate-Binding Nuclear Protein 1}, or PABPN1 (also known as PABP-2), is a protein encoded by the PABPN1 gene, which is involved in the addition of a Poly-A tail to the \gls{pmrna} during the splicing process (see figure \ref{fig:BB:splicing} on section \ref{sec:basics:transcription_process}) \cite{muniz2015poly} \\
    \hline
    SETD1A & \hl{SET Domain Containing 1A, Histone Lysine Methyltransferase}, or SETD1A. The protein encoded by this gene is a component of a histone methyltransferase (HMT) complex that produces mono-, di-, and trimethylated histone H3 at Lys4. Trimethylation of histone H3 at lysine 4 (H3K4me3) is a chromatin modification known to generally mark the transcription start sites of active genes \cite{SETD1A} \\
    \hline
    COIL & \hl{Coilin}, or COIL. The protein encoded by this gene is an integral component of Cajal bodies, which are nuclear suborganelles involved in the post-transcriptional modification of small nuclear and small nucleolar RNAs \cite{COIL} \\
    \hline
    EU & \hl{5-Ethynyl Uridine}, or EU, is a molecule that binds to newly transcribed RNA \cite{jao2008exploring}. This means that EU can be used to detect RNA synthesis in cells and/or predict \gls{tr} \\
    \hline
%  \end{tabular}
  \caption{\Acrlong{if} markers description. The first column shows the markers name, the second the identifier used on the implementation (parameters file) and the third a brief description of it.}
  \label{table:apendix:if_markers}
%\end{table}
\end{longtable}

\footnotetext[2]{The \gls{ns} (also known as \hl{Splicing speckles}) are structures inside the cell nucleus in which the \gls{pmrna} is transformed into a mature \gls{mrna} (see section \ref{sec:basics:transcription_process}) \cite{spector2011nuclear}.}

\footnotetext[3]{An \hl{antigen} is a molecule that triggers the formation of antibodies (by bounding to its specific antibody or B-cell antigen receptor) and can cause an immune response.}

\footnotetext[4]{DNA polymerase delta, or DNA Pol $\delta$, is an enzyme complex found in eukaryotes that is involved in DNA replication and repair.}


% all appendices behind backmatter will go without numbers
\backmatter

%% =============================================================================
%% Lists, glossaries, etc.
%% =============================================================================

%List of Figures
\listoffigures

\vspace*{1.5cm}

%List of Tables
\listoftables


%List of algorithms
% only in conjunction with algorithm2e
%\phantomsection% for hyperref
%\addcontentsline{toc}{chapter}{Algorithmenverzeichnis}%
%\markboth{Algorithmenverzeichnis}{Algorithmenverzeichnis}
%\listofalgorithms


%Algorithmenverzeichnis
% nur in Verbindung mit algorithm2e
%\phantomsection% fuer hyperref
%\addcontentsline{toc}{chapter}{Algorithmenverzeichnis}%
%\markboth{Algorithmenverzeichnis}{Algorithmenverzeichnis}
%\listofalgorithms

% Index
% Add index to table of contents
\addcontentsline{toc}{chapter}{Index}
\printindex

%Print the glossary
%\printglossaries
\printglossary[type=\acronymtype]

% Bibliography
\printbibliography[heading=bibintoc]

%In the final version, this should be empty and needn't be commented out. Someone who works sloppily should at least remember to comment out the fixme list, so that unfinished places are
%not so obvious for the examiners.
\listoffixmes
%
\end{document}
% ==============================================================================
% End of document
% ==============================================================================
